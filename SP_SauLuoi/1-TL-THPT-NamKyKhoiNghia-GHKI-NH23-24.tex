
\de{ĐỀ THI GIỮA HỌC KỲ I NĂM HỌC 2023-2024}{THPT NAM KỲ KHỞI NGHĨA}

%Câu 1...........................
\begin{bt}%[1D1H2-2][Dự án đề kiểm tra Toán 11 GHKI NH23-24- Bùi Thanh Cương]%[THPT Nam Kỳ Khởi Nghĩa]
	Cho $\sin x=-\dfrac{\sqrt{3}}{4}$ và $x \in\left(\pi ; \dfrac{3 \pi}{2}\right)$. Tính giá trị của $\cos x$, $\tan x$.
	\loigiai
	{Ta có $\sin^2x+\cos^2x=1$ nên $\cos^2x=1-\sin^2x=1-\dfrac{3}{16}=\dfrac{13}{16}\Rightarrow \cos x=\pm\dfrac{\sqrt{13}}{4}$.\\
		Do $x\in\left(\pi; \dfrac{3\pi}{2}\right)$ nên $\cos x<0$, ta chọn $\cos x=-\dfrac{\sqrt{13}}{4}$.\\
		Từ đó suy ra 
		$\tan x =\dfrac{\sin x }{\cos x}=\dfrac{-\dfrac{\sqrt{3}}{4}}{-\dfrac{\sqrt{13}}{4}}=\dfrac{\sqrt{39}}{13}$.\\
		Vậy giá trị  $\cos x= -\dfrac{\sqrt{13}}{4}$ và $\tan x=\dfrac{\sqrt{39}}{13}$.
	}
\end{bt}

%Câu 2...........................
\begin{bt}%[1D1H3-3][Dự án đề kiểm tra Toán 11 GHKI NH23-24- Bùi Thanh Cương]%[THPT Nam Kỳ Khởi Nghĩa]
	Cho $\tan x=\sqrt{2}$. Tính $\cos 2x$.
	\loigiai
	{Áp dụng công thức nhân đôi ta có $\cos 2x=\cos^2x-\sin^2x$. \hfill $(*)$\\
		Do $\tan x=\sqrt{2}$ nên $\cos x\neq 0$. Chia tử và mẫu biểu thức $(*)$ cho $\cos^2x$ ta được
		\[ \cos 2x=\dfrac{\dfrac{\cos^2x-\sin^2x}{\cos^2x}}{\dfrac{1}{\cos^2 x}}=\dfrac{1-\tan^2x}{1+\tan^2 x}=\dfrac{1-2}{1+2}=-\dfrac{1}{3}.\]  
		Vậy giá trị $\cos 2x=-\dfrac{1}{3}$.
	}
\end{bt}

%Câu 3...........................
\begin{bt}%[1D1V3-5]%[Dự án đề kiểm tra Toán 11 GHKI NH23-24- Bùi Thanh Cương]%[THPT Nam Kỳ Khởi Nghĩa]
	Rút gọn biểu thức $T=\dfrac{(\sin 4 x-\sin 2 x) \cdot \cos 5 x}{2 \cos ^2 4 x+\cos 2 x-1}$.
	\loigiai
	{Ta có
		\allowdisplaybreaks
		\begin{eqnarray*}
			T
			&=& \dfrac{(\sin 4 x-\sin 2 x) \cdot \cos 5 x}{2 \cos ^2 4 x+\cos 2 x-1} \\
			&=&\dfrac{(\sin 4 x-\sin 2 x) \cdot \cos 5 x}{(2\cos ^2 4 x-1)+\cos 2 x} \\
			&=& \dfrac{(\sin 4 x-\sin 2 x) \cdot \cos 5 x}{\cos 8x+\cos 2 x} \\
			&=&\dfrac{2\cos 3x\cdot \sin x\cdot \cos 5 x}{2\cos 5x\cdot \cos 3x} \\
			&=& \sin x.
		\end{eqnarray*}
		Vậy $T=\dfrac{(\sin 4 x-\sin 2 x) \cdot \cos 5 x}{2 \cos ^2 4 x+\cos 2 x-1}=\sin x$.
	}
\end{bt}

%Câu 4...........................
\begin{bt}% [1D1B5-5] [Dự án đề kiểm tra Toán 11 GHKI NH23-24- Phạm Văn Long]%[THPT Nam Kỳ Khởi Nghĩa]
	Giải phương trình sau $\cos 2 x+\cos \dfrac{3 \pi}{8}=0$.
	\loigiai{
		Ta có $$\cos 2 x+\cos \dfrac{3 \pi}{8}=0\Leftrightarrow \cos 2 x=\cos \dfrac{5\pi}{8} \Leftrightarrow \hoac{&x=\dfrac{5\pi}{16}+k\pi\\&x=-\dfrac{5\pi}{16}+k\pi} \quad (k \in \mathbb{Z}).$$
	}
\end{bt}

%Câu 5...........................
\begin{bt}%[1D2B2-5][Dự án đề kiểm tra Toán 11 GHKI NH23-24- Phạm Văn Long]%[THPT Nam Kỳ Khởi Nghĩa]
	Tìm số hạng đầu tiên và công sai của cấp số cộng $(u_n)$ biết $\heva{&u_1+2 u_5=0\\&S_4=14.}$
	\loigiai{
		Ta có $$\heva{&u_1+2 u_5=0\\&S_4=14}\Leftrightarrow \heva{&u_1+2(u_1+4d)=0\\&\dfrac{4\cdot (2u_1+3d)}{2}=14}\Leftrightarrow \heva{&3u_1+8d=0\\&2u_1+3d=7}\Leftrightarrow \heva{&u_1=8\\&d=-3.}$$
		Vậy $u_1=8$ và $d=-3$.
	}
\end{bt}

%Câu 6...........................
\begin{bt}%[1D2K2-6][Dự án đề kiểm tra Toán 11 GHKI NH23-24- Phạm Văn Long]%[THPT Nam Kỳ Khởi Nghĩa]
	Một công ty A tuyển nhân viên bằng hai phương án khi kí hợp đồng lao động.
	\begin{itemize}
		\item \textbf{Phương án 1:} Năm đầu tiên nhận mức lương $100$ triệu đồng, mỗi năm tiếp theo tăng thêm $12$ triệu đồng.
		\item \textbf{Phương án 2:} Quý đầu tiên nhận mức lương $15$ triệu đồng, mỗi quý tiếp theo tăng thêm $2{,}5$ triệu đồng.
	\end{itemize}
	Một người lao động quyết định kí hợp đồng lao động với công ty A trong $10$ năm. Để tổng tiền lương nhận được trong $10$ năm tốt hơn, người lao động nên chọn phương án nào?
	\loigiai{
		\begin{itemize}
			\item \textbf{Phương án 1:} Gọi $u_n$ là tiền lương nhận được sau $n$ năm.\\
			Ta có $u_n=u_{n-1}+12$ với $n\ge 2$ nên dãy $(u_n)$ là cấp số cộng với $u_1=100$ và $d=12$.\\
			Khi đó số tiền lương nhận được trong $10$ là $S_{10}=\dfrac{10\cdot (2\cdot 100+9\cdot 12)}{2}=1540$ (triệu).
			\item \textbf{Phương án 2:} Gọi $v_n$ là tiền lương nhận được sau $n$ quý.\\
			Ta có $v_n=v_{n-1} + 2{,}5$ với $n\ge 2$ nên dãy $(v_n)$ là cấp số cộng với $v_1=15$ và $d=2{,}5$.\\
			Một năm có $4$ quý nên $10$ năm có $40$ quý.\\
			Khi đó số tiền lương nhận được trong $10$ là $S_{40}=\dfrac{40\cdot (2\cdot 15+39\cdot 2{,}5)}{2}=2550$ (triệu).
		\end{itemize}
		Vậy người lao động nên chọn phương án 2 là tốt hơn.
	}
\end{bt}

%Câu 7...........................
\begin{bt}%[1H1B1-3]%[1H1B2-2]%[1H1B2-4]%[Dự án đề kiểm tra Toán 11 GHKI NH23-24 - Quan Ón]%[THPT Nam Kỳ Khởi Nghĩa - Tp HCM]
	Cho hình chóp $S.ABCD$ có đáy $ABCD$ là hình bình hành tâm $O$; $G$ là trọng tâm tam giác $SCD$; $H$ là trọng tâm tam giác $ACD$.
	\begin{enumerate}
		\item Tìm giao tuyến của hai mặt phẳng $(SAC)$ và $(SBD)$.
		\item Chứng minh hai đường thẳng $HG$ và $SA$ song song.
		\item Tìm giao điểm của đường thẳng $OG$ với mặt phẳng $(SAD)$.
	\end{enumerate}
	\loigiai{
		\begin{center}
			\begin{tikzpicture}[>=stealth,line join=round,line cap=round,font=\footnotesize,scale=0.9]
				\path 
				(0.57,3.34) coordinate (S)
				(0,0) coordinate (A)
				(-2,-2) coordinate (B)
				(3,-2) coordinate (C)
				(5,0) coordinate (D)
				(intersection of A--C and B--D) coordinate (O)
				($(C)!0.5!(D)$) coordinate (M)
				($(S)!0.5!(C)$) coordinate (N)
				($(A)!2/3!(M)$) coordinate (H)
				($(S)!2/3!(M)$) coordinate (G)
				($(D)-(O)+(N)$) coordinate (a)
				($(a)!1.6!(D)$) coordinate (x)
				(intersection of D--a and O--G) coordinate (E);
				\draw (S)--(B)--(C)--(D)--(S)--(C) (S)--(M) (D)--(N) (G)--(E)--(x);
				\draw[dashed] (B)--(A)--(D)--(B) (C)--(A)--(S)--(O)--(N) (A)--(M) (H)--(G)--(O);
				\foreach \l/\g in {A/180,B/-135,C/-45,D/0,O/-90,S/90,N/180,M/-45,G/85,H/-70,E/0}
				\draw[fill=black] (\l) circle (1pt) +(\g:.3) node{$\l$};
				\fill (x) node[right]{$x$};
			\end{tikzpicture}
		\end{center}
		\begin{enumerate}
			\item Ta có $S \in (SAC)\cap (SBD)$. $\hfill (1)$\\
			Vì $O$ là tâm của hình bình hành $ABCD$ nên $O \in AC$, $O \in BD$.\\
			Suy ra $\heva{&O \in AC,AC \subset (SAC)\\&O \in BD, BD \subset (SBD)} \Rightarrow O \in (SAC)\cap (SBD)$. $\hfill (2)$\\
			Từ $(1)$ và $(2)$, suy ra $SO = (SAC)\cap (SBD)$.
			\item Gọi $M$, $N$ lần lượt là trung điểm của các cạnh $CD$ và $SC$.\\
			Vì $G$ là trọng tâm tam giác $SCD$ nên $\dfrac{MG}{MS} = \dfrac{1}{3}$.\\
			Mặt khác, ta có $H$ là trọng tâm tam giác $ACD$ suy ra $\dfrac{MH}{MA} = \dfrac{1}{3}$.\\
			Do đó $\dfrac{MG}{MS} = \dfrac{MH}{MA} = \dfrac{1}{3} \Rightarrow HG \parallel SA$.
			\item Vì $G$ là trọng tâm tam giác $SCD$ nên $\dfrac{DG}{DN} = \dfrac{2}{3}$.\\
			Vì $H$ là trọng tâm tam giác $ACD$ nên $\dfrac{DH}{DO} = \dfrac{2}{3}$.\\
			Do đó $\dfrac{DG}{DN} = \dfrac{DH}{DO} = \dfrac{2}{3} \Rightarrow ON \parallel HG$.\\
			Mà $HG \parallel SA$ (cmt) nên $ON \parallel SA$.\\
			Xét $OG \subset (OGD)$.\\
			Ta có $\heva{&D \in (SAD)\cap (OGD)\\
			&ON \subset (OGD)\\&SA \subset (SAD)\\&ON \parallel SA \textrm{ (cmt)}} \Rightarrow (OGD)\cap (SAD) = Dx\parallel ON\parallel SA$.\\
			Trong $(OGD)$, gọi $E = OG \cap Dx$.\\
			Khi đó $\heva{&E \in OG\\&E \in Dx, Dx \subset (SAD)} \Rightarrow E = OG \cap (SAD)$.
		\end{enumerate}
	}
\end{bt}