
\de{ĐỀ THI GIỮA HỌC KỲ II NĂM HỌC 2022-2023}{THPT Lương Thế Vinh}
\begin{center}
	\textbf{PHẦN 1 - TRẮC NGHIỆM}
\end{center}
\Opensolutionfile{ans}[ans/ans]
\begin{ex}%[0T9Y1-3]%[Dự án đề kiểm tra GHK2 22-23-Nhật Thiện]%[THPT Lương Thế Vinh]
	Trong mặt phẳng tọa độ $Oxy$, cho hai điểm $A(5;7)$, $B(7;5)$. Tìm tọa độ của véc-tơ $\vec{AB}$.
	\choice
	{$\vec{AB}=(12;12)$}
	{$\vec{AB}=\left(\dfrac{7}{5};\dfrac{5}{7}\right)$}
	{\True $\vec{AB}=(2;-2)$}
	{$\vec{AB}=(-2;2)$}
\loigiai{
Ta có $\vec{AB}=(7-5;5-7)=(2;-2)$.
}
\end{ex}
\begin{ex}%[0T9Y3-2]%[Dự án đề kiểm tra GHK2 22-23-Nhật Thiện]%[THPT Lương Thế Vinh]
	Trong mặt phẳng tọa độ $Oxy$, phương trình đường tròn $(C)$ có tâm $I(-1;2)$, bán kính $R=6$ là
	\choice
	{\True $(x+1)^2+(y-2)^2=36$}
	{$x^2+y^2+2x-4y-36=0$}
	{$x^2+y^2+2x-4y-6=0$}
	{$(x+1)^2+(y-2)^2=6$}
\loigiai{
Phương trình đường tròn $(C)$ có tâm $I(-1;2)$, bán kính $R=6$ là
$$(x+1)^2+(y-2)^2=6^2=36.$$
}
\end{ex}
\begin{ex}%[0T7Y1-1]%[Dự án đề kiểm tra GHK2 22-23-Nhật Thiện]%[THPT Lương Thế Vinh]
	Tìm tất cả giá trị $m$ để đa thức $f(x)=(m-2)x^2-x+2$ là tam thức bậc hai.
	\choice
	{\True $m\ne 2$}
	{$m=3$}
	{$m=2$}
	{$m\ne 3$}
\loigiai{
Đa thức $f(x)$ là tam thức bậc hai khi và chỉ khi $m-2\ne 0\Leftrightarrow m\ne 2$.
}
\end{ex}
\begin{ex}%[0T9Y4-5]%[Dự án đề kiểm tra GHK2 22-23-Nhật Thiện]%[THPT Lương Thế Vinh]
	Trong các phương trình sau, phương trình nào là phương trình chính tắc của đường hypebol?
	\choice
	{$\dfrac{x^2}{5^2}-\dfrac{y^2}{3^2}=0$}
	{$\dfrac{x^2}{5^2}+\dfrac{y^2}{4^2}=1$}
	{$\dfrac{x^2}{4^2}-\dfrac{y^2}{5^2}=-1$}
	{\True $\dfrac{x^2}{5^2}-\dfrac{y^2}{4^2}=1$}
\loigiai{
Phương trình $\dfrac{x^2}{5^2}-\dfrac{y^2}{4^2}=1$ là phương trình chính tắc của đường hypebol.
}
\end{ex}
\begin{ex}%[0T9B3-2]%[Dự án đề kiểm tra GHK2 22-23-Nhật Thiện]%[THPT Lương Thế Vinh]
	Trong mặt phẳng tọa độ $Oxy$, cho hai điểm $A(-1;2)$ và $B(3;0)$. Khi đó đường tròn $(C)$ đường kính $AB$ có phương trình là
	\choice
	{$(x+2)^2+(y-1)^2=25$}
	{$(x-2)^2+(y+1)^2=5$}
	{\True $(x-1)^2+(y-1)^2=5$}
	{$(x-1)^2+(y-1)^2=2$}
\loigiai{
Ta có $\vec{AB}=(4;-2)$ suy ra bán kính $R=\dfrac{AB}{2}=\dfrac{\sqrt{4^2+(-2)^2}}{2}=\sqrt{5}$.\\
Tâm $I$ của đường tròn $(C)$ là trung điểm $AB$ suy ra $I(1;1)$.\\
Phương trình đường tròn $(C)$ có tâm $I(1;1)$, bán kính $R=\sqrt{5}$ là
$$(x-1)^2+(y-1)^2=(\sqrt{5})^2=5.$$
}
\end{ex}
\begin{ex}%[0T7Y2-1]%[Dự án đề kiểm tra GHK2 22-23-Nhật Thiện]%[THPT Lương Thế Vinh]
	Bất phương trình nào sau đây là bất phương trình bậc hai một ẩn?
	\choice
	{\True $x^2-3x+2\ge 0$}
	{$1-3x<0$}
	{$2x-y+1\le 0$}
	{$x^2+4x>x^2-3$}
\loigiai{
Bất phương trình $x^2-3x+2\ge 0$ là một bất phương trình bậc hai một ẩn.
}
\end{ex}
\begin{ex}%[0T9Y4-7]%[Dự án đề kiểm tra GHK2 22-23-Nhật Thiện]%[THPT Lương Thế Vinh]
	Tham số tiêu của parabol $y^2=3x$ là
	\choice
	{$p=\dfrac{3}{8}$}
	{$p=3$}
	{$p=\dfrac{3}{4}$}
	{\True $p=\dfrac{3}{2}$}
\loigiai{Phương trình chính tắc của parabol là $y^2=2px=3x$.\\
Vậy tham số tiêu của parabol là $p=\dfrac{3}{2}$.
}
\end{ex}
\begin{ex}%[0T7Y1-1]%[Dự án đề kiểm tra GHK2 22-23-Nhật Thiện]%[THPT Lương Thế Vinh]
	Với giá trị $x$ nào sau đây thì tam thức $f(x)=x^2-2x-3$ nhận giá trị dương?
	\choice
	{$x=0$}
	{$x=2$}
	{\True $x=4$}
	{$x=-1$}
\loigiai{
Thế $x=4$ vào $f(x)$, ta được $f(4)=4^2-2\cdot 4-3=1>0$ là mệnh đề đúng.
}
\end{ex}
\begin{ex}%[0T7B2-1]%[Dự án đề kiểm tra GHK2 22-23-Nhật Thiện]%[THPT Lương Thế Vinh]
	Tập nghiệm của bất phương trình $x^2-7x+12<0$ là
	\choice
	{$(-\infty;3]\cup [4;+\infty)$}
	{\True $(3;4)$}
	{$[3;4]$}
	{$(-\infty;3)\cup (4;+\infty)$}
\loigiai{
Tam thức bậc hai $f(x)=x^2-7x+12$ có hai nghiệm phân biệt $x_1=4$ và $x_2=3$, $a=1>0$ nên $f(x)$ âm với mọi $x$ thuộc khoảng $(3;4)$.\\
Vậy bất phương trình $x^2-7x+12<0$ có tập nghiệm là $(3;4)$.
}
\end{ex}
\begin{ex}%[0T9Y1-3]%[Dự án đề kiểm tra GHK2 22-23-Nhật Thiện]%[THPT Lương Thế Vinh]
	Trong mặt phẳng tọa độ $Oxy$, cho véc-tơ $\vec{a}=-2\vec{j}$. Khi đó tọa độ véc-tơ $\vec{a}$ là
	\choice
	{$\vec{a}=(-2;-2)$}
	{$\vec{a}=(-2;0)$}
	{$\vec{a}=(2;0)$}
	{\True $\vec{a}=(0;-2)$}
\loigiai{
Ta có $\vec{a}=-2\vec{j}=0\vec{i}-2\vec{j}$ suy ra tọa độ véc-tơ $\vec{a}=(0;-2)$.
}
\end{ex}
\begin{ex}%[0T9B2-2]%[Dự án đề kiểm tra GHK2 22-23-Nhật Thiện]%[THPT Lương Thế Vinh]
	Trong mặt phẳng tọa độ $Oxy$, cho điểm $A(-2;-3)$ và $B(5;1)$. Biết đường thẳng $\Delta$ đi qua gốc tọa độ và song song với đường thẳng $AB$. Phương trình tham số của đường thẳng $\Delta$ là
	\choice
	{$\heva{&x=1-2t\\&y=-3+t}$}
	{$\heva{&x=2-3t\\&y=4+5t}$}
	{$\heva{&x=1-3t\\&y=2t}$}
	{\True $\heva{&x=7t\\&y=4t}$}
\loigiai{
Đường thẳng $\Delta$ song song với đường thẳng $AB$ nên nhận $\vec{a}=\vec{AB}=(7;4)$ là một véc-tơ chỉ phương.\\
Phương trình tham số đường thẳng $\Delta$ qua gốc tọa độ $O(0;0)$, có véc-tơ chỉ phương $\vec{a}=(7;4)$ là
$$\heva{&x=7t\\&y=4t.}$$
}
\end{ex}
\begin{ex}%[0H2B1-4]%[Dự án đề kiểm tra GHK2 22-23-Nhật Thiện]%[THPT Lương Thế Vinh]
	Trong mặt phẳng tọa độ $Oxy$, cho hai véc-tơ $\vec{a}=(4;3)$, $\vec{b}=(-1;-7)$. Góc giữa hai véc-tơ $\vec{a}$ và $\vec{b}$ là
	\choice
	{$60^\circ$}
	{\True $135^\circ$}
	{$45^\circ$}
	{$30^\circ$}
\loigiai{
Gọi $\varphi$ là góc giữa hai véc-tơ $\vec{a}$ và $\vec{b}$. Ta có
$$\cos \varphi =\dfrac{\vec{a}\cdot \vec{b}}{|\vec{a}|\cdot |\vec{b}|}=\dfrac{4\cdot (-1)+3\cdot (-7)}{\sqrt{4^2+3^2}\cdot \sqrt{(-1)^2+(-7)^2}}=\dfrac{-\sqrt{2}}{2}.$$
Suy ra $\varphi=135^\circ$.
}
\end{ex}
\begin{ex}%[0T7B3-2]%[Dự án đề kiểm tra GHK2 22-23-Nhật Thiện]%[THPT Lương Thế Vinh]
	Tích các nghiệm của phương trình $\sqrt{3x^2-x-3}=\sqrt{x^2-x+1}$ là
	\choice
	{$0$}
	{\True $-2$}
	{$-\sqrt{2}$}
	{$2$}
\loigiai{
Ta có \begin{eqnarray*}
	& &\sqrt{3x^2-x-3}=\sqrt{x^2-x+1}\\
	&\Rightarrow& 3x^2-x-3=x^2-x+1\\
	&\Rightarrow& 2x^2-4=0\\
	&\Rightarrow& \hoac{&x=\sqrt{2}\\&x=-\sqrt{2}.}
\end{eqnarray*}
Thử lại, ta thấy $x=\sqrt{2}$ và $x=-\sqrt{2}$ đều là nghiệm của phương trình đã cho.\\
Vậy tích các nghiệm bằng $-2$.
}
\end{ex}
\begin{ex}%[0T7Y1-1]%[Dự án đề kiểm tra GHK2 22-23-Nhật Thiện]%[THPT Lương Thế Vinh]
	Biểu thức nào sau đây là tam thức bậc hai?
	\choice
	{$f(x)=x^3-3x+1$}
	{\True $f(x)=2x^2-5x+5$}
	{$f(x)=-3x+5$}
	{$f(x)=4x-7$}
\loigiai{
Biểu thức $f(x)=2x^2-5x+5$ là một tam thức bậc hai.
}
\end{ex}
\begin{ex}%[0T7B3-2]%[Dự án đề kiểm tra GHK2 22-23-Nhật Thiện]%[THPT Lương Thế Vinh]
	Tập nghiệm của phương trình $\sqrt{x+1}=x-1$ là
	\choice
	{$\{3;1\}$}
	{$\{3;2\}$}
	{$\emptyset$}
	{\True $\{3\}$}
\loigiai{
Ta có
\begin{eqnarray*}
	& &\sqrt{x+1}=x-1\\
	&\Rightarrow& x+1=(x-1)^2\\
	&\Rightarrow& x+1=x^2-2x+1\\
	&\Rightarrow& x^2-3x=0\\
	&\Rightarrow& \hoac{&x=0\\&x=3.}
\end{eqnarray*}
Thử lại, ta thấy $x=3$ là nghiệm phương trình đã cho.\\
Vậy tập nghiệm của phương trình là $\{3\}$.
}
\end{ex}
\begin{ex}%[0T9K2-5]%[Dự án đề kiểm tra GHK2 22-23-Nhật Thiện]%[THPT Lương Thế Vinh]
	Trong mặt phẳng tọa độ $Oxy$, cho đường thẳng $\Delta\colon x+y-2=0$ và hai điểm $A(1;3)$, $B(2;1)$. Biết điểm $M(a;b)$, $a>0$ thuộc đường thẳng $\Delta$ sao cho diện tích tam giác $MAB$ bằng $4$. Tổng $3a+5b$ bằng
	\choice
	{$60$}
	{\True $-12$}
	{$12$}
	{$-60$}
\loigiai{
Phương trình tổng quát đường thẳng đi qua $A$ và $B$ là $2x+y-5=0$.\\
Điểm $M$ thuộc $\Delta$ suy ra $a+b-2=0\Rightarrow b=2-a$.
Khoảng cách từ $M$ đến đường thẳng $AB$ là
$$h=\dfrac{|2a+b-5|}{\sqrt{2^2+1^2}}=\dfrac{|2a+(2-a)-5|}{\sqrt{5}}=\dfrac{|a-3|}{\sqrt{5}}.$$
Diện tích tam giác $MAB$ bằng
$$S=\dfrac{1}{2}\cdot h\cdot AB=\dfrac{1}{2}\cdot \dfrac{|a-3|}{\sqrt{5}}\cdot \sqrt{5}=\dfrac{|a-3|}{2}.$$
Mà $S=4\Rightarrow |a-3|=8\Rightarrow \hoac{&a=-5\\&a=11.}$ mà $a>0$ suy ra $a=11$ và $b=2-a=-9$.\\
Vậy tổng $3a+5b=3\cdot 11+5\cdot (-9)=-12$.
}
\end{ex}
\begin{ex}%[0T9B2-2]%[Dự án đề kiểm tra GHK2 22-23-Nhật Thiện]%[THPT Lương Thế Vinh]
	Trong mặt phẳng tọa độ $Oxy$, phương trình tổng quát đường thẳng $\Delta$ đi qua điểm $M(2;-7)$ và có véc-tơ pháp tuyến $\vec{n}=(-2;3)$ là
	\choice
	{\True $-2x+3y+25=0$}
	{$-2x+3y-17=0$}
	{$-2x+3y-25=0$}
	{$2x-3y-17=0$}
\loigiai{
Phương trình tổng quát đường thẳng $\Delta$ đi qua điểm $M(2;-7)$ và có véc-tơ pháp tuyến $\vec{n}=(-2;3)$ là
$$-2(x-2)+3(y+7)=0\Leftrightarrow -2x+3y+25=0.$$
}
\end{ex}
\begin{ex}%[0T9B1-3]%[Dự án đề kiểm tra GHK2 22-23-Nhật Thiện]%[THPT Lương Thế Vinh]
	Trong mặt phẳng tọa độ $Oxy$, cho hai véc-tơ $\vec{a}=(1;2)$ và $\vec{b}=(-3;4)$. Véc-tơ $2\vec{a}+3\vec{b}$ có tọa độ là
	\choice
	{$(11;16)$}
	{\True $(-7;16)$}
	{$(-4;16)$}
	{$(-11;16)$}
	\loigiai{
	Ta có $2\vec{a}+3\vec{b}=(2\cdot 1+3\cdot (-3);2\cdot 2+3\cdot 4)=(-7;16)$.
}
\end{ex}
\begin{ex}%[0T9G4-0]%[Dự án đề kiểm tra GHK2 22-23-Nhật Thiện]%[THPT Lương Thế Vinh]
	Một mảnh vườn hình elip có độ dài trục lớn bằng $12$ m, độ dài trục bé bằng $8$m. Người ta dự định trồng hoa trong một hình chữ nhật nội tiếp của elip như hình vẽ bên dưới. Hỏi diện tích trồng hoa lớn nhất có thể là bao nhiêu m$^2$?
	\begin{center}
		\begin{tikzpicture}[scale=1, font=\footnotesize, line join=round, line cap=round, >=stealth]
			\def\a{3}
			\def\b{1.25}
			\path 
			(0,0) coordinate (O)
			+(\a,0) coordinate (A)
			+(-\a,0) coordinate (A')
			+(0,\b) coordinate (B)
			+(0,-\b) coordinate (B')
			(A) arc (0:45:\a cm and \b cm) coordinate (M)
			(A) arc (0:135:\a cm and \b cm) coordinate (N)
			(A) arc (0:-45:\a cm and \b cm) coordinate (Q)
			(A) arc (0:-135:\a cm and \b cm) coordinate (P)
			;
			\draw (A) arc (0:180:\a cm and \b cm);
			\draw (A) arc (0:-180:\a cm and \b cm);
			\draw[pattern=north west lines] (M)--(N)--(P)--(Q)--cycle;
			\draw (A)--(A') (B)--(B');
			\node at ($(O)+(-3,-1.25)$) {$AA'=12\text{ m}$};
			\node at ($(O)+(-3,-1.75)$) {$BB'=8\text{ m}$};
			\foreach \p/\r in {A/0,A'/180,B/90,B'/-90}
			\fill (\p) circle (1.5pt) node[shift={(\r:3mm)}]{$\p$};
		\end{tikzpicture}
	\end{center}
	\choice
	{$\dfrac{576}{13}$ m$^2$}
	{$62$ m$^2$}
	{\True $48$ m$^2$}
	{$46$ m$^2$}
\loigiai{
\immini{Độ dài trục lớn bằng $12$ m suy ra $2a=12$ hay $a=6$.\\
Độ dài trục bé bằng $8$ m suy ra $2b=8$ hay $b=4$.\\
Phương trình Elip $(E)\colon \dfrac{x^2}{6^2}+\dfrac{y^2}{4^2}=1$.\\
Gọi $M(x_M;y_M)$ là một đỉnh hình chữ nhật với $x_M>0$; $y_M>0$.\\
Do $M\in (E)$ suy ra $\dfrac{x_M^2}{6^2}+\dfrac{y_M^2}{4^2}=1$.\\
Hình chữ nhật nội tiếp trong elip nên sẽ tạo ra $4$ hình chữ nhật nhỏ bằng nhau.\\
Diện tích một hình chữ nhật nhỏ là $x_M\cdot y_M$ (m$^2$).\\
Vậy diện tích trồng hoa là $4x_M\cdot y_M$ (m$^2$).}{
	\begin{tikzpicture}[scale=1, font=\footnotesize, line join=round, line cap=round, >=stealth]
		\def\a{3}
		\def\b{1.25}
		\path 
		(0,0) coordinate (O)
		+(\a,0) coordinate (A)
		+(-\a,0) coordinate (A')
		+(0,\b) coordinate (B)
		+(0,-\b) coordinate (B')
		(A) arc (0:45:\a cm and \b cm) coordinate (M)
		(A) arc (0:135:\a cm and \b cm) coordinate (N)
		(A) arc (0:-45:\a cm and \b cm) coordinate (Q)
		(A) arc (0:-135:\a cm and \b cm) coordinate (P)
		;
		\draw (A) arc (0:180:\a cm and \b cm);
		\draw (A) arc (0:-180:\a cm and \b cm);
		\draw[pattern=north west lines] (M)--(N)--(P)--(Q)--cycle;
		\draw (A)--(A') (B)--(B');
		\node at ($(O)+(-3,-1.25)$) {$AA'=12\text{ m}$};
		\node at ($(O)+(-3,-1.75)$) {$BB'=8\text{ m}$};
		\foreach \p/\r in {A/0,A'/180,B/90,B'/-90,M/45}
		\fill (\p) circle (1.5pt) node[shift={(\r:3mm)}]{$\p$};
	\end{tikzpicture}
}
\noindent Ta có 
\begin{eqnarray*}
	& &\dfrac{x_M^2}{6^2}+\dfrac{y_M^2}{4^2}\ge 2\cdot \dfrac{x_M}{6}\cdot \dfrac{y_M}{4}\\
	&\Rightarrow& 1\ge \dfrac{x_M\cdot y_M}{12}\\
	&\Rightarrow& x_M\cdot y_M\le 12\\
	&\Rightarrow& 4x_M\cdot y_M\le 48.
\end{eqnarray*}
Vậy diện tích trồng hoa lớn nhất là $48$ m$^2$.
}
\end{ex}
\begin{ex}%[0T9B4-1]%[Dự án đề kiểm tra GHK2 22-23-Nhật Thiện]%[THPT Lương Thế Vinh]
	Trong mặt phẳng tọa độ $Oxy$, elip $(E)\colon \dfrac{x^2}{20}+\dfrac{y^2}{11}=1$ có tiêu cự bằng
	\choice
	{$3$}
	{$9$}
	{\True $6$}
	{$18$}
\loigiai{
Ta có $a^2=20$, $b^2=11$ suy ra $c^2=a^2-b^2=9$ hay $c=3$.\\
Vậy tiêu cự của elip đã cho là $2c=6$.
}
\end{ex}



\Closesolutionfile{ans}
%\begin{center}
%	\textbf{ĐÁP ÁN}
%	\inputansbox{10}{ans/ans}	
%\end{center}
\begin{center}
	\textbf{PHẦN 2 - TỰ LUẬN}
\end{center}
\begin{bt}%[0T9Y2-5]%[Dự án đề kiểm tra HKII NH22-23- Nguyễn Sĩ Đạt]%[Lương Thế Vinh]
	Trong mặt phẳng tọa độ $Oxy$, cho điểm $M(-1;2)$ và đường thẳng $\Delta$ có phương trình $4x-3y-5=0$. Tính khoảng cách từ điểm $M$ đến đường thẳng $\Delta$.
	\loigiai
	{
		Ta có $\mathrm{d}(M,\Delta)=\dfrac{|4\cdot(-1)-3\cdot 2-5|}{\sqrt{4^2+(-3)^2}}=3$.
	}
\end{bt}

\begin{bt}%[0T9K2-2]%[Dự án đề kiểm tra HKII NH22-23- Nguyễn Sĩ Đạt]%[Lương Thế Vinh]
	Trong mặt phẳng tọa độ $Oxy$, viết phương trình đường thẳng $\Delta$ tiếp xúc với đường tròn $(C)\colon (x-1)^2+(y-2)^2=4$ biết đường thẳng $\Delta$ song song với đường thẳng $3x-4y+1=0$.
	\loigiai
	{
		Vì $\Delta\parallel d\colon 3x-4y+1=0$ nên phương trình đường thẳng $\Delta$ có dạng $3x-4y+c=0\,(c\neq 1)$.\\
		Đường tròn $(C)$ có tâm $I(1;2)$ và bán kính $R=2$. Khi đó
			\begin{center}
				$\mathrm{d}(I,\Delta)=R\Rightarrow \dfrac{|3\cdot1-4\cdot 2+c|}{\sqrt{3^2+(-4)^2}}=2 \Leftrightarrow |c-5|=10 \Leftrightarrow \hoac{&c=15\\&c=-5.}$
			\end{center}
		So với điều kiện, ta nhận $c=15,c=-5$.\\
		Vậy $\Delta\colon 3x-4y+15=0$ hay $\Delta\colon 3x-4y-5=0$.
	}
\end{bt}

