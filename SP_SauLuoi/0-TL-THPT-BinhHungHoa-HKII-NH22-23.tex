\de{ĐỀ THI HỌC KỲ II NĂM HỌC 2022-2023}{THPT Bình Hưng Hoà}

\begin{bt}[1,5 điểm]%[0T0B2-1]%[Dự án đề kiểm tra HKII NH22-23- Trương Đăng Khoa]%[THPT Bình Hưng Hòa]
	Sử dụng các dữ kiện trong khung để điền vào các chỗ trống $(1)$, $(2)$, $(3)$, $(4)$, $(5)$, $(6)$.
\begin{center}
		\begin{tabular}{|c|c|c|c|}
		\hline
		$3!$ & chỉnh hợp & $8!$ & tổ hợp \\
		\hline
		$\mathrm{C}_8^3$ & $24$ & hoán vị & $\mathrm{A}_8^3$ \\
		\hline
	\end{tabular}
\end{center}
	\begin{enumerate}
		\item Có bao nhiêu cách sắp xếp $8$ học sinh ngồi vào một dãy gồm $8$ ghế?\\
		Trả lời: Mỗi cách sắp xếp $8$ học sinh ngồi vào một dãy gồm $8$ ghế là một $(1)$ của $8$ học sinh. Do đó, số cách sắp xếp như vậy là $(2)$ cách.
		\item Từ tám chữ số $1$, $2$, $3$, $4$, $5$, $6$, $7$, $8$ có thể lập được bao nhiêu số tự nhiên có ba chữ số khác nhau?\\
		Trả lời: Mỗi số tự nhiên có ba chữ số khác nhau được lập từ các số $1$, $2$, $3$, $4$, $5$, $6$, $7$, $8$ là một $(3)$ chập $3$ của $8$ chữ số. Do đó, số số tự nhiên được lập là $(4)$ số.
		\item Tổ Một có $8$ thành viên. Có bao nhiêu cách cử $3$ bạn bất kì của tổ làm trực nhật?\\
		Trả lời: Mỗi cách phân công $3$ bạn từ $8$ bạn là một $(5)$ chập $3$ của $8$ bạn. Do đó, số cách phân công $3$ bạn đi trực nhật là $(6)$ cách.
	\end{enumerate}
\loigiai{
\begin{enumerate}
	\item Trả lời: Mỗi cách sắp xếp $8$ học sinh ngồi vào một dãy gồm $8$ ghế là một \textbf{hoán vị} của $8$ học sinh. Do đó, số cách sắp xếp như vậy là $\mathbf{8!}$ cách.
	\item Trả lời: Mỗi số tự nhiên có ba chữ số khác nhau được lập từ các số $1$, $2$, $3$, $4$, $5$, $6$, $7$, $8$ là một \textbf{chỉnh hợp} chập $3$ của $8$ chữ số. Do đó, số số tự nhiên được lập là $\mathbf{\mathrm{A}_8^3}$ số.
	\item Trả lời: Mỗi cách phân công $3$ bạn từ $8$ bạn là một \textbf{tổ hợp} chập $3$ của $8$ bạn. Do đó, số cách phân công $3$ bạn đi trực nhật là $\mathbf{\mathrm{C}_8^3}$ cách.
\end{enumerate}
}
\end{bt}

\begin{bt}[1,5 điểm]%[0T8B3-2]%[Dự án đề kiểm tra HKII NH22-23- Trương Đăng Khoa]%[THPT Bình Hưng Hòa]
	Khai triển nhị thức $(2x+1)^4$ và tìm hệ số $x^3$ trong khai triển.
	\loigiai{
		Ta có $\begin{aligned}[t]
			(2x+1)^4=&\ \mathrm{C}_4^0\cdot (2x)^4 +\mathrm{C}_4^1\cdot (2x)^3\cdot 1^1 +\mathrm{C}_4^2\cdot (2x)^2\cdot 1^2+\mathrm{C}_4^3\cdot (2x)\cdot 1^3+\mathrm{C}_4^4\cdot 1^4\\
			=&\ 16x^4+32x^3+24x^2+8x+1.
		\end{aligned}$\\
		Vậy hệ số $x^3$ trong khai triển là $32$.
	}
\end{bt}

\begin{bt}[1 điểm]%[0T8Y2-2]%[Dự án đề kiểm tra HKII NH22-23- Trương Đăng Khoa]%[THPT Bình Hưng Hòa]
 Một bệnh viện có $15$ bác sĩ nội khoa và $10$ bác sĩ ngoại khoa. Bệnh viện cần cử $3$ bác sĩ tham gia vào đội y tế hỗ trợ thiên tai. Có bao nhiêu lựa chọn sao cho đội y tế đó có cả bác sĩ nội khoa và bác sĩ ngoại khoa?
 \loigiai{
  Bệnh viện cần cử $3$ bác sĩ tham gia vào đội y tế có cả bác sĩ nội khoa và bác sĩ ngoại khoa hỗ trợ thiên tai, có các trường hợp sau.\\
 \textbf{Trường hợp 1.} Đội y tế gồm $1$ bác sĩ nội khoa và $2$ bác sĩ ngoại khoa.
 \begin{itemize}
 	\item Số cách chọn $1$ bác sĩ nội khoa có $15$ cách.
 	\item Số cách chọn $2$ bác sĩ ngoại khoa có $\mathrm{C}_{10}^2$ cách.
 \end{itemize}
Theo quy tắc nhân, ta có $15\cdot \mathrm{C}_{10}^2=675$ cách.\\
\textbf{Trường hợp 2.} Đội y tế gồm $2$ bác sĩ nội khoa và $1$ bác sĩ ngoại khoa.
 \begin{itemize}
	\item Số cách chọn $2$ bác sĩ nội khoa có $\mathrm{C}_{15}^2$ cách.
	\item Số cách chọn $1$ bác sĩ ngoại khoa có $10$ cách.
\end{itemize}
Theo quy tắc nhân, ta có $\mathrm{C}_{15}^2 \cdot 10=1050$ cách.\\
Vậy theo quy tắc cộng, ta có $675+1050=1725$.
}
\end{bt}

\begin{bt}[2 điểm]%[0T0K2-2]%[Dự án đề kiểm tra HKII NH22-23- Trương Đăng Khoa]%[THPT Bình Hưng Hòa]
 Trong hộp có $5$ quả cầu xanh, $3$ quả cầu đỏ, $2$ quả cầu vàng có kích thước và khối lượng như nhau. Lấy ngẫu nhiên từ trong hộp $4$ quả cầu. Tính xác suất để trong $4$ quả cầu được lấy ra có ít nhất $1$ quả cầu đỏ.
 \loigiai{
Số phần tử không gian mẫu là $n(\Omega)=\mathrm{C}_{10}^4$. \\
Gọi $A$ là biến cố \lq\lq có ít nhất $1$ quả cầu đỏ\rq\rq.\\
Khi đó $\overline{A}$ là biến có \lq\lq không có quả cầu đỏ\rq\rq.\\
Số cách chọn $4$ quả cầu không có quả cầu đỏ có $n(\overline{A})=\mathrm{C}_7^4$ cách.\\
Do đó xác suất để chọn được $4$ quả cầu không có quả cầu đó là $\mathrm{P}(\overline{A})=\dfrac{n(\overline{A})}{n(\Omega)}=\dfrac{\mathrm{C}_7^4}{\mathrm{C}_{10}^4}=\dfrac{1}{6}$.\\
Vậy $\mathrm{P} (A)=1-\dfrac{1}{6}=\dfrac{5}{6}$.
}
\end{bt}

\begin{bt}%[0H9B4-2]%[Dự án đề kiểm tra HKII NH22-23-Nguyễn Vương Hiển]%[Bình Hưng Hòa]
Trong mặt phẳng với hệ tọa độ $Oxy$, viết phương trình chính tắc của elip có độ dài trục lớn bằng $26$ và tiêu cự bằng $24$.
\loigiai{
\begin{itemize}
	\item Độ dài trục lớn $2a=26\Rightarrow a=13$.
	\item Tiêu cự $2c=24\Rightarrow c=12$.
	\item $b=\sqrt{a^2-c^2}=\sqrt{13^2-12^2}=5$.
\end{itemize}	
Vậy phương trình của elip là $(E)\colon\dfrac{x^2}{169}+\dfrac{y^2}{25}=1$.
}
\end{bt}
\begin{bt}%[0H9B3-3]%[Dự án đề kiểm tra HKII NH22-23-Nguyễn Vương Hiển]%[Bình Hưng Hòa]
Trong mặt phẳng với hệ tọa độ $Oxy$, cho đường tròn $(C)\colon(x-2)^2+(y+1)^2=8$. Viết phương trình tiếp tuyến $\Delta$ của $(C)$ biết $\Delta$ song song với đường thẳng $d\colon x-y+1=0$.
\loigiai{
\begin{itemize}
	\item $(C)$ có tâm $I(2;-1)$ và bán kính $R=2\sqrt{2}$.
	\item $d$ có véc-tơ pháp tuyến $\overrightarrow{n}=(1;-1)$.
	\item Vì $\Delta\parallel d$ nên $\Delta$ có véc-tơ pháp tuyến $\overrightarrow{n}=(1;-1)$.
	\item Phương trình $\Delta$ có dạng $x-y+c=0\,(c\ne1)$.\\
	$\Delta$ tiếp xúc $(C)$ khi và chỉ khi
	$$\mathrm{d}(I,\Delta)=\dfrac{|2+1+c|}{\sqrt{2}}=2\sqrt{2}\Leftrightarrow\hoac{&c=1\\&c=-7}\Leftrightarrow c=-7.$$
	Vậy phương trình tiếp tuyến $\Delta\colon x-y-7=0$.
\end{itemize}
}
\end{bt}
\begin{bt}%[0D0K2-2]%[Dự án đề kiểm tra HKII NH22-23-Nguyễn Vương Hiển]%[Bình Hưng Hòa]
Có ba chiếc hộp. Hộp thứ nhất chứa $10$ tấm thẻ đánh số từ $1$ đến $10$. Hộp thứ hai chứa $15$ tấm thẻ đánh số từ $1$ đến $15$. Hộp thứ ba chứa $20$ tấm thẻ đánh số từ $1$ đến $20$. Từ mỗi hộp rút ngẫu nhiên một tấm thẻ. Tính xác suất để tổng ba số ghi trên ba tấm thẻ là số chẵn.
\loigiai{
\begin{itemize}
\item Số phần tử của không gian mẫu là $n(\Omega)=\mathrm{C}_{10}^1\cdot\mathrm{C}_{15}^1\cdot\mathrm{C}_{20}^1=3000$.
\item Gọi $A$ là biến cố: \lq\lq Tổng ba số ghi trên ba tấm thẻ là số chẵn\rq\rq. 
\begin{itemize}
	\item Trường hợp $1\colon$ Có $1$ số chẵn và $2$ số lẻ.\\
Ta có $\mathrm{C}_5^1\cdot\mathrm{C}_7^1\cdot\mathrm{C}_{10}^1+\mathrm{C}_5^1\cdot\mathrm{C}_8^1\cdot\mathrm{C}_{10}^1+\mathrm{C}_5^1\cdot\mathrm{C}_8^1\cdot\mathrm{C}_{10}^1=1150$ cách.
\item Trường hợp $2\colon$ Có $3$ số chẵn\\
Ta có $\mathrm{C}_5^1\cdot\mathrm{C}_7^1\cdot\mathrm{C}_{10}^1=350$ cách.
\end{itemize}
\item Số phần tử của biến cố $A$ là $n(A)=1150+350=1500$.
\item Vậy xác suất của biến cố $A$ là
$$\mathrm{P}(A)=\dfrac{1500}{3000}=\dfrac{1}{2}.$$
\end{itemize}
}
\end{bt}
\begin{bt}%[0H9G2-6]%[Dự án đề kiểm tra HKII NH22-23-Nguyễn Vương Hiển]%[Bình Hưng Hòa]
Trong mặt phẳng với hệ tọa độ $Oxy$, cho đường tròn $(C)$ có phương trình\\ $x^2+y^2-6x+2y-15=0$. Tam giác $ABC$ nội tiếp đường tròn $C$ và đường phân giác trong của góc $A$ có phương trình $x+y-1=0$. Xác định tọa độ các đỉnh của tam giác $ABC$, biết đường thẳng $BC$ đi qua $M(4;-4)$ và điểm $A$ có hoành độ âm.
\loigiai{
\immini{\begin{itemize}
\item Đường tròn $(C)$ có tâm $I(3;-1)$ và bán kính $R=5$.
\item Giả sử $A$ và $E$ là giao điểm của đường phân giác trong góc $A$ và $(C)$.\\
Ta có $\heva{&x^2+y^2-6x+2y-15=0\\&x+y-1=0.}$\\
hay
$\heva{&2x^2+-10x-12=0\\&y=1-x}\Leftrightarrow\heva{&\hoac{&x=-1\\&x=6}\\&y=1-x.}$\\
Suy ra $A(-1;2)$ và $E(6;-5)$. 
\item Vì $BC\perp IE$ nên $BC$ có véc-tơ pháp tuyến $\overrightarrow{IE}=(3;-4)$.\\
Phương trình đường thẳng $BC$ là $$3(x-4)-4(y+4)=0\Leftrightarrow 3x-4y-28=0.$$
\item Tọa độ $B$ và $C$ là giao điểm của $BC$ và $(C)$.\\
Ta có $\heva{&x^2+y^2-6x+2y-15=0\\&3x-4y-28=0}$.\\ Suy ra $B\left(\dfrac{8}{5};-\dfrac{28}{5}\right)$ và $C(8;-1)$.
\end{itemize}
}{
\begin{tikzpicture}[>=stealth,line join=round,line cap=round,font=\footnotesize,xscale=.5,yscale=.5,transform shape]
	\pgfmathsetmacro\r{5}
	\path
	(3,-1)coordinate(I)
	(-1,2)coordinate(A)
	(6,-5)coordinate(E)
	(4,-4)coordinate(M)
	(8,-1)coordinate(C)
	($(E)!(C)!(I)$)coordinate(H)
	($2*(H)-(C)$)coordinate(B)
	;	
	\draw (I)circle(\r);
	\draw (A)--(E)--(I)(B)--(C) (B)--(A)--(C);
	\foreach \x/\g in {A/90,E/-90,I/90,M/90,C/0,B/-90}\fill(\x)circle(1pt)($(\x)+(\g:3mm)$)node{\x};
\end{tikzpicture}
}	
}
\end{bt}