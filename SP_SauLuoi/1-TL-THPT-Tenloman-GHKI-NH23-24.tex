
\de{ĐỀ THI GIỮA HỌC KỲ I NĂM HỌC 2023-2024}{THPT Ten lơ man}



\begin{bt}%[1D1B1-5]%ID%[Dự án đề kiểm tra Toán 11 GHKI NH23-24- DinhTri]%[THPT Tenloman - Tp HCM]
	\begin{enumerate}
		\item Cho $\cos a=\dfrac{1}{3}$. Tính giá trị của biểu thức $A=2-2 \cdot \sin ^2 a$.
		\item Cho $\sin a=\dfrac{4}{5}\left(\dfrac{\pi}{2}<a<\pi\right)$. Tính giá trị lượng giác: $\cos \left(\dfrac{\pi}{6}+a\right)$
		\item Tìm tập giá trị của hàm số $y=4 \cdot \sin 2 x+1$ trên tập hợp các số thực $\mathbb{R}$.
	\end{enumerate}
	\loigiai{
		\begin{enumerate}
			\item  $  \sin ^2 a=1-\left(\dfrac{1}{3}\right)^2=\dfrac{8}{9} \\
			 A=2 \cos ^2 a=2 \cdot\left(\dfrac{1}{3}\right)^2=\dfrac{2}{9}$
			\item $	 \cos a=\dfrac{-3}{5} \\
			 \cos \left(\dfrac{\pi}{6}+a\right)=\dfrac{-3}{5} \cdot \dfrac{\sqrt{3}}{2}-\dfrac{4}{5} \cdot \dfrac{1}{2}=-\dfrac{3 \sqrt{3}+4}{10}$
			\item $	 -1 \leq \sin 2 x \leq 1 \\
			 -3 \leq y \leq 5 \\
			 T=\left[-3;5\right]$
		\end{enumerate}
	}
\end{bt}
%%%%%%%%%%%%%%%%%%%%%%%%%%%%%%
%Câu 2...........................
\begin{bt}%[1D1K2-1]%ID%[Dự án đề kiểm tra Toán 11 GHKI NH23-24- DinhTri]%[THPT Tenloman - Tp HCM]
	Giải các phương trình lượng giác sau:
	\begin{enumerate}
		\item $\cos 2 x=\cos \dfrac{\pi}{5}$
		\item $\sin x=\cos 2 x$
	\end{enumerate}
	\loigiai{
		
		\begin{enumerate}
			\item $\cos 2x=\cos \dfrac{\pi}{5}\Leftrightarrow \hoac{&2x=\dfrac{\pi}{5}+k2\pi \\&2x=-\dfrac{\pi}{5}+k2\pi}\Leftrightarrow \hoac{&x=\dfrac{\pi}{10}+k\pi,k\in \mathbb{Z} \\&x=-\dfrac{\pi}{10}+k\pi,k\in \mathbb{Z}}$.
			\item $\cos \left(\dfrac{\pi}{2}-x\right)=\cos 2x\Leftrightarrow \hoac{&-3x=\dfrac{-\pi}{2}+k2\pi \\&x=-\dfrac{\pi}{2}+k2\pi}\Leftrightarrow \hoac{&x=\dfrac{\pi}{6}-\dfrac{k2\pi}{3},k\in \mathbb{Z} \\&x=-\dfrac{\pi}{2}+k2\pi,k\in \mathbb{Z}}$.
		\end{enumerate}
	}
\end{bt}
%%%%%%%%%%%%%%%%%%%%%%%%%%%
%Câu 3...........................
\begin{bt}%[1D1G1-5]%ID%[Dự án đề kiểm tra Toán 11 GHKI NH23-24- DinhTri]%[THPT Tenloman - Tp HCM]
	Chứng minh đẳng thức sau: $\cos x\left(\dfrac{\sin x}{1+\cos x}+\dfrac{1+\cos x}{\sin x}\right)=2 \cot x$
	\loigiai{
		$VT=\cos x\left(\dfrac{\sin^2x+(1+\cos x)^2}{(1+\cos x)\sin}\right)=\dfrac{2\cos x(1+\cos x)}{\sin x(1+\cos x)}=2\cot x=VP$.	
	}
\end{bt}
\begin{bt}%[1D1C4-8]
\immini{
	\begin{enumerate}
	\item Nếu độ cao của mực nước thủy triều được tính theo công thức $$h(t)=4+2\cos (\dfrac{\pi t}{12}+\dfrac{3\pi}{2})$$ với t là thời gian trong ngày tính theo giờ và độ cao mực nước tính theo đơn vị m. Tính độ cao của mực nước vào lúc 6h sáng.\\
	\item Tuabin gió chuyển đổi động lực di chuyển của gió thành năng lượng điện. Hình bên mô phỏng ba cánh quạt của tuabin gió, các điểm M, N và P là vị trí ban đầu của ba cánh quạt. Giả sử điểm O cách mặt đất 130m và mỗi cánh của tuabin gió dài 70m. Khi điểm N quay được 503 vòng thì khoảng cách từ M đến mặt đất là bao nhiêu biết quạt quay theo chiều kim đồng hồ.(Kết quả làm tròn đến chữ số thập phân thứ hai sau dấu phẩy).
\end{enumerate}}{\begin{tikzpicture}[scale=1, font=\footnotesize, line join=round, line cap=round,>=stealth]  
\def\R{3}
\path
(0,0) coordinate (O)
(20:\R) coordinate (M)
(120:\R) coordinate (N)
(-110:\R) coordinate (P)
(-\R-.2,0) coordinate (E)
(-\R-.2,-2*\R) coordinate (F)
;
\draw[dashed,thick,green] (O) circle (\R);
\draw (N)--(O)--(M) (O)--(P)(E)--(F);

\foreach \x/\g in {O/-45,M/45,N/120,P/-130}\draw[fill=black] (\x) circle (.05) +(\g:.3) node{$\x$};
\draw[fill=black] (E) circle (.05) ;
\draw[fill=black] (F) circle (.05) ;
\node[right] at (-\R-.2,-\R) {$130$ m};
\end{tikzpicture}}
\loigiai{\begin{enumerate}
		\item $\underline{\underline{h}(6)}=6 \mathrm{~m}$
		\item Ta có góc $\widehat{A O M}=15^{\circ}$.Quay 503 vòng thì các điểm vẫn quay lại vị trí ban đầu do đó khoảng cách từ khoảng cách từ $\mathrm{M}$ đến mặt đất là:
		$$
		130+70 \sin \left(15^{\circ}\right)=148.12 m
		$$
	\end{enumerate}
}
\end{bt}
\begin{bt}%[1H4H1-3]%[1H4C1-4]
	Cho hình chóp $S.ABCD$ có đáy $ABCD$ là hình bình hành. Gọi $M, N$ lần lượt là các điểm thuộc cạnh SA và SD sao cho $3SM=MA$, $SN=3ND$, G là trọng tâm tam giác ACD.\\
	a) Tìm giao tuyến của hai mặt phẳng $(SAC)$ và $(SBD)$.\\
	b) Tìm giao điểm H của đường thẳng BC và mặt phẳng $(GNM)$. Tính tỉ số $\dfrac{HB}{HC}$.
	\loigiai{
\immini{Ta có:\\
	$S \in(S A C) \cap(S B D)$\\
	$O=A C \cap B D$\\
	$O \in(S A C) \cap(S B D)$\\
	$\Rightarrow S O=(S A C) \cap(S B D)$\\
	$(G M N) \cap(A B C D)=G J$\\
	$G J \cap C B=H \Rightarrow(G M N) \cap B C=H$\\
	Tính tỉ lệ ra cho 0.25.}{
\begin{tikzpicture}
	\def\a{4}
	\path 	(0:0) coordinate (A)
			++(0:\a) coordinate (D)
			++(-130:\a/2) coordinate (C)
			($(A)+(C)-(D)$) coordinate (B)
			($(A)+(80:\a)$) coordinate (S)
	(intersection of A--C and B--D) coordinate (O)
	($(A)!3/4!(S)$) coordinate (M)
	($(S)!3/4!(D)$) coordinate (N)
	($(O)!1/3!(D)$) coordinate (G)
	(intersection of M--N and A--D) coordinate (I)
	(intersection of I--G and B--C) coordinate (H)
	(intersection of M--H and B--S) coordinate (J)
	(intersection of G--I and D--C) coordinate (I')
	(intersection of G--I and A--B) coordinate (J')
	 ;
\tkzDrawSegments[dashed](A,C B,D S,O M,G G,N N,M I',J')
\tkzDrawSegments(N,I I,I' J',H H,B)
	\draw[dashed,thick] 	(B)--(A)--(D)	(A)--(S);
	\draw[thick] 			(B)-- (C)--(D)
					(B)--(S)	(C)--(S)	(D)--(S);
	\foreach \x/\g in {A/135,B/-135,C/-45,D/45,S/90,M/180,N/0,O/0,G/0,I/0,H/180}
			\fill[black] 	(\x) circle (1pt)
			($(\g:4mm)+(\x)$) node {$\x$};	
\end{tikzpicture}}
}
\end{bt}