
\de{ĐỀ THI HỌC KỲ I NĂM HỌC 2022-2023}{THPT Gò Vấp}

\begin{bt}%[0T1B3-4]%[0T1B3-5]%Câu 1%[Dự Án Đề Kiểm Tra HKI NH22-23-DonLee]%[THPT Gò Vấp]
	Cho $A=(-\infty;3)$, $B=(1;5]$ và $C=[-2;4]$.
	\begin{enumEX}{2}
		\item Tìm $A\cap B$, $A\setminus B$.
		\item Tìm $B\cup C$, $A\cap C$.
	\end{enumEX}
	\loigiai{
		\begin{enumEX}{2}
			\item $A\cap B=(1;3)$, $A\setminus B=(-\infty;1]$.
			\item $B\cup C=[-2;5]$, $A\cap C=[-2;3)$.
	\end{enumEX}
	}
\end{bt}

\begin{bt}%[0T3B1-2]%Câu 2%[Dự Án Đề Kiểm Tra HKI NH22-23-DonLee]%[THPT Gò Vấp]
	Tìm tập xác định của hàm số $y=\dfrac{2}{(x+2)\sqrt{x+1}}$.
	\loigiai{
		Điều kiện xác định $\heva{&x+2\ne0\\&x+1>0}\Leftrightarrow\heva{&x\ne-2\\&x>-1}\Leftrightarrow x>-1$.\\
		Vậy tập xác định của hàm số là $\mathscr{D}=(-1;+\infty)$.
	}
\end{bt}

\begin{bt}%[0T3K1-4]%Câu 3%[Dự Án Đề Kiểm Tra HKI NH22-23-DonLee]%[THPT Gò Vấp]
	Xét sự đồng biến và nghịch biến của hàm số $y=f(x)=\dfrac{2x-1}{x+2}$ trên khoảng $(-\infty;-2)$.
	\loigiai{
		Tập xác định của hàm số là $\mathscr{D}=\mathbb{R}\setminus\{-2\}$.\\
		Ta có $y=f(x)=2-\dfrac{5}{x+2}$.\\
		$\forall x_1,\, x_2\in(-\infty;-2): x_1<x_2$, ta có\\
		$f(x_1)-f(x_2)=2-\dfrac{5}{x_1+2}-2+\dfrac{5}{x_2+2}=\dfrac{5}{x_2+2}-\dfrac{5}{x_1+2}$.\\
		Do $x_1<x_2$ nên $x_1+2<x_2+2$ suy ra\\
		\[\dfrac{5}{x_1+2}>\dfrac{5}{x_2+2} \Rightarrow f(x_1)-f(x_2)<0 \Rightarrow f(x_1)<f(x_2).\]
		Vậy hàm số đồng biến trên $(-\infty;-2)$.
	}
\end{bt}

\begin{bt}%[0T3B2-2]%Câu 4%[Dự Án Đề Kiểm Tra HKI NH22-23-DonLee]%[THPT Gò Vấp]
	Xác định parabol $(P)\colon y=ax^2+bx+2$, biết rằng $(P)$ đi qua điểm $M(1;5)$ và có trục đối xứng là đường thẳng $x=-\dfrac{1}{4}$.
	\loigiai{
		Do $M\in(P)$ suy ra $a+b+2=5\Rightarrow a+b=3$. \tagEX{1} \noindent
		$(P)$ có trục đối xứng là đường thẳng $x=-\dfrac{1}{4}\Rightarrow -\dfrac{b}{2a}=-\dfrac{1}{4}\Rightarrow 2a-4b=0$. \tagEX{2} \noindent
		Từ $(1)$ và $(2)$ ta có $\heva{&a+b=3\\&2a-4b=0}\Leftrightarrow\heva{&a=2\\&b=1.}$\\
		Vậy $(P):y=2x^2+x+2$.
	}
\end{bt}

\begin{bt}%[0T2K1-2]%Câu 5%[Dự Án Đề Kiểm Tra HKI NH22-23-DonLee]%[THPT Gò Vấp]
	Trong mặt phẳng $Oxy$, hãy xác định miền nghiệm của bất phương trình $-x+4y\le3(y-x+1)$.
	\loigiai{
		Ta có $-x+4y\le3(y-x+1)\Leftrightarrow2x+y-3\le0$.\\
		Ta vẽ đường thẳng $2x+y-3=0 \Leftrightarrow y=-2x+3$.\\
		Bảng giá trị 
		\begin{center}
			\begin{tabular}{|c|c|c|}
				\hline
				$x$ & $0$ & $1$ \\
				\hline
				$y$ & $3$ & $1$ \\
				\hline
			\end{tabular}
		\end{center}
		Ta có đồ thị như sau 
		\begin{center}
			\begin{tikzpicture}[scale=1, font=\footnotesize, line join=round, line cap=round,>=stealth]%<DTools>
				%Định nghĩa số liệu.
				\def\xmin{-4};\def\ymin{-4};\def\xmax{4};\def\ymax{4};
				%Định nghĩa điểm.
				\coordinate (O) at (0,0);
				%Trục Oxy.
				\draw[->] (\xmin,0)--(\xmax,0) node[below]{$x$};
				\draw[->] (0,\ymin)--(0,\ymax) node[left]{$y$};
				\fill (O) node[below left]{$O$} circle(1pt);
				%Giới hạn đồ thị.
				\clip ({\xmin-0.1},{\ymin-0.1}) rectangle ({\xmax+0.1},{\ymax+0.1});
				\foreach \x in {-3,-2,-1,1,2,3}{\draw (\x,0.1)--(\x,-0.1) node [below] {\footnotesize $\x$};}
				\foreach \y in {-3,-2,-1,1,2,3}{\draw (0.1,\y)--(-0.1,\y) node [left] {\footnotesize $\y$};}
				\fill (0,3) circle(2pt);
				\fill (1,1) circle(2pt);
				\draw[dashed] (1,0)--(1,1)--(0,1);
				%Vẽ đồ thị.
				\draw[thick,samples=100] plot[domain=-0.5:3.5](\x,{-2*\x+3});
				\fill[pattern=north east lines, smooth,opacity=0.5,pattern color=blue] (-0.5,4)--(3.5,-4)--(4,-4)--(4,4);	
			\end{tikzpicture}
		\end{center}
		Chọn điểm $O(0;0)$ thay vào bất phương trình $2x+y-3\le0$ ta được 
		$$2\cdot0+0-3\le0 \Leftrightarrow -3\le0\; (\text{đúng}).$$
		Suy ra điểm $O(0;0)$ thuộc miền nghiệm của bất phương trình $2x+y-3\le0$.\\
		Vậy miền nghiệm của bất phương trình $-x+4y\le3(y-x+1)$ là những miền không bị gạch nằm dưới đường thẳng và cả đường thẳng đó.
	}
\end{bt}


\begin{bt}%[0T6K4-1]%[Dự Án Đề Kiểm Tra HKI NH22-23-Trương Đăng Khoa]%[THPT Gò Vấp]%Câu 6
	Mẫu số liệu tuổi của $16$ nhân viên trong công ty $X$ như sau
	\begin{center}
		\begin{tabular}{cccccccc}
			$18$ & $30$ & $20$ & $22$ & $21$ & $40$ & $47$ & $22$\\
			$20$ & $19$ & $25$ & $30$ & $44$ & $40$ & $35$ & $47$
		\end{tabular}
	\end{center}
	Tìm số trung bình và tứ phân vị của mẫu số liệu trên.
	\loigiai{
		Sấp xếp lại mẫu số liệu theo thứ tự không giảm, ta được
		\begin{center}
			\begin{tabular}{cccccccccccccccc}
				$18$ & $19$ & $20$ & $20$ & $21$ & $22$ & $22$ & $25$
				& $30$ & $30$ & $35$ & $40$ & $40$ & $44$ & $47$ & $47$
			\end{tabular}
		\end{center}
		Số trụng bình là $$\overline{x}=\dfrac{18+19+20+20+21+22+22+25+30+30+35+40+40+44+47+47}{16}=30.$$
		Vì $n=16$ là số chẵn nên giá trị tứ phân vị thứ hai là $Q_2=\dfrac{25+30}{2}=27{,}5$.\\
		Tứ phân vị thứ nhất là trung vị của mẫu
		\begin{tabular}{cccccccc}
			$18$ & $19$ & $20$ & $20$ & $21$ & $22$ & $22$ & $25$.
		\end{tabular}\\
		Do đó $Q_1=\dfrac{20+21}{2}=20{,}5$.\\
		Tứ phân vị thứ ba là trung vị của mẫu
		\begin{tabular}{cccccccc}
			$30$ & $30$ & $35$ & $40$ & $40$ & $44$ & $47$ & $47$.
		\end{tabular}\\
		Do đó $Q_3=\dfrac{40+40}{2}=40$.}
\end{bt}

\begin{bt}%[0T3T2-5]%[Dự Án Đề Kiểm Tra HKI NH22-23-Trương Đăng Khoa]%[THPT Gò Vấp]%Câu 7
	Khi một vật được ném lên thì chiều cao $h$ $\mathrm{(m)}$ so với mặt đất theo thời gian $t$ (giây) được tính bởi hàm số $h(t)=-5t^2+v_0t+h_0$ với $v_0$ là vận tốc ban đầu, $h_0$ $\mathrm{(m)}$ là độ cao ban đầu của vật. Một quả bóng được cầu thủ Messi đá lên từ mặt đất với vận tốc ban đầu là $20$ $\mathrm{(m/s)}$. Hỏi
	\begin{enumEX}{1}
		\item Độ cao lớn nhất của quả bóng so với mặt đất?
		\item Sau bao lâu thì bóng chạm đất?
	\end{enumEX}
	\loigiai{
		Ta có $h_0=0$ (m) và $v_0=20$ (m/s).\\
		$\Rightarrow h(t)=-5t^2+20t$.
		\begin{enumEX}{1}
			\item Quả bóng có độ cao lớn nhất so với mặt đất sau khi nó di chuyển được $$t_1=-\dfrac{20}{2\cdot(-5)}=2\, (\text{giây}).$$
			Vậy độ cao lớn nhất của quả bóng so với mặt đất là
			$$h_1 =-5t_1^2+20t_1
			= -5\cdot 2^2+20\cdot 2=20\, \mathrm{(m)}.$$
			\item Ta có $h_2=0\Rightarrow -5t_2^2+20t_2=0\Leftrightarrow \hoac{&t_1=4\,(\text{nhận})\\&t_2=0\,(\text{loại}).}$\\
			Vậy sau $4$ giây thì bóng chạm đất.
	\end{enumEX}}
\end{bt}

\begin{bt}%[0T4B3-1]%[Dự Án Đề Kiểm Tra HKI NH22-23-Trương Đăng Khoa]%[THPT Gò Vấp]%Câu 8
	Cho tam giác $ABC$ có $a=6$, $b=5$, $c=8$. Tính $\cos A$ và diện tích của tam giác $ABC$.
	\loigiai{
		Xét tam giác $ABC$, ta có
		\begin{eqnarray*}
			&&a^2=b^2+c^2-2bc\cos A\\
			&\Leftrightarrow& 6^2=5^2+8^2-2\cdot 5\cdot 8\cdot \cos A\\
			&\Leftrightarrow& \cos A=\dfrac{53}{80}.
		\end{eqnarray*}
		Ta có $p=\dfrac{a+b+c}{2}=\dfrac{6+5+8}{2}=\dfrac{19}{2}$.\\
		Vậy $S_{ABC}=\sqrt{p(p-a)(p-b)(p-c)}=\sqrt{\dfrac{19}{2}\cdot\left(\dfrac{19}{2}-6\right)\cdot\left(\dfrac{19}{2}-5\right)\cdot\left(\dfrac{19}{2}-8\right)}=\dfrac{3\sqrt{399}}{4}$.}
\end{bt}

\begin{bt}%[0T5K2-5]%[Dự Án Đề Kiểm Tra HKI NH22-23-Trương Đăng Khoa]%[THPT Gò Vấp]%Câu 9
	Cho hình thoi $ABCD$ có đường chéo $AC=2a$, $BD=4a$. Tính $\left|\overrightarrow{CA}+\overrightarrow{AD}\right|$.
	\loigiai{
		\immini{Ta có $\left|\overrightarrow{CA}+\overrightarrow{AD}\right|=\left|\overrightarrow{CD}\right|=CD$.\\
			Gọi $O$ là giao điểm của $AC$ và $BD$\\
			Khi đó $\heva{&OC=a\\ &OD=2a}$\\
			Xét tam giác $OCD$ vuông tại $O$\\
			$ CD=\sqrt{OC^2+OD^2}=a\sqrt{5}$.\\
			Vậy $\left|\overrightarrow{CA}+\overrightarrow{AD}\right|=a\sqrt{5}$.}
		{\begin{tikzpicture}[scale=1]
				\coordinate [label=above:$A$] (A) at (2,1);
				\coordinate [label=left:$B$] (B) at (0,0);
				\coordinate [label=below:$C$] (C) at (2,-1);
				\coordinate [label=right:$D$] (D) at (4,0);
				\coordinate [label=above left:$O$] (O) at ($(A)!1/2!(C)$);
				\foreach \point in {A,B,C,D,O} \fill[black] (\point) circle (1pt);
				\draw (A)--(B)--(C)--(D)--(A) (A)--(C) (B)--(D);
				\draw ($(O)!6pt!(C)$)--($(O)!2!($($(O)!6pt!(C)$)!.5!($(O)!6pt!(D)$)$)$)--($(O)!6pt!(D)$);
		\end{tikzpicture}}
	}
\end{bt}

\begin{bt}%[0T5K3-1]%[Dự Án Đề Kiểm Tra HKI NH22-23-Trương Đăng Khoa]%[THPT Gò Vấp]%Câu 10
	Một người dùng một lực $\overrightarrow{F}$ tạo với mặt đất một góc $45^\circ$ và có độ lớn là $20$ $\mathrm{N}$ để kéo một vật dịch chuyển theo đường thẳng một đoạn $80\,\mathrm{m}$. Tính công sinh bởi lực $\overrightarrow{F}$.
	\begin{center}
		\begin{tikzpicture}[scale=1,font=\footnotesize,line join=round,line cap=round,>=stealth]
			\coordinate (A) at (-3,0.75);
			\coordinate (B) at (-3,-0.75);
			\coordinate (C) at (0,-0.75);
			\coordinate (D) at (0,0.75);
			\coordinate (E) at (0,0);
			\coordinate (F) at (45:2.83);
			\coordinate (G) at (4,0);
			\coordinate [label=below:$80\,\mathrm{m}$] (T) at ($(E)!.5!(G)$);
			\coordinate [label=right:$45^\circ$,yshift=-5pt,xshift=0pt] (U) at ($(E)!.25!(F)$);
			\coordinate [label=above:$\overrightarrow{F}$] (V) at ($(E)!.7!(F)$);
			\draw (-2.3,-1) circle (0.25);
			\draw (-0.7,-1) circle (0.25);
			\draw[->] (E)--(F);
			\draw[->] (E)--(G);
			\draw (A)--(B)--(C)--(D)--(A) (-3.5,-1.25)--(4.5,-1.25);
			\draw pic[draw,angle radius=6mm]{angle=G--E--F};
		\end{tikzpicture}
	\end{center}
	\loigiai{
		Công sinh bởi lực $\overrightarrow{F}$ là $A=\left|F\right|\cdot s\cdot\cos\alpha=20\cdot80\cdot\cos45^\circ=800\sqrt{2}\,\,\mathrm{J}$.}
\end{bt}


