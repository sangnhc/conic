
\de{ĐỀ THI GIỮA HỌC KỲ I NĂM HỌC 2022-2023}{THPT Trường Chinh}

\begin{bt}%[Dự án đề kiểm tra GHKI NH22-23-Võ Thị Thuỳ Trang]%[0T1Y1-3]
   Cho mệnh đề $P\colon$ \lq\lq $ \forall x\in\mathbb{R},\ x^2\geq 0 $\rq\rq . \\
   	Xét tính đúng, sai của mệnh đề trên và nêu mệnh đề phủ định của nó.
\loigiai{Mệnh đề $P\colon$  \lq\lq $ \forall x \in \mathbb{R}, \ x^2\geq 0 $\rq\rq ~(đúng).\\
Mệnh đề phủ định của $P$ là $\overline{P}\colon$ \lq\lq $\exists x \in \mathbb{R}, \ x^2< 0 $\rq\rq .	
}
\end{bt}
\begin{bt}%[Dự án đề kiểm tra GHKI NH22-23-Võ Thị Thuỳ Trang]%[0T1B2-1]
	Viết lại tập hợp $C=\left\{x \in \mathbb{N}  \Big| \left(x^2-5x+6\right)\left(2x+1\right)=0 \right\}$ dưới dạng liệt kê các phần tử.
	\loigiai{ Ta có 
		\[\left(x^2-5x+6\right)\left(2x+1\right)=0 \Leftrightarrow \hoac{&x^2-5x+6=0\\&2x+1=0}\Leftrightarrow \hoac{&x=3 \in \mathbb{N}\\&x=2\in \mathbb{N}\\&x=-\dfrac{1}{2}\notin \mathbb{N}. }\]
		Vậy $C=\left\{2;3\right\}$.
	}
\end{bt}

\begin{bt}%[Dự án đề kiểm tra GHKI NH22-23-Võ Thị Thuỳ Trang]%[0T1B3-2]
	Cho hai tập hợp $A=\left[-5;3\right)$ và $B=(1;+\infty)$. 
	Tìm $A \cup B$, $A\cap B$, $A \backslash B$, $A \backslash B$. 
	\loigiai{ 
		Ta có $A \cup B= [-5;+\infty)$ , $A\cap B= (1;3)$, $A \backslash B =[-5;1]$, $B \backslash A=[3;+\infty)$.	}
\end{bt}

\begin{bt}%[0T2B1-3]
	Biểu diễn miền nghiệm của bất phương trình $3x+2y\geq 300$ trên mặt phẳng tọa độ.
	\loigiai{\immini{ Vẽ đường thẳng $d\colon 3x+2y-300=0$ đi qua hai điểm $A(0;150)$, $B(100;0)$.\\
		Lấy điểm $O(0;0) \notin d$ thay vào bất phương trình $3x+2y\geq 300$ ta được $3\cdot 0+2 \cdot 0 \geq 300$ (sai).\\
		Vậy miền nghiệm của bất phương trình đã cho là miền không chứa điểm $O(0;0)$ (là miền không bị gạch chéo kể cả bờ $d$).
	}{
		\begin{tikzpicture}[line join = round, line cap = round,>=stealth,font=\footnotesize,scale=0.7]
			\def\xt{-2} \def\xp{4} \def\yd{-2} \def\yt{5}
			\def\a{-1.5} 
			\def\b{3}
			\draw[->] (\xt,0)--(\xp,0) node[below]{$x$};
			\draw[->] (0,\yd)--(0,\yt) node[left]{$y$};
			\fill (0,0) circle (1.5pt) node[below left]{$O$};
			\fill (0,3) circle (1.5pt) node[right]{$150$};
			\fill (0,2) circle (1.5pt) node[right]{$ $};
			\fill (0,1) circle (1.5pt) node[right]{$ $};
			\fill (2,0) circle (1.5pt) node[above right]{$100$};
			\fill (1,0) circle (1.5pt) node[above]{$ $};
		\begin{scope}
		\clip (\xt+0.1,\yd+0.1) rectangle (\xp-0.1,\yt-0.1);
		\def\a{-1.5} 
		\def\b{3}
		\draw[samples=150,smooth,domain=\xt:\xp] plot(\x,{(\a)*\x+(\b)});
		\fill[pattern=north east lines] (-2,-2) --
		plot[domain=\xt:\xp] (\x,{(\a)*\x+(\b)})
		-- (2,-2) -- cycle;
	\end{scope}
\end{tikzpicture}
}	
}
\end{bt}
\begin{bt}%[0T4B3-1]
	Cho tam giác $ABC$ có $a=6$, $b=5$, $c=8$. 
\begin{enumerate}
	\item Tính góc $\widehat{BAC}$.
	\item  Tính diện tích tam giác $ABC$.
\end{enumerate}
\loigiai{ 
	\begin{enumerate}
		\item Tính góc $\widehat{BAC}$.\\
		 Ta có $\cos A= \dfrac{b^2+c^2-a^2}{2bc}=\dfrac{5^2+8^2-6^2}{2\cdot 5 \cdot 8}=\dfrac{53}{80}$. Suy ra $\widehat{A}\simeq 48^\circ 30'$.
		\item  Tính diện tích tam giác $ABC$.\\
		Nửa chu vi tam giác $ABC$ là $p=\dfrac{6+5+8 }{2 }=\dfrac{19}{2}$.\\
		Diện tích tam giác $ABC$ 
		\[S=\sqrt{\dfrac{19}{2}\left(\dfrac{19}{2}-6\right)\left(\dfrac{19}{2}-5\right)\left(\dfrac{19}{2}-8\right)}=\dfrac{3\sqrt{399}}{4}.\]
	\end{enumerate}
}
\end{bt}

\begin{bt}%[Dự án đề kiểm tra GHKI NH22-23-Võ Thị Thuỳ Trang]%[0T5B2-5]
	Cho tam giác đều $ABC$ có cạnh bằng $a$. Tính $\left|\overrightarrow{AB}+\overrightarrow{AC}\right|$.
	\loigiai{\immini{ Dựng hình bình hành $ABDC$, ta có: $\overrightarrow{AB}+\overrightarrow{AC}=\overrightarrow{AD}$.\\
		Gọi $I$ là giao điểm của $AD$ và $BC$, ta có $I$ là trung điểm của $AC$ và $BD$. Vì tam giác $ABC$ đều nên
		$$AI\perp BC\Rightarrow AI=AB \cdot \sin B=a \cdot  \sin 60^\circ=\dfrac{a\sqrt{3}}{2}.$$
		Do đó  $\left|\overrightarrow{AB}+\overrightarrow{AC}\right|=\left|\overrightarrow{AD}\right|=AD=2AI=a\sqrt{3}$.
	}{\begin{tikzpicture}[scale=0.8, font=\footnotesize, line join=round, line cap=round, >=stealth]
		\coordinate [label=left:$A$]  (A) at (0,0);
		\coordinate [label=right:$B$] (B) at (3,0);
		\draw (A)--(B)-- ([turn]120:3) coordinate [label=above:$C$] (C)--(A);
		\draw (B)-- ([turn]60:3) coordinate [label=above:$D$] (D)--(C) (A)--(D);
		\coordinate[label= above:$I$] (I) at ($(B)!.5!(C)$);
		\foreach \diem in {A,B,C,D,I} \fill (\diem) circle(1.5pt);
\end{tikzpicture}}
	}
\end{bt}

\begin{bt}%[Dự án đề kiểm tra GHKI NH22-23-Võ Thị Thuỳ Trang]%[0T2T2-3]
	Bạn An kinh doanh hai mặt hàng Handmade là vòng tay và vòng đeo cổ. Mỗi vòng tay làm trong $4$ giờ và bán với giá $40$ nghìn đồng. Mỗi vòng đeo cổ làm trong $6$ giờ và bán với giá $80$ nghìn đồng. Mỗi tuần bạn An bán được không quá $15$ vòng tay và $4$ vòng đeo cổ. Tính số giờ làm tối thiểu trong tuần để số tiền bán được ít nhất  $400$ nghìn đồng.
	\loigiai{ Gọi $x$, $y$ lần lượt là số vòng tay và vòng đeo cổ cần làm (điều kiện $x\geq 0, y\geq 0$).\\
	Tổng số tiền bán ít nhất  $400$ nghìn đồng nên 
	\[40x+80\geq 400\Leftrightarrow x+2y\geq 10.\]
	Mỗi tuần bạn An bán được không quá $15$ vòng tay và $4$ vòng đeo cổ nên \[x\leq 15, y\leq 4.\]
	Do đó ta cần tìm $x,y$ thỏa
	\[\heva{&0\leq x\leq 15\\&0\leq y\leq 4\\&x+2y\geq 10.}\]
   Tổng số giờ làm $F=4x+6x$ nhỏ nhất.
   Biểu diễn miền nghiệm của hệ bất phương trình trên ta được miền nghiệm là phần không bị gạch bỏ như hình vẽ.\\
   
   \begin{center}
   	\begin{tikzpicture}[line join = round, line cap = round,>=stealth,font=\footnotesize,scale=0.7]
   		\def\xt{-1} \def\xp{18} \def\yd{-1} \def\yt{7}
   		\def\a{-1.5} 
   		\def\b{3}
   		\draw[->] (\xt,0)--(\xp,0) node[below]{$x$};
   		\draw[->] (0,\yd)--(0,\yt) node[left]{$y$};
   		\fill (0,0) circle (1.5pt) node[below left]{$O$};
   		\fill (0,4) circle (1.5pt) node[below right]{$4$};
   		\fill (0,5) circle (1.5pt) node[right]{$5 $};
   		\fill (15,0) circle (1.5pt) node[above right]{$15 $};
   		\fill (2,0) circle (1.5pt) node[below right]{$2$};
   		\fill (10,0) circle (1.5pt) node[above]{$10 $};
   		\fill (10,0) circle (1.5pt) node[below ]{$B $};
   		\fill (15,0) circle (1.5pt) node[below left ]{$C $};
   		\fill (15,4) circle (1.5pt) node[below left ]{$D $};
   		\fill (2,4) circle (1.5pt) node[above right ]{$A $};
   		\draw[dashed] (2,0)--(2,4)--(0,4);
   		\begin{scope}
   			\clip (\xt+0.1,\yd+0.1) rectangle (\xp-0.1,\yt-0.1);
   			\def\a{-0.5} 
   			\def\b{5}
   			\draw[samples=150,smooth,domain=\xt:\xp] plot(\x,{(\a)*\x+(\b)});
   			\fill[pattern=north east lines] (-2,-2) --
   			plot[domain=\xt:\xp] (\x,{(\a)*\x+(\b)})
   			-- (2,-2) -- cycle;
   			\fill[pattern=north east lines] (-1,6) --
   			plot[domain=\xt:\xp] (\x,{4})
   			-- (17,6) -- cycle;
   			\fill[pattern=north east lines] (-1,-1) --
   			plot[domain=\xt:\xp] (\x,{0})
   			-- (17,-1) -- cycle;
   			\fill[pattern=north east lines] (17,-1) --
   			(15,-1)-- (15,6)--(17,6) -- cycle;
   			\draw[samples=150,smooth,domain=\xt:\xp] plot(\x,{4});
   			\draw (15,-1)--(15,6);
   		\end{scope}
   	\end{tikzpicture}
   \end{center}
Tại $A(2;4)$ ta có $F=4\cdot 2+6\cdot 4=32$.\\
Tại $B(10;0)$ ta có $F=4\cdot 10+6\cdot 0=40$.\\
Tại $C(15;0)$ ta có $F=4\cdot 15+6\cdot 0=60$.\\
Tại $D(15;4)$ ta có $F=4\cdot 15+6\cdot 4=84$.\\
Ta thấy $F$ đạt giá trị nhỏ nhất bằng $32$ tại $A(2;4)$.\\
Vậy tổng số giờ làm nhỏ nhất bằng $32$.

}
\end{bt}

\begin{bt}%[Dự án đề kiểm tra GHKI NH22-23-Võ Thị Thuỳ Trang]%[0T4T3-1]
	\immini{Trên nóc tòa nhà có một cột Ăng-ten cao $5$ m. Từ một vị trí quan sát $A$ cao $7$m so với mặt đất có thể nhìn thấy đỉnh $B$ và chân $C$ của cột Ăng-ten, với hai góc tương ứng là $\widehat{DAB}=50^\circ$ và $\widehat{DAC}=40^\circ$ so với phương nằm ngang $AD$. Tính chiều cao $CH$ của tòa nhà.
}{\begin{tikzpicture}[scale=.5, font=\footnotesize, line join=round, >=stealth,thick]
	\coordinate (O) at (0,0);
	\def\nhathap{1} %so tang nha thap
	\def\nhacao{5} %so tang nha cao
	\def\kc{10} %khoang cach 2 nha
	\def\dai{1} %chieu dai 1 khoi
	\def\cao{2} %chieu cao 1 khoi
	\coordinate (H) at (\kc,0);
	\newcommand{\xaynha}[2]{
			\draw[very thick] (#1,#2)rectangle(#1+\dai,#2+\cao);
			\draw[very thick,fill=black] (#1,#2)rectangle(#1+\dai,#2+0.25*\cao);
			\draw[very thick] (#1+0.2*\dai,#2+0.25*\cao)rectangle(#1+0.8*\dai,#2+0.9*\cao);
		}
%	%Xay nha thap
	\foreach \j in {1,...,\nhathap}{
			\foreach \i in {0,...,4}{
					\xaynha{\i*1.1}{\j*\cao}
				}
		}
\draw[line width=0.2cm] (-0.1,\cao*\nhathap+\cao)--(5*\dai*1.1,\cao*\nhathap+\cao);
	\foreach \j in {1,...,\nhacao}{
			\foreach \i in {0,...,4}{
					\xaynha{\i*1.1+\kc}{\j*\cao}
				}
		}
	\draw[line width=0.2cm] (-0.1+\kc,\cao*\nhacao+\cao)--(5*\dai*1.1+\kc,\cao*\nhacao+\cao);
	\coordinate (A) at (5*\dai*1.1,\cao*\nhathap+\cao+0.1);
	\draw (A) node[above left]{$A$};
	\coordinate (C) at (\kc,\cao*\nhacao+\cao);
	\draw (C) node[above right]{$C$};
	\coordinate (B) at ($(C)+(0,2)$);
	\draw (B) node[above]{$B$};
	\coordinate (H) at (\kc,\cao);
	\draw (H) node[below]{$H$};
	\coordinate (D) at (\kc,\cao*\nhathap+\cao+0.1);
	\draw (D) node[above left]{$D$};
	\draw[very thick] (-0.5,\cao)--(\kc+5*\dai*1.2,\cao);
	\draw [dashed](A)--(D); \draw (B)--(A)--(C);\draw[line width=0.07cm] (C)--(B);
	\draw ($(B)+(135:0.5cm)$)--($(B)+(-45:0.5cm)$);
	\draw pic[draw,angle radius=7mm] {angle = D--A--B};
	\draw pic[draw,angle radius=3mm] {angle = D--A--C};

\end{tikzpicture}}
	\loigiai{ Ta có $\widehat{DAC}=40^\circ$. Suy ra $\widehat{DCA}=90^\circ-40^\circ=50^\circ$.\\
	 Do đó $\widehat{ACB}=180^\circ-50^\circ=130^\circ$.\\
	Lại có $\widehat{CAB}=50^\circ-40^\circ=10^\circ$.\\
		Áp dụng định lý sin cho tam giác $ABC$
		\[\dfrac{BC}{\sin A}=\dfrac{AB}{\sin C}\Leftrightarrow \dfrac{5}{\sin 10^\circ}=\dfrac{AB}{\sin 130^\circ}\Leftrightarrow AB=\dfrac{5\cdot \sin 130^\circ}{\sin 10^\circ}\simeq 22,06~ m.\]
		Trong tam giác vuông $ABD$ có $BD=AB\cdot \sin 50^\circ=22,06 \cdot \sin 50^\circ\simeq 16,89~m$.\\
		$CD=BD-BC=16,89-5=11,89~m$.\\
		Vậy chiều cao của tòa nhà 
		\[CH=CD+DH=11,89+7=18,89~m.\]
	}
\end{bt}

