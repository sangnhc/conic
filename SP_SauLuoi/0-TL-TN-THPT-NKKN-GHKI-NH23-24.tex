
\de{ĐỀ THI GIỮA HỌC KỲ I NĂM HỌC 2023-2024}{THPT Nam Kỳ Khởi Nghĩa}
\begin{bt}%[0D1H1-5]%[Dự án đề kiểm tra Toán 10 GHKI NH23-24- Phạm Phương]%[THPT Nam Kì Khởi Nghĩa]
	%(1.0 điểm)
	Xét tính đúng, sai của mỗi mệnh đề sau và phát biểu mệnh đề phủ định của nó.
	\begin{multicols}{2}
		\begin{enumerate}
			\item $P$: ``$6$ là số nguyên tố''.
			\item $Q$: ``$\exists x \in \mathbb{R}, x^2+2 x>0$''.
		\end{enumerate}
	\end{multicols}
	\loigiai{
		\begin{enumerate}
			\item $P$ là mệnh đề sai vì $6$ chia hết cho $2$, $3$.\\
			Mệnh đề phủ định của $P$ là $\overline{P}:$ \lq\lq$6$ không phải là số nguyên tố\rq\rq.
			\item $Q$ là mệnh đề đúng vì với $x=1$ ta có $1^2+2\cdot 1=3>0$.\\
			Mệnh đề phủ định của $Q$ là $\overline{Q}:$ \lq\lq$\forall x \in \mathbb{R}, x^2+2 x \leq 0$\rq\rq.
		\end{enumerate}
	}
\end{bt}
%Câu 2...........................
\begin{bt}%[0D1H2-1]%[Dự án đề kiểm tra Toán 10 GHKI NH23-24- Phạm Phương]%[THPT Nam Kì Khởi Nghĩa]
	%(1.0 điểm)
	Viết các tập hợp sau đây dưới dạng liệt kê các phần tử:
	\begin{enumerate}
		\item $A=\{x \mid x=2k+1, k \in \mathbb{N}, k \leq 4\}$.
		\item $B=\left\{x \in \mathbb{Z} \mid\left(x^2-5x+4\right)\left(x^2-\sqrt{2}x\right)=0\right\}$.
		\item $C=\left\{\dfrac{m}{m+5} \mid m \in \mathbb{Z},|m| \leq 2\right\}$.
		\item $D=\left\{x \in \mathbb{N} \mid \dfrac{6}{6-x} \in \mathbb{N}\right\}$.
	\end{enumerate}
	\loigiai{
		\begin{enumerate}
			\item Vì $k \in \mathbb{N}, k \leq 4\Rightarrow k=0;1;2;3;4$. Ta có bảng giá trị
			\begin{center}
				\begin{tabular}{|c|c|c|c|c|c|}
					\hline
					$k$& $0$ &$1$  &$2 $ & $3$& $4$ \\
					\hline
					$x=2k+1$&$1$  &$3$  &$5$  &$7$ &$9$ \\
					\hline
				\end{tabular}
			\end{center}
			Vậy $A=\{1;3;5;7;9\}$.
			\item $\left(x^2-5x+4\right)\left(x^2-\sqrt{2}x\right)=0 \Leftrightarrow \hoac{&x^2-5x+4=0\\& x^2-\sqrt{2}x=0}\Leftrightarrow\hoac{&x=1\\&x=4\\&x=0\\&x=\sqrt{2}.}$\\
			Mà $x \in \mathbb{Z}$ nên ta có 
			$B=\left\{0;1;4\right\}$.
			\item Vì $m \in \mathbb{Z},|m| \leq 2 \Rightarrow m=-2;-1;0;1;2$. Ta có bảng giá trị
			\begin{center}
				\begin{tabular}{|c|c|c|c|c|c|}
					\hline
					$m$& $-2$ &$-1$  &$0$ & $1$& $2$ \\
					\hline
					$\dfrac{m}{m+5}$&$-\dfrac{2}{3}$ &$-\dfrac{1}{4}$  &$0$  &$\dfrac{1}{6}$ &$\dfrac{2}{7}$ \\
					\hline
				\end{tabular}
			\end{center}
			$C=\left\{-\dfrac{2}{3};-\dfrac{1}{4};0;\dfrac{1}{6};\dfrac{2}{7}\right\}$.
			\item Vì $x \in \mathbb{N}$ và $\dfrac{6}{6-x} \in \mathbb{N}$ nên $6\,\vdots\,(6-x)$, do đó $(6-x)\in\{1;2;3;6\}\Rightarrow x\in\{5;4;3;0\}$.\\
			Vậy $D=\left\{0;3;4;5\right\}$.
		\end{enumerate}
	}
\end{bt}
%Câu 3...........................
\begin{bt}%[0D1H3-2]%[Dự án đề kiểm tra Toán 10 GHKI NH23-24- Phạm Phương]%[THPT Nam Kì Khởi Nghĩa]
	%(1.0 điểm)
	Cho các tập hợp $A=\{1;3;5;7;9\}$, $B=\{1;2;3;4;5\}$, $C=\{3;4;5;6\}$. Hãy xác định các tập hợp: $A \cap B$, $A \cup B$, $(A \cup B) \setminus C$, $(A \cap B) \setminus C$.
	\loigiai{
		\begin{itemize}
			\item $A \cap B=\{1;3;5\}$.
			\item $A \cup B=\{1;2;3;4;5;7;9\}$.
			\item $(A \cup B) \setminus C=\{1;2;7;9\}$.
			\item $(A \cap B) \setminus C=\{1\}$.
		\end{itemize}
	}
\end{bt}
\begin{bt}%[0D1H3-3]%[Dự án đề kiểm tra Toán 10 GHKI NH23-24- Lê Minh Thiện Anh]%[THPT Nam Kì Khởi Nghĩa]
	Xác định các tập hợp sau đây:
	\begin{multicols}{2}
		\begin{enumerate}
			\item $A=(-\infty ; 2] \cap(-3 ; 3)$;
			\item $B=[-2 ; 1) \cup(0 ; 4]$;
			\item $C=(-3 ; 4) \setminus(1 ; 5)$;
			\item $D=C_{\mathbb{R}}(-3 ; 2)$.
		\end{enumerate}
	\end{multicols}
	\loigiai{
		\begin{enumerate}
			\item Ta có
			\begin{center}
				\begin{tikzpicture}[line join = round, line cap = round,>=stealth,font=\footnotesize,scale=1]
					\draw[->] (-5,0)node[below]{$-\infty$}--(5,0);
					\path[pattern=north east lines] (-5,-3pt)rectangle(-3,3pt) (2,-3pt)rectangle(5,3pt);
					\foreach \so/\ngoac in {-3/(,2/],3/)} \path (\so,0) node{$\big\ngoac$} node[below=2pt]{$\so$};
				\end{tikzpicture}\\
				$A=(-\infty ; 2] \cap(-3 ; 3)=(-3;2]$;
			\end{center}
			\item Ta có
			\begin{center}
				\begin{tikzpicture}[line join = round, line cap = round,>=stealth,font=\footnotesize,scale=1]
					\draw[->] (-4,0)--(6,0);
					\path[pattern=north east lines] (-4,-3pt)rectangle(-2,3pt) (4,-3pt)rectangle(6,3pt);
					\foreach \so/\ngoac in {-2/[,0/(,1/),4/]} \path (\so,0) node{$\big\ngoac$} node[below=2pt]{$\so$};
				\end{tikzpicture}\\
				$B=[-2 ; 1) \cup(0 ; 4]=[-2;4]$;
			\end{center}	
			\item Ta có
			\begin{center}
				\begin{tikzpicture}[line join = round, line cap = round,>=stealth,font=\footnotesize,scale=1]
					\draw[->] (-4,0)--(6,0);
					\path[pattern=north east lines] (-4,-3pt)rectangle(-3,3pt) (1,-3pt)rectangle(6,3pt);
					\foreach \so/\ngoac in {-3/(,1/(,1/],4/),5/)} \path (\so,0) node{$\big\ngoac$} node[below=2pt]{$\so$};
				\end{tikzpicture}\\
				$C=(-3 ; 4)\setminus(1 ; 5)=(-3;1]$;
			\end{center}	
			\item Ta có
			\begin{center}
				\begin{tikzpicture}[line join = round, line cap = round,>=stealth,font=\footnotesize,scale=1]
					\draw[->] (-5,0) node[below]{$-\infty$}--(4,0) node[below]{$+\infty$};
					\path[pattern=north east lines]  (-3,-3pt)rectangle(2,3pt);
					\foreach \so/\ngoac in {-3/],2/[} \path (\so,0) node{$\big\ngoac$} node[below=2pt]{$\so$};
				\end{tikzpicture}\\
				$D=C_{\mathbb{R}}(-3 ; 2)=(-\infty;-3]\cup[2;+\infty)$.
			\end{center}			
		\end{enumerate}		
	}
\end{bt}

\begin{bt}%[0D1H3-5]%[Dự án đề kiểm tra Toán 10 GHKI NH23-24- Lê Minh Thiện Anh]%[THPT Nam Kì Khởi Nghĩa]
	Để phục vụ cho một hội nghị quốc tế, ban đầu ban tổ chức huy động $35$ người phiên dịch Tiếng Anh, $30$ người phiên dịch Tiếng Pháp, trong đó có $16$ người phiên dịch được cả Tiếng Anh và Tiếng Pháp. Hỏi ban tổ chức đã huy động bao nhiêu người cho hội nghị?
	\loigiai{
		\immini{Kí hiệu $A$ là tập hợp các người phiên dịch Tiếng Anh, $B$ là tập hợp các người phiên dịch Tiếng Pháp.\\
			Ta có $n(A)=35$, $n(B)=30$, $n(A\cap B)=16$.\\
			Số người huy động cho hội nghị là $$n(A\cup B)=n(A)+n(B)-n(A\cap B)=35+30-16=49.$$}
		{		
			\begin{tikzpicture}[scale=1, font=\footnotesize, line join=round, line cap=round, >=stealth]
				\tkzDefPoints{0/0/A,-1/-0.5/B,0.5/-0.5/C}
				\tkzDrawCircle[R,green,fill=cyan,fill opacity=0.4](B,1.5 cm)
				\tkzDrawCircle[R,green,fill=yellow,fill opacity=0.4](C,1.2 cm)
				\path (A) +(260:0.2) node{$A\cap B$};
				\path (A) +(200:1.5) node{$A$};
				\path (A) +(-30:1.2) node{$B$};
			\end{tikzpicture}
		}
	}
\end{bt}

\begin{bt}%[0D2H1-2]%[Dự án đề kiểm tra Toán 10 GHKI NH23-24- Lê Minh Thiện Anh]%[THPT Nam Kì Khởi Nghĩa]
	Biểu diễn miền nghiệm của bất phương trình $3x+2y-6 \geq 0$ trên mặt phẳng tọa độ.
	\loigiai{
		\immini{Vẽ đường thẳng $d\colon 3x+2y=6$ đi qua hai điểm $A(2;0)$ và $B(0;3)$.
			Xét gốc tọa độ $O(0;0)$.\\
			Ta thấy $O\notin d$ và $3\cdot0+2\cdot0-6=-6<0$.
			Suy ra $(0;0)$ không là nghiệm của bất phương trình $3x+2y-6 \geq 0$.\\
			Do đó miền nghiệm của bất phương trình $3x+2y-6 \geq 0$ là nửa mặt phẳng kể bờ $d$ không chứa điểm $O$ (miền không bị gạch).}
		{\begin{tikzpicture}[scale=.8,line join=round, line cap=round,>=stealth,font=\footnotesize,thick]
				\tikzset{label style/.style={font=\footnotesize}}
				\draw[->] (-2,0)--(4,0) node[below]{$x$};
				\draw[->] (0,-1.5)--(0,5) node[left]{$y$};
				\draw (0,0) node[below left=-2pt]{$O$};
				\clip (-2,-1.5) rectangle (4,5);
				\draw[samples=200,domain=-2:4] plot(\x,{3-1.5*(\x)});
				\fill[pattern=north east lines,opacity=0.6] plot[domain=-2:4](\x,{3-1.5*(\x)})--(4,-1.5)--(-2,-1.5)--cycle;
				\node at (1.4,1.8)[rotate=-56.3]{$d\colon 3x+2y=6$};
				\node at (2,0)[above]{$A$};
				\node at (0,3)[right]{$B$};
				\foreach \x in {-1,1,2,3}\draw (\x,0.05)--(\x,-0.05) node[below] {\footnotesize $\x$};
				\foreach \y in {-1,1,2,3,4}\draw (0.05,\y)--(-0.05,\y) node [left] {\footnotesize $\y$};
		\end{tikzpicture}}
	}
\end{bt}
\begin{bt}%[0H4H2-2]%[0H4H2-1]%[Dự án đề kiểm tra Toán 10 GHKI NH23-24- Thinh Ngo]%[THPT Nam Kì Khởi Nghĩa]
	Cho tam giác $ABC$ có $AB = 10$, $AC = 12$, $\widehat{BAC} = 60^\circ$.
	\begin{enumerate}
		\item Tính diện tích tam giác $ABC$.
		\item Tính độ dài cạnh $BC$, bán kính đường tròn ngoại tiếp và bán kính đường tròn nội tiếp của tam giác $ABC$.
	\end{enumerate}
	\loigiai{
	\begin{enumerate}
		\item $S_{ABC}=\dfrac{1}{2}\cdot AB\cdot AC \cdot \sin A = \dfrac{1}{2} \cdot 10 \cdot 12 \cdot \dfrac{\sqrt{3}}{2}=30\sqrt{3}$.
		\item 
		\begin{itemize}
			\item $BC^2 = AB^2 + AC^2 - 2AB\cdot AC \cdot \cos A = 10^2 + 12^2 - 2\cdot 10 \cdot 12 \cdot \dfrac{1}{2}= 124\\ \Rightarrow BC = 2\sqrt{31}$.
			\item $\dfrac{BC}{\sin A} = 2R \Rightarrow R = \dfrac{2\sqrt{31}}{2\sin 60^\circ}= \dfrac{2\sqrt{93}}{3}$.
			\item Ta có $p = \dfrac{AB+BC+AC}{2}=11+\sqrt{31} \Rightarrow r = \dfrac{S_{ABC}}{p}= \dfrac{30\sqrt{3}}{11+\sqrt{31}}$. 
			
		\end{itemize}
		
	\end{enumerate}
	}
\end{bt}
\begin{bt}%[0H5H1-2]%[Dự án đề kiểm tra Toán 10 GHKI NH23-24- Thinh Ngo]%[THPT Nam Kì Khởi Nghĩa]
	Cho tam giác $ABC$. Gọi $M$, $N$, $P$ lần lượt là trung điểm của $BC$, $AC$, $AB$.
	\begin{enumerate}
		\item Tìm các vec-tơ khác vec-tơ không có điểm đầu và điểm cuối là các điểm $A$, $B$, $C$, $M$, $N$, $P$ cùng phương với vec-tơ $\overrightarrow{MN}$.
		\item Tìm các vec-tơ cùng hướng với vec-tơ $\overrightarrow{AC}$.
		
	\end{enumerate}
	\loigiai{
	\immini
	{
		\begin{enumerate}
			\item Các vec-tơ khác vec-tơ không cùng phương với vec-tơ $\overrightarrow{MN}$ là
			$\overrightarrow{NM}$,
			$\overrightarrow{AB}$, $\overrightarrow{BA}$, $\overrightarrow{AP}$, $\overrightarrow{PA}$,
			$\overrightarrow{PB}$,
			$\overrightarrow{BP}$.
			\item Các vec-tơ cùng hướng với vec-tơ $\overrightarrow{AC}$ là $\overrightarrow{AN}$,
			$\overrightarrow{NC}$, $\overrightarrow{PM}$. 
		\end{enumerate}
	}
	{
		\begin{tikzpicture}[line join=round, line cap=round,thick]
			\coordinate (A) at (0,3);
			\coordinate (B) at (-2,0);
			\coordinate (C) at (3,0);
			\coordinate (M) at ($(B)!0.5!(C)$);
			\coordinate (N) at ($(A)!0.5!(C)$);
			\coordinate (P) at ($(A)!0.5!(B)$);
			\draw [->] (M) -- (N);
			\draw [->] (A) -- (C);
			\draw [->] (P) -- (M);
			\draw(A)--(B)--(C)--cycle;
			\foreach \i/\g in {A/90,B/-90,C/-90, M/-90, N/30, P/120}{\draw[fill=white](\i) circle (1.0pt) ($(\i)+(\g:3mm)$) node[scale=1]{$\i$};}
		\end{tikzpicture}
	}
	
		
	}
\end{bt}
\begin{bt}%[0H5H2-1]%[Dự án đề kiểm tra Toán 10 GHKI NH23-24- Thinh Ngo]%[THPT Nam Kì Khởi Nghĩa]
	Cho các điểm $A$, $B$, $C$, $D$, $E$. Tìm các vec-tơ sau:
	\begin{enumerate}
		\item $\overrightarrow{u}=\overrightarrow{AB}- \overrightarrow{CD}+\overrightarrow{BD}-\overrightarrow{AC}$.
		\item $\overrightarrow{v}=\overrightarrow{AC}+ \overrightarrow{DE}-\overrightarrow{DC}-\overrightarrow{CE}+\overrightarrow{CB}$.
	\end{enumerate}
	\loigiai{
	\begin{enumerate}
		\item
		\begin{eqnarray*}
		\overrightarrow{u}&=&\overrightarrow{AB}- \overrightarrow{CD}+\overrightarrow{BD}-\overrightarrow{AC}\\
		&=&\left( \overrightarrow{AB}+\overrightarrow{BD}\right) -\left(\overrightarrow{AC}+ \overrightarrow{CD}\right)\\
		&=& \overrightarrow{AD} -\overrightarrow{AD}\\
		&=&\overrightarrow{0}.	
		\end{eqnarray*}
		
		\item
		\begin{eqnarray*}
		\overrightarrow{v}&=&\overrightarrow{AC}+ \overrightarrow{DE}-\overrightarrow{DC}-\overrightarrow{CE}+\overrightarrow{CB}\\
		&=&\left( \overrightarrow{AC}+\overrightarrow{CB}\right) - \left(\overrightarrow{DC}+\overrightarrow{CE} \right)+ \overrightarrow{DE}\\
		&=&\overrightarrow{AB}- \overrightarrow{DE}+ \overrightarrow{DE}\\
		&=& \overrightarrow{AB}.	
		\end{eqnarray*}
		
	\end{enumerate}	
	}
\end{bt}
