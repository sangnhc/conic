
\de{ĐỀ THI GIỮA HỌC KỲ II NĂM HỌC 2022-2023}{THPT Nguyễn Hữu Cầu}
\begin{center}
	\textbf{PHẦN 1 - TRẮC NGHIỆM}
\end{center}
\Opensolutionfile{ans}[ans/ans]
%Câu 1...........................
\begin{ex}%[0T6Y1-2]%[Dự án đề kiểm tra HKII NH22-23- Nguyễn Văn Sơn]%[THPT NGUYEN HUU CAU]
	Cho số gần đúng $a=3,1475$ với độ chính xác $d=0,001$. Xác định số quy tròn của số $a$.
	\choice
	{$3,14$}
	{\True $3,15$}
	{$3,148$}
	{$3,147$}
	\loigiai{Vì độ chính xác $d=0,001$ đến hàng phần nghìn nên ta quy tròn số $a$ đến hàng phần trăm. Khi đó số quy tròn của số $a=3,15$.}
\end{ex}
%Câu 2...........................
\begin{ex}%[0T6B4-2]%[Dự án đề kiểm tra HKII NH22-23- Nguyễn Văn Sơn]%[THPT NGUYEN HUU CAU]
	Giả sử ta có mẫu số liệu là $x_1,x_2,.........x_n$. Phương sai của mẫu số liệu, kí hiệu là $S^2$, được tính bởi công thức nào sau đây?
	\choice
	{\True $S^2=\dfrac{1}{n}(x_1^2+x_2^2+.......+x_n^2) - \overline{x}^2$}
	{$S^2=n.(x_1^2+x_2^2+.......+x_n^2) - \overline{x}$}
	{$S^2=\dfrac{1}{n}(x_1^2+x_2^2+.......+x_n^2) - \overline{x}$}
	{$S^2=n.(x_1^2+x_2^2+.......+x_n^2) - \overline{x}^2$}

	\loigiai{Theo công thức tính phương sai của mẫu số liệu là $S^2=\dfrac{1}{n}(x_1^2+x_2^2+.......+x_n^2) - \overline{x}^2$.}
\end{ex}

%Câu 3...........................
\begin{ex}%[0T6B1-1]%[Dự án đề kiểm tra HKII NH22-23- Nguyễn Văn Sơn]%[THPT NGUYEN HUU CAU]
	Cho số $\overline{a}=5,21\pm 0,03$. Ước lượng sai số tương đối của số gần đúng $a$.  
	\choice
	{$\delta_a\leq 0,45\%$}
	{$\delta_a\leq 0,03\%$}
	{$\delta_a\leq 3\%$}
	{\True $\delta_a\leq 0,58\%$}
	\loigiai{Ước lượng sai số tính theo công thức $\delta_a\leq\dfrac{d}{\left|{a}\right|}$. Nên ta có $\delta_a\leq\dfrac{0,03}{\left|{5,21}\right|}$=0,58$\%$ }
\end{ex}
%Câu 4...........................
\begin{ex}%[0T9Y2-1]%[Dự án đề kiểm tra HKII NH22-23- Nguyễn Văn Sơn]%[THPT NGUYEN HUU CAU]
	Trong mặt phẳng $Oxy$ cho đường thẳng $d$ có phương trình tham số $\heva{&x=2+t\\ &y=3-2t}$. Trong các điểm sau đây, điểm nào thuộc đường thẳng $d$?
	\choice
	{$N(3;2)$}
	{$M(1;-2)$}
	{\True $P(3;1)$}
	{ $Q(3;5))$}
	\loigiai
	{
		Thế tọa độ từng điểm vào phương trình của đường thẳng ta thấy chỉ khi thế $x=3$ và $y=1$ vào phương trình ta được $t=1$ nên điểm $P(3;1)$ thuộc đường thẳng $d$. (Các điểm khác thế vào phương trình sẽ ra hai kết quả $t$ khác nhau)
	}
\end{ex}
%Câu 5...........................
\begin{ex}%[0T9Y1-1]%[Dự án đề kiểm tra HKII NH22-23- Nguyễn Văn Sơn]%[THPT NGUYEN HUU CAU]
	Trong mặt phẳng $Oxy$ cho hai điểm $A(3;-4)$, $B(1;0)$. Xác định toạ độ điểm $I$ là trung điểm của đoạn thẳng $AB$.
	\choice
	{$I(1;-2)$}
	{$I(-2;4)$}
	{\True $I(2;-2)$}
	{$I(1;-1))$}
	\loigiai{
		Do $I$ là trung điểm của đoạn thẳng $AB$ nên:
		$\heva{&x_I=\dfrac{x_A+x_B}{2}=\dfrac{3+1}{2}=2\\&y_I=\dfrac{y_A+y_B}{2}=\dfrac{-4+0}{2}=-2.}$\\
		Vậy  $I(2;-2)$.}
\end{ex}
%Câu 6...........................
\begin{ex}%[0T6B4-2]%[Dự án đề kiểm tra HKII NH22-23- Nguyễn Văn Sơn]%[THPT NGUYEN HUU CAU]
	Cho mẫu số liệu sau:
	\begin{center}
		\begin{tabular}{|c|c|c|c|c|c|}
			\hline
			Giá trị &$25$&$18$&$35$&$7$&$12$\\
			\hline
			Tần số &$14$&$27$&$11$&$35$&$13$\\
			\hline
		\end{tabular}
	\end{center}
	Tìm phương sai của mẫu số liệu trên.
	\choice
	{ $85{,}2$}
	{\True $82{,}5$}
	{$80{,}5$}
	{$83{,}5$}
	\loigiai{
		Số trung bình của mẫu số liệu:\\
		$\overline{x}=\dfrac{25\cdot 14+ 18\cdot 27+ 35\cdot 11 +7\cdot 35+ 12\cdot 13}{14+27+11+35+13}=16{,}22$.\\
		Tính phương sai
		$s_x^2=\dfrac {1} {n}=\dfrac 1 n\left(n_1x_1^2+n_2x_2^2+\ldots+n_5 x_5^2\right)-(\overline{x})^2 \approx 82{,}5.$}	
\end{ex}
%Câu 7...........................
\begin{ex}%[Dự án đề kiểm tra HKII NH22-23- Nguyễn Văn Sơn]%[THPT NGUYEN HUU CAU]
	Trong mặt phẳng $Oxy$, cho đường thẳng $d$ có phương trình tổng quát $2x+y-3=0$. Tìm một véc-tơ pháp tuyến $\vec{n}$ của đường thẳng $d$.
	\choice
	{\True $\vec{n}=(2;1)$}
	{$\vec{n}=(1;2)$}
	{$\vec{n}=(-2;1)$}
	{$\vec{n}=(-1;2)$}
	\loigiai{
		Do đường thẳng $d$ có phương trình tổng quát $2x+y-3=0$ nên có một véc-tơ pháp tuyến $\vec{n}=(2;1)$.}
\end{ex}
%Câu 8...........................
\begin{ex}%[0T9Y1-3]%[Dự án đề kiểm tra HKII NH22-23- Nguyễn Văn Sơn]%[THPT NGUYEN HUU CAU]
 	Trong mặt phẳng tọa độ $Oxy$, cho hai véc-tơ $\vec{a}=\left(-1;3\right)$ và $\vec{b}=\left(4;2\right)$. Tinhs $\vec{a}.\vec{b}$.
 	\choice
 	{$\vec{a}.\vec{b}=4$}
 	{$\vec{a}.\vec{b}=10$}
 	{$\vec{a}.\vec{b}=12$}
 	{\True $\vec{a}.\vec{b}=2$}
 	\loigiai
 	{
 		$\vec{a}.\vec{b}=-1.4+3.2=2$
 	}
\end{ex}
%Câu 9...........................
\begin{ex}%[0T6Y4-2]%[Dự án đề kiểm tra HKII NH22-23- Nguyễn Văn Sơn]%[THPT NGUYEN HUU CAU]
	Cho mẫu số liệu có phương sai là $5$. Xác định độ lệch chuẩn của mẫu số liệu đó.
	\choice
	{$25$}
	{\True $\sqrt{5}$}
	{$10$}
	{$5$}
	\loigiai
	{
		Độ lệch chuẩn là căn bậc hai của phương sai nên là $\sqrt{5}$.
	}
\end{ex}
%Câu 10...........................
\begin{ex}%[0T6Y4-1]%[Dự án đề kiểm tra HKII NH22-23- Nguyễn Văn Sơn]%[THPT NGUYEN HUU CAU]
	Cho mẫu số liệu sau: $11,34,7,26,10,42$. Xác định khoảng biến thiên của mẫu số liệu trên.
	\choice
	{$31$}
	{$27$}
	{$32$}
	{\True $35$}
	\loigiai
	{
		Mẫu dữ liệu được sắp xếp là: $7,10,11,26,34,42$.\\
		Khoảng biến thiên của mẫu số liệu trên là $R=42-7=35$.
	}
\end{ex}

%Câu 11...........................
\begin{ex}%[0T6Y3-4]%[Dự án đề kiểm tra HKII NH22-23- Thành Lê]%[THPT Nguyễn Hữu Cầu]
	Cho mẫu số liệu sau
	\begin{center}
		\begin{tabular}{|c|c|c|c|c|c|}
			\hline
			\textbf{Giá trị} & $12$ & $7$ & $21$ & $42$ & $5$\\
			\hline
			\textbf{Tần số} & $23$ & $32$ & $13$ & $15$ & $8$\\
			\hline
		\end{tabular}
	\end{center}
	Xác định mốt của mẫu số liệu trên.
	\choice
	{\True $7$}
	{$42$}
	{$15$}
	{$32$}
	\loigiai{
		Mốt là $7$.
	}
\end{ex}

%Câu 12
\begin{ex}%[0T6Y3-2]%[Dự án đề kiểm tra HKII NH22-23- Thành Lê]%[THPT Nguyễn Hữu Cầu]
	Cho mẫu số liệu sau: $12$, $30$, $8$, $15$, $42$. Tìm trung vị của mẫu số liệu đã cho.
	\choice
	{\True$15$}
	{$12$}
	{$30$}
	{$8$}
	\loigiai{
		Sắp xếp mẫu số liệu, ta có $8$, $12$, $15$, $30$, $42$, nên trung vị là $15$.
	}
\end{ex}

%Câu 13
\begin{ex}%[0T6B1-1] %[Dự án đề kiểm tra HKII NH22-23- Thành Lê]%[THPT Nguyễn Hữu Cầu]
	Chiều cao của cột cờ được ghi là $6{,}5\pm0{,}2$ (m). Khi đó chiều cao đúng của cột cờ nằm trong đoạn nào dưới đây?
	\choice
	{$[6{,}3;6{,}5]$}
	{\True $[6{,}3;6{,}7]$}
	{$[6{,}5;6{,}7]$}
	{$[6{,}4;6{,}6]$}
	\loigiai{
		Chiều cao đúng của cột cờ nằm trong đoạn $[6{,}3;6{,}7]$.
	}
\end{ex}

%Câu 14
\begin{ex}%[0T6B1-2] %[Dự án đề kiểm tra HKII NH22-23- Thành Lê]%[THPT Nguyễn Hữu Cầu]
	Bạn An tính chu vi của đường tròn có bán kính $r=4$ cm bằng công thức $P=3{,}145\cdot8=25{,}16$. Biết rằng $3{,}14<\pi<3{,}15$, hãy ước lượng độ chính xác $d$ của $P$.
	\choice
	{$d=0{,}02$}
	{$d=0{,}05$}
	{\True$d=0{,}04$}
	{$d=0{,}01$}
	\loigiai{
		Ta có $3{,}14<\pi<3{,}15\Rightarrow 25{,}12<8\pi<25{,}2$.\\
		Vậy $d=0{,}04$.
	}
\end{ex}

%Câu 15
\begin{ex}%[0T6Y3-1] %[Dự án đề kiểm tra HKII NH22-23- Thành Lê]%[THPT Nguyễn Hữu Cầu]
	Cho mẫu số liệu sau
	\begin{center}
		\begin{tabular}{|c|c|c|c|c|c|}
			\hline
			\textbf{Giá trị} & $11$ & $7$ & $21$ & $42$ & $5$\\
			\hline
			\textbf{Tần số} & $23$ & $31$ & $13$ & $15$ & $8$\\
			\hline
		\end{tabular}
	\end{center}
	Tính số trung bình của mẫu số liệu trên.
	\choice
	{$16$}
	{$16{,}5$}
	{\True$15{,}7$}
	{$17{,}5$}
	\loigiai{
		Ta có $\overline{x}=\dfrac{11\cdot23+7\cdot31+21\cdot13+42\cdot15+5\cdot8}{23+31+13+15+8}=15{,}7$.
	}
\end{ex}

%Câu 16
\begin{ex}% [0T6K3-3] %[Dự án đề kiểm tra HKII NH22-23- Thành Lê]%[THPT Nguyễn Hữu Cầu]
	Cho mẫu số liệu sau
	\begin{center}
		\begin{tabular}{|c|c|c|c|c|c|}
			\hline
			\textbf{Giá trị} & $25$ & $18$ & $35$ & $7$ & $12$\\
			\hline
			\textbf{Tần số} & $14$ & $27$ & $11$ & $35$ & $13$\\
			\hline
		\end{tabular}
	\end{center}
	Hãy tìm tứ vị của mẫu số liệu trên.
	\choice
	{$Q_1=35$; $Q_2=18$; $Q_3=7$}
	{$Q_1=7$; $Q_2=18$; $Q_3=35$}
	{$Q_1=7$; $Q_2=18$; $Q_3=25$}
	{\True$Q_1=7$; $Q_2=18$; $Q_3=21{,}5$}
	\loigiai{
		Sắp xếp mẫu số liệu theo thứ tự không giảm, ta có
		\begin{center}
				\begin{tabular}{|c|c|c|c|c|c|}
				\hline
				\textbf{Giá trị} & $7$ & $12$ & $18$ & $25$ & $35$\\
				\hline
				\textbf{Tần số} & $35$ & $13$ & $27$ & $14$ & $11$\\
				\hline
			\end{tabular}
		\end{center}
		\begin{itemize}
			\item Vì $n=100$ nên $Q_2=\dfrac{1}{2}(18+18)=18$.
			\item Cỡ mẫu là $n=100$.
			\begin{itemize}
				\item Số liệu thứ $25$ và $26$ là 7 nên $Q_1=\dfrac{1}{2}(7+7)=7$.
				\item Số liệu thứ $75$ và $76$ là $18$ và $25$ nên $Q_3=\dfrac{1}{2}(18+25)=21{,}5$.
			\end{itemize}
		\end{itemize}
	}
\end{ex}

%Câu 17
\begin{ex}%[0T9B1-3] %[Dự án đề kiểm tra HKII NH22-23- Thành Lê]%[THPT Nguyễn Hữu Cầu]
	Trong mặt phẳng tọa độ $Oxy$, cho điểm $M(3;-2)$. Tìm điểm đối xứng $M'$ của $M$ qua trục $Oy$.
	\choice
	{$M'(-3;2)$}
	{$M'(3;2)$}
	{\True$M'(-3;-2)$}
	{$M'(-2;3)$}
	\loigiai{
		Hình chiếu của $M$ lên $Oy$ là $H(0;-2)$.\\
		$H$ là trung điểm của $MM'$ nên
		\[
		\heva{& x_H=\dfrac{x_M+x_{M'}}{2} \\ & y_H=\dfrac{y_M+y_{M'}}{2}}\Rightarrow\heva{& x_{M'}=2x_H-x_M=2\cdot0-3=-3 \\ & y_{M'}=2y_H-y_M=2\cdot(-2)-(-2)=-2}\Rightarrow M'(-3;-2).
		\]
	}
\end{ex}

%Câu 18
\begin{ex}%[0T9B1-6] %[Dự án đề kiểm tra HKII NH22-23- Thành Lê]%[THPT Nguyễn Hữu Cầu]
	Trong mặt phẳng tọa độ $Oxy$, cho đường thẳng $d_1$ có một véc-tơ pháp tuyến $\vec{n_1}=\left(a_1;b_1\right)$ và đường thẳng $d_2$ có một véc-tơ pháp tuyến $\vec{n_2}=\left(a_2;b_2\right)$ (các véc-tơ $\vec{n_1}$ và $\vec{n_2}$ đều khác véc-tơ không). Trong các khẳng định sau, khẳng định nào \textbf{đúng}?
	\choice
	{$\cos\left(d_1, d_2\right)=\dfrac{\left|a_1b_1 + a_2b_2\right|}{\sqrt{a_1^2 + a_2^2}\cdot\sqrt{b_1^2 + b_2^2}}$}
	{$\cos\left(d_1, d_2\right)=\dfrac{\left|a_1a_2 + b_1b_2\right|}{\sqrt{a_1^2 + a_2^2}\cdot\sqrt{b_1^2 + b_2^2}}$}
	{$\cos\left(d_1, d_2\right)=\dfrac{\left|a_1a_2 + b_1b_2\right|}{\sqrt{a_1^2 + b_2^2}\cdot\sqrt{a_2^2 + b_1^2}}$}
	{\True$\cos\left(d_1, d_2\right)=\dfrac{\left|a_1a_2 + b_1b_2\right|}{\sqrt{a_1^2 + b_1^2}\cdot\sqrt{a_2^2 + b_2^2}}$}
	\loigiai{
		Khẳng định đúng là $\cos\left(d_1, d_2\right)=\dfrac{\left|a_1a_2 + b_1b_2\right|}{\sqrt{a_1^2 + b_1^2}\cdot\sqrt{a_2^2 + b_2^2}}$.
	}
\end{ex}

%Câu 19
\begin{ex}%[0T9B2-3]%[Dự án đề kiểm tra HKII NH22-23- Thành Lê]%[THPT Nguyễn Hữu Cầu]
	Trong mặt phẳng tọa độ $Oxy$, cho đường thẳng $\Delta$ có phương trình tổng quát $2x-y+3=0$ và $d\perp\Delta$. Trong các khẳng định sau, khẳng định nào \textbf{sai}?
	\choice
	{\True Đường thẳng $d$ có một véc-tơ pháp tuyến là $(2;-1)$}
	{Đường thẳng $d$ có một véc-tơ chỉ phương là $(-2;1)$}
	{Đường thẳng $d$ có một véc-tơ chỉ phương là $(2;-1)$}
	{Đường thẳng $d$ có một véc-tơ pháp tuyến là $(1;2)$}
	\loigiai{
		Đường thẳng $\Delta$ có một véc-tơ pháp tuyến là $\vec{n_{\Delta}}=(2;-1)$.\\
		Vì $d\perp\Delta\Rightarrow \vec{n_{\Delta}}$ là véc-tơ chỉ phương của $d$.\\
		Vậy véc-tơ chỉ phương của $d$ là $\vec{n_{\Delta}}=(2;-1)$.\\
		Đáp án sai là đường thẳng $d$ có một véc-tơ pháp tuyến là $(2;-1)$.
	}
\end{ex}

%Câu 20
\begin{ex}%[0T9B1-5]%[Dự án đề kiểm tra HKII NH22-23- Thành Lê]%[THPT Nguyễn Hữu Cầu]
	Trong mặt phẳng tọa độ $Oxy$, cho hai điểm $A(-3;1)$ và $B(0;5)$. Trong các véc-tơ sau đây, véc-tơ nào cùng phương với $\vec{AB}$?
	\choice
	{$\vec{d}=(4;3)$}
	{\True$\vec{b}=(-6;-8)$}
	{$\vec{a}=(3;-4)$}
	{$\vec{c}=(3;-1)$}
	\loigiai{
		Ta có $\vec{AB}=(0+3;5-1)=(3;4)$ mà $\vec{b}=-2\vec{a}$ nên $\vec{b}$ cùng phương $\vec{AB}$.
	}
\end{ex}


\Closesolutionfile{ans}
%\begin{center}
%	\textbf{ĐÁP ÁN}
%	\inputansbox{10}{ans/ans}	
%\end{center}
