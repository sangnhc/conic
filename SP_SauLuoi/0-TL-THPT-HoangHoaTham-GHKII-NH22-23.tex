
\de{ĐỀ THI GIỮA HỌC KỲ II NĂM HỌC 2022-2023}{THPT Hoàng Hoa Thám}


\begin{bt}%[0T8B1-2]%[Dự án đề kiểm tra giữa HKII NH22-23- Nguyễn Ngọc Nguyên]%[THPT Hoàng Hoa Thám]
Từ các chữ số $1$, $2$, $3$, $4$, $5$, $6$, $7$ có thể lập được bao nhiêu số tự nhiên gồm $5$ chữ số khác nhau?
\loigiai{Mỗi cách lấy ra $5$ trong $7$ chữ số từ các chữ số $1$, $2$, $3$, $4$, $5$, $6$, $7$ và sắp xếp chúng theo một hàng ngang ta được một số tự nhiên có $5$ chữ số khác nhau. \\
Do đó số các số tự nhiên có $5$ chữ số khác nhau được lấy từ tập $\{1;2;3;4;5;6;7\}$ là $\mathrm{A}_7^5=2520$ số.}
\end{bt}

\begin{bt}%[0T8B2-1]%[Dự án đề kiểm tra giữa HKII NH22-23- Nguyễn Ngọc Nguyên]%[THPT Hoàng Hoa Thám]
	Một nhóm có $5$ người: Bình, An, Toàn, Dũng và Nam. Có bao nhiêu cách xếp $5$ người này đứng theo một hàng ngang sao cho Bình luôn đứng ngoài cùng.
	\loigiai{
Để xếp $5$ người Bình, An, Toàn, Dũng và Nam theo một hàng ngang sao cho Bình Luôn đứng ngoài cùng ta thực hiện theo hai phương án sau:
\begin{itemize}
	\item Phương án 1: Bình đứng ngoài cùng bên phải. Khi đó ta sắp xếp $4$ người còn lại vào $4$ vị trí bên trái của Bình: có $4!$ cách.
	\item Phương án 2: Bình đứng ngoài cùng bên trái. Khi đó ta sắp xếp $4$ người còn lại vào $4$ vị trí bên phải của Bình: có $4!$ cách.
\end{itemize} 	
Theo quy tắc cộng ta có $4!+4!=48$ cách.
}
\end{bt}

\begin{bt}%[0T8K2-1]%[Dự án đề kiểm tra giữa HKII NH22-23- Nguyễn Ngọc Nguyên]%[THPT Hoàng Hoa Thám]
	Một hộp có $15$ viên bi gồm $6$ viên bi xanh, $4$ viên bi vàng và $5$ viên bi đỏ. Hỏi có bao nhiêu cách chọn ra $6$ viên bi sao cho mỗi loại có ít nhất một viên?
	\loigiai{
		Ta có số cách chọn ra $6$ viên bi bất kì trong $15$ viên bi là $\mathrm{C}_{15}^6$.\\
		Ta có số cách chọn ra $6$ viên bi chỉ có đúng một màu xanh hoặc có đúng hai màu xanh và vàng là $\mathrm{C}_{10}^6$ (trong đó có đúng $1$ cách chọn ra $6$ bi cùng màu xanh).\\
		Ta có số cách chọn ra $6$ bi chỉ có đúng hai màu đỏ và vàng là $\mathrm{C}_{9}^{6}$.\\
		Ta có số cách chọn ra $6$ bi chỉ có đúng một màu xanh hoặc hai màu xanh và đỏ là $\mathrm{C}_{11}^6$ (trong đó có đúng $1$ cách chọn ra $6$ bi cùng màu xanh).\\
Theo quy tắc bao hàm loại trừ ta có số cách chọn ra $6$ viên bi sao cho mỗi loại có ít nhất một viên là
\begin{eqnarray*}
	\mathrm{C}_{15}^6 - \mathrm{C}_{10}^6 - \mathrm{C}_{9}^6-\mathrm{C}_{11}^6+\mathrm{C}_6^6=4250.
\end{eqnarray*}	
}
\end{bt}

\begin{bt}%[0T8B3-1]%[Dự án đề kiểm tra giữa HKII NH22-23- Nguyễn Ngọc Nguyên]%[THPT Hoàng Hoa Thám]
	Khai triển biểu thức $(x+2)^5$.
	\loigiai{
Ta có
\begin{eqnarray*}
	(x+2)^5&=&x^5+\mathrm{C}_{5}^1 x^4 \cdot 2 + \mathrm{C}_{5}^2 x^3 \cdot 2^2+\mathrm{C}_{5}^3 x^2 \cdot 2^3+\mathrm{C}_{5}^4 x \cdot 2^4+2^5 \\
	&=&x^5+10x^4+ 40x^3+ 80x^2+80x+32.
\end{eqnarray*}	
}
\end{bt}

%Câu 5...........................
\begin{bt}%[0T9B1-3]%[Dự án đề kiểm tra HKII NH22-23- Thy Nguyen]%[Hoàng Hoa Thám]
Trong mặt phẳng tọa độ $Oxy$, cho hai điểm $A(2;1)$, $B(-4;0)$. Tìm tọa độ điểm $M$ thỏa hệ thức $\overrightarrow{AM}+\overrightarrow{BM}+2\overrightarrow{AB}=\vec{0}$.
	\loigiai{
		Gọi $M(x_M;y_M)$ là tọa độ điểm $M$ cần tìm.\\
		Ta có $\overrightarrow{AM}=(x_M-2;y_M-1)$; $\overrightarrow{BM}=(x_M+4;y_M)$; $\overrightarrow{AB}=(-6;-1)$.
		\allowdisplaybreaks{\begin{eqnarray*}
			\overrightarrow{AM}+\overrightarrow{BM}+2\overrightarrow{AB}=\vec{0}
				&\Leftrightarrow& \heva{&x_M-2+x_M+4+2\cdot (-6)=0\\&y_M-1+y_M+2\cdot (-1)=0} \\
				&\Leftrightarrow& \heva{&2x_M-10=0\\&2y_M-3=0} \\
				&\Leftrightarrow& \heva{&x_M=5\\&y_M=\dfrac{3}{2}.}
		\end{eqnarray*}}
	Vậy $M\left(5;\dfrac{3}{2}\right)$ là điểm cần tìm.
	}
\end{bt}

%Câu 6...........................
\begin{bt}%[0T9B1-5]%[Dự án đề kiểm tra HKII NH22-23- Thy Nguyen]%[Hoàng Hoa Thám]
	Trong mặt phẳng tọa độ $Oxy$, cho hai điểm $A(-1;-2)$, $B(3;1)$. Tìm tọa độ điểm $M$ thuộc trục $Ox$ sao cho 3 điểm $A$, $B$, $M$ thẳng hàng.
	\loigiai{
	$M \in Ox \Rightarrow M(x_M;0)$.\\
	Ta có $\overrightarrow{AB}=(4;3)$; $\overrightarrow{AM}=(x_M+1;2)$.\\
	$A$, $B$, $M$ thẳng hàng $\Leftrightarrow \overrightarrow{AB}$ cùng phương $\overrightarrow{AM} \Leftrightarrow \dfrac{x_M+1}{4}=\dfrac{2}{3} \Leftrightarrow x_M=\dfrac{5}{3}$.\\
	Vậy $M\left(\dfrac{5}{3};0\right)$ là điểm cần tìm.
	}
\end{bt}

%Câu 7...........................
\begin{bt}%[0T9B2-2]%[Dự án đề kiểm tra HKII NH22-23- Thy Nguyen]%[Hoàng Hoa Thám]
	Trong mặt phẳng tọa độ $Oxy$, cho tam giác $ABC$ với $A(-1;-2)$, $B(3;1)$ và $C(5;-3)$. Viết phương trình tổng quát đường trung tuyến $AI$ của $\triangle ABC$.
	\loigiai{
		Tọa độ trung điểm $I$ của $AB$ thỏa $\heva{&x_I=\dfrac{x_A+x_B}{2}=\dfrac{-1+3}{2}=1\\&y_I=\dfrac{y_A+y_B}{2}=\dfrac{-2+1}{2}=-\dfrac{1}{2}.}$\\
		Ta có $I\left(1;-\dfrac{1}{2}\right)$ và $\overrightarrow{AI}=\left(2;\dfrac{3}{2}\right)$.\\
		Đường thẳng $AI$ đi qua điểm $A(-1;-2)$ và nhận $\vec{u}=(4;3)$ làm véc-tơ chỉ phương. Do đó $AI$ nhận $\vec{n}=(3;-4)$ làm véc-tơ pháp tuyến.\\
		$AI \colon 3(x+1)-4(y+2)=0 \Leftrightarrow 3x-4y-5=0$.
	}
\end{bt}