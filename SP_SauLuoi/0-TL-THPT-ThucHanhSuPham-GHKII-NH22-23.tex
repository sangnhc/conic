
\de{ĐỀ THI GIỮA HỌC KỲ II NĂM HỌC 2022-2023}{THPT Thực Hành Sư Phạm}




\begin{bt}%[0T7B2-1]%[0T7B3-2]%[Dự án đề kiểm tra GHKII NH22-23- Phạm Duy Phương]%[Trường Trung Học Thực Hành]
	Giải các phương trình và bất phương trình ẩn $x$ sau:
	\begin{listEX}[3]
		\item $x^2+x-12<0$.
		\item $\sqrt{x^2+3}=3x-1$.
		\item $\sqrt{x^2+2x+4}=\sqrt{2-x}$.
	\end{listEX}
\loigiai{
	\begin{enumerate}
		\item Tam thức $f(x)=x^2+x-12$ có hai nghiệm phân biệt $x=3$ và $x=-4$; hệ số $a=1>0$ nên $f(x)<0$ khi $x \in \left(-4;3\right)$.\\
		Vậy bất phương trình $x^2+x-12<0$ có tập nghiệm là $T=\left(-4;3\right)$.
		\item Ta có
		\allowdisplaybreaks\begin{eqnarray*}
			& &\sqrt{x^2+3}=3x-1 \\
			&\Rightarrow & x^2+3=\left(3x-1\right)^2 \\
			&\Rightarrow & x^2+3=9x^2-6x+1 \\
			&\Rightarrow & 8x^2-6x-2=0 \\
			&\Rightarrow & \hoac{&x=1\\&x=-\dfrac{1}{4}.}
		\end{eqnarray*}
	Thay lần lượt các giá trị trên vào phương trình đã cho ta thấy chỉ có $x=1$ thoả mãn.\\
	Vậy phương trình có tập nghiệm là $S=\{1\}$.
		\item Ta có
		\allowdisplaybreaks\begin{eqnarray*}
			& & \sqrt{x^2+2x+4}=\sqrt{2-x} \\
			&\Rightarrow & x^2+2x+4=2-x \\
			&\Rightarrow & x^2+3x+2=0 \\
			&\Rightarrow & \hoac{&x=-1\\&x=-2.}
		\end{eqnarray*}
	Thay lần lượt các giá trị trên vào phương trình đã cho ta thấy $x=-1$, $x=-2$ đều thoả mãn.\\
	Vậy phương trình có tập nghiệm là $S=\left\lbrace -2;-1\right\rbrace$.
	\end{enumerate}
}
\end{bt}
\begin{bt}%[0T7K2-1]%[Dự án đề kiểm tra GHKII NH22-23- Phạm Duy Phương]%[Trường Trung Học Thực Hành]
	Tìm tất cả các giá trị của tham số $m$ để bất phương trình sau nghiệm đúng với mọi $x \in \mathbb{R}$.
	$$
	x^2-2(m-1)x+4m+8 \geq 0.
	$$
	\loigiai{
Tam thức $f(x)=x^2-2(m-1)x+4m+8$ có $\Delta'=(m-1)^2-1(4m+8)=m^2-6m-7$, hệ số $a=1$. Do đó
\allowdisplaybreaks\begin{eqnarray*}
	& &x^2-2(m-1)x+4m+8 \geq 0,\ \forall x \in \mathbb{R} \\
	&\Leftrightarrow & \heva{&a>0\\&\Delta'\leq 0} 
	\Leftrightarrow  \heva{& 1>0 \ \text{(thoả mãn)}\\&m^2-6m-7 \leq 0}
	\Leftrightarrow  -1\leq m \leq 7.
\end{eqnarray*}
Vậy tập hợp các giá trị của tham số $m$ thoả mãn yêu cầu đề bài là $T=\left[-1;7\right]$.
}
\end{bt}
\begin{bt}%[0T9B2-2]%[0T9B2-5]%[0T9B2-4]%[Dự án đề kiểm tra GHKII NH22-23- Phạm Duy Phương]%[Trường Trung Học Thực Hành]
	Trong mặt phẳng $Oxy$, cho điểm $M(2;1)$ và đường thẳng $\Delta\colon 2x-y+1=0$.
	\begin{enumerate}
		\item Lập phương trình tổng quát của đường thẳng $d$ đi qua điểm $M$ và vuông góc với $\Delta$.
		\item Tính khoảng cách từ $M$ đến đường thẳng $\Delta$.
		\item Tính cô-sin của góc giữa hai đường thẳng $\Delta$ và $\Delta'\colon 4x+2y-5=0$.
	\end{enumerate}
	\loigiai{
\begin{enumerate}
	\item Vì $d$ vuông góc với $\Delta\colon 2x-y+1=0$ nên $d$ có một véc-tơ pháp tuyến là $\vec{n_d}=\left(1;2\right)$.\\
	$d$ đi qua điểm $M(2;1)$ và có véc-tơ pháp tuyến $\vec{n_d}=\left(1;2\right)$ có phương trình tổng quát dạng
	\allowdisplaybreaks\begin{eqnarray*}
		& &1(x-2)+2(y-1)=0 \\
		&\Leftrightarrow &x-2+2y-2=0  \\
		&\Leftrightarrow & x+2y-4=0.
	\end{eqnarray*}
	\item Khoảng cách từ $M(2;1)$ đến đường thẳng $\Delta\colon 2x-y+1=0$ là
		$$
		\mathrm{d}\left(M,\Delta\right)= \dfrac{\left|2\cdot 2+(-1)\cdot 1+1\right|}{\sqrt{2^2+(-1)^2}}=\dfrac{4\sqrt{5}}{5}.
		$$
		\item $\Delta\colon 2x-y+1=0$ có véc-tơ pháp tuyến là $\vec{n}=\left(2;-1\right)$; $\Delta'\colon 4x+2y-5=0$ có véc-tơ pháp tuyến là $\vec{n'}=\left(4;2\right)$. Cô-sin góc giữa $\Delta$ và $\Delta'$ là
		$$\cos\left(\Delta,\Delta'\right)=\dfrac{\left|\vec{n}\cdot \vec{n'}\right|}{\left|\vec{n}\right|\cdot\left|\vec{n'}\right|}=\dfrac{\left|2\cdot 4+(-1)\cdot 2\right|}{\sqrt{2^2+(-1)^2}\cdot\sqrt{4^2+2^2}}=\dfrac{3}{5}.$$
\end{enumerate}	
}
\end{bt}


\begin{bt}%[0T9K2-2]%[0T9B2-6]%[Dự án đề kiểm tra GHK2 NH22-23- NguyenHuynh]%[TTSG]
	Trong mặt phẳng $Oxy$, cho ba điểm $M(-1;-4)$, $N(0;-2)$, $P(-3;4)$.
	\begin{enumerate}
		\item Viết phương trình tham số của đường thẳng $NP$.
		\item Tìm tọa độ điểm $S$ di động trên đường thẳng $NP$ biết khoảng cách từ $M$ đến $S$ là ngắn nhất.
	\end{enumerate}
	\loigiai{
		\begin{enumerate}
			\item Ta có $\overrightarrow{NP} = (-3;6) = -3(1;-2)$. Chọn $\vec{u}=(1;-2)$ là véc-tơ chỉ phương của đường thẳng $NP$.\\
			Vậy phương trình tham số của đường thẳng $NP$ là $\heva{
				& x = 0 + 1\cdot t \\
				& y = -2 + (-2)\cdot t
			}$ hay 
			$\heva{
				& x = t \\
				& y = -2 - 2t.	
			}$
			\item Gọi $S(t;-2-2t) \in NP \Rightarrow \overrightarrow{MS}=(t+1;2-2t)$.\\
			Do $M \notin NP$ nên $MS$ ngắn nhất khi và chỉ khi $$MS \perp NP \Leftrightarrow \overrightarrow{MS} \perp \overrightarrow{{u}} \Leftrightarrow 1(t+1)-2(2-2t)=0 \Leftrightarrow t = \dfrac{3}{5}.$$
			Vậy $S\left( \dfrac{3}{5}; -\dfrac{16}{5}\right) $.
		\end{enumerate}
	}
\end{bt}
\begin{bt}%[0T9B3-2]%[Dự án đề kiểm tra GHK2 NH22-23- NguyenHuynh]%[TTSG]
	Trong mặt phẳng $Oxy$, viết phương trình đường tròn $(C)$
	\begin{enumerate}[a)]
		\item Đường kính $AB$ với $A(1;1)$, $B(7;5)$.
		\item Tâm $I(1;1)$ và tiếp xúc với đường thẳng $(d)\colon 3x+4y-2=0$.
	\end{enumerate}
	\loigiai{
		\begin{enumerate}[a)]
			\item Gọi $E$ là trung điểm của $AB $, suy ra $E(4;3)$.\\
			Ta có $EA=\sqrt{(4-1)^2+(3-1)^2}=\sqrt{13}$.\\
			$(C)$ có tâm $E(4;3)$ và bán kính $R=EA=\sqrt{13}$.\\
			Suy ra phương trình $(C)\colon (x-4)^2+(y-3)^2=13$.
			\item $(C)$ tiếp xúc với $(d)$ nên $R=\mathrm d(I,(d))=\dfrac{\left|3\cdot 1+4\cdot1-2\right|}{\sqrt{3^2+4^2}}=1$.\\
			Suy ra phương trình $(C)\colon (x-1)^2+(y-1)^2=1$.
		\end{enumerate}
	}
\end{bt}
\begin{bt}%[0T7B2-1] %[Dự án đề kiểm tra GHK2 NH22-23- NguyenHuynh]%[TTSG]
	Một quả bóng được bắn thẳng lên từ độ cao $2$ (m) với vận tốc ban đầu $30$ (m/s). Khoảng cách của bóng so với mặt đất sau $t$ giây được cho bởi hàm số $h(t)=-4{,}9t^2+30t+2$ (m). Hỏi quả bóng nằm ở độ cao trên $40$ (m) trong thời gian bao lâu? Làm tròn đến hàng phần mười.
	\loigiai{
		Quả bóng nằm trên độ cao $40$m $ \Leftrightarrow h(t)>40\Leftrightarrow -4{,}9t^2+30t -38>0 \Leftrightarrow 1{,}79 <t <4{,}33$.\\
		Thời gian quả bóng ở độ cao trên $40$ (m) là $4{,}33-1{,}79=2{,}54\approx 2{,}5$ giây.
	}
\end{bt}
