\de{ĐỀ THI GIỮA HỌC KỲ I NĂM HỌC 2023-2024}{THPT LONG TRƯỜNG}
%\begin{center}
%	\textbf{PHẦN 1 - TRẮC NGHIỆM}
%\end{center}
\Opensolutionfile{ans}[ans/ans]

%Câu 1...........................
\begin{bt}[0.5 điểm]%[0D1B2-1]%[Dự án đề kiểm tra Toán 10 GHKI NH23-24 - Xuan Vy Pham]%[THPT Long Trường- Tp HCM]
Viết tập hợp $B= \lbrace x \in \mathbb{N} \; | -4 <x \le 3 \rbrace$ dưới dạng liệt kê các phần tử.
\loigiai{
$B=\lbrace 0;1;2;3 \rbrace$.
}
\end{bt}

%Câu 2...........................
\begin{bt}[1.5 điểm]%[0D1B3-1]%[Dự án đề kiểm tra Toán 10 GHKI NH23-24 - Xuan Vy Pham]%[THPT Long Trường- Tp HCM]
Cho hai tập hợp $A=\lbrace 1;2;3;7 \rbrace$ và $B=\lbrace 2;4;6;7;8 \rbrace$. Xác định các tập hợp $A \cap B$, $A \cup B$, $B \setminus A$.
\loigiai{\begin{itemize}
\item $A \cap B = \lbrace 2;7 \rbrace.$
\item $A \cup B = \lbrace 1;2;3;4;6;7;8 \rbrace$.
\item $B \setminus A = \lbrace 4;6;8 \rbrace$.
\end{itemize}
}
\end{bt}

%Câu 3..........................
\begin{bt}[2.0 điểm]%[0D1B3-4]%[Dự án đề kiểm tra Toán 10 GHKI NH23-24- Xuan Vy Pham]%[THPT Long Trường - Tp HCM]
Cho hai tập hợp $A=(-\infty ;3)$ và $B=(1;5]$. Xác định các tập hợp $A \cap B$, $A \cup B$, $A \setminus B$, $C_\mathbb{R}B.$
\loigiai{\begin{itemize}
\item $A \cap B =(1,3).$
\item $A \cup B = (-\infty;5]$.
\item $A \setminus B = (-\infty;1]$.
\item $C_\mathbb{R}B = (-\infty;1] \cup (5;+\infty)$.
\end{itemize}
}
\end{bt}

%Câu 4..........................
\begin{bt}[1.5 điểm]%[0H1K1-2]%[Dự án đề kiểm tra Toán 10 GHKI NH23-24- Xuan Vy Pham]%[THPT Long Truong - Tp HCM]
Cho $\sin \alpha =\dfrac{3}{5}$ với $90^\circ < \alpha < 180^\circ$. Tính $\cos \alpha$, $\tan \alpha$, $3\cos^2 \alpha+ 5 \tan^2 \alpha$.
\loigiai{\begin{itemize}
\item Ta có $\sin^2 \alpha + \cos^2 \alpha =1$. Do đó $\cos^2 \alpha = 1 - \sin^2 \alpha =1 -\left( \dfrac{3}{5} \right)^2=\dfrac{16}{25}$.

Mà $90^\circ < \alpha < 180^\circ$ nên $\cos \alpha <0$. Suy ra $\cos \alpha = -\sqrt{\dfrac{16}{25}}=-\dfrac{4}{5}$.
\item  $\tan \alpha = \dfrac{\sin \alpha}{\cos \alpha}=\dfrac{\dfrac{3}{5}}{-\dfrac{4}{5}}=-\dfrac{3}{4}$.
\item $3 \cos^2 \alpha +5 \tan^2 \alpha = 3. \dfrac{16}{25}+5.\left( -\dfrac{3}{4} \right)^2=\dfrac{1893}{400}$.
\end{itemize}
}
\end{bt}

%Câu 5
\begin{bt}[0.5 điểm]%[0H1B1-3]%[Dự án đề kiểm tra Toán 10 GHKI NH23-24- Tacgia]%[THPT LONG TRƯỜNG - Tp HCM]
	Cho tam giác $ABC$. Chứng minh đẳng thức: $\tan A+\tan(B+C)=0$ $\left(A\neq 90^{\circ}\right)$. 
	
	\loigiai{
		Trong tam giác $ABC$ ta có $\widehat{A} +\widehat{B}+\widehat{C}=180^{\circ} \Rightarrow \widehat{B}+\widehat{C}= 180^{\circ}-\widehat{A}$.\\
		Lại có $\tan \left( 180^{\circ} -A\right)=-\tan A $ nên $\tan(B+C)=\tan\left(180^{\circ} -A\right)=-\tan A$.\\
		Do đó, $\tan A+\tan(B+C)=\tan A- \tan A=0$ (đpcm).
	}
\end{bt}

%Câu6
\begin{bt}[3 điểm]%[0H1B2-1]%[Dự án đề kiểm tra Toán 10 GHKI NH23-24- Tacgia]%[THPT - Tp HCM]
	Cho tam giác $ABC$ biết $AB=6$ cm, $BC=8$ cm, góc $ABC=60^{\circ}$. 
	\begin{enumerate}[a)]
		\item  Tính độ dài $AC$, góc $C$ và diện tích tam giác $ABC$.
		\item  Tính độ dài đường cao $AH$ và bán kính đường tròn ngoại tiếp $R$ của tam giác $ABC$.
		
		(Chú ý: số đo góc làm tròn đến đơn vị độ; độ dài và diện tích lấy số đúng hoặc làm tròn đến chữ số thập phân thứ hai).
	\end{enumerate}
	\loigiai{
		\begin{enumerate}[a)]
			\item  Áp dụng định lý $\cos$ trong tam giác $ABC$ ta có
			\begin{eqnarray*}
				AC^2&=&AB^2+BC^2-2\cdot AB\cdot BC\cdot\cos ABC\\
				AC^2&=&6^2+8^2-2\cdot6\cdot8\cdot \cos 60^{\circ}=52\\
				AC&=&\sqrt{52}.
			\end{eqnarray*}
			Áp dụng định lý $\sin$ trong tam giác  $ABC$ ta có
			\begin{eqnarray*}
				&&\dfrac{AB}{\sin C}=\dfrac{BC}{\sin A}=\dfrac{CA}{\sin B}=2R\\
				&\Rightarrow& \sin C=\dfrac{AB\cdot\sin B}{CA}=\dfrac{6\cdot\sin 60^{\circ}}{\sqrt{52}}\\
				&\Rightarrow& \widehat{C}\approx 46^{\circ}.
			\end{eqnarray*}
			Diện tích tam giác $ABC$ là $S=\dfrac{1}{2}\cdot AB\cdot BC\cdot\sin ABC=\dfrac{1}{2}\cdot6\cdot8\cdot\sin 60^{\circ}=12\sqrt{3}$ (cm$^2$).
			\item  Ta có $S=\dfrac{1}{2}\cdot AH \cdot BC \Rightarrow AH=\dfrac{2S}{BC}=\dfrac{2\cdot12\sqrt{3}}{8}=3\sqrt{3}$ (cm).\\
			Áp dụng định lý $\sin$ trong tam giác  $ABC$ ta có
			
			$\dfrac{CA}{\sin B}=2R \Rightarrow R=\dfrac{CA}{2\cdot\sin B}=\dfrac{\sqrt{52}}{2\cdot \sin 60^{\circ}}=\dfrac{2\sqrt{39}}{3}$ (cm).
			
	\end{enumerate}}
\end{bt}

%Câu 7
\begin{bt}[1 điểm]%[0H1B3-2]%[Dự án đề kiểm tra Toán 10 GHKI NH23-24- Tacgia]%[THPT - Tp HCM]
	Giả sử $CD=h$ là chiều cao của tháp trong đó $C$ là chân tháp (hình vẽ). Chọn hai điểm $A$, $B$ trên mặt đất sao cho ba điểm $A$, $B$ và $C$ thẳng hàng. Ta đo được $AB=25$ m, $CAD=60^{\circ}$, $CBD=45^{\circ}$. Tính chiều cao $h$ của tháp ?
	\begin{center}
		
		\begin{tikzpicture}[scale=1, font=\footnotesize, line join=round, line cap=round,>=stealth]
			\path
			(8,0) coordinate (B)
			(6,0) coordinate (A)
			(4,4) coordinate (D)
			(4,0) coordinate (C)
			(3.75,0) coordinate (U)
			(4.25,0) coordinate (V)
			;
			\draw[thick] (U)--(C)--(V); 
			\draw (C)--(B)--(A)--(D)--(B);
			\draw[dashed] (D)--(C) ;
			\draw[thick] (D)--(U)--(V)--(D);
			\draw  
			($(D)!{1}!(U)$)--($(D)!{0.9}!(V)$)
			--($(D)!{0.8}!(U)$)--($(D)!{0.7}!(V)$)
			--($(D)!{0.6}!(U)$)--($(D)!{0.5}!(V)$)
			--($(D)!{0.4}!(U)$)--($(D)!{0.3}!(V)$)
			--($(D)!{0.2}!(U)$)--($(D)!{0.1}!(V)$)
			;
			\tkzMarkAngle[arc=l, size=0.6,mark=0](D,A,C)
			\tkzLabelAngle[pos=1](D,A,C) {$60^{\circ}$}
			\tkzMarkAngle[arc=ll, size=0.6,mark=0](D,B,C)
			\tkzLabelAngle[pos=1](D,B,C) {$45^{\circ}$}
			\foreach \x/\g in {A/-90,B/-90,C/-90,D/90} 
			\fill[black] (\x) circle (1pt)+(\g:3mm) node {$\x$};
		\end{tikzpicture}
	\end{center}
	
	\loigiai{
		$\tan DAC=\dfrac{CD}{CA}\Rightarrow CD =\tan DAC \cdot CA.\\
		\tan DBC=\dfrac{CD}{BC} \Rightarrow CD =\tan DBC \cdot BC.\\
		\Rightarrow\tan DAC \cdot CA=\tan DBC \cdot BC. \\
		\Rightarrow \dfrac{CA}{BC}=\dfrac{\tan DBC}{\tan DAC}=\dfrac{\tan 45^{\circ}}{\tan 60^{\circ}}=\dfrac{1}{\sqrt{3}}=\dfrac{\sqrt{3}}{3}.
		\\ \Rightarrow CA=\dfrac{\sqrt{3}}{3}BC$.\\
		Lại có $BC-CA=AB=25. \\\Rightarrow BC-\dfrac{\sqrt{3}}{3}BC=25.\\
		\Rightarrow \left( 1-\dfrac{\sqrt{3}}{3}\right) \cdot BC=25.\\
		\Rightarrow BC= 25 : \left( 1-\dfrac{\sqrt{3}}{3}\right)=\dfrac{75+25\sqrt{3}}{2}.\\
		\Rightarrow h=CD=\tan DBC \cdot BC=\tan 45^{\circ}\cdot\dfrac{75+25\sqrt{3}}{2}\approx 59$ (m).}
\end{bt}





























\Closesolutionfile{ans}
%\begin{center}
%	\textbf{ĐÁP ÁN}
%	\inputansbox{10}{ans/ans}	
%\end{center}
%\begin{center}
%	\textbf{PHẦN 2 - TỰ LUẬN}
%\end{center}
%
%%Câu 1...........................
%\begin{bt}%ID%[Dự án đề kiểm tra Toán 11 GHKI NH23-24- Tacgia]%[THPT - Tp HCM]
%
%\loigiai{}
%\end{bt}


