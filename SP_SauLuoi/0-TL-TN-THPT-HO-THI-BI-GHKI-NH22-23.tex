 
\de{ĐỀ THI GIỮA HỌC KỲ I NĂM HỌC 2023-2024}{THPT HỒ THỊ BI}
\begin{center}
	\textbf{PHẦN 1 - TRẮC NGHIỆM}
\end{center}
\Opensolutionfile{ans}[ans/ans]
%%==========Câu 1
\begin{ex} %[0D1N1-1]
	Phát biểu nào sau đây là mệnh đề toán học?
	\choice 
	{Trận bóng rổ này hay quá!}
	{Hôm nay bạn có học tiếng Anh không?}
	{\True $5$ là số nguyên tố}
	{Tỉnh Hải Dương thuộc vùng Đồng Bằng Bắc Bộ}
	\loigiai{		
		Mệnh đề toán học là mệnh đề khẳng định một sự kiện trong toán học.
	}
\end{ex} 
%%==========Câu 2
\begin{ex}%%[0D1Y2-2]
	Số tập con của tập hợp $N=\{x ; y ; z \}$ có hai phần tử là 
	\choice
	{$3$}
	{$2$}
	{$8$}
	{\True $4$}
	\loigiai{Có $3$ tập con của $N$ gồm hai phần tử là
		$\{x;\,y\}, \{y;\,z\}, \{x;\,z\}$
	}
\end{ex}
%%==========Câu 3
\begin{ex}%%[0D1B3-1]
	Cho hai tập hợp $A=(-2;\,3)$, $B=[-1;\,5]$. Xác định tập $A\cup B$
	\choice
	{$A\cup B = (-2;\,3)$}
	{\True $A\cup B = (-2;\,5]$}
	{$A\cup B = [3;\,5]$}
	{$A\cup B = [-1;\,3)$}
	\loigiai{
		Ta có $A\cup B = (-2;\,5]$.
	}
\end{ex}
%%==========Câu 4
\begin{ex}%[0D1Y1-1]
	Phát biểu nào dưới đây là một mệnh đề?
	\choice 
	{Đảo Phú Quốc rất đẹp vào buổi sáng} 
	{Một ngày nào đó tôi sẽ đến đảo Phú Quốc} 
	{Bạn có bao giờ đến Phú Quốc chưa?}
	{\True Phú Quốc là một huyện đảo của tỉnh Kiên Giang}
	\loigiai{
		Mệnh đề là câu khẳng định có thể xác định được tính đúng hay sai của nó. Một mệnh đề không thể vừa đúng, vừa sai.
	}
\end{ex}
%%==========Câu 5
\begin{ex} %[0H1Y1-3]
	Cho góc $\alpha\left(90^\circ <\alpha<180^\circ,\,\alpha \ne 90^\circ \right )$
	\choice 
	{$\cos\left(180^{\circ} -\alpha \right )=\cos\alpha$} 
	{$\tan\left(180^{\circ} -\alpha \right )=\tan\alpha$} 
	{$\cot\left(180^{\circ} -\alpha \right )=\cot\alpha$} 
	{\True $\sin\left(180^{\circ} -\alpha \right )=\sin\alpha$} 
	\loigiai{
		sin bù $$\sin\left(180^{\circ} -\alpha \right )=\sin\alpha$$ 
	}
\end{ex}
%%==========Câu 6
\begin{ex}%%[0D1B2-1]
	Cho $x$ là một phần tử của tập hợp $X$. Cách ghi nào sau đây đúng?
	\choice
	{$X\in x$}
	{$\{x\}\in X$}
	{\True $x\in X$}
	{$x\subset X$}
	\loigiai{
		Phần tử ký hiệu chữ thường: $x$\\
		Tập hợp ký hiệu trong dấu $\{...\}$  hoặc chữ cái in hoa: $X$\\
		Phần tử chỉ có mối liên hệ $\in$ và $\notin $, còn tập hợp có mối liên hệ $\supset$, $\subset$, $\not\subset $ và $\not\supset$
	}
\end{ex}
%%==========Câu 7
\begin{ex} %[0D1B2-1]
	Cho tập hợp $X=\{x \in \mathbb{N}\mid 2x^2-3x+1=0 \}$. Viết lại tập hợp $X$ dưới dạng liệt kê,  kết quả nào dưới đây là đúng? 
	\choice
	{\True $X=\{1\}$}
	{$X=\left\{1;\dfrac 12\right\}$}
	{$X=\left\{1;\dfrac 32\right\}$}
	{$X=\left\{\dfrac 12\right\}$}
	\loigiai{
		Ta có $2x^2-3x+1=0\Leftrightarrow \hoac{&x=1\\&x=\frac{1}{2}(l)} $
		Vậy $X=\{1\}$
	}
\end{ex}
%%==========Câu 8
\begin{ex}%[0D1B3-3]
	Lớp 10A có $35$ học sinh, trong đó có $13$ học sinh giỏi môn Văn, $15$ học sinh giỏi môn Toán, $10$ học sinh không học giỏi môn nào trong hai môn Văn và Toán. Có bao nhiêu học sinh lớp 10A giỏi cả hai môn Toán và Văn.
	\choice
	{$13$ học sinh}
	{$11$ học sinh}
	{$4$ học sinh}
	{\True $3$ học sinh}
	\loigiai{
		Gọi $A$, $B$ lần lượt là tập hợp các bạn học giỏi Toán và Văn.\\
		Theo giả thiết ta có $n(A)=15$, $n(B)=13$ và $n(A\cup B)=35-10=25$\\
		Vậy số học sinh thi học sinh giỏi cả hai môn Toán và Ngữ văn (phần giao nhau) là 
		$$n(A\cap B)=n(A)+n(B)-n(A\cup B)=15+13-25=3$$	.	
	}
\end{ex}

%%==========Câu 9
\begin{ex}%[0H1Y1-1]
	Cho góc $\alpha\left(0^{\circ} <\alpha<90^{\circ} \right )$
	\choice 
	{\True$\sin\alpha>0$} 
	{$\cos\alpha>1$} 
	{$\cot\alpha<0$} 
	{$\tan\alpha<0$} 
	\loigiai{
		$0^{\circ} <\alpha<90^{\circ}$ nên các giá trị lượng giác góc $\alpha$ đều dương 
	}
\end{ex}
%%==========Câu 10
\begin{ex}%[0D1Y3-1]
	\immini[thm]{
		Cho hai tập hợp $A$, $B$ như hình bên. Phần gạch chéo trong hình biểu thị phép toán nào giữa hai tập hợp $A$, $B$ ?
		\choice
		{$A\backslash B$}
		{$B\backslash A$}
		{$A\cup B$}
		{\True $A\cap B$}}
	{\begin{tikzpicture}[scale=.75, line join=round, line cap=round]
			\coordinate (A) at (0,0);
			\coordinate (B) at (2,0);
			\begin{scope}
				\clip[rotate=45] (A) ellipse (2cm and 1cm); %cat theo elip A
				\fill[rotate=40,pattern=north west lines] (B) ellipse (2.2cm and 1.2cm); %to elip B nhung bi cat mat theo phan giao voi elip A
			\end{scope}
			\draw[rotate=45,ultra thick] (A) ellipse (2cm and 1cm);
			\draw(-1.25,0)--(-2,0) node[left]{$A$}; %ve lai elip A
			\draw[rotate=40,ultra thick] (B) ellipse (2.2cm and 1.2cm);
			\draw(3.55,0)--(4.25,0) node[right]{$B$}; %ve lai elip B
	\end{tikzpicture}}
	\loigiai{
		
	}
\end{ex}
%%%% Câu 11..................
\begin{ex}%[0D1N3-2]%[Dự án đề kiểm tra Toán 10 GHKI NH23-24- Viết Tường]%[THPT Hồ Thị Bi]
	Cho $A=\{0;1;2;3;4\}$ và $B=\{2;3;4;5;6\}$. Tìm tập hợp $A\setminus B$.
	\choice 
	{$A\setminus B=\{1;2\}$} 
	{$A\setminus B=\{5;6\}$}
	{$A\setminus B=\{0\}$} 
	{\True $A\setminus B=\{0;1\}$}
	\loigiai
	{
		Ta có $A\setminus B=\{0;1\}$.
	}
\end{ex}
%%%% Câu 12..................
\begin{ex}%[0D1N3-1]%[Dự án đề kiểm tra Toán 10 GHKI NH23-24- Viết Tường]%[THPT Hồ Thị Bi]
	Cho hai tập hợp $A=\{-2;-1;0;1;2;3;4;5\}$ và $B=\{-1;2;3;5;6;7;8;9\}$. Tìm tập hợp $A\cap B$.
	\choice 
	{$A\cap B=\{6;7;8;9\}$} 
	{\True $A\cap B=\{-1;2;3;5\}$}
	{$A\cap B=\{-2;0;1;4\}$} 
	{$A\cap B=\{-2;-1;0;1;2;3;4;5;6;7;8;9\}$}
	\loigiai
	{
		Ta có $A\cap B=\{-1;2;3;5\}$.
	}
\end{ex}
%%%% Câu 13..................
\begin{ex}%[0D1N1-5]%[Dự án đề kiểm tra Toán 10 GHKI NH23-24- Viết Tường]%[THPT Hồ Thị Bi]
	Phát biểu nào sau đây là mệnh đề chứa biến?
	\choice 
	{\True $\forall n\in\mathbb{N}$ ta có $n^2-2x+1$ là số chính phương} 
	{Hình vuông có độ dài cạnh bằng $2$ thì chu vi của hình vuông là $8$}
	{Sông Sê-rê-pok chảy ngang qua thành phố Buôn Ma Thuột} 
	{$\pi$ là một số hữu tỉ}
	\loigiai
	{
		Mệnh đề chứa biến là \lq\lq$\forall n\in\mathbb{N}$ ta có $n^2-2x+1$ là số chính phương.\rq\rq
	}
\end{ex}
%%%% Câu 14..................
\begin{ex}%[0D1H1-2]%[Dự án đề kiểm tra Toán 10 GHKI NH23-24- Viết Tường]%[THPT Hồ Thị Bi]
	Khẳng định nào sau đây đúng?
	\choice 
	{$\sqrt{2}\in\mathbb{Z}$} 
	{\True $\sqrt{2}\in\mathbb{R}$}
	{$\sqrt{2}\in\mathbb{Q}$} 
	{$\sqrt{2}\in\mathbb{N}$}
	\loigiai
	{
		$\sqrt{2}$ là một số vô tỉ, do đó $\sqrt{2}\in\mathbb{R}$.
	}
\end{ex}
%%%% Câu 15..................
\begin{ex}%[0D1H1-5]%[Dự án đề kiểm tra Toán 10 GHKI NH23-24- Viết Tường]%[THPT Hồ Thị Bi]
	Phủ định của mệnh đề \lq\lq$\forall x\in\mathbb{R}\,:\,x^2>0$\rq\rq\, là mệnh đề nào trong các mệnh đề sau?
	\choice 
	{$\exists x\in\mathbb{R}\,:\, x^2\ne 0$} 
	{\True $\exists x\in\mathbb{R}\,:\,x^2\le 0$}
	{$\exists x\in\mathbb{R}\,:\,x^2<0$} 
	{$\forall x\in\mathbb{R}\,:\,x^2\le 0$}
	\loigiai
	{
		Phủ định của mệnh đề đã cho là mệnh đề \lq\lq $\exists x\in\mathbb{R}\,:\,x^2\le 0$.\rq\rq
	}
\end{ex}
%%%% Câu 16..................
\begin{ex}%[0D1H2-2]%[Dự án đề kiểm tra Toán 10 GHKI NH23-24- Viết Tường]%[THPT Hồ Thị Bi]
	Cho hai tập hợp $A=\{1;3\}$, $B=\{0;1;2;3\}$. Khẳng định nào dưới đây đúng?
	\choice 
	{$A=B$} 
	{$B\subset A$}
	{\True $A\subset B$} 
	{$A\in B$}
	\loigiai
	{
		Ta có $1\in A$, $1\in B$ và $3\in A$, $3\in B$.\\
		Do đó $A\subset B$.
	}
\end{ex}
\Closesolutionfile{ans}
\begin{center}
	\textbf{PHẦN 2 - TỰ LUẬN}
\end{center}
%%%% Câu 17..................
\begin{bt}%[0D2H1-2]%[Dự án đề kiểm tra Toán 10 GHKI NH23-24- Viết Tường]%[THPT Hồ Thị Bi]
	Biểu diễn miền nghiệm của bất phương trình $3x-y+2\le 0$ trên mặt phẳng tọa độ $Oxy$.
	\loigiai
	{
		\immini{
			Vẽ đường thẳng $d\colon 3x-y+2=0$.\\ Vì $3\cdot 0-0+2=2>0$ nên tọa độ $O(0;0)$ không thỏa mãn bất phương trình $3x-y+2\le 0$. \\Do đó, miền nghiệm của bất phương trình $3x-y+2\le 0$ là nửa mặt phẳng bờ $d$ không chứa gốc tọa độ $O$ (phần không bị gạch chéo trên hình kể cả đường thẳng $d$).
		}{
			\begin{tikzpicture}[font=\footnotesize ,line cap=round,line join=round,scale=1,>=stealth]
				\draw[->] (-2,0)--(2,0) node[below left] {$x$};
				\draw[->] (0,-1)--(0,4) node[above left] {$y$};
				\draw (0,0) node [below left] {$O$};
				\begin{scope}
					\clip (-2,-1) rectangle (2,4);
					\draw[domain=-2:2,smooth,variable=\x] plot (\x,{3*(\x)+2});
					\fill[pattern=north east lines,pattern color=blue!50](-1,-1)--(0.667,4)--(2,4)--(2,-1)--cycle;
				\end{scope}
				\draw(.7,3.5)node[right]{$d\colon 3x-y+2=0$};
			\end{tikzpicture}
		}
	}
\end{bt}
%Câu 18...........................
\begin{bt}%[[0D2T1-3]]%[Dự án đề kiểm tra Toán 10 GHKI NH23-24- Đào Hoàng Vũ]%[THPT Hồ Thị Bi ]
Một công ty sản xuất tương ớt trong kho còn $12$ tấn ớt và $8$ tấn cà chua dùng để sản xuất hai loại tương ớt: siêu cay và bình thường. Để sản xuất một tấn loại tương ớt siêu cay cần dùng $6$ tấn ớt và $2$ tấn cà chua, khi bán lãi được $10$ triệu. Để sản xuất một tấn loại tương ớt bình thường cần dùng $2$ tấn ớt và $2$ tấn cà chua, khi bán lãi được $8$ triệu đồng.
\begin{enumerate}[a)]
	\item Gọi $x$, $y$ (đơn vị: tấn) lần lượt là số tấn tương ớt loại siêu cay và số tấn tương ớt loại bình thường công ty có thể sản xuất được. Lập hệ bất phương trình mô tả các điều kiện ràng buộc đối với $x$, $y$.
	\item Công ty cần sản xuất bao nhiêu tấn tương ớt mỗi loại để thu được tổng số tiền lãi cao nhất?
\end{enumerate}
\loigiai{
\begin{enumerate}[a)]
	\item Ta có điều kiện ràng buộc đối với $x$, $y$ như sau:
	\begin{itemize}
		\item $x\geq 0$, $y\geq 0$.
		\item Số tấn ớt không vượt quá $12$ tấn nên $6x+2y\leq 12$.
		\item Số tấn cà chua không vượt quá $8$ tấn nên $2x+2y\leq 8$.
	\end{itemize}
Khi đó, ta có hệ bất phương trình thỏa điều kiện ràng buộc $\heva{&6x+2y\leq 12\\ & 2x+2y \leq 8\\ &x\geq 0\\&y\geq 0}$.
	\item Gọi $F=10x+8y$ là số tiền lãi của công ty.\\
Vẽ đường thẳng $d_1\colon 6x+2y=12$ và $d_2\colon 2x+2y=8$.
\begin{center}
	\begin{tikzpicture}[font=\footnotesize ,line cap=round,line join=round,scale=.7,>=stealth]
		\begin{scope} 
			\clip (-2,-1.5) rectangle (9.5,9.5);
			\fill[pattern=north east lines,pattern color=blue!50](2.5,-1.5)--(10,-1.5)--(10,9)--(-1,9)--cycle;
			\fill[pattern=north east lines,pattern color=red!50](6,-2)--(10,-2)--(10,9)--(-2,9)--(-2,6)--cycle;
			\fill[pattern=north west lines,pattern color=yellow!50] (0,-2) rectangle (-2,9);
			\fill[pattern=north west lines,pattern color=purple!50](-2,-1.5) rectangle (10,0);
			\draw[blue,smooth,domain=-1:10]plot(\x,-3*\x+6);
			\draw[red,smooth,domain=-2:10]plot(\x,-\x+4);
		\end{scope}
		\draw (-1,6.5) node [above,rotate=-70] {$6x+2y-12=0$};
		\draw (4,0) node [above,rotate=-45] {$2x+2y-8=0$};
		\draw[->] (-1.5,0)--(10.5,0) node[below left] {$x$};
		\draw[->] (0,-1.5)--(0,10) node[below left] {$y$};
		\draw (0,0) node [below left] {$O$};
		\draw [fill=black] (2,0) circle (2pt) node [below right] {$A$};
		\draw [fill=black] (1,3) circle (2pt) node [above right] {$B$};
		\draw [fill=black] (0,4) circle (2pt) node [below left] {$C$};
	\end{tikzpicture}
\end{center}
	Vậy miền nghiệm của hệ bất phương trình trên là miền trong của đa giác $OABC$ (kể cả biên), với $O(0,0)$, $A(2,0)$, $B(1,3)$, $C(0,4)$.Khi đó:
	\begin{itemize}
		\item Tại $O(0,0)\Rightarrow F=0$.
		\item Tại $A(2,0)\Rightarrow F=10\cdot2=20$.
		\item Tại $B(1,3)\Rightarrow F=10\cdot1+8\cdot3=34$.
		\item Tại $C(0,4)\Rightarrow F=8\cdot4=32$.
	\end{itemize}
Vậy công ty thu được số tiền lãi cao nhất là $34$ triệu đồng khi sản xuất $1$ tấn tương ớt siêu cay và $3$ tấn tương ớt loại bình thường.	
\end{enumerate}}
\end{bt}
%%% Câu 19.......................
\begin{bt}%[[0H1K3-1]]%[Dự án đề kiểm tra Toán 10 GHKI NH23-24- Đào Hoàng Vũ]%[THPT Hồ Thị Bi ]
	Cho tam giác $ABC$ có $\widehat{ABC}=60^\circ{,} \  BC=a=8{,} \ AB=c=5$.
\begin{enumerate}[a)]
	\item Tính chính xác độ dài cạnh $AC$ và diện tích $S$ của tam giác $ABC$.
	\item Gọi $I$ là tâm đường tròn nội tiếp của tam giác $ABC$. Tính chính xác diện tích $S_1$ của tam giác $ABI$.
\end{enumerate}
\begin{center}
	\begin{tikzpicture}[line join=round, line cap=round,>=stealth,thick]
		\path
		(0,0) coordinate (A) node [below left] {$A$}
		(3,0) coordinate (B) node [below right] {$B$}
		(2,3) coordinate (C) node [above] {$C$}
		;
		\coordinate (Tempta) at ($($(A)!1cm!(B)$)!0.5!($(A)!1cm!(C)$)$);
		\coordinate (Temptb) at ($($(B)!1cm!(C)$)!0.5!($(B)!1cm!(A)$)$);
		\coordinate (I) at (intersection of A--Tempta and B--Temptb);
		\coordinate (Temptc) at ($(A)!(I)!(B)$);
		\path let \p1=(I),\p2=(Temptc),\n1={veclen(\x2-\x1,\y2-\y1)} in \pgfextra{\xdef\Tempt{\n1}};
		\draw (I) node[above] {$I$} circle (\Tempt);
		\draw (A)--(B)--(C)--cycle;
		\path ($(A)!(I)!(B)$) coordinate (H) node [below] {$H$};
		\draw (I)--(H) (I)--(A) (I)--(B);
		\draw pic[draw,angle radius=0.2cm]{right angle=I--H--A};
	\end{tikzpicture}
\end{center}
\loigiai{
\begin{enumerate}[a)]
	\item Ta có $AC^2=AB^2+BC^2-2AC\cdot BC\cos A=5^2+8^2-2\cdot5\cdot8\cdot\cos 60^\circ=49\Rightarrow AC=7$.\\
	Nửa chu vi tam giác $ABC$
	\begin{center}
		 $p=\dfrac{a+b+c}{2}=\dfrac{8+5+7}{2}=10$.
	\end{center}
	Diện tích tam giác $ABC$
	\begin{center}
		$S_{ABC}=\sqrt{p(p-a)(p-b)(p-c)}=\sqrt{10\cdot2\cdot3\cdot5}=10\sqrt{3}$.
	\end{center}
	\item Gọi $H$ là hình chiếu vuông góc của $I$ lên $AB$.\\
	 Bán kính đường tròn nội tiếp tam giác $ABC$
	 \begin{center}
	 	$r=IH=\dfrac{S}{p}=\sqrt{3}$.
	 \end{center}
	 Diện tích tam giác $IAB$
	 \begin{center}
	 	$S_{IAB}=\dfrac{1}{2}\cdot IH\cdot AB=\dfrac{1}{2}\cdot \sqrt{3}\cdot 5=\dfrac{5\sqrt{3}}{2}$.
	 \end{center}
\end{enumerate}}
\end{bt}
%%% Câu 20.............................
\begin{bt}%[[0H1B3-2]]%[Dự án đề kiểm tra Toán 10 GHKI NH23-24- Đào Hoàng Vũ]%[THPT Hồ Thị Bi ]
Một ô tô muốn đi từ địa điểm $A$ đến địa điểm $C$ nhưng giữa hai địa điểm là một ngọn núi cao nên để tránh ngọn núi ô tô phải chạy thành hai đoạn đường: từ địa điểm $A$ đến địa điểm $B$ và từ điểm $B$ đến địa điểm $C$ (như hình vẽ). Biết $AB=12$ km, $BC=18$ km,   $\widehat{ABC}=115^\circ$. Giả sử người ta khoan hầm qua núi để tạo ra một con đường thẳng từ địa điểm $A$ đến địa điểm $C$ thì ô tô chạy trên con đường này tiết kiệm được bao nhiêu $\mathrm{km}$ so với đường cũ (đáp án làm tròn đến chữ số thập phân thứ nhất)? 
\begin{center}
	\begin{tikzpicture}
		\path
		(0,0) coordinate (A)
		(30:4) coordinate (B)
		(10,0) coordinate (C)
		($(A)!.5!(B)$) coordinate (M) node [above, rotate=30] {$12\mathrm{km}$}
		($(B)!.5!(C)$) coordinate (N) node [above, rotate=-20] {$18\mathrm{km}$}
		($(B)+(0,-.2)$) node [below] {$115^\circ$}
		;
		\draw
		(A)--(B)--(C)--cycle;
		\foreach \x/\g in{A/180,B/90,C/0} {\draw [fill=white] (\x) circle (2pt) ($(\x)+(\g:4mm)$) node {$\x$};}
	\end{tikzpicture}
\end{center}
\loigiai{
Quãng đường ô tô đi được nếu đi qua hầm
\begin{center}
	$AC=\sqrt{AB^2+BC^2-2AB\cdot BC\cdot\cos B}=\sqrt{12^2+18^2-2\cdot12\cdot18\cdot{\cos115^\circ}}$ (km).
\end{center} 
Quãng đường ô tô tiết kiệm được khi so với đường cũ
\begin{center}
	$AB+BC-AC=12+18-\sqrt{12^2+18^2-2\cdot12\cdot18\cdot\cos115^\circ}\approx 4.5$ (km).
\end{center}
Vậy ô tô sẽ tiết kiệm khoảng $4{,}5$ km so với đường cũ nếu đi qua hầm.}
\end{bt}