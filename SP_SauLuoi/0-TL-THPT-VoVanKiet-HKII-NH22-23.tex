\de{ĐỀ THI HỌC KỲ II NĂM HỌC 2022-2023}{THPT Võ Văn Kiệt}


\begin{bt}%[0T7K3-2]%[Dự án đề kiểm tra HKII NH22-23- BCTuan]%[THPT Võ Văn Kiệt TPHCM]
\begin{enumerate}
	\item Giải bất phương trình $(x-4)(x^2+5x-6)\ge 0$.
	\item Giải phương trình $\sqrt{2x^2-9x-4}=\sqrt{x^2-3x-4}$.
	\item Giải phương trình $\sqrt{3x^2+6x+7}+\sqrt{5x^2+10x+14}=4-2x-x^2$.
\end{enumerate}
\loigiai{
\begin{enumerate}
	\item Ta có
	\begin{eqnarray*}
	&&	(x-4)(x^2+5x-6)\ge 0\\
	&\Leftrightarrow &\hoac{&\heva{& x-4\ge 0 \\ & x^2+5x-6\ge 0}\\&\heva{& x-4\le 0 \\ & x^2+5x-6\le 0}} \Leftrightarrow \hoac{& \heva{& x\ge 4 \\ & \hoac{& x\ge 1 \\ & x\le -6}} \\ & \heva{& x\le 4 \\ & -6\le x\le 1}}\\
	&\Leftrightarrow & \hoac{& x\ge 4 \\ & -6\le x\le 1.}
	\end{eqnarray*}
Vậy tập nghiệm của bất phương trình là $S=[-6;1]\cup [4;+\infty)$.
\item Ta có
\begin{eqnarray*}
&&\sqrt{2x^2-9x-4}=\sqrt{x^2-3x-4}\\
&\Rightarrow & 2x^2-9x-4=x^2-3x-4\\
&\Leftrightarrow & x^2-6x=0\Leftrightarrow \hoac{& x=0 \\ & x=6.}
\end{eqnarray*}
Thay các nghiệm $x=0$, $x=6$ vào phương trình ban đầu ta thấy chỉ có nghiệm $x=6$ thỏa mãn.\\
Vậy tập nghiệm của phương trình là $S=\{6\}$.
\item Ta có 
\begin{itemize}
	\item $\sqrt{3x^2+6x+7}=\sqrt{3(x+1)^2+4}\ge 2$,\ $\forall x\in\mathbb{R}$.
	\item $\sqrt{5x^2+10x+14}=\sqrt{5(x+1)^2+9}\ge 3$,\ $\forall x\in\mathbb{R}$.
	\item $4-2x-x^2=5-(x+1)^2\le 5$,\ $\forall x\in\mathbb{R}$.
\end{itemize}
Vậy $\sqrt{3x^2+6x+7}+\sqrt{5x^2+10x+14}\ge 5\ge 4-2x-x^2$,\ $\forall x\in\mathbb{R}$.\\
Vậy phương trình đã cho tương đương $$\heva{& \sqrt{3x^2+6x+7}+\sqrt{5x^2+10x+14}=5 \\ & 4-2x-x^2=5}\Leftrightarrow x=-1.$$
Vậy tập nghiệm của phương trình là $S=\{-1\}$.
\end{enumerate}
}
\end{bt}
\begin{bt}%[0T7B1-1]%[Dự án đề kiểm tra HKII NH22-23- BCTuan]%[THPT Võ Văn Kiệt TPHCM]
	\begin{enumerate}
		\item Tìm $m$ để hàm số $y=\sqrt{x^2+2(m-1)x+m+5}$ xác định với mọi $x \in \mathbb{R}$.
		\item Tìm $m$ để phương trình $x^2-(m-1)x+2m-2=0$ có $2$ nghiệm dương phân biệt.
	\end{enumerate}
\loigiai{
\begin{enumerate}
	\item Để hàm số $y=\sqrt{x^2+2(m-1)x+m+5}$ xác định với mọi $x \in \mathbb{R}$ thì $$x^2+2(m-1)x+m+5\ge 0,\ \forall x\in\mathbb{R}.$$ Hay
	$$\heva{& a=1>0 \\ & \Delta'=(m-1)^2-m-5\le 0}\Leftrightarrow m^2-3m-4\le 0\Leftrightarrow -1\le m\le 4.$$
	\item Để phương trình $x^2-(m-1)x+2m-2=0$ có $2$ nghiệm dương phân biệt $x_1, x_2$ thì
	$$\heva{& \Delta >0 \\ & x_1+x_2>0\\ &x_1x_2>0}\Leftrightarrow \heva{& (m-1)^2-4(2m-2)>0 \\ & m-1>0\\ &2m-2>0}\Leftrightarrow \heva{& m^2-10m+9>0 \\ & m>1}\Leftrightarrow m>9.$$
\end{enumerate}
}
\end{bt}
\begin{bt}%[0T0B2-2]%[Dự án đề kiểm tra HKII NH22-23- BCTuan]%[THPT Võ Văn Kiệt TPHCM]
	\begin{enumerate}
		\item Có $15$ bác sĩ, có bao nhiêu cách lập nhóm công tác gồm $5$ bác sĩ từ $15$ bác sĩ đó.
		\item Một tập thể gồm $10$ học sinh ưu tú, người ta cần đề cử một đoàn đi dự trại hè quốc tế, trong đó có $1$ trưởng đoàn, $1$ phó đoàn và $3$ đoàn viên. Hỏi có bao nhiêu cách đề cử như vậy.
		\item Tìm số hạng chứa $x^{12}$ trong khai triển $\left(x^2+x\right)^{10}$.
		\item Trong một hộp gồm $5$ bi xanh, $4$ bi đỏ, $6$ bi vàng (các bi có cùng kích cỡ và khối lượng). Lấy ngẫu nhiên $4$ bi. Tính xác suất sao cho các bi được lấy có đủ $3$ loại màu.
		\item Từ một hộp chứa $20$ thẻ được đánh số từ $1$ đến $20$, chọn ngẫu nhiên $4$ thẻ. Tính xác suất để $4$ thẻ được chọn đều được đánh số chẵn.
	\end{enumerate}
\loigiai{
\begin{enumerate}
	\item Số cách lập nhóm công tác gồm $5$ bác sĩ từ $15$ bác sĩ là $\mathrm{C}_{15}^5=3003$ (cách).
	\item Số cách chọn trưởng đoàn là $10$ cách.\\
	Số cách chọn phó đoàn là $9$ cách.\\
	Số cách chọn $3$ đoàn viên là $\mathrm{C}_8^3=56$ cách.\\
	Theo quy tắc nhân, số cách lập đoàn là $10\cdot 9\cdot 56=5040$ (cách).
	\item Số hạng tổng quát của khai triển trên là 
	$$T_{k+1}=\mathrm{C}_{10}^k\left(x^2\right)^{10-k}\cdot x^k=\mathrm{C}_{10}^k x^{20-k}.$$
	Số hạng chứa $x^{12}$ ứng với $k$ thỏa mãn điều kiện $20-k=12\Leftrightarrow k=8$.\\
	Vậy số hạng chứa $x^{12}$ là $\mathrm{C}_{10}^8x^{12}$.
	\item Số phần tử của không gian mẫu $n(\Omega)=\mathrm{C}_{15}^4=1365$.\\
	Gọi $A$ là biến cố \lq\lq Lấy được $4$ viên bi có đủ ba màu\rq\rq.\\
	{\bf TH1:} $2$ bi xanh, $1$ bi đỏ, $1$ bi vàng: $\mathrm{C}_5^2\mathrm{C}_4^1\mathrm{C}_6^1$ (cách).\\
	{\bf TH2:} $1$ bi xanh, $2$ bi đỏ, $1$ bi vàng: $\mathrm{C}_5^1\mathrm{C}_4^2\mathrm{C}_6^1$ (cách).\\
	{\bf TH3:} $1$ bi xanh, $1$ bi đỏ, $2$ bi vàng: $\mathrm{C}_5^1\mathrm{C}_4^1\mathrm{C}_6^2$ (cách).\\
	Vậy $n(A)=\mathrm{C}_5^2\mathrm{C}_4^1\mathrm{C}_6^1+\mathrm{C}_5^1\mathrm{C}_4^2\mathrm{C}_6^1+\mathrm{C}_5^1\mathrm{C}_4^1\mathrm{C}_6^2=720$.\\
	Vậy $\mathrm{P}(A)=\dfrac{n(A)}{n(\Omega)}=\dfrac{48}{91}$.
	\item Số phần tử của không gian mẫu $n(\Omega)=\mathrm{C}_{20}^4$.\\
	Gọi $A$ là biến cố \lq\lq Lấy được $4$ tấm thẻ đều được đánh số chẵn\rq\rq.\\
	Khi đó $n(A)=\mathrm{C}_{10}^4$.\\
	Vậy $\mathrm{P}(A)=\dfrac{n(A)}{n(\Omega)}=\dfrac{\mathrm{C}_{10}^4}{\mathrm{C}_{20}^4}=\dfrac{14}{323}$.
\end{enumerate}
}
\end{bt}

\begin{bt}%[0T3B1-5]%[0T3B1-2]%[0T3B2-3]% [0T3K2-2]%[Dự án đề kiểm tra HKII NH22-23- Nguyễn Ngọc Nguyên]%[Trường THPT Võ Văn Kiệt]
Trong mặt phẳng tọa độ $Oxy$
	\begin{enumerate}
		\item Tính khoảng cách từ điểm $A(2;3)$ đến đường thẳng $\Delta \colon 4x+3y+3=0$.
		\item Cho tam giác $ABC$ có $A(6;-2)$, $B(4;2)$, $C(5;-5)$. Viết phương trình đường cao $AH$ của tam giác $ABC$.
		\item Viết phương trình tiếp tuyến của đường tròn $(C) \colon (x-1)^2+(y-2)^2=25$ tại điểm $M(5;-1)$.
		\item Viết phương trình đường tròn $(C)$ đi qua hai điểm $A(1;0)$, $B(3;0)$ và tiếp xúc với đường thẳng $\Delta \colon x-y+1=0$.
	\end{enumerate}
	\loigiai{
\begin{enumerate}
	\item Ta có $\mathrm{d} (A,\Delta)=\dfrac{|4 \cdot 2 +3 \cdot 3 +3|}{\sqrt{4^2+3^2}}=4$.
	\item Đường cao $AH$ đi qua điểm $A(6;-2)$ nhận $\overrightarrow{BC}=(1;-7)$ làm véc-tơ pháp tuyến có phương trình là 
	\begin{eqnarray*}
		1(x-6)-7(y+2)=0 \Leftrightarrow x-7y-20=0.
	\end{eqnarray*}
\item Đường tròn $(C)$ có tâm $I(1;2)$.\\
Gọi $\Delta$ là tiếp tuyến của $(C)$ tại $M$.\\
$\Delta$ đi qua $M(5;-1)$ và nhận $\overrightarrow{IM}=(4;-3)$ làm véc-tơ pháp tuyến có phương trình là
\begin{eqnarray*}
	\Delta \colon 4(x-5)-3(y+1)=0 \Leftrightarrow 4x-3y-23=0.
\end{eqnarray*}
\item Phương trình đường tròn $(C)$ có dạng $x^2+y^2-2ax-2by+c=0$.\\
Ta có 
\begin{eqnarray*}
	\heva{&A \in (C) \\ &B \in (C)} \Leftrightarrow \heva{&-2a+c=-1 \\ & -6a+c=-9} \Leftrightarrow \heva{&a=2 \\&c=3.}
\end{eqnarray*}
Mặt khác, $d_1$ tiếp xúc với $(C)$ khi và chỉ khi
\begin{eqnarray*}
	\mathrm{d}(I,\Delta)=R &\Leftrightarrow& \dfrac{|a-b+1|}{\sqrt{1+1}}=\sqrt{a^2+b^2-c} \Leftrightarrow |2-b+1| = \sqrt{2} \sqrt{4+b^2-3} \\ &\Leftrightarrow& |3-b|=\sqrt{2}\sqrt{b^2+1} \Leftrightarrow b^2+6b-7=0 \Leftrightarrow \hoac{&b=1 \\ &b=-7.}
\end{eqnarray*}
Vậy có hai đường tròn cần tìm là
\begin{eqnarray*}
	&&(C_1) \colon x^2+y^2-4x-2y+3=0 \\
	&&(C_2) \colon x^2+y^2-4x+14y+3=0.
\end{eqnarray*}
\end{enumerate}	

}
\end{bt}

\begin{bt}%[0T3B3-1]%%[Dự án đề kiểm tra HKII NH22-23- Nguyễn Ngọc Nguyên]%[Trường THPT Võ Văn Kiệt]
	Trong mặt phẳng tọa độ $Oxy$,
	\begin{enumerate}
		\item Cho elip $(E) \colon \dfrac{x^2}{25}+\dfrac{y^2}{16}=1$. Tính độ dài trục lớn và độ dài trục nhỏ của elip $(E)$.
		\item Viết phương trình chính tắc của Hypebol $(H)$ biết độ dài trục ảo bằng $\sqrt{28}$ và tọa độ của một tiêu điểm là $F(4;0)$.
	\end{enumerate}
	\loigiai{
\begin{enumerate}
\item Từ phương trình $(E) \colon \dfrac{x^2}{25}+\dfrac{y^2}{16}=1$ ta có $a^2=25$ và $b^2=16$. Từ đó ta có $a=5$ và $b=4$. Do đó độ dài trục lớn của $(E)$ là $2a=10$ và độ dài trục nhỏ của $(E)$ là $2b=8$.
\item $(H)$ có độ dài trục ảo bằng $\sqrt{28}$ nên $b=\dfrac{\sqrt{28}}{2}=\sqrt{7}$.\\
$(H)$ có tiêu điểm $F(4;0)$ nên $c=4$.\\
Ta có $c^2=a^2+b^2 \Leftrightarrow a^2=c^2-b^2 = 16-7=9 \Rightarrow a =3$. \\
 Vậy phương trình chính tắc của $(H)$ là $\dfrac{x}{9}-\dfrac{y^2}{7}=1$.
\end{enumerate}	
}
\end{bt}

\begin{bt}%[0T1G4-3]%[Dự án đề kiểm tra HKII NH22-23- Nguyễn Ngọc Nguyên]%[Trường THPT Võ Văn Kiệt]
Cho đường thẳng $d \colon x+2y-4=0$ và hai điểm $P(1;4)$, $Q \left( 8;\dfrac{1}{2} \right)$. Tìm điểm $K$ thuộc $d$ sao cho $5KP^2+KQ^2$ nhỏ nhất.
	\loigiai{
Ta có $d \colon \heva{x=4-2m \\ y=m}.$ Mà $K \in (C) \Rightarrow K(4-2m;m)$.\\
Từ đó ta tính được
\begin{eqnarray*}
	&&\overrightarrow{KP}=(2m-3;4-m) \Rightarrow 5KP^2=5 \left[ \left(2m-3\right)^2+(4-m)^2 \right] \\
	&&\overrightarrow{KQ}=\left( 2m+4; \dfrac{1}{2}-m \right) \Rightarrow 2KQ^2=2 \left[ (2m+4)^2+\left( \dfrac{1}{2}-m \right)^2 \right].
\end{eqnarray*}
Khi đó $5KP^2+2LQ^2=35m^2-70m+\dfrac{315}{2}=35(m-1)^2+\dfrac{245}{2} \ge \dfrac{245}{2}$.\\
Dấu ``$=$'' xảy ra khi và chỉ khi $m=1$. Vậy $K(2;1)$.


}
\end{bt}

\begin{bt}%[0T3G2-5] %[Dự án đề kiểm tra HKII NH22-23- Nguyễn Ngọc Nguyên]%[Trường THPT Võ Văn Kiệt]
Cho tam giác $ABC$ có $C(4;-1)$, đường cao và đường trung tuyến kẻ từ đỉnh $A$ có phương trình lần lượt là $d_1 \colon 2x-3y+12=0$ và $d_2 \colon 2x+3y=0$. Tìm tọa độ trực tâm $H$ của tam giác $ABC$.
	\loigiai{
Ta có $A=d_1 \cap d_2$.\\
Tọa độ điểm $A$ là nghiệm của hệ phương trình
\begin{eqnarray*}
	\heva{&2x-3y+12=0 \\ &2x+3y=0} \Leftrightarrow \heva{&x=-3 \\ &y=2} \Rightarrow A(-3;2).
\end{eqnarray*}	
Đường thẳng $BC$ đi qua $C(4;-1)$ và vuông góc với $d_1$ nên $BC \colon 3x+2y-10=0$.\\
Gọi $M$ là trung điểm của $BC$. Khi đó $M=BC \cap d_2 \Rightarrow M(6;-4) \Rightarrow B(8;-7)$.\\
Gọi $d_3$ là đường cao kẻ từ đỉnh $C$, khi đó ta có $d_3 \colon 11x-9y-53=0$.\\
Ta có $H=d_1 \cap d_3$ do đó $H \left(\dfrac{89}{5}; \dfrac{238}{15} \right)$.
}
\end{bt}