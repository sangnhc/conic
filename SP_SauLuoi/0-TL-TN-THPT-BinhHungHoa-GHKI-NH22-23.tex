
\de{ĐỀ THI GIỮA HỌC KỲ I NĂM HỌC 2023-2024}{BÌNH HƯNG HÒA}
\begin{center}
	\textbf{PHẦN 2 - TỰ LUẬN}
\end{center}

%%==========Bài 1
\begin{bt}%[ĐỀ GIỮA HỌC KÌ I- THPT Bình Hưng Hòa-HCM]%[Nguyễn Tài Tuệ]%Câu 1%[0D1H1-5]
	Xét tính đúng sai và viết mệnh đề phủ định của mệnh đề $P\colon $  \lq\lq $\forall x \in \mathbb{R}$, $x^2+8>4 x$\rq\rq.
	\loigiai{
		Ta có $x^2+8>4x\Leftrightarrow x^2-4x+8>0\Leftrightarrow (x-2)^2+4>0$ luôn đúng $\forall x \in \mathbb{R}$.\\
		Do đó mệnh đề $P$ là mệnh đề đúng.\\
		Mệnh đề $\overline{P}\colon $\lq\lq $\exists x \in \mathbb{R}, x^2+8\le 4x$\rq\rq.
	}
\end{bt}

%%==========Bài 2
\begin{bt}%Câu 2%[ĐỀ GIỮA HỌC KÌ I- THPT Bình Hưng Hòa-HCM]%[Nguyễn Tài Tuệ] %[0D1H3-4]
	Xác định các tập hợp sau
	\begin{enumerate}
		\item $\{0 ; 3 ; 6 ; 9\}\cup\{2 ; 4 ; 6 ; 8\}$.
		\item $A \setminus B$ với $A=\{x \in \mathbb{Z}  \big| \mid x \mid \leq 2\}$ và $B=\{n \in \mathbb{N}\mid n<4\}$.
		\item $C \cap D$ với $C=\{x \in \mathbb{R}\mid-3<x \leq 4\}$ và $D=\{x \in \mathbb{R}\mid-1<x<6\}$.
		\item $C_{\mathbb{R}}E$ với $E=\{x \in \mathbb{R}\mid 5-3 x<0\}$.
	\end{enumerate}
	\loigiai{
		\begin{enumerate}
			\item Ta có $\{0 ; 3 ; 6 ; 9\}\cup\{2 ; 4 ; 6 ; 8\} =\{0;2;3;4;6;8;9\}$.
			\item Có $A=\{-2; -1 ;0 ;1 ;2\}$, $B=\{0; 1; 2; 3\}$ nên $A\setminus B=\{-2; -1\}$.
			\item Có $C=(-3;4]$, $D=(-1;6)$. Do đó $C\cap D =(-1;4]$.
			\item Có $5-3x<0\Leftrightarrow x>\dfrac{5}{3}$. Nên $E=\left(\dfrac{5}{3};+\infty\right) $.\\
			Do đó   $C_{\mathbb{R}}E =\left(-\infty;\dfrac{5}{3}\right] $.
		\end{enumerate}
	}
\end{bt}

%%==========Bài 3
\begin{bt}%Câu 3%[ĐỀ GIỮA HỌC KÌ I- THPT Bình Hưng Hòa-HCM]%[Nguyễn Tài Tuệ]%[0D2H1-2]
	Biểu diễn miền nghiệm của bất phương trình $x+2 y-3 \leq 0$ trên mặt phẳng tọa độ $O x y$.
	\loigiai{
		\begin{itemize}
			\item Vẽ đường thẳng $x+2y-3=0$.\\
			Với $x=0\Rightarrow y=\dfrac{3}{2}$, với $y=0\Rightarrow x=3$.
			\item Thay $O(0;0)$ vào bất phương trình đã cho ta thấy thõa mãn.\\
			Do đó miền nghiệm của bất phương trình là nửa mặt chứa điểm $O$ có bờ là đường thẳng $x+2y-3=0$ (kể cả bờ).
			\begin{center}
				\begin{tikzpicture}[line join=round, line cap=round,>=stealth,thick]
					\tikzset{every node/.style={scale=0.9}}
					\begin{scope}
						\clip (-1,-1) rectangle (4,3);
						\fill[pattern=north east lines] (-4,3.5)--(6,3.5)--(6,-1.5)--cycle;
						\draw (-3,3)--(5,-1) node [pos=0.55, below, sloped] {$x+2y-3=0$};
					\end{scope}
					\draw[->] (-1,0)--(4,0) node[below]{$x$};
					\draw[->] (0,-1)--(0,3) node[left]{$y$};
					\draw (0,0) node[below left]{$O$};
				\end{tikzpicture}
			\end{center}
		\end{itemize}
	}
\end{bt}
%%==========Bài 4
\begin{bt}%[0H4V3-1]%[Dự án đề kiểm tra Toán 10 GHKI NH23-24- Lê Văn Toàn]%[Bình Hưng Hòa]
	Cho tam giác $ABC$ có $\widehat{A}=70^\circ$, $\widehat{B}=80^\circ$ và $AB=5$.
	\begin{enumerate}
		\item Giải tam giác $ABC$.
		\item Tính bán kính đường tròn ngoại tiếp và diện tích của tam giác $ABC$.
	\end{enumerate}
	\loigiai{
		\begin{enumerate}
			\item Ta có $\widehat{C}=180^\circ -\left(\widehat{A}+\widehat{B}\right)=180^\circ-\left(70^\circ+80^\circ\right)=30^\circ$.\\
			Áp dụng định lý $\sin$ ta có $\dfrac{BC}{\sin A}=\dfrac{AC}{\sin B}=\dfrac{5}{\sin 30^\circ}$.\\
			Suy ra $BC=\dfrac{5\sin 70^\circ}{\sin 30^\circ}\approx 9{,}3969$; $AC=\dfrac{5\sin 80^\circ}{\sin 30^\circ}\approx 9{,}8481$.\\
			\item Theo công thức diện tích tam giác ta có $$S=AC \cdot AB \cdot \sin A\approx 9{,}8481 \cdot 5 \cdot \sin 30^\circ \approx 24{,}6202.$$
			Ta có $S=\dfrac{abc}{4R} \Rightarrow \dfrac{abc}{4S}=\dfrac{9{,}3969 \cdot 9{,}8481 \cdot 5}{4\cdot 24{,}6202} \approx 4{,}6985$.
		\end{enumerate}
		
	}
\end{bt}

%%==========Bài 5
\begin{bt}%[0D1V3-5]%[Dự án đề kiểm tra Toán 10 GHKI NH23-24- Lê Văn Toàn]%[Bình Hưng Hòa]
	Lớp $10C$ có $20$ bạn chơi bóng đá, $25$ bạn chơi cầu lông, $12$ bạn chơi cả hai môn và $7$ bạn không chơi môn nào trong hai môn thể thao này.
	\begin{enumerate}
		\item Lớp $10C$ có bao nhiêu học sinh chơi ít nhất một trong hai môn thể thao bóng đá và cầu lông?
		\item Lớp $10C$ có bao nhiêu học sinh?
	\end{enumerate}
	\loigiai{
		\begin{enumerate}
			\item Gọi $A$, $B$ lần lượt là tập hợp các bạn chơi bóng đá và cầu lông.\\
			Khi đó, $n(A\cup B)=n(A)+n(B)-n(A \cap B)=20+25-12=33$.\\
			Vậy có $33$ bạn chơi ít nhất một trong hai môn thể thao.
			\item Số học sinh của lớp là bao gồm các bạn chơi ít nhất một trong hai môn thể thao và các bạn không chơi môn nào.\\
			Vậy lớp $10C$ có $33+7=40$ học sinh.
		\end{enumerate}
	}
\end{bt}
%%==========Bài 6
\begin{bt}%%[0D2C2-3]%[Dự án đề kiểm tra Toán 10 GHKI NH23-24- Nguyễn Hoàng Việt]%[Bình Hưng Hòa][1.0 điểm] 
	Một công ty cần thuê xe để chở $120$ thùng sơn và $6{,}5$ tấn bột bả. Nơi thuê xe có hai loại xe $A$ và $B$, trong đó loại xe $A$ có $9$ chiếc và loại xe $B$ có $8$ chiếc. Một chiếc xe loại $A$ cho thuê với giá 4 triệu đồng, một chiếc xe loại $B$ cho thuê với giá 3 triệu đồng. Biết rằng mỗi chiếc xe loại $A$ có thể chở tối đa $20$ thùng sơn và $0{,}5$ tấn bột bả; mỗi chiếc xe loại $B$ có thể chở tối đa $10$ thùng sơn và $2$ tấn bột bả. Hỏi phải thuê bao nhiêu xe mỗi loại để chi phí bỏ ra là thấp nhất?
	\loigiai{Gọi $x$(xe), $y$(xe) lần lượt là số xe loại $A$ và loại $B$ cần phải thuê.\\
		Số tiền cần bỏ ra để thuê xe là: $ f\left(x;y\right)=4x+3y$ (triệu đồng).
		\immini{
			
			Ta có $x$ xe loại $A$ và $y$ xe loại $B$ sẽ chở được $20x+10y$ thùng sơn và $0{,}5x+2y$ tấn bột bả.\\
			Theo đề bài, ta có hệ bất phương trình: 
			$\heva{&0 \le x \le 9\\ &0 \le y \le 8\\ &20x+10y \ge 120\\ &0{,}5x+2y \ge 6.5}
			\Leftrightarrow 
			\heva{ &0\le x\le9\\ &0\le y\le8\\ &2x+y\ge12\\ &x+4y\ge13} $.\\
			Miền nghiệm của hệ bất phương trình trên là tứ giác $ABCD$(kể cả biên) với $A\left(5;2\right)$, $B\left(9;1\right)$, $C\left(9;8\right)$, $D\left(2;8\right)$ như hình vẽ.\\		
			Ta có: $ f\left(5;2\right)=26$; $ f\left(9;1\right)=39$; $ f\left(9;8\right)=60$; $ f\left(2;8\right)=32$.\\
			Suy ra $ f\left(x;y\right)$ nhỏ nhất khi $\left(x;y\right)=\left(5;2\right)$\\
			Vậy để chi phí thuê là thấp nhất thì cần thuê $5$ xe loại $A$ và $2$ xe loại $B$.	
		}{
			\begin{tikzpicture}[scale=0.45,thick,>=stealth']
				% Tiến hành vẽ hai trục tọa độ
				\draw[->,color=black, thick] (-2,0) -- (14.3,0)node[above]{$x$}; 
				\foreach \x in {-2,2,4,6,8,10,12,14}
				\draw[shift={(\x,0)},color=black] (0pt,2pt) -- (0pt,-2pt) node[below] {\footnotesize $\x$};
				\draw[->,color=black] (0,-2) -- (0,14.3)node[right]{$y$};
				\foreach \y in {-2,2,4,6,8,10,12,14}
				\draw[shift={(0,\y)},color=black] (2pt,0pt) -- (-2pt,0pt) node[left] {\footnotesize $\y$};
				\node[below left] at (0,0) {$O$};				
				\clip(-2,-2) rectangle (14.3,14.3); 
				\draw[line width=1.2pt,smooth,samples=100,domain=-2:14.3] plot(\x,{8+0*(\x)});
				\fill[pattern=north west lines,smooth,opacity=0.5,pattern color=green] (-2,8)--(-2,14.3)--(14.3,14.3)--(14.3,8)-- cycle;
				\draw[line width=1.2pt,smooth,samples=100,domain=-2:14.6] plot(\x,{0*(\x)});
				\fill[pattern= north east lines,smooth,opacity=0.5,pattern color=yellow] (-2,-2)--(-2,0) --(14.3,0)--(14.3,-2)-- cycle;
				\fill[pattern=north east lines,smooth,opacity=0.5,pattern color=orange] (9,16.3)--(14.3,14.3)--(14.3,-2)--(9,-2)-- cycle;
				\draw[line width=1.2pt,smooth,samples=100](9,-2)--(9,14.3);
				\draw[line width=1.2pt,smooth,samples=100](0,-2)--(0,14.3); 
				\fill[pattern=north east lines,smooth,opacity=0.5,pattern color=blue] (-2,-2)--(-2,14.3)--(0,14.3)--(0,-2)-- cycle; 
				\draw[line width=1.2pt,smooth,samples=100,domain=-2:14.3] plot(\x,{12-2*(\x)}); 
				\fill[pattern=north east lines,smooth,opacity=0.5,pattern color=red] (-2,-2)--(-2,14.3)--(-1.15,14.3) -- (7,-2)-- cycle;	
				\draw[line width=1.2pt,smooth,samples=100,domain=-2:14.3] plot(\x,{13/4-1/4*(\x)});
				\fill[pattern=north west lines,smooth,opacity=0.5,pattern color=blue] (-2,-2)--(-2,15/4)--(14.3,-1.3/4)--(14.3,-2)-- cycle;
				\node[above right] at (5,2) {$A$};
				\node[below right] at (2,8) {$B$};
				\node[below left] at (9,8) {$C$};
				\node[above left] at (9,1) {$D$};
			\end{tikzpicture}
		}
		
	}
\end{bt}




%%==========Bài 7
\begin{bt}%%[0H1V2-2]%[Dự án đề kiểm tra Toán 10 GHKI NH23-24- Nguyễn Hoàng Việt]%[Bình Hưng Hòa][1.0 điểm] 
	\immini{
		Cho hình thang $ABCD$ có đáy cạnh đáy bé $AB=12$ và cạnh bên $BC=20$ 
		(minh họa như hình vẽ bên dưới). Biết rằng $\cos\widehat{ABC}=\dfrac{4}{5}$ và $\dfrac{AC}{BD}=\dfrac{\sqrt{290}}{15}$. Tính chính xác
		diện tích hình thang $ABCD$.		
	}{\begin{tikzpicture}[>=stealth,line join=round, line cap=round, scale=0.7]
			\coordinate (A) at (2,3);
			\coordinate (B) at (5.5,3);
			\coordinate (C) at (9,-1);
			\coordinate (D) at (0,-1);
			\coordinate (E) at (5.5,-1);
			\draw (A) node[above]{$A$}--(B) node[above]{$B$}--(C) node[below right]{$C$}--(D) node[below]{$D$}--(A);
			\node [blue, above] at (3.75,3) {$12$};
			\draw (5.5,3)--(9,-1) node [blue, pos=0.5, above, sloped] {$20$};				
			\draw 	pic["", draw=black, angle eccentricity=1.5, angle radius=0.7cm]{angle=A--B--C};			
			\fill (2,3) circle (1.5pt);
			\fill (5.5,3) circle (1.5pt);
			\fill (9,-1) circle (1.5pt);
		%	\fill (5.5,-1) circle (1.5pt);
			\fill (0,-1) circle (1.5pt);
		\end{tikzpicture}
	}
	\loigiai{
		\immini{
			\allowdisplaybreaks
			Áp dụng định lí hàm số cô-sin ta có
			\begin{eqnarray*}
				{{AC}^2}&=&{{AB}^2}+{{BC}^2}-2\cdot AB\cdot BC\cdot cos\widehat{ABC}\\&=&928\\
				\Rightarrow AC&=&4\sqrt{58}
			\end{eqnarray*}		
			Do 
		\allowdisplaybreaks
		$\begin{aligned}[t]
			\dfrac{AC}{BD}&=\dfrac{\sqrt{290}}{15}\\
			\Rightarrow BD&=\dfrac{15\cdot AC}{\sqrt{290}}
			&=\dfrac{15\cdot 4\sqrt{58}}{\sqrt{290}}
			&=12\sqrt{5}
		\end{aligned}$\\
		}{
			\begin{tikzpicture}[>=stealth,line join=round, line cap=round, scale=0.8]
				\coordinate (A) at (2,3); 
				\coordinate (B) at (5.5,3);
				\coordinate (C) at (9,-1);
				\coordinate (D) at (0,-1);
				\coordinate (E) at (5.5,-1);
				\draw (A) node[above]{$A$}--(B) node[above]{$B$}--(C) node[below right]{$C$}--(D) node[below]{$D$}--(A)--(C) (B)--(D);
				\draw[dashed] (E) node[below]{$E$}--(B); 
				\node [blue, above] at (3.75,3) {$12$};
				\draw (5.5,3)--(9,-1) node [blue, pos=0.5, above, sloped] {$20$}; \draw 	pic["", draw=black, angle eccentricity=1.5, angle radius=0.7cm]{angle=A--B--C};
				
				\draw pic[draw,angle radius=3mm]{right angle= B--E--D};
				\fill (2,3) circle (1.5pt);
				\fill (5.5,3) circle (1.5pt);
				\fill (9,-1) circle (1.5pt);
				\fill (5.5,-1) circle (1.5pt);
				\fill (0,-1) circle (1.5pt);
			\end{tikzpicture}
		}
	
		Kẻ $BE\perp DC$ ta có: 
		\allowdisplaybreaks	%\begin{eqnarray*}
			$\begin{aligned}[t]	
			{{S}_{ABC}}&=&{{S}_{ABE}} \\
			\Leftrightarrow 
			\dfrac{1}{2}\cdot AB\cdot BC\cdot \sin \widehat{ABC}&=&\dfrac{1}{2}\cdot AB\cdot BE \\
			\Rightarrow BE&=&BC\cdot \sin\widehat{ABC}
			&=&20\cdot \dfrac{3}{5}&=&12
			\end{aligned}$\\
		%\end{eqnarray*}	
		Mặt khác
		$DE=\sqrt{B{{D}^2}-B{{E}^2}}=24$ và $EC=\sqrt{B{{C}^2}-B{{E}^2}}=16$ \\
		suy ra $DC=DE+EC=24+16=40$\\
		Vậy ${{S}_{ABCD}}=\dfrac{\left(AB+DC\right).BE}{2}=\dfrac{\left(12+40\right)\cdot 12}{2}=312$
	}
\end{bt}



