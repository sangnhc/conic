\de{ĐỀ THI HỌC KỲ II NĂM HỌC 2022-2023}{THPT Marie Curie}

\begin{ex}%[0T7B2-1]%[Dự án đề kiểm tra HKII NH22-23- Lê Hung Thắng]%[THPT Marie-Curie]
	Tìm tập xác định của hàm số
	$y=\sqrt{-x^2+6 x-5}+\dfrac{2}{2 x-x^2}$.
	\loigiai{
Điều kiện $\heva{&-x^2+6 x-5 \geq 0 \\& 2 x-x^2 \neq 0} \Leftrightarrow \heva{&1\leq x\leq 5 \\& x\ne 0 \\& x \ne 2} \Leftrightarrow \heva{&1\leq x\leq 5 \\& x \ne 2.}$\\
Vậy tập xác định là $\mathscr{D} = [1;5]\setminus \{2\}$.
	}
\end{ex}

\begin{ex}%[0T7B3-2]%[Dự án đề kiểm tra HKII NH22-23- Lê Hung Thắng]%[THPT Marie-Curie]
	Giải phương trình $\sqrt{2x^2-3x-1}=3x+5$.
	\loigiai{
	Bình phương hai vế ta được	\allowdisplaybreaks \vspace*{-0.8cm}
	\begin{eqnarray*} 
		&  &  \sqrt{2x^2-3x-1}=3 x+5\\
		&  \Rightarrow  & 2x^2-3x-1= 9x^2 +30x +25\\
		&  \Rightarrow  & -7x^2-33x-26=0 \\
		&  \Rightarrow  & \hoac{&x=-1 \\& x= -\dfrac{26}{27}.}
	\end{eqnarray*}
	Thế lần lượt $x=-1 ; x=-\dfrac{26}{7}$ vào phương trình ban đầu ta thấy chỉ có $x=-1$ thỏa.\\
	Vậy tập nghiệm của phương trình là $S=\{-1\}$.
	}
\end{ex}

\begin{ex}%[0T8K2-2]%[Dự án đề kiểm tra HKII NH22-23- Lê Hung Thắng]%[THPT Marie-Curie]
	Từ các chữ số $1 ; 2 ; 3 ; 4 ; 5 ; 6$ lập đuợc bao nhiêu số tự nhiên chẵn có $5$ chữ số khác nhau mà chữ số chính giữa phải là chữ số lẻ?
	\loigiai{
Gọi số tự nhiên thỏa đề bài có dạng $n=\overline{abcde}$.\\
	$\bullet$ $n$ là số chẵn nên lập $e$ có $3$ cách.\\
	$\bullet$ chữ số chính giữa là chữ số lẻ nên lập $c$ có $3$ cách.\\
	$\bullet$ Số cách lập ba vị trí $a,b,d$ còn lại là $\mathrm{A}_4^3=24$ cách.\\
	Theo quy tắc nhân suy ra số các số $n$ thỏa đề là $3\cdot 3\cdot 24 = 216$ (số).
	}
\end{ex}

\begin{ex}%[0T8K2-2]%[Dự án đề kiểm tra HKII NH22-23- Lê Hung Thắng]%[THPT Marie-Curie]
	Một đội cứu hộ động đất quốc tế gồm $5$ người Việt Nam, $7$ người Thái Lan và $9$ người Hàn Quốc.
	
	$a$) Có bao nhiêu cách phân công $3$ trong $5$ người Việt Nam làm trưởng nhóm, phó nhóm và truyền thông tin?
	
	$b$) Có bao nhiêu cách chọn ra $2$ người Thái Lan và $3$ người Hàn Quốc và phân công $5$ nhiệm vụ khác nhau cho $5$ người này?
	\loigiai{
		
	$a$) Số cách phân công $3$ trong $5$ nguời Việt Nam làm trưởng nhóm, phó nhóm và truyền thông tin là $\mathrm{A}_5^3=60$ cách.
	
	$b$) Số cách chọn $2$ người Thái Lan và $3$ người Hàn Quốc là $\mathrm{C}_7^2 \mathrm{C}_9^3=1764$ cách.\\
	Số cách phân công $5$ nhiệm vụ cho $5$ người là $5 !=120$ cách.\\
	Vậy số cách chọn và phân công là $1764 \cdot 120 = 211680$ cách.
	}
\end{ex}

\begin{ex}%[0T0K2-2]%[Dự án đề kiểm tra HKII NH22-23- Lê Hung Thắng]%[THPT Marie-Curie]
	Để lì xì đầu năm cho các cháu, bà Hoa chuẩn bị $30$ bao lì xì có họa tiết khác nhau gồm $15$ bao mỗi bao mệnh giá $50$ nghìn, $10$ bao mỗi bao mệnh giá $100$ nghìn và $5$ bao mỗi bao mệnh giá $200$ nghìn. Bạn Long nhỏ tuổi nhất nên được bốc trước $3$ bao lì xì. Tính xác suất để bạn Long nhận được $300$ nghìn tiền lì xì.
	\loigiai{
	Số cách chọn $3$ bao lì xì từ $30$ bao lì xì là $\mathrm{C}_{30}^3=4040$.\\
	Suy ra $n(\Omega)=4040$.\\
	Gọi biến cố $A:$ \lq\lq Long nhận đuợc $300$ nghìn tiền lì xì\rq\rq.\\
	Ta có $300 = 3 \cdot 100 = 2\cdot 50 +200$. Do đó $n(A)=\mathrm{C}_{10}^3+\mathrm{C}_{15}^2\cdot \mathrm{C}_5^1=645$.\\
	Xác suất của biến cố $A$ là $P(A)=\dfrac{n(A)}{n(\Omega)}=\dfrac{645}{4040}=\dfrac{129}{808}$.
	}
\end{ex}

\begin{ex}%[0T8B3-1]%[Dự án đề kiểm tra HKII NH22-23- Lê Hung Thắng]%[THPT Marie-Curie]
	Khai triển biểu thức $(3x+2)^4$.
	\loigiai{Ta có
	$$\begin{aligned}
	(3 x+2)^4 
	& =\mathrm{C}_4^0(3 x)^4+\mathrm{C}_4^1(3 x)^3 2+\mathrm{C}_4^2(3 x)^2 2^2+\mathrm{C}_4^3(3 x) 2^3+\mathrm{C}_4^4 2^4 \\
	& =81 x^4+216 x^3+216 x^2+96x+16.
	\end{aligned}$$
	}
\end{ex}

\begin{ex}%[0T9B3-2]%[Dự án đề kiểm tra HKII NH22-23- Lê Hung Thắng]%[THPT Marie-Curie]
	Trong mặt phẳng $Oxy$, viết phương trình đường tròn tâm $I(1;2)$ và có bán kính $R=AB$ với $A(-1;0)$, $B(0;2)$.
	\loigiai{
	Bán kính là $R=AB=\sqrt{1^2+2^2}=\sqrt{5}$.\\
	Phương trình của đường tròn là $(x-1)^2+(y-2)^2=5$.
	}
\end{ex}

\begin{ex}%[0T9K3-3]%[Dự án đề kiểm tra HKII NH22-23- Lê Hung Thắng]%[THPT Marie-Curie]
	Trong mặt phẳng $Oxy$, cho đường tròn $(C)\colon x^2+y^2+4 x-2 y-4=0$. Viết phương trình tiếp tuyến $\Delta$ của đường tròn $(C)$, biết tiếp tuyến vuông góc với đường thẳng $d\colon 6x-8y+1=0$.
	\loigiai{
	Đường tròn $(C)$ có tâm là $I(-2;1)$ và bán kính là $R=3$.\\
$\bullet$ $\Delta \perp d \Rightarrow \Delta\colon 8x+6y+c=0$.\\
$\bullet$ $\mathrm{d}(I,\Delta)=R \Leftrightarrow|c-10|=30	\Leftrightarrow\hoac{&c=40 \\&c=-20.}$\\
Vậy có $2$ tiếp tuyến là\\
 $\Delta_1: 8x+6y+40=0 \Leftrightarrow 4 x+3y+20=0$\\
 $\Delta_2: 8x+6y-20=0 \Leftrightarrow 4 x+3y-10=0$.
}
\end{ex}

\begin{ex}%[0T9B4-1]%[Dự án đề kiểm tra HKII NH22-23- Lê Hung Thắng]%[THPT Marie-Curie]
	Trong mặt phẳng $Oxy$, cho elip $(E)\colon \dfrac{x^2}{9}+\dfrac{y^2}{4}=1$. Tìm tọa độ đỉnh và tiêu điểm của $(E)$.
	\loigiai{
	$\bullet$ Theo đề ta có $\heva{&a^2=9 \\& b^2=4} \Leftrightarrow \heva{&a=3\\ &b=2.}$\\
	$\bullet$ $c^2=a^2-b^2= 9-4=5 \Rightarrow c=\sqrt{5}$.\\
	$\bullet$ Các đỉnh là $A_1(-3;0); A_2(3;0); B_1(0;-2); B_2(0;2)$.\\
	$\bullet$ Các tiêu điểm là $F_1(-\sqrt{5};0); F_2(\sqrt{5};0)$.
	}
\end{ex}

\begin{ex}%[0T7T2-1]%[Dự án đề kiểm tra HKII NH22-23- Lê Hung Thắng]%[THPT Marie-Curie]
	Một quán ăn bán buffet trưa với giá $250.000$ đồng/người. Bạn Nam đến đặt tiệc cho công ty và hỏi quản lý nếu đi nhiều người có khuyến mãi gì không? Quản lý trả lời: Nếu đoàn khách có nhiều hơn $40$ người thì cứ có thêm $1$ người, giá vé sẽ giảm $5.000$ đồng/người cho toàn bộ đoàn khách nhưng chỉ được áp dụng tối đa cho $m$ khách (vì nếu vượt quá số $m$ thì quán sẽ lỗ do chi phí quán bỏ ra để kinh doanh một ngày là $7.000.000$ đồng). Vậy hãy cho biết con số $m$ mà quản lý muốn nói là bao nhiêu?
	\loigiai{
	Gọi $x$ là số người tăng thêm $(x>0)$.\\
	Thêm $1$ người thì giá vé là $250.000-5000\cdot 1$\\
	Thêm $x$ người thì giá vé là $250.000-5000 \cdot x$\\
	Tổng doanh thu tương ứng của quán là $(40+x)(250.000-5000 \cdot x)$\\
	Để quán không bị lỗ (có thể hòa vốn) thì
	$$\begin{aligned}
	& (40+x)(250.000-5000 \cdot x) \geq 7.000 .000 \\
	 \Leftrightarrow & -x^2+10 x+600 \geq 0 \\
	 \Leftrightarrow & -20 \leq x \leq 30.
	\end{aligned}$$
	Vậy số khách tối đa được áp dụng giảm giá vé là $m=40+30=70$ khách.
	}
\end{ex}


