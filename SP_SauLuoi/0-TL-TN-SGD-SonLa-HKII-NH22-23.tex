
\de{ĐỀ THI HỌC KỲ I NĂM HỌC 2022-2023}{SGD Sơn La}
\begin{center}
	\textbf{PHẦN 1 - TRẮC NGHIỆM}
\end{center}
\Opensolutionfile{ans}[ans/ans]
%Câu 1...........................
\begin{ex}%[0D3Y2-1]%[Dự án đề kiểm tra HKII NH22-23 -Ngô Quang Anh]%[SGD Sơn La]
	Tung một đồng xu cân đối và đồng chất một lần. Xác suất của biến cố:  \lq\lq Kết quả tung đồng xu được mặt sấp\rq\rq\, là
	\choice
	{$\dfrac{1}{3}$}
	{\True $\dfrac{1}{2}$}
	{$\dfrac{3}{4}$}
	{$\dfrac{1}{4}$}
	\loigiai{
	Số phần tử của không gian mẫu là $2$.\\
	Tung đồng xu một lần được mặt sấp có $1$ cách.\\
	Suy ra xác suất cần tìm là $\dfrac{1}{2}$.
		}
\end{ex}
\begin{ex}%[0H4Y1-1]%[Dự án đề kiểm tra HKII NH22-23 -Ngô Quang Anh]%[SGD Sơn La]
	Trong mặt phẳng toạ độ $Oxy$, cho đường thẳng $\Delta\colon 3x+4y+1=0$. Một véc-tơ pháp tuyến của đường thẳng $\Delta$ có tọa độ là
	\choice
	{\True $(3; 4)$}
	{$(3; -4)$}
	{$(4; -3)$}
	{$(-3; 4)$}
	\loigiai{
	Một véc-tơ pháp tuyến của đường thẳng $\Delta$ có tọa độ là $(3; 4)$.
	}
\end{ex}
\begin{ex}%[0D2Y2-8]%[Dự án đề kiểm tra HKII NH22-23 -Ngô Quang Anh]%[SGD Sơn La]
	Số các hoán vị của $6$ phần tử là
	\choice
	{\True $6!$}
	{$5!$}
	{$5$}
	{$6$}
	\loigiai{
	Số các hoán vị của $6$ phần tử là $6!$.
	}
\end{ex}
\begin{ex}%[0H3Y1-3]%[Dự án đề kiểm tra HKII NH22-23 -Ngô Quang Anh]%[SGD Sơn La]
	Trong mặt phẳng toạ độ $Oxy$, cho $M(-2; 3)$ và $N(4; 5)$. Tọa độ trung điểm $I$ của đoạn thẳng $MN$ là
	\choice
	{ $(-2; 8)$}
	{$(2; 8)$}
	{\True $(1; 4)$}
	{$(-1; 4)$}
	\loigiai{
	Tọa độ trung điểm $I$ của đoạn thẳng $MN$ là  $(1; 4)$.
	}
\end{ex}
\begin{ex}%[0D2Y2-8]%[Dự án đề kiểm tra HKII NH22-23 -Ngô Quang Anh]%[SGD Sơn La]
	Cho $n$ là một số nguyên dương, $n \geq 8$. Số các chỉnh hợp chập 8 của $n$ phần tử là
	\choice
	{$\mathrm{C}_{n}^8$}
	{$8!$}
	{\True $\mathrm{A}_{n}^8$}
	{$n!$}
	\loigiai{
	Số các chỉnh hợp chập 8 của $n$ phần tử là $\mathrm{A}_{n}^8$.
	}
\end{ex}
\begin{ex}%[0D2Y1-1]%[Dự án đề kiểm tra HKII NH22-23 -Ngô Quang Anh]%[SGD Sơn La]
	Một công việc được hoàn thành bởi một trong hai hành động. Nếu hành động thứ nhất có $m$ cách thực hiện, hành động thứ hai có $n$ cách thực hiện (các cách thực hiện của cả $2$ hành động là khác nhau đôi một) thì số cách hoàn thành công việc đó là
	\choice
	{\True $m+n$}
	{$m-n$}
	{$\dfrac{m}{n}$}
	{$m \cdot n$}
	\loigiai{
	Số cách hoàn thành công việc đó là $m+n$.
	}
\end{ex}
\begin{ex}%[0D3Y1-1]%[Dự án đề kiểm tra HKII NH22-23 -Ngô Quang Anh]%[SGD Sơn La]
	Tung một đồng xu cân đối và đồng chất hai lần. Số phần tử của biến cố cả hai lần tung đều xuất hiện mặt ngửa là
	\choice
	{$2$}
	{$3$}
	{\True $1$}
	{$4$}
	\loigiai{
	Ta có $n\left(\Omega\right)=2\cdot2=4$.\\
	Gọi $A$ là biến cố \lq\lq Cả hai lần tung đều xuất hiện mặt ngửa \rq\rq.\\
	Suy ra $n\left(A\right)=1$.
	}
\end{ex}
\begin{ex}%[0D3Y1-2]%[Dự án đề kiểm tra HKII NH22-23 -Ngô Quang Anh]%[SGD Sơn La]
	Gieo một con xúc xắc cân đối và đồng chất ba lần. Số phần tử của không gian mẫu là
	\choice
	{\True $6^3$}
	{$6^2$}
	{$6$}
	{$6^4$}
	\loigiai{
	Ta có $n\left(\Omega\right)=6\cdot 6\cdot 6=6^3$.
	}
\end{ex}
\begin{ex}%[0H4Y1-3]%[Dự án đề kiểm tra HKII NH22-23 -Ngô Quang Anh]%[SGD Sơn La]
	Trong mặt phẳng tọa độ $Oxy$, cho hai đường thẳng $d_1$ và $d_2$ có phương trình lần lượt là $d_1\colon a_1x+b_1y+c_1=0$ và $d_2\colon a_2x+b_2y+c_2=0$. Hệ phương trình $\heva{&a_1 x+b_1 y+c_1=0 \\ & a_2 x+b_2 y+c_2=0}$ có vô số nghiệm. Khi đó $d_1$ và $d_2$
	\choice
	{song song}
	{cắt nhau nhưng không vuông góc nhau}
	{\True trùng nhau}
	{vuông góc}
	\loigiai{
	Hệ phương trình vô số nghiệm khi $d_1$ và $d_2$ trùng nhau.
	}
\end{ex}
\begin{ex}%[0H4Y2-1]%[Dự án đề kiểm tra HKII NH22-23 -Ngô Quang Anh]%[SGD Sơn La]
	Trong mặt phẳng tọa độ $Oxy$, cho đường tròn $(C)\colon x^2+y^2-4x+2y+1=0$. Tâm $I$ của $(C)$ có tọa độ là
	\choice
	{$I(-2 ; 1)$}
	{$I(4 ;-2)$}
	{$I(-4 ; 2)$}
	{\True $I(2 ;-1)$}
	\loigiai{
	Tâm $I$ của $(C)$ có tọa độ là $(2 ;-1)$.
	}
\end{ex}
\begin{ex}%[0H4Y2-2]%[Dự án đề kiểm tra HKII NH22-23 -Ngô Quang Anh]%[SGD Sơn La]
	Trong mặt phẳng toạ độ $Oxy$, phương trình đường tròn có tâm $I(-2; 1)$ và bán kính $R=2$ là
	\choice
	{$(x+2)^2+(y-1)^2=2$}
	{\True $(x+2)^2+(y-1)^2=4$}
	{$(x-2)^2+(y+1)^2=2$}
	{$(x-2)^2+(y+1)^2=4$}
	\loigiai{
	Phương trình đường tròn có tâm $I(-2; 1)$ và bán kính $R=2$ là $(x+2)^2+(y-1)^2=4$.
	}
\end{ex}

\begin{ex}%[0D1Y1-3]%[Dự án đề kiểm tra HKII NH22-23 -Ngô Quang Anh]%[SGD Sơn La]
	Giá trị gần đúng của $\sqrt{5}$ chính xác đến hàng phần trăm là
	\choice
	{$2{,}2$ }
	{\True $2{,}24$}
	{$2{,}3$}
	{$2{,}236$}
	\loigiai{
	Ta có $\sqrt{5}\approx 2{,}24$.
	}
\end{ex}
\begin{ex}%[0D1Y4-3]%[Dự án đề kiểm tra HKII NH22-23 -Ngô Quang Anh]%[SGD Sơn La]
	Cho phương sai của một mẫu số liệu là $0{,}4$ thì độ lệch chuẩn của mẫu số liệu đó là
	\choice
	{$\dfrac{\sqrt{2}}{5}$}
	{\True $\dfrac{\sqrt{10}}{5}$}
	{$\dfrac{1}{\sqrt{5}}$}
	{$\dfrac{2}{5}$}
	\loigiai{
	Độ lệch chuẩn của mẫu số liệu đó là $\sqrt{0{,}4}=\dfrac{\sqrt{10}}{5}$.
	}
\end{ex}
\begin{ex}%[0D3Y1-1]%[Dự án đề kiểm tra HKII NH22-23 -Ngô Quang Anh]%[SGD Sơn La]
	Từ một hộp chứa $5$ quả cầu trắng, $6$ quả cầu đỏ, $7$ quả cầu vàng, các quả cầu có kích thước và khối lượng giống nhau. Xét phép thử: "Lấy ngẫu nhiên đồng thời $3$ quả cầu". Biến cố đối của biến cố $A\colon$ "$3$ quả cầu lấy ra có ít nhất một quả màu đỏ” là
	\choice
	{$\bar{A}\colon$ "3 quả cầu lấy ra cùng màu trắng"}
	{\True $\bar{A}\colon$ "3 quả cầu lấy ra không có màu đỏ"}
	{$\bar{A}\colon$ "3 quả cầu lấy ra cùng màu vàng"}
	{$\bar{A}\colon$ "3 quả cầu lấy ra có 3 màu khác nhau"}
	\loigiai{
	Biến cố đối của biến cố $A$ là $\bar{A}\colon$ "3 quả cầu lấy ra không có màu đỏ".
	}
\end{ex}
\begin{ex}%[0H3Y1-3]%[Dự án đề kiểm tra HKII NH22-23 -Ngô Quang Anh]%[SGD Sơn La]
	Trong mặt phẳng tọa độ $Oxy$, cho $\vec{u}=(4;-3)$. Biểu diễn của véc-tơ $\vec{u}$ qua véc-tơ $\vec{i}$ và $\vec{j}$ là
	\choice
	{$\vec{u}=-4 \vec{i}-3 \vec{j}$}
	{\True $\vec{u}=4 \vec{i}-3 \vec{j}$}
	{$\vec{u}=4 \vec{i}+3 \vec{j}$}
	{$\vec{u}=-4 \vec{i}+3 \vec{j}$}
	\loigiai{
	Ta có $\vec{u}=(4;-3)\Rightarrow \vec{u}=4 \vec{i}-3 \vec{j}$.
	}
\end{ex}
\begin{ex}%[0H4Y1-1]%[Dự án đề kiểm tra HKII NH22-23 -Ngô Quang Anh]%[SGD Sơn La]
	Trong mặt phẳng toạ độ $Oxy$, cho đường thẳng $d\colon 4x+3y-28=0$. Điểm nào sau đây thuộc đường thẳng $d$?
	\choice
	{$A(4; 0)$}
	{\True $C(7; 0)$}
	{$B(-7; 0)$}
	{$D(3; 0)$}
	\loigiai{
	Vì $4\cdot 7+3\cdot 0-28=0$ nên $C(7; 0)\in d$.
	}
\end{ex}
\begin{ex}%[0D2Y2-8]%[Dự án đề kiểm tra HKII NH22-23 -Ngô Quang Anh]%[SGD Sơn La]
	Số các tổ hợp chập $3$ của $7$ phần tử là
	\choice
	{\True $3!$}
	{$\mathrm{A}_{7}^3$}
	{$\mathrm{C}_{7}^3$}
	{$7!$}
	\loigiai{
	Số các tổ hợp chập $3$ của $7$ phần tử là $\mathrm{C}_{7}^3$.
	}
\end{ex}
\begin{ex}%[0D1Y3-1]%[Dự án đề kiểm tra HKII NH22-23 -Ngô Quang Anh]%[SGD Sơn La]
	Điểm kiểm tra môn Văn của một nhóm gồm $7$ học sinh như sau:
	$$
	\begin{array}{llllllll}
		5 & 6 & 8 & 6 & 7 & 8 & 5
	\end{array}
	$$
	Điểm trung bình môn Văn của $7$ học sinh đó (kết quả làm tròn đến hàng phần mười) là
	\choice
	{$6{,}2$}
	{\True $6{,}4$}
	{$6{,}6$}
	{$6{,}7$}
	\loigiai{
	Điểm trung bình môn Văn của $7$ học sinh đó là
	$$\bar x=\dfrac{5+6+8+6+7+8+5}{7}=6{,}4.$$
	}
\end{ex}
\begin{ex}%[0H4Y1-1]%[Dự án đề kiểm tra HKII NH22-23 -Ngô Quang Anh]%[SGD Sơn La]
	Trong mặt phẳng hệ toạ độ $Oxy$, một véc-tơ chỉ phương của đường thẳng $\heva{&x=3+5t \\ &y=-2+t}$ có tọa độ là
	\choice
	{$(3; -2)$}
	{\True $(5; 1)$}
	{$(-5; 1)$}
	{$(3; 2)$}
	\loigiai{
	Một véc-tơ chỉ phương của đường thẳng đã cho có tọa độ là $(5; 1)$.
	}
\end{ex}
\begin{ex}%[0D1Y4-1]%[Dự án đề kiểm tra HKII NH22-23 -Ngô Quang Anh]%[SGD Sơn La]
	Mẫu số liệu sau cho biết cân nặng (đơn vị kg) của $7$ học sinh lớp 10B
	$$
	\begin{array}{lllllll}
		58 & 45 & 46 & 42 & 38 & 44 & 40
	\end{array}
	$$
	Khoảng biến thiên của mẫu số liệu trên là
	\choice
	{\True $20$}
	{$22$}
	{$42$}
	{$38$}
	\loigiai{
	Ta có giá trị lớn nhất là $58$, giá trị nhỏ nhất là $38$.\\
	Khoảng biến thiên là $R=58-38=20$.	
	}
\end{ex}
\begin{ex}%[0D2Y3-1]%[Dự án đề kiểm tra HKII NH22-23 -Ngô Quang Anh]%[SGD Sơn La]
	Khai triển nào sau đây đúng?
	\choice
	{$(1+x)^5=1+5 x+12 x^2+12 x^3+5 x^4+x^5$}
	{$(1+x)^5=1+x+x^2+x^3+x^4+x^5$}
	{$(1+x)^5=1+4 x+9 x^3+9 x^4+4 x^4+x^5$}
	{\True $(1+x)^5=1+5 x+10 x^2+10 x^3+5 x^4+x^5$}
	\loigiai{
	Ta có $(1+x)^5=1+5 x+10 x^2+10 x^3+5 x^4+x^5$.
	}
\end{ex}
\begin{ex}%[0D3Y1-2]%[Dự án đề kiểm tra HKII NH22-23 -Ngô Quang Anh]%[SGD Sơn La]
	Lớp 10A có $40$ học sinh với $24$ nữ và $16$ nam. Giáo viên chủ nhiệm cần chọn ra $4$ học sinh. Số phần tử của biến cố $A\colon$ "trong số 4 học sinh được chọn có ít nhất $2$ nam" là
	\choice
	{$48830$}
	{\True $48380$}
	{$46560$}
	{$33120$}
	\loigiai{
	 Số phần tử của biến cố $A$ là
	 $$n(A)=\mathrm{C}_{40}^4-(\mathrm{C}_{24}^4+\mathrm{C}_{24}^3\cdot\mathrm{C}_{16}^1)=48380.$$
	}
\end{ex}
\begin{ex}%[0D3Y2-4]%[Dự án đề kiểm tra HKII NH22-23 -Ngô Quang Anh]%[SGD Sơn La]
	Từ một hộp chứa $5$ quả cầu trắng, $6$ quả cầu đỏ, $7$ quả cầu vàng, các quả cầu có kích thước và khối lượng giống nhau, lấy ngẫu nhiên đồng thời $3$ quả cầu. Xác suất lấy được $3$ quả cầu có ba màu khác nhau là
	\choice
	{$\dfrac{35}{68}$}
	{\True $\dfrac{35}{136}$}
	{$\dfrac{11}{24}$}
	{$\dfrac{11}{36}$}
	\loigiai{
	Ta có $n\left(\Omega\right)=\mathrm{C}_{18}^3$.\\
	Gọi $A$ là biến cố \lq\lq lấy được $3$ quả cầu có ba màu khác nhau \rq\rq.\\
	Suy ra $n\left(A\right)=5\cdot 6\cdot 7$.\\
	Do đó $\mathrm{P}(A)=\dfrac{n(A)}{n(\Omega)}=\dfrac{35}{136}$.
	}
\end{ex}
\begin{ex}%[0H4Y1-4]%[Dự án đề kiểm tra HKII NH22-23 -Ngô Quang Anh]%[SGD Sơn La]
	Trong mặt phẳng tọa độ $Oxy$, góc giữa hai đường thẳng $d_1\colon 3x+y-6=0$ và $d_2\colon 2x-y+5=0$ là
	\choice
	{$30^{\circ}$}
	{\True $45^{\circ}$}
	{$90^{\circ}$}
	{$135^{\circ}$}
	\loigiai{
	Ta có $d_1\colon 3x+y-6=0$ có véc-tơ pháp tuyến là $\overrightarrow{n}_1=(3;1)$.\\
	$d_2\colon 2x-y+5=0$ có véc-tơ pháp tuyến là $\overrightarrow{n}_2=(2;-1)$.\\
	$$\cos (d_1,d_2)=\dfrac{\left|\overrightarrow{n}_1\cdot \overrightarrow{n}_2\right|}{\left|\overrightarrow{n}_1\right|\cdot \left|\overrightarrow{n}_2\right|}=\dfrac{\left|2\cdot3+1\cdot(-1)\right|}{\sqrt{9+1}\cdot \sqrt{4+1}}=\dfrac{1}{\sqrt{2}}.$$
	Suy ra $\left(d_1,d_2\right)=45^\circ$.
	}
\end{ex}
\begin{ex}%[0D3B2-6]%[Dự án đề kiểm tra HKII NH22-23 -Ngô Quang Anh]%[SGD Sơn La]
	Cho tập hợp $A=\{0;1;2;3;4;5;6\}$ và $S$ là tập hợp các số tự nhiên có ba chữ số khác nhau được lập từ các chữ số của tập hợp $A$. Lấy ngẫu nhiên một số thuộc tập hợp $S$. Xác suất của biến cố: "Số được chọn là một số chia hết cho $5$" là
	\choice
	{$\dfrac{11}{35}$}
	{$\dfrac{11}{42}$}
	{\True $\dfrac{11}{36}$}
	{$\dfrac{11}{24}$}
	\loigiai{
	Gọi số cần tìm trong tập $S$ có dạng $\overline{abc}$.\\
	Ta có
	\begin{itemize}
		\item Chọn $a\in A\setminus\{0\}$ có $6$ cách.
		\item Chọn $b\in A\setminus\{a\}$ có $6$ cách.
		\item Chọn $c\in A\setminus\{a;b\}$ có $5$ cách.
	\end{itemize}
	Do đó tập $S$ có $6\cdot 6\cdot 5=180$ phần tử.\\
	Suy ra số phần tử của không gian mẫu là $n(\Omega)=\mathrm{C}_{180}^1=180$.\\
	Gọi $X$ là biến cố \lq\lq Số được chọn là một số chia hết cho $5$\rq\rq.\\
	Khi đó ta có 
	\begin{itemize}
		\item \textbf{TH1.} $c=0$.\\
		Số cách chọn $\overline{ab}$ là $\mathrm{A}_{6}^2=30$ cách.\\
		Suy ra trường hợp này có $30$ số chia hết cho $5$.
		\item \textbf{TH2.} $c=5$.
			\begin{itemize}
			\item Chọn $a\in A\setminus\{0\}$ có $5$ cách.
			\item Chọn $b\in A\setminus\{a, c\}$ có $5$ cách.
		\end{itemize}
	Suy ra trường hợp này có $5\cdot5=25$ số chia hết cho $5$.
	\end{itemize}
	Do đó số phần tử của biến cố $X$ là $n(X)=30+25=55$.\\
	Vậy xác suất cần tính là $\mathrm{P}(X)=\dfrac{n(X)}{n(\Omega)}=\dfrac{55}{180}=\dfrac{11}{36}$.
	}
\end{ex}
\begin{ex}%[0H3Y1-3]%[Dự án đề kiểm tra HKII NH22-23 -Ngô Quang Anh]%[SGD Sơn La]
	Trong mặt phẳng tọa độ $Oxy$, cho tam giác $ABC$ có trọng tâm $G$. Tìm tọa độ điểm $C$ biết $A(2; 1), B(-3; 0), G(1; 1)$?
	\choice
	{$C(4;-2)$}
	{$C(2; 0)$}
	{\True $C(4; 2)$}
	{$C(-2; 0)$}
	\loigiai{
	Vì $G$ là trọng tâm tam giác $ABC$ nên
	$$\heva{&x_G=\dfrac{x_A+x_B+x_C}{3}\\&y_G=\dfrac{y_A+y_B+y_C}{3}}\Leftrightarrow \heva{&x_C=3x_G-x_B-x_A=4\\&y_C=3y_G-y_B-y_A=2.}$$
	Vậy $C(4; 2)$.	
	}
\end{ex}
\begin{ex}%[0D3Y2-1]%[Dự án đề kiểm tra HKII NH22-23 -Ngô Quang Anh]%[SGD Sơn La]
	Gieo một con xúc xắc cân đối và đồng chất hai lần liên tiếp. Xác suất của biến cố: “Tổng số chấm xuất hiện trong hai lần gieo bằng $6$" là
	\choice
	{\True $\dfrac{5}{36}$}
	{$\dfrac{5}{6}$}
	{$\dfrac{7}{36}$}
	{$\dfrac{1}{6}$}
	\loigiai{
	Ta có $n\left(\Omega\right)=36$.\\
	Gọi $A$ là biến cố \lq\lq có tổng số chấm xuất hiện trong hai lần gieo bằng $6$ \rq\rq.\\
	$1+5=6$, $2+4=6$, $3+3=6$.
	Suy ra $n\left(A\right)=5$.\\
	Do đó $\mathrm{P}(A)=\dfrac{n(A)}{n(\Omega)}=\dfrac{5}{36}$.		
	}
\end{ex}
\begin{ex}%[0D1Y3-2]%[Dự án đề kiểm tra HKII NH22-23 -Ngô Quang Anh]%[SGD Sơn La]
	 Với $x$ nguyên dương, cho mẫu số liệu sau (đã sắp xếp theo thứ tự)
	$$
	\begin{array}{lllllllll}
		4 & 13 & 15 & 4 x-3 & 4 x-1 & 4 x & 22 & 23 & 25
	\end{array}
	$$
	Tìm $x$, biết rằng số trung vị trong mẫu số liệu trên bằng $19$.
	\choice
	{$x=12$}
	{$x=14$}
	{\True $x=5$}
	{$x=15$}
	\loigiai{
	Ta có bảng số liệu trên đã sắp thứ tự không giảm.\\
	Vì tổng số liệu của mẫu là $9$ nên trung vị $M_e=4x-1=19\Leftrightarrow x=5$.\\
	}
\end{ex}
\begin{ex}%[0D1B4-1]%[Dự án đề kiểm tra HKII NH22-23 -Ngô Quang Anh]%[SGD Sơn La]
	Cho mẫu số liệu sau
	$$
	\begin{array}{lllllllll}
		146 & 152 & 158 & 170 & 154 & 175 & 160 & 155 & x
	\end{array}
	$$
	Khi đó $x$ nhận giá trị nào sau đây để mẫu số liệu này có khoảng biến thiên là $30$?
	\choice
	{$130$}
	{$160$}
	{$180$}
	{\True $176$}
	\loigiai{
	Ta có $x$ chỉ có thể là giá trị lớn nhất hoặc giá trị nhỏ nhất.\\
	Vì khoảng biến thiên là $30$ nên $\hoac{&x-146=30 \\ &175-x=30}\Leftrightarrow \hoac{& x=176 \\ &x=145.}$
	}
\end{ex}
\begin{ex}%[0H4Y1-5]%[Dự án đề kiểm tra HKII NH22-23 -Ngô Quang Anh]%[SGD Sơn La]
	Trong mặt phẳng toạ độ $Oxy$, khoảng cách từ điểm $A(1; 3)$ đến đường thẳng \\$\Delta\colon x-y-3=0$ là
	\choice
	{$\dfrac{1}{\sqrt{2}}$}
	{$\sqrt{2}$}
	{\True $\dfrac{5}{\sqrt{2}}$}
	{$1$}
	\loigiai{
	Ta có $\mathrm{d}(A,\Delta)=\dfrac{\big|1-3-3\big|}{\sqrt{1^2+1^2}}=\dfrac{5}{\sqrt{2}}$.
	}
\end{ex}


\begin{ex}%[0H4B2-2]
	Trong mặt phẳng $Oxy$, phương trình đường tròn nhận $AB$ làm đường kính biết $A(-2; 2)$ và $B(2;-4)$ là
	\choice
	{$(x+1)^{2}+y^{2}=13$}
	{$(x+1)^{2}+y^{2}=9$}
	{$x^{2}+(y+1)^{2}=9$}
	{\True $x^{2}+(y+1)^{2}=13$}
	\loigiai{
		Đường tròn đường kính $AB$ nhận trung điểm $I(0;-1)$ của $AB$ làm tâm và bán kính \linebreak $R=\dfrac{AB}{2}=\sqrt{13}$.\\
		Phương trình đường tròn đường kính $AB$ là $x^{2}+(y+1)^{2}=13$.
	}
\end{ex}

\begin{ex}%[0X3B2-1]
	Tung một đồng xu cân đối và đồng chất ba lần liên tiếp. Xác suất của biến cố: \lq\lq Mặt ngửa xuất hiện ít nhất hai lần\rq\rq\, là
	\choice
	{\True $\dfrac{1}{2}$}
	{$\dfrac{3}{4}$}
	{$\dfrac{1}{3}$}
	{$\dfrac{1}{4}$}
	\loigiai{
		Xét biến cố $A\colon$ \lq\lq Mặt ngửa xuất hiện ít nhất hai lần\rq\rq.\\
		Ta có $\mathrm{P}(A)=\mathrm{C}_{3}^{2}\cdot\left(\dfrac{1}{2}\right)^3+\left(\dfrac{1}{2}\right)^3=\dfrac{1}{2}$.
	}
\end{ex}

\begin{ex}%[0X2B2-8]
	Cho $n \in \mathbb{N}, n \geq 5$ và $2\mathrm{C}_{n}^{5}=252$. Giá trị của $n$ bằng
	\choice
	{\True $9$}
	{$8$}
	{$6$}
	{$7$}
	\loigiai{
		Ta có 
		\allowdisplaybreaks
		$\begin{aligned}[t]
			&\quad 2\mathrm{C}_{n}^{5}=2\dfrac{n!}{(n-5)!\cdot 5!}=252\\
			&\Leftrightarrow n(n-1)(n-2)(n-3)(n-4)=15120=9\cdot 8\cdot 7\cdot 6\cdot 5\\
			&\Leftrightarrow n=9.
		\end{aligned}$
	}
\end{ex}

\begin{ex}%[0X1B3-3]
	Cho mẫu số liệu sau
	$$\begin{array}{lllllllll}3 & 6 & 9 & 10 & 12 & 14 & 15 & 18 & 20\end{array}$$
	Khoảng tứ phân vị của mẫu số liệu trên là
	\choice
	{$16{,}5$}
	{\True $9$}
	{$12$}
	{$7{,}5$}
	\loigiai{
	
	}
\end{ex}

\begin{ex}%[0H4B1-2]
	Trong mặt phẳng $Oxy$, phương trình tổng quát của đường thẳng đi qua $A(2;-1)$, $B(3;-2)$ là
	\choice
	{$x-y-3=0$}
	{$x-y-1=0$}
	{$x+y+3=0$}
	{\True $x+y-1=0$}
	\loigiai{
		Ta có $\overrightarrow{AB}=(1;-1)$, chọn $\overrightarrow{n}=(1;1)$ là một véc-tơ pháp tuyến của đường thẳng $AB$.\\
		Phương trình tổng quát của đường thẳng $AB$ là $x+y-1=0$.
	}
\end{ex}


\Closesolutionfile{ans}
%\begin{center}
%	\textbf{ĐÁP ÁN}
%	\inputansbox{10}{ans/ans}	
%\end{center}
\begin{center}
	\textbf{PHẦN 2 - TỰ LUẬN}
\end{center}


\begin{bt}%[1,0 điểm]%[0X2B3-2]%[0X1B3-1]
	\begin{enumerate}[a)]
		\item Tìm hệ số của $x^{2}$ trong khai triển $(2x+5)^{5}$.
		\item Bốn bạn Dũng, Phương, Linh, Đăng cùng thi vào lớp $10$. Kết quả thi được thống kê bởi bảng sau
			\begin{center}
				\begin{tabular}{|l|c|c|c|}
					\hline
					Học sinh & Điểm Toán & Điểm Ngữ Văn & Điểm tiếng Anh\\
					\hline
					Dũng & 9 & 4 & 6\\
					\hline
					Đăng & 8 & 8 & 2\\
					\hline
					Linh & 5 & 8 & 3\\
					\hline
					Phương & 8 & 5 & 6\\
					\hline
				\end{tabular}
			\end{center}
			Tính điểm trung bình kết quả thi $3$ môn Toán, Ngữ Văn, tiếng Anh của mỗi bạn (kết quả làm tròn đến hàng phần trăm) và cho biết bạn nào trúng tuyển. Biết rằng, nếu muốn trúng tuyển thì điểm trung bình các môn thi phải lớn hơn hoặc bằng $5$ và không môn nào dưới $3$ điểm.
	\end{enumerate}
	\loigiai{
		\begin{enumerate}[a)]
			\item Tìm hệ số của $x^{2}$ trong khai triển $(2x+5)^{5}$.\\
				Số hạng tổng quát trong khai triển là $\mathrm{C}_{5}^{k}2^{5-k}5^{k}x^{k}$.\\
				Số hạng chứa $x^2$ có hệ số là $ \mathrm{C}_{5}^{2}2^{3}5^{2}=50000$.
			\item Điểm trung bình kết quả thi $3$ môn Toán, Ngữ Văn, tiếng Anh của
			\begin{itemize}
				\item bạn Dũng là $\dfrac{9+4+6}{3}=6{,}33$.
				\item bạn Đăng là $\dfrac{8+8+2}{3}=6{,}0$.
				\item bạn Linh là $\dfrac{5+8+3}{3}=5{,}33$.
				\item bạn Phương là $\dfrac{8+5+6}{3}=6{,}33$.
			\end{itemize}
			Ba bạn trúng tuyển là Dũng, Linh, Phương.
		\end{enumerate}
	}
\end{bt}

\begin{bt}%[1,0 điểm]%[0H4B1-2]%[0H4B2-6]
	Trong mặt phẳng $Oxy$, cho tam giác $ABC$ với $A(2; 6)$, $B(-3;-4)$ và $C(5; 1)$.
	\begin{enumerate}[a)]
		\item Viết phương trình tổng quát đường cao $AH$ của tam giác $ABC$.
		\item Tìm tọa độ tâm đường tròn ngoại tiếp $\triangle ABC$.
	\end{enumerate}
	\loigiai{
		\begin{enumerate}[a)]
			\item Viết phương trình tổng quát đường cao $AH$ của tam giác $ABC$.\\
				Đường cao $AH$ nhận véc-tơ $\overrightarrow{BC}=(8;5)$ là véc-tơ pháp tuyến.\\
				Phương trình của $AH\colon 8x+5y-46=0$.
			\item Tìm tọa độ tâm đường tròn ngoại tiếp $\triangle ABC$.\\
				Gọi phương trình đường tròn ngoại tiếp $\triangle ABC$ là $x^2+y^2+2ax+2by+c=0$.\\
				Đường tròn đi qua $3$ điểm $A$, $B$, $C$ suy ra
				\[\heva{&-4a-12b+c=-40\\&6a+8b+c=-25\\&-10a-2b+c=-26} \Leftrightarrow \heva{&a=-\dfrac{13}{22}\\&b=\dfrac{23}{22}\\&c=-\dfrac{328}{11}.}\]
				Vậy tọa độ tâm đường tròn ngoại tiếp $\triangle ABC$ là $I\left(-\dfrac{13}{22}; \dfrac{23}{22}\right)$.
		\end{enumerate}
	}
\end{bt}

\begin{bt}%[1,0 điểm]%[0X3B2-3]
	Một tổ của lớp $10A$ có $12$ học sinh gồm $4$ học sinh nữ trong đó có Mai và $8$ học sinh nam trong đó có Minh. Chia tổ thành $3$ nhóm, mỗi nhóm gồm $4$ học sinh và phải có ít nhất $1$ học sinh nữ. Tính xác suất để Mai và Minh cùng một nhóm.
	\loigiai{
		\begin{itemize}
			\item Chia $12$ học sinh thành $3$ nhóm theo yêu cầu.\\
				Nhóm $1$: Gồm $2$ nam và $2$ nữ, có $\mathrm{C}_{8}^{2}\cdot \mathrm{C}_{4}^{2}$ cách chọn.\\
				Nhóm $2$: Gồm $3$ nam và $1$ nữ, có $\mathrm{C}_{6}^{3}\cdot \mathrm{C}_{2}^{1}$ cách chọn.\\
				Các học sinh còn lại xếp thành một nhóm, có $1$ cách chọn.\\
				Số phần tử của không gian mẫu là $\mathrm{C}_{8}^{2}\cdot \mathrm{C}_{4}^{2}\cdot \mathrm{C}_{6}^{3}\cdot \mathrm{C}_{2}^{1}\cdot 1=6720$.
			\item Gọi $A$ là biến cố \lq\lq Chia tổ thành $3$ nhóm, mỗi nhóm gồm $4$ học sinh, phải có ít nhất $1$ học sinh nữ và Mai và Minh cùng một nhóm\rq\rq.
				\begin{itemize}
					\item \textit{Trường hợp 1}. Mai và Minh xếp vào nhóm có $2$ nữ và $2$ nam\\
						Mai và Minh được xếp vào nhóm $1$, có $\mathrm{C}_{7}^{1}\cdot \mathrm{C}_{3}^{1}$ cách chọn.\\
						Xếp người vào nhóm $2$, có $\mathrm{C}_{6}^{3}\cdot \mathrm{C}_{2}^{1}$ cách chọn.\\
						Các học sinh còn lại xếp thành một nhóm, có $1$ cách chọn.\\
						Vậy trường hợp 1, có $\mathrm{C}_{7}^{1}\cdot \mathrm{C}_{3}^{1}\cdot \mathrm{C}_{6}^{3}\cdot \mathrm{C}_{2}^{1}\cdot 1=840$ cách xếp.
					\item \textit{Trường hợp 2}. Mai và Minh xếp vào nhóm có $1$ nữ và $3$ nam\\
						Mai và Minh được xếp vào nhóm $2$, có $\mathrm{C}_{7}^{2}\cdot \mathrm{C}_{3}^{0}$ cách chọn.\\
						Xếp người vào nhóm $1$, có $\mathrm{C}_{5}^{2}\cdot \mathrm{C}_{3}^{2}$ cách chọn.\\
						Các học sinh còn lại xếp thành một nhóm, có $1$ cách chọn.\\
						Vậy trường hợp 2, có $\mathrm{C}_{7}^{2}\cdot \mathrm{C}_{3}^{0}\cdot \mathrm{C}_{5}^{2}\cdot \mathrm{C}_{3}^{2}\cdot 1=630$ cách xếp.\\
				\end{itemize}
			Số phần tử của biến cố $A$ là $n(A)=840+630=1470$.
			\item Xác suất của biến cố $A$ là $\mathrm{P}(A)=\dfrac{1470}{6720}=\dfrac{7}{32}$.
		\end{itemize}
	}
\end{bt}