
\de{ĐỀ THI HỌC KỲ I NĂM HỌC 2022-2023}{SGD Bắc Ninh}
\begin{center}
	\textbf{PHẦN 1 - TRẮC NGHIỆM}
\end{center}
\Opensolutionfile{ans}[ans/ans]
%Câu 1...........................
\begin{ex}%[0T8Y1-2] %[Dự án đề kiểm tra HKII NH22-23- Phạm Văn Long]%[Sở Giáo dục Bắc Ninh]
Một hộp có $2$ quả cầu màu xanh và $3$ quả cầu màu đỏ. Có bao nhiêu cách lấy ra $2$ quả cầu không cùng màu từ hộp đó?
	\choice
	{$5$}
	{\True $6$}
	{$8$}
	{$1$}
	\loigiai{
		Số cách lấy ra $2$ quả cầu không cùng màu từ hộp là $2\cdot 3=6$.
	}
\end{ex}
%Câu 2...........................
\begin{ex}%[0T8Y2-1] %[Dự án đề kiểm tra HKII NH22-23- Phạm Văn Long]%[Sở Giáo dục Bắc Ninh]
	Từ các chữ số $1, 2, 3, 4$ lập được bao nhiêu số nguyên dương có $4$ chữ số đôi một phân biệt?
	\choice
	{$10$}
	{$64$}
	{\True $24$}
	{$256$}
	\loigiai{
	Số các số nguyên dương có $4$ chữ số đôi một phân biệt là $4!=24$ số.	
	}
\end{ex}
%Câu 3...........................
\begin{ex}%[0T8B2-1] %[Dự án đề kiểm tra HKII NH22-23- Phạm Văn Long]%[Sở Giáo dục Bắc Ninh]
	Trên mặt phẳng cho 8 điểm đôi một phân biệt. Có bao nhiêu véc-tơ (khác $\overrightarrow{0}$) có điểm đầu và điểm cuối được chọn từ $8$ điểm đã cho?
	\choice
	{$64$}
	{\True $56$}
	{$28$}
	{$72$}
	\loigiai{
	Số véc-tơ (khác $\overrightarrow{0}$) có điểm đầu và điểm cuối được chọn từ $8$ điểm đã cho $\mathrm{A}_8^2=56$ véc-tơ.	
	}
\end{ex}
%Câu 4...........................
\begin{ex}%[0T8B3-2]%[Dự án đề kiểm tra HKII NH22-23- Phạm Văn Long]%[Sở Giáo dục Bắc Ninh]
	Số hạng chứa $x^4$ trong khai triển của $\left(3x-2\right)^5$ là
	\choice
	{\True $-810 x^4$}
	{$-810$}
	{$405 x^4$}
	{$-162 x^4$}
	\loigiai{
	Ta có \allowdisplaybreaks
	\begin{eqnarray*}
		\left(3x-2\right)^5&=&(3 x)^5+5 \cdot(3 x)^4 \cdot(-2)+10 \cdot(3 x)^3 \cdot(-2)^2+10 \cdot(3 x)^2 \cdot(-2)^3\\
		& & +5 \cdot(3 x) \cdot(-2)^4+(-2)^5\\
		&=&243 x^5-810 x^4+1080 x^3-720 x^2+240 x-32
	\end{eqnarray*}	
	Vậy số hạng chứa $x^4$ trong khai triển	là $-810 x^4$.
	}
\end{ex}
%Câu 5...........................
\begin{ex}%[0T6B1-1]%[Dự án đề kiểm tra HKII NH22-23- Phạm Văn Long]%[Sở Giáo dục Bắc Ninh]
	Trái Đất quay một vòng quanh Mặt Trời mất $365$ ngày. Kết quả này có độ chính xác là $\dfrac{1}{4}$ ngày. Sai số tương đối tối đa là
	\choice
	{\True $0{,}07 \%$}
	{$0{,}08 \%$}
	{$0{,}27 \%$}
	{$0{,}25 \%$}
	\loigiai{
	Sai số tương đối là $\delta \le \dfrac{1}{4}\cdot \dfrac{1}{365}\approx 0{,}000684$ nên sai số tương đối tối đa là	$0{,}07 \%$.
	}
\end{ex}
%Câu 6...........................
\begin{ex}%[0T6B4-1]%[Dự án đề kiểm tra HKII NH22-23- Phạm Văn Long]%[Sở Giáo dục Bắc Ninh]
	Sĩ số $8$ lớp $10$ ở một trường Trung học được cho bởi bảng dưới đây
	\begin{center}
		\begin{tabular}{|c|c|c|c|c|c|c|c|c|}
		\hline Lớp & $10 \mathrm{A} 1$ & $10 \mathrm{A} 2$ & $10 \mathrm{A} 3$ & $10 \mathrm{A} 4$ & $10 \mathrm{A} 5$ & $10 \mathrm{A} 6$ & $10 \mathrm{A} 7$ & $10 \mathrm{A} 8$ \\
		\hline Sĩ số & 36 & 37 & 40 & 38 & 40 & 38 & 41 & 39 \\
		\hline
	\end{tabular}
	\end{center}
	Khoảng biến thiên của mẫu số liệu đã cho bằng
	\choice
	{$4$}
	{$3$}
	{\True $5$}
	{$6$}
	\loigiai{
	Khoảng biến thiên của mẫu số liệu đã cho là $41-36=5$.	
	}
\end{ex}
%Câu 7...........................
\begin{ex}%[0T6B3-4]%[Dự án đề kiểm tra HKII NH22-23- Phạm Văn Long]%[Sở Giáo dục Bắc Ninh]
	Chọn 22 bạn học sinh của một lớp và thống kê số lượt đạt hoa điểm tốt của mỗi bạn trong đợt thi đua chào mừng ngày $26 / 3$, ta được bảng số liệu sau đây
	\begin{center}
		\begin{tabular}{|c|c|c|c|c|c|c|c|c|c|c|}
		\hline 1 & 8 & 5 & 2 & 7 & 8 & 10 & 2 & 2 & 2 & 9 \\
		\hline 1 & 3 & 4 & 5 & 3 & 6 & 6 & 5 & 5 & 5 & 5 \\
		\hline
	\end{tabular}
	\end{center}
	Mốt của mẫu số liệu trên bằng
	\choice
	{\True $5$}
	{$6$}
	{$2$}
	{$8$}
	\loigiai{
		Mốt của mẫu là $5$.
	}
\end{ex}
%Câu 8...........................
\begin{ex}%[0T0B1-1]%[Dự án đề kiểm tra HKII NH22-23- Phạm Văn Long]%[Sở Giáo dục Bắc Ninh]
	Gieo một lần một con xúc xắc cân đối và đồng chất có $6$ mặt. Gọi $M$ là biến cố xuất hiện mặt có số chấm là số lẻ. Biểu diễn biến cố $M$ ở dạng tập hợp con của không gian mẫu, ta được
	\choice
	{$M=\{1;2;3\}$}
	{$M=\{1;3;5;7\}$}
	{$M=\{1;2;3;4;5;6\}$}
	{\True $M=\{1;3;5\}$}
	\loigiai{
	Vì $M$ là biến cố xuất hiện mặt có số chấm là số lẻ nên	$M=\{1;3;5\}$.
	}
\end{ex}
%Câu 9...........................
\begin{ex}%[0T9B2-3]%[Dự án đề kiểm tra HKII NH22-23- Phạm Văn Long]%[Sở Giáo dục Bắc Ninh]
	Trong mặt phẳng tọa độ $O x y$, cho đường thẳng $d_1$ có một véc-tơ chỉ phương là $\vec{u}_1$ và đường thẳng $d_2$ có một véc-tơ chỉ phương là $\vec{u}_2$. Biết rằng hai đường thẳng $d_1, d_2$ vuông góc với nhau. Khẳng định nào sau đây đúng?
	\choice
	{$\vec{u}_1 \cdot \vec{u}_2=1$}
	{\True $\vec{u}_1 \cdot \vec{u}_2=0$}
	{$\vec{u}_1 \cdot \vec{u}_2=-1$}
	{$\vec{u}_1=-\vec{u}_2$}
	\loigiai{
		Có $d_1, d_2$ vuông góc với nhau nên $\vec{u}_1 \cdot \vec{u}_2=0$.
	}
\end{ex}
%Câu 10...........................
\begin{ex}%[0T9B4-7]%[Dự án đề kiểm tra HKII NH22-23- Phạm Văn Long]%[Sở Giáo dục Bắc Ninh]
	Trong mặt phẳng tọa độ $Oxy$, cho parabol $(P)\colon y^2=3 x$. Tiêu điểm của parabol $P$ là điểm nào sau đây?
	\choice
	{\True $F_1\left(\dfrac{3}{4} ; 0\right)$}
	{$F_2\left(-\dfrac{3}{4} ; 0\right)$}
	{$F_3\left(\dfrac{3}{2} ; 0\right)$}
	{$F_4\left(0 ; \dfrac{3}{4}\right)$}
	\loigiai{
	Có $(P)\colon y^2=3 x$ suy ra $p=\dfrac{3}{2}$ nên tiêu điểm $F\left(\dfrac{3}{4} ; 0\right)$.	
	}
\end{ex}
%Câu 11...........................
\begin{ex}%[0T9B4-3]%[Dự án đề kiểm tra HKII NH22-23- Phạm Văn Long]%[Sở Giáo dục Bắc Ninh]
	Trong mặt phẳng tọa độ $O x y$, cho elip $(E)\colon \dfrac{x^2}{36}-\dfrac{y^2}{25}=1$. Điểm nào sau đây thuộc elip $E$?
	\choice
	{$M(36;0)$}
	{$N(\sqrt{6};0)$}
	{\True $P(0;-5)$}
	{$Q(0;-625)$}
	\loigiai{
	Khi $x=0$ thì $y=\pm 5$ nên $P(0;-5)$ thuộc elip $(E)$.
	}
\end{ex}
%Câu 12...........................
\begin{ex}%[0T9K4-6]%[Dự án đề kiểm tra HKII NH22-23- Phạm Văn Long]%[Sở Giáo dục Bắc Ninh]
	Trong mặt phẳng tọa độ $O x y$, cho hypebol $(H)\colon \dfrac{x^2}{4}-\dfrac{y^2}{9}=1$. Hai điểm $M, N$ lần lượt chạy trên hai nhánh khác nhau của hypebol $(H)$. Độ dài đoạn thẳng $M N$ có giá trị nhỏ nhất bằng
	\choice
	{$2$}
	{\True $4$}
	{$9$}
	{$3$}
	\loigiai{
	Ta có $a=2$, $b=3$ suy ra đỉnh $A_1(-2;0)$, $A_2(2;0)$.\\
	Hai điểm $M, N$ lần lượt chạy trên hai nhánh khác nhau của hypebol $(H)$. Độ dài đoạn thẳng $MN$ có giá trị nhỏ nhất bằng độ dài của hai đỉnh $A_1A_2$ suy ra $MN=F_1F_2=A_1A_2=4$.	
	}
\end{ex}


\Closesolutionfile{ans}
%\begin{center}
%	\textbf{ĐÁP ÁN}
%	\inputansbox{10}{ans/ans}	
%\end{center}


\begin{center}
	\textbf{PHẦN 2 - TỰ LUẬN}
\end{center}


\begin{bt}%[0X3B2-6]%[Dự án đề kiểm tra HKII NH22-23- Lương Như Quỳnh]%[Sở Bắc Ninh]
Các bạn Bắc và Ninh mỗi bạn chọn ngẫu nhiên một số nguyên dương có một chữ số. Tính xác suất của các biến cố sau đây
	\begin{enumerate}
		\item Bắc và Ninh chọn được hai số giống nhau;
		\item Bắc và Ninh chọn được hai số có tích là một số chẵn.
	\end{enumerate}
\loigiai{
\begin{enumerate}
	\item Số phần tử của không gian mẫu $n (\Omega)=9 \cdot 9=81$.\\
Gọi $A$ là biến cố: \lq\lq Hai bạn Bắc, Ninh cùng chọn được các số giống nhau\rq\rq.\\
Ta có $n(A)=9 \cdot 1=9$.\\
Vậy $\mathrm{P}(A)=\dfrac{n(A)}{n(\Omega)}=\dfrac{9}{81}=\dfrac{1}{9}$.
	\item Gọi $B$ là biến cố cần tính xác suất thì $\overline{B}$ là biến cố: \lq\lq Hai bạn Bắc, Ninh chọn được các số có tích là một số lẻ\rq\rq.\\ Ta có $n\left(\overline{B}\right)=5 \cdot 5=25$.\\
Do đó $\mathrm{P}\left(\overline{B}\right)=\dfrac{n\left(\overline{B}\right)}{n(\Omega)}=\dfrac{25}{81}$.\\
Vậy $\mathrm{P}(B)=1-\mathrm{P}\left(\overline{B}\right)=1-\dfrac{25}{81}=\dfrac{56}{81}$.
\end{enumerate}
}
\end{bt}

\begin{bt}%[0X1B3-1] %[Dự án đề kiểm tra HKII NH22-23- Lương Như Quỳnh]%[Sở Bắc Ninh]
Một người thực hành đo chiều dài một cái bàn $5$ lần. Biết rằng trung bình cộng của cả $5$ lần đo bằng $119{,}6$ cm, trung bình cộng của $4$ lần đo trước bằng $119{,}5$ cm. Tính kết quả của lần đo thứ $5$.
\loigiai{
Gọi kết quả của $5$ lần đo lần lượt là $x_1$, $x_2$,$x_3$, $x_4$, $x_5$ (là các số thực dương, đơn vị: cm).\\
Vì trung bình cộng của cả $5$ lần đo bằng $119{,}6 $ cm nên
\[\dfrac{1}{5} \left(x_1+x_2+x_3+x_4+x_5\right)=119{,}6 .\]
Vì trung bình cộng của $4$ lần đo trước bằng $119{,}5$ cm nên
\[\dfrac{1}{4} \left(x_1+x_2+x_3+x_4\right)=119{,}5.\]
Kêt quả của lần đo thứ $5$ là
\[x_5=x_1+x_2+x_3+x_4+x_5-x_1+x_2+x_3+x_4=119{,}6 \cdot 5-119,5 \cdot 4=120 \text{ cm.}\]
}
\end{bt}

\begin{bt}%[0H4B2-2]%[0H4B2-3]%[Dự án đề kiểm tra HKII NH22-23- Lương Như Quỳnh]%[Sở Bắc Ninh]
Trong mặt phẳng tọa độ $Oxy$, cho ba điểm $A(4;0)$, $B(0;3)$, $C(2;4)$.
	\begin{enumerate}
		\item Viết phương trình tổng quát của đường thẳng $AB$.
		\item Viết phương trình đường tròn $(T)$ tâm $C$ và tiếp xúc với đường thẳng $AB$.
		\item Một điểm có cả hoành độ và tung độ là số nguyên được gọi là điểm nguyên. Biết rằng trên đường tròn $(T)$ có bốn điểm nguyên. Bốn điểm này là bốn đỉnh của một tứ giác lồi. Tính diện tích của tứ giác lồi đó.
	\end{enumerate}
\loigiai{
\begin{enumerate}
	\item Ta có $\overrightarrow{AB}=(-4;3)$ là một véc-tơ chỉ phương của đường thẳng $AB$.\\
Suy ra $\overrightarrow{n}=(3;4)$ là một véc-tơ pháp tuyến của đường thẳng $AB$.\\
Đường thẳng $AB$ đi qua $A(4;0)$, có một véc-tơ pháp tuyến $\overrightarrow{n}=(3;4)$ nên có phương trình là \[3(x-4)+4(y-0)=0 \Leftrightarrow 3 x+4 y-12=0.\]
{\bf Cách khác.} Ta có phương trình đường thẳng $AB$ theo đoạn chắn là $\dfrac{x}{4}+\dfrac{y}{3}=1\Leftrightarrow 3x+4y-12=0$.
	\item Đường tròn $(T)$ tâm $C(2;4)$ và tiếp xúc với đường thẳng $AB$ nên có bán kính là
	\[R=\mathrm{d} (C, AB)=\dfrac{|3 \cdot 2+4 \cdot 4-12|}{\sqrt{3^2+4^2}}=2.\]
Vậy $(T)$ có phương trình là $(x-2)^2+(y-4)^2=4$.
	\item Số $4$ được phân tích thành tổng của hai số chính phương là $4=0+4=4+0$.\\
Với $(x-2)^2+(y-4)^2=4$ và $x$, $y \in \mathbb{Z}$ ta có
\[\heva{&(x-2)^2=0\\&(y-4)^2=4}
\Leftrightarrow \heva{&x=2\\&y=6}\text{ hoặc } \heva{&x=2\\&y=2.}\]
\[
\heva{&(x-2)^2=4\\&(y-4)^2=0}
\Leftrightarrow \heva{&x=0\\&y=4}\text{ hoặc } \heva{&x=4\\&y=4.}
\]
Trên $T$ có bốn điểm nguyên là $M(0;4)$, $N(2;6)$, $P(4;4)$, $Q(2;2)$.\\
Dễ thấy $MNPQ$ là hình vuông có đường chéo $MP=NQ=4$. \\
Diện tích hình vuông $MNPQ$ là $S=\dfrac{1}{2}\cdot 4^2=8$ (đơn vị diện tích).
\end{enumerate}
}
\end{bt}

\begin{bt}%[0X2K2-8]%[Dự án đề kiểm tra HKII NH22-23- Lương Như Quỳnh]%[Sở Bắc Ninh]
Tìm số nguyên dương $n$ thỏa mãn
\[\left(\mathrm{C}_{2 n}^1\right)^2-\left(\mathrm{C}_{2 n-1}^1\right)^2+\ldots+\left(\mathrm{C}_4^1\right)^2-\left(\mathrm{C}_3^1\right)^2+\left(\mathrm{C}_2^1\right)^2-\left(\mathrm{C}_1^1\right)^2=78.\]
\loigiai{
Với số nguyên dương $n$ ta có 
\allowdisplaybreaks
\begin{eqnarray*}
 && \left(\mathrm{C}_{2 n}^1\right)^2-\left(\mathrm{C}_{2 n-1}^1\right)^2+\ldots+\left(\mathrm{C}_4^1\right)^2-\left(\mathrm{C}_3^1\right)^2+\left(\mathrm{C}_2^1\right)^2-\left(\mathrm{C}_1^1\right)^2=78 \\ 
  &\Leftrightarrow& (2n)^2-(2n-1)^2+\ldots+4^2-3^2+2^2-1^2=78\\ 
  &\Leftrightarrow& [2n-(2n-1)] \cdot (2n+2n-1)+\ldots+(4-3)\cdot (4+3)+(2-1)\cdot (2+1)=78 \\ 
  &\Leftrightarrow& 2 n+2 n-1+\ldots+4+3+2+1=78 \\ 
  &\Leftrightarrow& 1+2+3+4+\ldots+2 n-1+2 n=78\quad \quad(1).
\end{eqnarray*}
Vì $1+2+3+\ldots+n=\dfrac{n(n+1)}{2}$, $\forall n \in \mathbb{N}^*$, nên
\[1+2+3+4+\ldots+2 n-1+2 n=n(2n+1),\,\forall n \in \mathbb{N}^*.\]
Suy ra $(1)\Leftrightarrow n(2n+1)=78\Leftrightarrow 2 n^2+n-78=0 \Leftrightarrow \hoac{&n=6&\text{(nhận, vì }n \in \mathbb{N}^*\text{)}&\\&n=-\dfrac{13}{2}&\text{(loại).}}$\\
Vậy $n=6$ là giá trị cần tìm.
}
\end{bt} 
