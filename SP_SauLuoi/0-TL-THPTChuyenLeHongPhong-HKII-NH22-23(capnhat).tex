

\de{ĐỀ THI ÔN TẬP  HỌC KỲ II NĂM HỌC 2022-2023}{THPT Chuyên Lê Hồng Phong}


%%%Câu 1
\begin{bt}%[0D7B2-2]%[Dự án đề kiểm tra HKII đợt 1- DonLee]%[THPTChuyenLeHongPhong]
	Giải bất phương trình $(x^2-3x+2)(x^2-1)\le 0$.
	\loigiai{
		Ta có
		\allowdisplaybreaks
		$\begin{aligned}[t]
			&\quad (x^2-3x+2)(x^2-1)\le 0 \Leftrightarrow (x-1)^{2}(x+1)(x-2)\le 0\\
			&\Leftrightarrow \hoac{&x-1=0\\&(x+1)(x-2)\le 0} \Leftrightarrow \hoac{&x=1\\&-1\le x\le 2.}
		\end{aligned}$\\
		Vậy bất phương trình có tập nghiệm $S=[-1;2]$.
	}
\end{bt}

%%%Câu 2
\begin{bt}%[0D7B3-2]%[Dự án đề kiểm tra HKII đợt 1- DonLee]%[THPTChuyenLeHongPhong]
	Giải phương trình $\sqrt{2x^{2}-5x+2}=x-2$.
	\loigiai{
		Ta có 
		\allowdisplaybreaks
		$\begin{aligned}[t]
			&\quad \sqrt{2x^{2}-5x+2}=x-2 &\Leftrightarrow \heva{&x-2\ge 0\\&2x^{2}-5x+2=(x-2)^{2}}\\
			&\Leftrightarrow \heva{&x\ge 2\\&x^{2}-x-2=0} &\Leftrightarrow \heva{&x\ge 2\\&x=-1;\, x=2} \Leftrightarrow x=2.
		\end{aligned}$\\
		Vậy phương trình có nghiệm $x=2$.
	}
\end{bt}

%%%Câu 3
\begin{bt}%[0D7B1-1]%[Dự án đề kiểm tra HKII đợt 1- DonLee]%[THPTChuyenLeHongPhong]
	Tìm tất cả các giá trị của tham số $m$ để bất phương trình $mx^{2}-2(m+1)x+m+3>0$ thỏa mãn với mọi $x\in\mathbb{R}$.
	\loigiai{
		Xét $mx^{2}-2(m+1)x+m+3>0$. \quad $(1)$
		\begin{itemize}
			\item Với $m=0$ thì $(1)$ trở thành $-2x+3>0 \Leftrightarrow x<\dfrac{3}{2}$. Suy ra $m=0$ không thỏa mãn.
			\item Với $m\ne 0$ thì $(1)$ luôn đúng với mọi $x\in\mathbb{R}$ khi và chỉ khi
			\[\heva{&m>0\\&\Delta'=(m+1)^{2}-m(m+3)<0} \Leftrightarrow \heva{&m>0\\&-m+1<0} \Leftrightarrow \heva{&m>0\\&m>1} \Leftrightarrow m>1.\]
		\end{itemize}
		Vậy $m>1$ là giá trị cần tìm.
	}
\end{bt}


%%%Câu 4
\begin{bt}%[0T8B2-2]%[Dự án đề kiểm tra HKII đợt 1- TinDatTran]%[THPTChuyenLeHongPhong]
Có bao nhiêu số tự nhiên lẻ có $4$ chữ số đôi một khác nhau và chia hết cho $5$?
\loigiai{
Gọi số cần tìm có dạng $\overline{abcd}$ với $a\neq 0$.\\
Vì $\overline{abcd}$ lẻ và $\overline{abcd}\,\vdots\, 5$ nên $d=5$.\\
Ta có
\begin{itemize}
\item Chọn $d$ có 1 cách chọn ($d=5$).
\item Chọn $a \neq 0$ và $a \neq d$ có $8$ cách chọn.
\item Chọn $b, c \neq a$ và $b, c \neq d$ có $\mathrm{A}_8^2$ cách chọn.
\end{itemize}
Theo quy tắc nhân, có $1\cdot 8\cdot \mathrm{A}_8^2=448$ số thỏa yêu cầu bài toán.
}
\end{bt}
%%%Câu 5
\begin{bt}%[0T9B3-5]%[Dự án đề kiểm tra HKII đợt 1- TinDatTran]%[THPTChuyenLeHongPhong]
Trong mặt phẳng $Oxy$, cho điểm $A(2;3)$ và đường thẳng $(d)\colon x-2 y+1=0$.
\begin{enumerate}
\item Viết phương trình đường thẳng $(\Delta)$ qua $A$ và vuông góc với $d$.
\item Viết phương trình đường tròn $(C)$ tâm $A$ tiếp xúc với $d$.
\end{enumerate}
\loigiai{
\begin{enumerate}
\item Đường thẳng $(d)\colon x-2 y+1=0$ có VTPT $\vec{n}=(1 ;-2)$ suy ra $d$ có VTCP $\vec{u}=(2 ; 1)$.\\
Đường thẳng $(\Delta)$ vuông góc với $d$ nên $(\Delta)$ có VTPT $\overrightarrow{n}_{\Delta}=(2;1)$ và đường thẳng $(\Delta)$ qua $A$ nên phương trình của $(\Delta)$ là $2\cdot (x-2)+1\cdot(y-3)=0 \Leftrightarrow 2 x+y-7=0$.\\
Vậy $(\Delta)\colon2 x+y-7=0$
\item Đường tròn $(C)$ tâm $A$ tiếp xúc với $d$ suy ra bán kính đường tròn là
\[
R=\mathrm{d}(A, d)=\dfrac{\left|2-6+1\right|}{\sqrt{5}}=\dfrac{3 \sqrt{5}}{5}.
\]
Phương trình đường tròn $(C)$ là $(x-2)^2+(y-3)^2=\dfrac{9}{5}$.
\end{enumerate}
}
\end{bt}

%%%Câu 6
\begin{bt}%[0T9B4-2]%[Dự án đề kiểm tra HKII đợt 1- TinDatTran]%[THPTChuyenLeHongPhong]
Trong mặt phẳng $Oxy$, viết phương trình chính tắc của elip $(E)$ có tiêu điểm $F_1(-4 ; 0)$, $F_2(4 ; 0)$ và đi qua điểm $M\left(\dfrac{5}{2} ; \dfrac{-3 \sqrt{3}}{2}\right)$.
\loigiai{
Gọi phương trình chính tắc của $(E)$ có dạng $\dfrac{x^2}{a^2}+\dfrac{y^2}{b^2}=1$ ($a>b>0$). $(E)$ có tiêu điểm $F_1(-4 ; 0)$, $F_2(4 ; 0) \Rightarrow c=4$.\\
Lại có $a^2=b^2+c^2 \Leftrightarrow a^2=b^2+16$.\\
$(E)$ đi qua điểm $M\left(\dfrac{5}{2} ; \dfrac{-3 \sqrt{3}}{2}\right)$ ta có $\dfrac{25}{4 a^2}+\dfrac{27}{4 b^2}=1 \Leftrightarrow 27 a^2+25 b^2=4 a^2 b^2$.\\
Ta có 
\begin{eqnarray*}
	&&\heva{&a^2=b^2+16\\&27 a^2+25 b^2=4 a^2 b^2}\\
	&\Leftrightarrow&\heva{&a^2=b^2+16\\&27\left(b^2+16\right)+25 b^2=4\left(b^2+16\right) b^2}\\
	&\Leftrightarrow&\heva{&a^2=b^2+16\\&4 b^4+12 b^2-432=0}\\
	&\Leftrightarrow&\heva{&\hoac{&b^2=-12\text { (loại)} \\ &b^2=9\text{ (thỏa mãn)}}\\&a^2=b^2+16}\\
	&\Leftrightarrow&\heva{&a^2=25\\&b^2=9.}
\end{eqnarray*}
Vậy phương trình chính tắc của $(E)$ là $\dfrac{x^2}{25}+\dfrac{y^2}{9}=1$.
}
\end{bt}


%%%Câu 7
\begin{bt}%[0H7GK-1]%[Dự án đề kiểm tra HKII đợt 1- Vũ Ngọc Hảo]%[THPTChuyenLeHongPhong]
\immini{	Tính bán kính đường tròn trong hình vẽ trên biết hình vuông có cạnh 
bằng $2$.}{
	\begin{tikzpicture}
\def\r{2.304886114323222}
\path
(0,0) coordinate (A)
(0,1) coordinate (B)
(1,0) coordinate (C)
(1,2) coordinate (D)
(2,2) coordinate (E)
(2,1) coordinate (K)
(2,3) coordinate (G)
(3,3) coordinate (I)
(3,2) coordinate (F)
(4,2) coordinate (M)
(4,1) coordinate (N)
(3,1) coordinate (L)
(3,2) coordinate (F)
(1.75,1.5) coordinate (O)
($(M)!0.5!(N)$)coordinate (H);
\draw[black] (O) circle (\r);
\draw 
(A)--(B)--(K)--(G)--(I)--(L)--(N)--(L)  (A)--(C)--(D)--(M)--(N);
\draw [fill=black] (A) circle (.05);
\draw [fill=black] (B) circle (.05);
\draw [fill=black] (C) circle (.05);
\draw [fill=black] (D) circle (.05);
\draw [fill=black] (E) circle (.05);
\draw [fill=black] (K) circle (.05);
\draw [fill=black] (G) circle (.05);
\draw [fill=black] (I) circle (.05);
\draw [fill=black] (F) circle (.05);
\draw [fill=black] (M) circle (.05);
\draw [fill=black] (N) circle (.05);
\draw [fill=black] (L) circle (.05);
\draw [fill=black] (F) circle (.05);
\draw [fill=black] (1,1) circle (.05);
\end{tikzpicture}
}
\loigiai{
\immini{Xét tam giác $ABC$ nội tiếp đường tròn có:\\
$BC=2$ (cạnh hình vuông)\\
$AC=\sqrt{(a\cdot 2)^2+(2\cdot 2)^2}=4\sqrt{5}$ (pytago)\\
$AB=\sqrt{(a\cdot 2)^2+(1\cdot 2)^2}=2\sqrt{17}$ (pytago)\\
$S_{ABC}=\dfrac{1}{2}\cdot BC\cdot DC=\dfrac{1}{2}\cdot 2\cdot (4\cdot 2)=8$ do $DC=d(A;BC)$\\
Mà $S_{ABC}=\dfrac{AB\cdot AC\cdot BC}{4R}$\\
Vậy $R_{(ABC)}=\dfrac{AB\cdot AC\cdot BC}{4S_{ABC}}=\dfrac{2\sqrt{17}\cdot 4\sqrt{5}\cdot 2}{4\cdot 8}=\dfrac{\sqrt{85}}{2}$}
		{\begin{tikzpicture}
\def\r{2.304886114323222}
\path
(0,0) coordinate (A)
(0,1) coordinate (N)
(1,0) coordinate (M)
(1,2) coordinate (Z)
(2,2) coordinate (E)
(2,1) coordinate (K)
(2,3) coordinate (G)
(3,3) coordinate (I)
(3,2) coordinate (F)
(4,2) coordinate (C)
(4,1) coordinate (B)
(3,1) coordinate (L)
(3,2) coordinate (F)
(1.75,1.5) coordinate (O)
(0,2) coordinate (D)
($(M)!0.5!(N)$)coordinate (H);
\draw[black] (O) circle (\r);
\draw 
(A)--(N)--(K)--(G)--(I)--(L)--(B)--(L)  (A)--(M)--(Z)--(C)--(B);
\draw[red] (A)--(B)--(C)--(A);
\draw[dashed] (0,1)--(0,2)--(1,2) (2,1)--(3,1);
\draw [fill=black] (Z) circle (.05);
\draw [fill=black] (E) circle (.05);
\draw [fill=black] (K) circle (.05);
\draw [fill=black] (G) circle (.05);
\draw [fill=black] (I) circle (.05);
\draw [fill=black] (F) circle (.05);
\draw [fill=black] (M) circle (.05);
\draw [fill=black] (N) circle (.05);
\draw [fill=black] (L) circle (.05);
\draw [fill=black] (F) circle (.05);
\draw [fill=black] (1,1) circle (.05);
\draw [fill=black] (0,2) circle (.05);
	\foreach \x/\g in {A/180,B/0,C/0,D/90}
\draw [fill= black] (\x) circle (.05)+(\g:.3) node{$\x$};
	\end{tikzpicture}}
}
\end{bt}

%%%Câu 8
\begin{bt}%[0D9KQ-5]%[Dự án đề kiểm tra HKII đợt 1- Vũ Ngọc Hảo]%[THPTChuyenLeHongPhong]
	Một hộp chứa $5$ bi xanh và $6$ bi đỏ. Lấy ngẫu nhiên $4$ viên bi từ hộp. 
	Tính xác suất để $4$ viên bi lấy được có đủ $2$ màu.
\loigiai{Ta có $n(\Omega)= \mathrm{C}^4_{11}=330$.\\
Gọi $A$ là biến cố \lq\lq$4$ viên bi lấy được có đủ $2$ màu\rq\rq.
\begin{itemize}
	\item TH1: Lấy $1$ bi xanh và $3$ bi đỏ có $\mathrm{C}^1_5 \cdot \mathrm{C}^3_6=100.$ 
	\item TH2: Lấy $2$ bi xanh và $2$ bi đỏ có $\mathrm{C}^2_5 \cdot \mathrm{C}^2_6=150.$ 
	\item TH3: Lấy $3$ bi xanh và $1$ bi đỏ có $\mathrm{C}^3_5 \cdot \mathrm{C}^1_6=60.$  
\end{itemize} 
	Theo quy tắc cộng suy ra $n(A)=100+150+60=310$.\\
Vậy xác suất để $4$ viên bi lấy được có đủ $2$ màu là 
$\mathrm{P}(A)=\dfrac{n(A)}{n(\Omega)}$.	}
\end{bt}

%%%Câu 9
\begin{bt}%[0D9GQ-4]%[Dự án đề kiểm tra HKII đợt 1- Vũ Ngọc Hảo]%[THPTChuyenLeHongPhong]
	Xếp ngẫu nhiên $10$ học sinh và $2$ giáo viên thành một hàng. Tính xác suất 
	để giữa $2$ giáo viên có đúng $5$ em học sinh.
	\loigiai{Ta có $n(\Omega)= 12!$.
	\begin{itemize}
		\item Gọi $A$ là biến cố \lq\lq Giữa $2$ giáo viên có đúng $5$ em học sinh \rq\rq.
		\item Xếp $2$ giáo viên có $2!$ cách xếp. 
		\item Chọn và sắp xếp $5$ trong $10$ học sinh vào giữa $2$ giáo viên có $A^5_{10}$.
		\item 5 học sinh còn lại có $5!$ cách xếp.
		\item Theo quy tắc nhân $n(A)=2!\cdot \mathrm{A}^5_{10}\cdot 5!.$
		\item Vậy xác suất để $4$ viên bi lấy được có đủ $2$ màu là 
		$\mathrm{P}(A)=\dfrac{n(A)}{n(\Omega)}=\dfrac{1}{66}$.	
	\end{itemize}
	 }
\end{bt}
