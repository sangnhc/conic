
\de{ĐỀ THI GIỮA HỌC KỲ I NĂM HỌC 2023-2024}{THPT Phạm Phú Thứ}
\Opensolutionfile{ans}[ans/ans]
\Closesolutionfile{ans}

\setcounter{bt}{0}
%Câu 1...........................
\begin{bt}%[1D1K3-3]%[Dự án đề kiểm tra Toán 11 GHKI NH23-24- AHL]%[THPT Pham Phu Thu - Tp HCM]
	Cho $\sin \alpha=-\dfrac{3}{5}$ và $\pi<\alpha<\dfrac{3 \pi}{2}$. Tính
	\begin{multicols}{3}
		\begin{enumerate}
			\item $\cos \alpha$;
			\item $\cos 2 \alpha$;
			\item $\sin \left(\alpha-\dfrac{\pi}{3}\right)$.
		\end{enumerate}
	\end{multicols}
	\loigiai{
		\begin{enumerate}
			\item Do $\pi<\alpha<\dfrac{3\pi}{2} \Rightarrow \cos \alpha<0$.\\
			Ta có $\sin ^2 \alpha+\cos ^2 \alpha=1 \Rightarrow \cos \alpha= - \sqrt{1-\sin ^2 \alpha}=-\sqrt{1-\left(-\dfrac{3}{5}\right)^2}=-\dfrac{4}{5}$.
			\item  Ta có $\cos 2 \alpha=2 \cos ^2 \alpha-1=2\left(-\dfrac{4}{5}\right)^2-1=\dfrac{7}{25}$.
			\item Ta có
			\begin{eqnarray*}
				\sin \left(\alpha-\dfrac{\pi}{3}\right) & = &\sin \alpha \cos \dfrac{\pi}{3}- \cos \alpha \sin \dfrac{\pi}{3} \\
				& = & -\dfrac{3}{5} \cdot \dfrac{1}{2} + \dfrac{4}{5} \cdot \dfrac{\sqrt{3}}{2} \\
				& = & \dfrac{-3+4\sqrt{3}}{10}.
			\end{eqnarray*}
		\end{enumerate}
	}
\end{bt}

%Câu 2...........................
\begin{bt}%[1D1K3-4]%[Dự án đề kiểm tra Toán 11 GHKI NH23-24- AHL]%[THPT Pham Phu Thu - Tp HCM]
	Với $x$ là giá trị sao cho hai vế đều có nghĩa, chứng minh đẳng thức sau
	$$
	\dfrac{1-(\sin 2 x-\cos 2 x)^2}{\sin 10 x-\sin 2 x}=\dfrac{1}{2 \cos 6 x}.
	$$
	\loigiai{
		Ta có
		\begin{eqnarray*}
			VT &=&\dfrac{1-\left( \sin^2{2x}+\cos^2{2x}-2\sin2x \cos2x \right)}{2\cos6x \sin4x}\\
			&=&\dfrac{1-\left( 1-\sin4x \right)}{2 \cos6x \sin4x}\\
			&=&\dfrac{1}{2 \cos6x}\\
			&=&VP.
		\end{eqnarray*}	
	}
\end{bt}

%Câu 3...........................
\begin{bt}%[1D1B4-1]%[Dự án đề kiểm tra Toán 11 GHKI NH23-24- AHL]%[THPT Pham Phu Thu - Tp HCM]
	Tìm tập xác định của các hàm số sau
	\begin{multicols}{2}
		\begin{enumerate}
			\item $y=\sin x+\tan \left(x+\dfrac{\pi}{3}\right)$;
			\item $y=\dfrac{\cos x}{1-\sin 4 x}$.
		\end{enumerate}
	\end{multicols}
	\loigiai{
		\begin{enumerate}
			\item Hàm số xác định khi
			$$x+\dfrac{\pi}{3} \ne \dfrac{\pi}{2} + k\pi  \Leftrightarrow x \ne \dfrac{\pi}{6} +k\pi \quad \left( k \in \mathbb{Z} \right). $$
			Vậy tập xác định của hàm số là $\mathscr{D}= \mathbb{R} \setminus \left\{ \dfrac{\pi}{6} +k\pi \mid k \in \mathbb{Z}  \right\}$.
			\item  Hàm số xác định khi
			$$ 1- \sin4x \ne 0 \Leftrightarrow 4x \ne \dfrac{\pi}{2} +k2\pi \Leftrightarrow x \ne \dfrac{\pi}{8} + \dfrac{k\pi}{2} \quad  \left( k \in \mathbb{Z} \right). $$
			Vậy tập xác định của hàm số là $\mathscr{D}= \mathbb{R} \setminus \left\{ \dfrac{\pi}{8} +\dfrac{k\pi}{2} \mid k \in \mathbb{Z}   \right\}$.
		\end{enumerate}
	}
\end{bt}
%Câu 4...........................
\begin{bt}%%[1D1B4-5]%[Dự án đề kiểm tra Toán 11 GHKI NH23-24- An Do]%[THPT - Tp HCM]
	Tìm tập giá trị của hàm số $y = 2\cos^2 x + 5$ trên $\mathbb{R}$.
	\loigiai{
		Ta có $0\leq \cos^2 x \leq 1 \Leftrightarrow 0\leq 2\cos^2x \leq 2 \Leftrightarrow 5\leq 2\cos^2x + 5 \leq 6\Leftrightarrow 5\leq y \leq 7$.
		\\Vậy tập giá trị của hàm số là $\left[5; 7\right]$.
	}
\end{bt}
%Câu 5
\begin{bt}%%[1D1B5-3]%[Dự án đề kiểm tra Toán 11 GHKI NH23-24- An Do]%[THPT - Tp HCM]
	Giải các phương trình sau:
	\begin{listEX}[2]
		\item $\cot 3x - \dfrac{\sqrt{3}}{3} = 0$;
		\item $\sin \left(2x - \dfrac{\pi}{5}\right)  -\cos x =0$.
	\end{listEX}
	\loigiai{
		\begin{listEX}[2]
			\item \begin{eqnarray*}
				&&\,\cot 3x - \dfrac{\sqrt{3}}{3} = 0\\
				&\Leftrightarrow&\cot 3x = \dfrac{\sqrt{3}}{3}\\
				&\Leftrightarrow&\cot 3x = \cot \dfrac{\pi}{3}\\
				&\Leftrightarrow&3x = \dfrac{\pi}{3} + k\pi \qquad (k \in \mathbb{Z})\\
				&\Leftrightarrow&x = \dfrac{\pi}{9} + \dfrac{k\pi}{3}.\qquad (k \in \mathbb{Z})
			\end{eqnarray*}
			\item \begin{eqnarray*}
				&&\sin \left(2x - \dfrac{\pi}{5}\right)  -\cos x =0\\
				&\Leftrightarrow&\sin \left(2x - \dfrac{\pi}{5}\right) = \cos x\\
				&\Leftrightarrow& \sin \left(2x - \dfrac{\pi}{5}\right) = \sin \left(\dfrac{\pi }{2} - x\right)\\
				&\Leftrightarrow&\hoac{&2x - \dfrac{\pi}{5} = \dfrac{\pi}{2}-x + k2\pi\\&2x - \dfrac{\pi}{5} = \pi - \dfrac{\pi}{2}+x + k2\pi} \quad(k\in \mathbb{Z})\\
				&\Leftrightarrow&\hoac{&x = \dfrac{7\pi}{30} + \dfrac{k2\pi}{3}\\&x = \dfrac{7\pi}{10} + k2\pi.}\quad(k\in \mathbb{Z})
			\end{eqnarray*}
		\end{listEX}
	}
\end{bt}
%Câu 6%%==========================%%%
\begin{bt}%[1D1K5-6][Dự án đề kiểm tra Toán 11 GHKI NH23-24- An Do]%[THPT - Tp HCM]
	Huyết áp là áp lực cần thiết tác động lên thành của động mạch để đưa máu từ tim đên nuôi dưỡng các mô trong cơ thể. Huyết áp được tạo ra do lực co bóp của cơ tim và sức cản của thành động mạch. Mỗi lần tim đập, huyết áp của chúng ta tăng rồi giảm giữa các nhịp. Huyết áp tối đa và huyết áp tối thiểu được gọi tương ứng là huyết áp tâm thu và tâm trương. Chi số huyết áp của chúng ta được viết là huyết áp tâm thu/huyêt áp tâm trương. Chi số huyết áp $120/80$ là bình thường. Giả sử huyết áp của một người nào đó được mô hình hóa bởi hàm số
	$$
	P(t)=100+20 \sin \left(\dfrac{5 \pi}{2} t\right),
	$$
	Trong đó $P\ (t)$ là huyết áp tính theo đơn vị mmHg (milimét thủy ngân) và thời gian $t$ tính theo giây.
	\begin{enumerate}
		\item  Hãy xác định các thời điểm huyết áp của người đó là $80 \mathrm{mmHg}$
		\item  Trong khoảng từ 0 đến 2 giây, hãy xác định số lần huyết áp của người đó là $80 \mathrm{mmHg}$
	\end{enumerate}
	\loigiai{
		\begin{enumerate}
			\item Giải phương trình \begin{eqnarray*}
				&&100+20 \sin \left(\dfrac{5 \pi}{2} t\right)  =  80\\
				&\Leftrightarrow& \sin \left(\dfrac{5 \pi}{2} t\right) =  -1\\
				&\Leftrightarrow& \dfrac{5\pi}{2}t = \dfrac{-\pi}{2} + k2\pi \qquad (k \in \mathbb{Z})\\
				&\Leftrightarrow& t =-\dfrac{1}{5} + \dfrac{4k}{5}. \qquad (k \in \mathbb{Z})\\
			\end{eqnarray*}\\
			Vì $t \geq 0$ nên $t  = -\dfrac{1}{5} + \dfrac{4k}{5}$ với $k \in \mathbb{N^*}$.
			\\Vậy các thời điểm huyết áp người đó là $80$mmHg được biểu diễn bởi công thức $$t  = -\dfrac{1}{5} + \dfrac{4k}{5} \text{~với~} k \in \mathbb{N^*}.$$
			\item Theo yêu cầu đề bài ta cần tìm $k$ thỏa \begin{eqnarray*}
				&&0\leq -\dfrac{1}{5} + \dfrac{4k}{5} \leq 2 \qquad\text{với } k\in \mathbb{N^*}\\
				&\Leftrightarrow& \dfrac{1}{5} \leq \dfrac{4k}{5} \leq \dfrac{11}{5} \qquad\text{với } k\in \mathbb{N^*}\\
				&\Leftrightarrow&\dfrac{1}{4} \leq k \leq \dfrac{11}{4} \qquad\text{với } k\in \mathbb{N^*}\\
				&\Leftrightarrow& k \in \left\{1;2\right\}.
			\end{eqnarray*}
			Vậy có hai lần huyết áp của người đó là $80$mmHg trong khoảng thời gian từ $0$ đến $2$ giây.
		\end{enumerate}
	}
\end{bt}
%Câu 7...........................
\begin{bt}%[1H1K2-4]%[Dự án đề kiểm tra Toán 11 GHKI NH23-24- Nguyễn Sơn]%[THPT Phạm Phú Thứ - Tp HCM]
Cho hình chóp $ S.ABCD $ có đáy $ ABCD $ là hình bình hành tâm $O$. Gọi $M, N$ lần lượt là trung điểm của $SB, SD$. Trên đoạn thẳng $SC$ lấy điểm $P$ sao cho $SP > PC$. Biết $MN$ cắt $SO$.
\begin{listEX}[1]
	\item Tìm giao tuyến của hai mặt phẳng $ (SAD) $ và $ (SBC) $.
	\item Tìm giao tuyến của hai mặt phẳng $ (SAC) $ và $(MNP)$.
	\item Tìm giao điểm $Q$ của đường thẳng $SA$ và mặt phẳng $(MNP)$.
	\item Gọi $I,J$ lần lượt là giao điểm của $QM$ và $AB, QP$ và $AC$. Chứng minh $IJ, QN$ và $AD$ đồng quy.
\end{listEX}
\loigiai{
\begin{center}
\begin{tikzpicture}[
	>=stealth, line join=round, line cap=round, 
	scale=0.7,
	declare function={r=3; a=75;}]
% ve diem
	\path 
		(0,0) coordinate (A)
		(-3,-3) coordinate (D)
		(6,-3) coordinate (C)
		(9,0) coordinate (B)
		(2,6) coordinate (S)
		(-3,1) coordinate (x1)
		(4,8) coordinate (x)
		($ (S)!.5!(B) $) coordinate (M)
		($ (S)!.5!(D) $) coordinate (N)
		($ (S)!.85!(C) $) coordinate (P)
		(intersection of A--C and B--D) coordinate (O)	
		(intersection of S--O and M--N) coordinate (K)
		(intersection of P--K and S--A) coordinate (Q)
		(intersection of Q--M and A--B) coordinate (I)	
		(intersection of Q--P and A--C) coordinate (J)	
		(intersection of I--J and A--D) coordinate (E)	
		;
	\draw[thick] (S)--(B)--(C)--(D)--(S)--(C) (x1)--(x) (M)--(P) (N)--(P) (B)--(I)--(M) (P)--(J)--(C);
	\draw[dashed](S)--(A)--(B)--(D)	(D)--(A)--(C) (S)--(O) (M)--(N) (P)--(Q) (P)--(J)--(C) (P)--(Q)--(M) (Q)--(N);
	\draw  (I)--(E)--(A) (E)--(Q);
	\foreach \a/\b/\c in {}	{
		\draw pic [draw=black,angle radius = 10] {right angle = \a--\b--\c};	}
% danh dau diem
	\foreach \x/\g in {S/90, A/135, B/-90, C/-140, D/180, M/90, N/180, O/-90, P/0, x/90, K/-120, Q/20, I/-90, J/-90, E/-90}{
		\draw[fill=black] (\x) circle (1.5pt)+(\g:.4) node{$\x$};}
\end{tikzpicture}
\end{center}
\begin{listEX}[1]
	\item Ta có $\heva{&S\in (SAD)\cap (SBC) \\ & AD\subset (SAD)\\&BC\subset(SBC)\\&AD\parallel BC}\Rightarrow (SAD)\cap (SBC) = Sx $ với $Sx \parallel AD \parallel BC$.
	\item Trong mặt phẳng $(SBD)$, gọi $K$ là giao điểm của $MN$ và $SO$.\\
	 Suy ra  $K\in (SAC)\cap (MNP)$.\\
	Ta lại có  $P\in (SAC)\cap (MNP)$.\\
	Suy ra $KP=(SAC)\cap(MNP)$.
	\item Trong mặt phẳng $(SAC)$ gọi $Q$ là giao điểm của $SA$ và $KP$.\\
	Suy ra $\heva{&Q\in SA\\&Q\in KP\subset (MNP)}\Rightarrow Q=SA\cap (MNP)$.
	\item Trong mặt phẳng $(SAD)$ gọi $E$ là giao điểm của $QN$ và $AD$.\\
	Ta cần chứng minh ba điểm $E, J, I $ thẳng hàng.\\
	Ta có $\heva{&E\in AD\subset (ABCD)\\&E\in QN\subset(MNP)}\Rightarrow E\in (ABCD)\cap (MNP)$\quad (1).\\
	Ta có $\heva{&J\in AC\subset (ABCD)\\&J\in QP\subset(MNP)}\Rightarrow J\in (ABCD)\cap (MNP)$\quad (2).\\
	Ta có $\heva{&I\in AB\subset (ABCD)\\&I\in QM\subset(MNP)}\Rightarrow I\in (ABCD)\cap (MNP)$\quad (3).\\
	Từ (1) (2) và (3) suy ra $E, I, J$ cùng thuộc giao tuyến hai mặt phẳng $(ABCD)$ và $(MNP)$. Suy ra $E, J, I $ thẳng hàng. Vậy ba đường thẳng $IJ, QN$ và $AD$ đồng quy.
\end{listEX}
}
\end{bt}

