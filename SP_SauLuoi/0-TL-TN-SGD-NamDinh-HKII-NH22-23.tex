
\de{ĐỀ THI HỌC KỲ I NĂM HỌC 2022-2023}{SGD Nam Định}
\begin{center}
	\textbf{PHẦN 1 - TRẮC NGHIỆM}
\end{center}
\Opensolutionfile{ans}[ans/ans]
%Câu 1...........................
\begin{ex}%[0D1Y3-1]%[Dự án đề kiểm tra HKII NH22-23- Nguyễn Cường]%[Sở Giáo Dục và Đào Tạo Nam Định]
Cho hai tập hợp $X=\{-1;1;3;5;8\}$; $Y=\{3;5;7;9\}$. Tập hợp $X\cup Y$ bằng tập hợp nào sau đây?
	\choice
	{$\{3;5\}$}
	{$\{1;7;9\}$}
	{$\{1;3;5\}$}
	{\True $\{-1;1;3;5;7;8;9\}$}
	\loigiai{
	Ta có $X\cup Y=\{-1;1;3;5;7;8;9\}$.
	}
\end{ex}
%Câu 2...........................
\begin{ex}%[0D2B1-2]%[Dự án đề kiểm tra HKII NH22-23- Nguyễn Cường]%[Sở Giáo Dục và Đào Tạo Nam Định]
	\immini{
	Nửa mặt phẳng không tô màu (kể cả bờ) trong hình vẽ bên là miền nghiệm của bất phương trình nào?
	\choice
	{$x+2y\ge 2$}
	{\True $2x+y\le 2$}
	{$x+2y\le 2$}
	{$2x+y\ge 2$}
}
{
\begin{tikzpicture}[line join=round, line cap=round,>=stealth,thick]
	\tikzset{every node/.style={scale=0.9}}
	\begin{scope}
		\clip (-1,-1) rectangle (2,3);
		\fill[pattern=crosshatch dots gray] (-2,6)--(3,6)--(3,-4)--cycle;
		\draw (-0.5,3)--(1.5,-1) 
		;
	\end{scope}
	\draw[->] (-1,0)--(2,0) node[below]{$x$};
	\draw[->] (0,-1)--(0,3) node[left]{$y$};
	\draw (0,0) node[below left]{$O$};
	\foreach \x in {1}
	\draw[thin] (\x,1pt)--(\x,-1pt) node [below] {$\x$};
	\foreach \y in {2}
	\draw[thin] (1pt,\y)--(-1pt,\y) node [left] {$\y$};
\end{tikzpicture}
}
	\loigiai{
		Đường thẳng $y=-2x+2$ đi qua điểm $A(1;0)$ và $B(0;2)$.\\
		Gốc tọa độ $O$ thuộc miền nghiệm của bất phương trình nên hình vẽ chính là miền nghiệm của bất phương trình $2x+y\le 2$.
	}
\end{ex}
%Câu 3...........................
\begin{ex}%[0D3Y2-1]%[Dự án đề kiểm tra HKII NH22-23- Nguyễn Cường]%[Sở Giáo Dục và Đào Tạo Nam Định]
	Biểu thức nào sau đây là một tam thức bậc hai?
		\choice
		{$f(x)=x^4-2x^2+2023$}
		{\True $f(x)=\sqrt{2}x^2-2x+2023$}
		{$f(x)=\left(\dfrac{1}{x}\right)^2-\dfrac{2}{x}-3$}
		{$f(x)=x+3\sqrt{x}-2023$}
	\loigiai{
		Tam thức bậc hai có dạng $y=ax^2+bx+c$ với $a\ne 0$.\\
		Do đó, $f(x)=\sqrt{2}x^2-2x+2023$ là một tam thức bậc hai.
	}
\end{ex}
%Câu 4...........................
\begin{ex}%[0H3Y1-2]%[Dự án đề kiểm tra HKII NH22-23- Nguyễn Cường]%[Sở Giáo Dục và Đào Tạo Nam Định]
	Trong mặt phẳng tọa độ $Oxy$, cho hai véc-tơ $\overrightarrow{a}=(1;2)$, $\overrightarrow{b}=(3;4)$. Tọa độ của véc-tơ $\overrightarrow{c}=\overrightarrow{a}+\overrightarrow{b}$ là
	\choice
	{$\overrightarrow{c}=(2;2)$}
	{\True $\overrightarrow{c}=(4;6)$}
	{$\overrightarrow{c}=(6;4)$}
	{$\overrightarrow{c}=(-2;2)$}
	\loigiai{
		Ta có $\overrightarrow{c}=\overrightarrow{a}+\overrightarrow{b}=(4;6)$.
	}
\end{ex}
%Câu 5...........................
\begin{ex}%[0D2Y1-1]%[Dự án đề kiểm tra HKII NH22-23- Nguyễn Cường]%[Sở Giáo Dục và Đào Tạo Nam Định]
	Trên giá sách có $5$ cuốn truyện ngắn, $6$ cuốn tiểu thuyết (tất cả đều khác nhau). Có bao nhiêu cách chọn một cuốn để đọc?
	\choice
	{\True $11$}
	{$30$}
	{$6$}
	{$5$}
	\loigiai{
		Áp dụng quy tắc cộng ta có $5+6=11$ cách chọn một cuốn sách để đọc.
	}
\end{ex}
%Câu 6...........................
\begin{ex}%[0D2B2-5]%[Dự án đề kiểm tra HKII NH22-23- Nguyễn Cường]%[Sở Giáo Dục và Đào Tạo Nam Định]
	Từ các chữ số $1$; $2$; $3$; $4$; $5$; $6$; $7$; $8$ có thể lập được bao nhiêu số tự nhiên có ba chữ số khác nhau?
	\choice
	{$\mathrm{P}_3$}
	{$\mathrm{C}_8^3$}
	{\True $\mathrm{A}_8^3$}
	{$8^3$}
	\loigiai{
		Mỗi số tự nhiên có ba chữ số khác nhau được lập từ $8$ chữ số đã cho là một chỉnh hợp chập $3$ của $8$.\\
		Do đó, có $\mathrm{A}_8^3$ số tự nhiên được lập.
	}
\end{ex}
%Câu 7...........................
\begin{ex}%[0D1B3-3]%[Dự án đề kiểm tra HKII NH22-23- Nguyễn Cường]%[Sở Giáo Dục và Đào Tạo Nam Định]
	Điểm học kỳ II môn Toán khối 10 của một nhóm học sinh được thống kê như sau 
	\[5;5;6;6;6;7;7;8;9;9.\]
	Tìm tứ phân vị thứ ba
	\choice
	{$Q_3=7$}
	{$Q_3=6{,}5$}
	{$Q_3=6$}
	{\True $Q_3=8$}
	\loigiai{
	\begin{itemize}
		\item Sắp xếp thứ tự không giảm mẫu số liệu ta có $5;5;6;6;6;7;7;8;9;9$.
		\item Vì cỡ mẫu là $n=10$ nên tứ phân vị thứ hai là $Q_2=\dfrac{6+7}{2}=6{,}5$.
		\item Tứ phân vị thứ ba là trung vị của mẫu $7;7;8;9;9$. Do đó, $Q_3=8$.
	\end{itemize}
	}
\end{ex}
%Câu 8...........................
\begin{ex}%[0H3Y2-2]%[Dự án đề kiểm tra HKII NH22-23- Nguyễn Cường]%[Sở Giáo Dục và Đào Tạo Nam Định]
	Trong mặt phẳng tọa độ $Oxy$, cho hai đường thẳng $d_1$ và $d_2$ với các véc-tơ pháp tuyến $\overrightarrow{n}_1=(a_1;b_1)$ và $\overrightarrow{n}_2=(a_2;b_2)$ tương ứng. Gọi góc $\varphi$ là góc giữa hai đường thẳng $d_1$ và $d_2$. Khẳng định nào sau đây đúng?
	\choice
	{$\sin\varphi=\dfrac{|a_1a_2+b_1b_2|}{\sqrt{a^2_1+b^2_1}\cdot\sqrt{a_2^2+b_2^2}}$}
	{\True $\cos\varphi=\dfrac{|a_1a_2+b_1b_2|}{\sqrt{a^2_1+b^2_1}\cdot\sqrt{a_2^2+b_2^2}}$}
	{$\cos\varphi=\dfrac{|a_1a_2+b_1b_2|}{\sqrt{a^2_1+b^2_1}+\sqrt{a_2^2+b_2^2}}$}
	{$\sin\varphi=\dfrac{|a_1a_2+b_1b_2|}{\sqrt{a^2_1+b^2_1}+\sqrt{a_2^2+b_2^2}}$}
	\loigiai{
	Ta có $\cos\varphi=\dfrac{|a_1a_2+b_1b_2|}{\sqrt{a^2_1+b^2_1}\cdot\sqrt{a_2^2+b_2^2}}$.
	}
\end{ex}
%Câu 9...........................
\begin{ex}%[0H4Y1-3]%[Dự án đề kiểm tra HKII NH22-23- Nguyễn Cường]%[Sở Giáo Dục và Đào Tạo Nam Định]
	Trong mặt phẳng tọa độ $Oxy$, cho hai đường thẳng $d_1\colon 3x-4y+1=0$ và $d_2\colon x-y+2023=0$. Khẳng định nào sau đây đúng?
	\choice
	{$d_1$ song song với $d_2$}
	{$d_1$ trùng với $d_2$}
	{$d_1$ vuông góc với $d_2$}
	{\True $d_1$ cắt $d_2$}
	\loigiai{
		Do $\dfrac{3}{1}\ne \dfrac{-4}{-1}$ nên hai đường thẳng $d_1$ và $d_2$ cắt nhau.
	}
\end{ex}
%Câu 10...........................
\begin{ex}%[0H4Y2-2]%[Dự án đề kiểm tra HKII NH22-23- Nguyễn Cường]%[Sở Giáo Dục và Đào Tạo Nam Định]
	Trong mặt phẳng tọa độ $Oxy$, cho đường tròn $(C)\colon x^2+y^2-2x-2y-2=0$. Điểm nào sau đây thuộc đường tròn $(C)$?
	\choice
	{$M(1;0)$}
	{\True $N(1;3)$}
	{$Q(2;2)$}
	{$P(3;-1)$}
	\loigiai{
		Tọa độ điểm $N(1;3)$ thỏa mãn phương trình đường tròn $(C)$. Do đó, $N\in (C)$.
	}
\end{ex}
%Câu 11...........................
\begin{ex}%[0H4Y3-3]%[Dự án đề kiểm tra HKII NH22-23- Nguyễn Cường]%[Sở Giáo Dục và Đào Tạo Nam Định]
	Trong mặt phẳng tọa độ $Oxy$, phương trình nào sau đây là phương trình chính tắc của elip?
	\choice
	{\True $\dfrac{x^2}{9}+\dfrac{y^2}{4}=1$}
	{$\dfrac{x^2}{9}-\dfrac{y^2}{4}=1$}
	{$\dfrac{x}{9}+\dfrac{y}{4}=1$}
	{$\dfrac{x^2}{9}+\dfrac{y^2}{4}=0$}
	\loigiai{
		Phương trình chính tắc của elip có dạng $\dfrac{x^2}{a^2}+\dfrac{y^2}{b^2}=1$ với $a>b>0$.\\
		Do đó, $\dfrac{x^2}{9}+\dfrac{y^2}{4}=1$ là phương trình chính tắc của elip.
	}
\end{ex}
%Câu 12...........................
\begin{ex}%[0H4B3-4]%[Dự án đề kiểm tra HKII NH22-23- Nguyễn Cường]%[Sở Giáo Dục và Đào Tạo Nam Định]
	Trong mặt phẳng tọa độ $Oxy$, cho hypebol có phương trình chính tắc $\dfrac{x^2}{9}-\dfrac{y^2}{16}=1$. Tìm tọa độ các tiêu điểm của hypebol.
	\choice
	{$F_1(-3;0)$, $F_2(3;0)$}
	{$F_1(-4;0)$, $F_2(4;0)$}
	{\True $F_1(-5;0)$, $F_2(5;0)$}
	{$F_1(0;-5)$, $F_2(0;5)$}
	\loigiai{
		Ta có $a^2=9$ và $b^2=16$, suy ra $c^2=a^2+b^2=25\Rightarrow c=5$.\\
		Do đó, hai tiêu điểm của hypebol là $F_1(-5;0)$, $F_2(5;0)$.
	}
\end{ex}
\Closesolutionfile{ans}

%\begin{center}
%	\textbf{ĐÁP ÁN}
%	\inputansbox{10}{ans/ans}	
%\end{center}


\begin{center}
	\textbf{PHẦN 2 - TỰ LUẬN}
\end{center}

\begin{bt}%[0T3B2-2]%[Dự án đề kiểm tra HKII NH22-23- Nguyễn Sĩ Đạt]%[Sở Nam Định]
Xác định parabol $(P)$ có phương trình $y=ax^2+bx+1$, biết rằng $(P)$ có trục đối xứng là đường thẳng $x=1$ và đi qua điểm $A(1;-2)$.
\loigiai{
Vì $(P)$ có trục đối xứng là $x=1$ nên 
$\dfrac{-b}{2a}=1 \Leftrightarrow -b =2a \Leftrightarrow 2a+b=0 $.\hfill (1)\\
Vì $(P)$ đi qua điểm $A(1;-2)$ nên $-2=a+b+1 \Leftrightarrow a+b=-3$.\hfill (2)\\
Từ (1) và (2), ta có hệ phương trình
$\heva{& 2a+b=0\\& a+b=-3} \Leftrightarrow  \heva{& a=3\\& b=-6.}$ \\
Vậy $(P) \colon  y=3x^2-6x+1$.
}
\end{bt}

\begin{bt}%[0T6Y4-2]%[0T6Y3-1]%[Dự án đề kiểm tra HKII NH22-23- Nguyễn Sĩ Đạt]%[Sở Nam Định]
Tốc độ tăng trưởng GDP của Việt Nam từ năm 2014 đến năm 2022 được cho trong bảng sau:
\begin{center}
{\renewcommand\arraystretch{1.5}
\begin{tabular}{|>{\centering\arraybackslash}m{2.3cm}|>{\centering\arraybackslash}m{1cm}|>{\centering\arraybackslash}m{1cm}|>{\centering\arraybackslash}m{1cm}|>{\centering\arraybackslash}m{1cm}|>{\centering\arraybackslash}m{1cm}|>{\centering\arraybackslash}m{1cm}|>{\centering\arraybackslash}m{1cm}|>{\centering\arraybackslash}m{1cm}|>{\centering\arraybackslash}m{1cm}|}
\hline
Năm & $2014$ & $2015$ & $2016$ & $2017$ & $2018$ & $2019$ & $2020$ & $2021$ & $2022$\\
\hline
GDP & $6{,}42$ & $6{,}99$ & $6{,}69$ & $6{,}94$ & $7{,}47$ & $7{,}36$ & $2{,}87$ & $2{,}56$ & $8{,}02$\\
\hline 
\end{tabular}
}
\end{center}
Tính số trung bình và phương sai của mẫu số liệu này (kết quả tính phương sai làm tròn đến hàng 
phần trăm).
\loigiai{
Số trung bình là
\begin{center}
$\overline{x} = \dfrac{6{,}42 + 6{,}99 + 6{,}69 + 6{,}94 + 7{,}47 + 7{,}36 +2{,}87 +2{,}56 +8{,}02}{9} = \dfrac{461}{75} $.
\end{center}
Phương sai là  
$s^2 = \left[ \left(6{,}42 - \dfrac{461}{75}\right)^2 + \left(6{,}99 - \dfrac{461}{75}\right)^2 + \left(6{,}69 - \dfrac{461}{75}\right)^2 + \left(6{,}94 - \dfrac{461}{75}\right)^2\right.$\\ $\left.+\left(7{,}47 - \dfrac{461}{75}\right)^2 + \left(7{,}36 - \dfrac{461}{75}\right)^2 + \left(2{,}87 - \dfrac{461}{75}\right)^2 + \left(2{,}56 - \dfrac{461}{75}\right)^2 + \left(8{,}02 - \dfrac{461}{75}\right)^2 \right] :9$\\$\approx 3{,}56$. 
}
\end{bt}

\begin{bt}%[0T4T2-1]%[Dự án đề kiểm tra HKII NH22-23- Nguyễn Sĩ Đạt]%[Sở Nam Định]
Để đo khoảng cách từ một điểm $A$ trên bờ sông đến gốc cây $C$ trên cù lao giữa sông. Người ta chọn một điểm $B$ cùng ở trên bờ với $A$ sao cho từ $A$ và $B$ có thể nhìn thấy điểm $C$. Người ta đo được khoảng cách $AB = 50$ m, $\widehat{CAB}=45^\circ$ và $\widehat{CBA}=70^\circ$ (như hình dưới đây). Tính khoảng cách từ $A$ đến gốc cây $C$ trên cù lao (kết quả làm tròn đến hàng phần trăm).
\begin{center}
\begin{tikzpicture}[scale=0.8, font=\footnotesize, line join = round, line cap = round,>=stealth]
\clip (-4.39,3.2) rectangle (4,-4);
\draw[pattern=north east lines,opacity=0.3] plot[smooth] coordinates{(-4.39,1.43)(-2.43,1.46) (-1.48,1.67)(1.43,2.12)(2.01,2.09)(4,1.64) (4.1,-1.06)(3.02,-0.77)(1.85,-1.38)(0.95,-1.48)(0.13,-1.72)(-1.48,-1.67)(-2.46,-1.51)(-3.57,-0.98)(-4.37,-1.08)};
\draw[opacity=3] plot[smooth] coordinates{(-4.39,1.43)(-2.43,1.46) (-1.48,1.67)(1.43,2.12)(2.01,2.09)(4,1.64) (4.1,-1.06)(3.02,-0.77)(1.85,-1.38)(0.95,-1.48)(0.13,-1.72)(-1.48,-1.67)(-2.46,-1.51)(-3.57,-0.98)(-4.37,-1.08)};
\draw[fill=white] plot[smooth cycle] coordinates{(-2.75,-0.08)(-1.83,0.32) (-0.03,0.42) (1,0) (0.24,-0.48)(-0.58,-0.79)(-1.38,-0.77)(-1.91,-0.71)};
\draw[fill=black!70] plot[smooth cycle] coordinates{(-2.14,0.58)(-2.09,0.24)(-2.33,-0.13) (-1.19,-0.13) (-1.38,-0.13) (-1.69,0.21)(-1.75,0.53)};
\draw[fill=blue!30] plot[smooth cycle] coordinates{(-1.75,0.53)(-1.46,0.79) (-1.01,0.58) (-1.08,1.06) (-0.64,1.08)(-0.93,1.59)(-0.53,1.85)(-0.93,2.2)(-1.38,2.7)(-2.04,3.18)(-2.49,2.91)(-3.07,2.91)(-3.2,2.22)(-3.73,1.96)(-3.1,1.3)(-3.31,1.01)(-2.83,1.01)(-2.99,0.69)(-2.14,0.58)};
\path
(3,-2)coordinate (A)node[below]{$A$}
(-2,-3)coordinate (B)node[below]{$B$} 
(-1.38,-0.13)coordinate (C)node[above right]{$C$};
\foreach \diem in {A,B,C}\fill (\diem)circle(1.5pt);
\draw (A)--(B)node[below,pos=.5]{$50$ m}--(C)--(A);
\draw pic[draw, angle radius=6mm]{angle=C--A--B} (A) node[shift={(175:.9)}]{$45^\circ$};
\draw pic[draw,angle radius=5mm]{angle=A--B--C} (B) pic[draw,angle radius=6mm]{angle=A--B--C} (B) node[shift={(45:.9)}]{$70^\circ$};
\end{tikzpicture}
\end{center}
\loigiai{
Ta có $\widehat{BCA}=180^\circ - 45^\circ - 70^\circ = 65^\circ$. \\ 
Áp dụng định lý sin trong tam giác $ABC$, ta có $\dfrac{AC}{\sin B} = \dfrac{AB}{\sin C}$. \\
Suy ra $AC = \dfrac{AB \cdot \sin B}{\sin C} = \dfrac{50 \cdot \sin 70^\circ}{\sin 65^\circ} \approx 51{,}84$ (m).
} 

\end{bt}

\begin{bt}%[0T8Y3-2]%[0T8Y2-1]%[Dự án đề kiểm tra HKII NH22-23- Nguyễn Sĩ Đạt]%[Sở Nam Định]
\begin{enumerate}
\item Một lớp có $15$ học sinh nam và $20$ học sinh nữ. Có bao nhiêu cách chọn $5$ bạn học sinh sao cho trong đóng có đúng $3$ học sinh nữ?
\item Tìm hệ số của số hạng chứa $x^2$ trong khai triển của $(1+3x)^4$.
\end{enumerate} 
\loigiai{
\begin{enumerate}
\item Số cách chọn $5$ bạn học sinh trong đó có đúng $3$ học sinh nữ $\mathrm{C}_{20}^3 \cdot \mathrm{C}_{15}^2=119700$.
\item Xét khai triển $(1+3x)^4=\mathrm{C}_4^0 + \mathrm{C}_4^1 \cdot 3x +\mathrm{C}_4^2 \cdot (3x)^2 +\mathrm{C}_4^3 \cdot (3x)^3 + \mathrm{C}_4^4 \cdot (3x)^4$.\\
Vậy khai hệ số của số hạng chứa $x^2$ trong khai triển là $\mathrm{C}_4^2 \cdot 3^2=54$.
\end{enumerate}
}
\end{bt}

\begin{bt}%[0T0K2-2]%[Dự án đề kiểm tra HKII NH22-23- Nguyễn Sĩ Đạt]%[Sở Nam Định]
Chọn ngẫu nhiên $4$ số khác nhau từ $40$ số nguyên dương đầu tiên. Tính xác suất để chọn được $4$ số có tích chia hết chia $4$.
\loigiai{
Gọi $\Omega$ là không gian mẫu. Số phần tử của không gian mẫu $n(\Omega)=C_{40}^4=91390$.\\
Gọi $A$ là biến cố: \lq\lq Chọn được $4$ số có tích chia hết cho $4$\rq\rq.\\
Suy ra $\overline{A}$ là biến cố: \lq\lq Chọn ra $4$ số có tích không chia hết cho $4$\rq\rq.
\begin{itemize}
\item Trường hợp 1: $4$ số lẻ, có $C_{20}^4$ cách chọn.
\item Trường hợp 2: $3$ số lẻ, $1$ số chẵn không chia hết cho $4$, có $C_{20}^3 \cdot C_{10}^1$ cách chọn.
\end{itemize}
Ta có $n(\overline{A})=C_{20}^4+C_{20}^3 \cdot C_{10}^1=16245 \Rightarrow P(A)=1-P(\overline{A})=1-\dfrac{16245}{91390}=\dfrac{791}{962}$. }
\end{bt}

\begin{bt}%[0T9K3-5]%[Dự án đề kiểm tra HKII NH22-23- Nguyễn Sĩ Đạt]%[Sở Nam Định]
Trong mặt phẳng tọa độ, cho hai điểm $A(1;1)$, $B(3;3)$ và đường thẳng $\Delta$ có phương trình $x-y-2=0$.
\begin{enumerate}
\item Viết phương trình tham số của đường thẳng đi qua hai điểm $A$, $B$.
\item Tính khoảng cách từ $A$ đến đường thẳng $\Delta$.
\item Lập phương trình đường tròn đường kính $AB$.
\item Tìm tọa độ điểm $M$ nằm trên đường thẳng $\Delta$ sao cho tam giác $MAB$ vuông tại $M$.
\end{enumerate}
\loigiai{
\begin{enumerate}
\item Ta có $\vec{AB}=(2;2)$.\\
Đường thẳng đi qua hai điểm $A$, $B$ có vectơ chỉ phương là $\vec{AB}=(2;2)$ và đi qua điểm $A(1;1)$ nên có phương trình tham số là  $\heva{&x=1+2t \\&y=1+2t},\, t \in \mathbb{R}$.
\item Khoảng cách từ $A$ đến đường thẳng $\Delta$ là $\mathrm{d}\left (A, \Delta\right )=\dfrac{\left | 1-1-2 \right |}{\sqrt{1^2+(-1)^2}}=\sqrt{2}$.
\item Gọi $I$ là trung điểm của $AB$ suy ra $I(2;2)$ và $IB=\sqrt{(3-2)^2+(3-2)^2}=\sqrt{2}$.\\
Phương trình đường tròn đường kính AB có tâm $I(2;2)$ và bán kính $IB=\sqrt{2}$ là $(x-2)^2+(y-2)^2=2$.
\item Ta có $\widehat{AMB}=90^\circ$ nên điểm $M$ thỏa mãn yêu cầu bài toán khi $M$ thuộc vào đường tròn đường kính $AB$. Khi đó, $M$ là giao điểm của đường thẳng $\Delta$ và đường tròn đường kính $AB$.\\
Tọa độ điểm $M$ là nghiệm của hệ $\heva{&x-y-2=0 \\&(x-2)^2+(y-2)^2=2.}$\\
Giải hệ trên ta được tọa độ điểm $M(3;1)$.
\end{enumerate}}
\end{bt}

