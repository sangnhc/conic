
\de{ĐỀ THI GIỮA HỌC KỲ II NĂM HỌC 2022-2023}{THPT Bà Điểm}



\begin{bt}%[0T7B3-2]%[Dự án đề kiểm tra giữa HKII NH22-23- Nguyễn Ngọc Nguyên]%[THPT Bà Điểm]
Giải các phương trình sau
\begin{listEX}[2]
\item $\sqrt{x^2-6x+6}=2x-1$.
\item $\sqrt{2x^2-3x-1}=\sqrt{2x+3}$.
\end{listEX}
\loigiai{
\begin{enumerate}
	\item Ta có
	\begin{eqnarray*}
		\sqrt{x^2-6x+6}=2x-1 &\Rightarrow& x^2-6x+6 =(2x-1)^2 \Rightarrow x^2-6x+6 = 4x^2-4x+1  \\
		&\Rightarrow& -3x^2-2x+5=0 \Rightarrow \hoac{&x=1 \\ &x=-\dfrac{5}{3}.}
	\end{eqnarray*}
Thử lại ta có $x=1$ là nghiệm của phương trình.
\item Ta có
\begin{eqnarray*}
	\sqrt{2x^2-3x-1}=\sqrt{2x+3} \Rightarrow 2x^2-3x-1=2x+3 \Rightarrow 2x^2 -5x-4=0  \Rightarrow \hoac{&x=\dfrac{5-\sqrt{57}}{4} \\ & x=\dfrac{5+\sqrt{57}}{4}.}
\end{eqnarray*}
Thử lại ta có $x=\dfrac{5-\sqrt{57}}{4}$ và $x=\dfrac{5+\sqrt{57}}{4}$ là hai nghiệm của phương trình đã cho. 
\end{enumerate}
}
\end{bt}

\begin{bt}%[0T7K1-1] %[Dự án đề kiểm tra giữa HKII NH22-23- Nguyễn Ngọc Nguyên]%[THPT Bà Điểm]
	Cho phương trình $(m+3)x^2+2(m+1)x+m+3=0$ ($m$ là tham số). Xác định tất cả các giá trị của $m$ để phương trình đã cho có $2$ nghiệm phân biệt.	
	\loigiai{
 Phương trình đã cho có $2$ nghiệm phân biệt khi và chỉ khi
 \begin{eqnarray*}
 	\heva{&m+3 \ne 0 \\ & \Delta ' >0} \Leftrightarrow \heva{& m \ne -3 \\ & (m+1)^2 -(m+3)^2>0} \Leftrightarrow \heva{&m \ne -3 \\ & -4m-8>0} \Leftrightarrow \heva{&m \ne -3 \\& m<-2.}
 \end{eqnarray*}
Vậy $m \in (-\infty;-2) \setminus \{-3\}$ thỏa yêu cầu bài toán.
	}
\end{bt}

\begin{bt}%[0T8K2-1] %[Dự án đề kiểm tra giữa HKII NH22-23- Nguyễn Ngọc Nguyên]%[THPT Bà Điểm]
	Một lớp có $10$ học sinh nam và $25$ học sinh nữ. Có bao nhiêu cách chọn $5$ bạn học sinh sao cho trong đó có đúng $3$ học sinh nữ?
	\loigiai{
Để chọn được $5$ bạn học sinh sao cho trong đó có đúng $3$ học sinh nữ ta thực hiện các hành động liên tiếp sau:
\begin{itemize}
	\item[B1:] Chọn $3$ học sinh nữ trong $25$ học sinh nữ: có $\mathrm{C}_{25}^3$ (cách).
	\item[B2:] Chọn $2$ học sinh nam trong $10$ học sinh nam: có $\mathrm{C}_{10}^2$ (cách).
\end{itemize}		
Theo quy tắc nhân, ta có $\mathrm{C}_{25}^3 \cdot \mathrm{C}_{10}^2 =103500$ cách chọn $5$ bạn học sinh sao cho trong đó có đúng $3$ học sinh nữ .
	}
\end{bt}

\begin{bt}%[0T8K2-2]%[Dự án đề kiểm tra giữa HKII NH22-23- Nguyễn Ngọc Nguyên]%[THPT Bà Điểm]
	Từ các chữ số $0$, $1$, $2$, $3$, $4$, $5$, $6$, $7$ có thể lập được bao nhiêu số tự nhiên chẵn và có $6$ chữ số đôi một khác nhau?
	\loigiai{
		Gọi $\overline{a_1a_2a_3a_4a_5a_6}$ ($a_1 \ne 0$) là số tự nhiên chẵn có $6$ chữ số đôi một khác nhau được lấy từ tập $S=\{0;1;2;3;4;5;6;7\}$.\\
		Để thiết lập được số tự nhiên chẵn và có $6$ chữ số đôi một khác nhau được lấy từ tập $S=\{0;1;2;3;4;5;6;7\}$ ta chia thành các trường hợp như sau
		\begin{itemize}
			\item[TH1:] Số có dạng $\overline{a_1a_2a_3a_4a_50}$. Khi đó để thiết lập số tự nhiên thỏa yêu cầu ta sẽ chọn $5$ số trong tập $S \setminus \{0\}$ và điền vào các vị trí $a_1a_2a_3a_4a_5$: có $\mathrm{A}_7^5$ (cách). \\
			Vậy trong trường hợp này này có $\mathrm{A}_7^5$ (số).
			\item[TH2:] Số có dạng $\overline{a_1a_2a_3a_4a_5a_6}$ với $a_6 \ne 0$. Khi đó để thiết lập số tự nhiên thỏa yêu cầu ta thực hiện liên tiếp các bước sau
			\begin{itemize}
				\item[B1:] Chọn $1$ số trong tập $\{2;4;6\}$ để điền vào vị trí $a_6$: có $3$ (cách).
				\item[B2:] Chọn $1$ số trong tập $S \setminus \{0;a_6\}$ để điền vào vị trí $a_1$: có $6$ (cách).
				\item[B3:] chọn $4$ số trong tập $S \setminus \{a_1;a_6\}$ và điền vào $4$ vị trí $a_2a_3a_4a_5$ : có $\mathrm{A}^4_6$ (cách).
			\end{itemize}
		Theo quy tắc nhân, trong trường hợp này ta có $3 \cdot 6 \cdot \mathrm{A}^4_6$ (số).
		\end{itemize}
	Theo quy tắc cộng, ta có $\mathrm{A}_7^5+3 \cdot 6 \cdot \mathrm{A}^4_6=9000$ số thỏa yêu cầu bài toán.
	}
\end{bt}


\begin{bt}%[0T8B2-1]%[Dự án đề kiểm tra HKII NH22-23- Nguyễn Ngọc Dũng]%[THPT Bà Điểm]
Một trường cấp 3 của TP.HCM có $8$ giáo viên Toán gồm có $3$ nữ và $5$ nam, giáo viên Vật lý thì có $4$ giáo viên nam. Hỏi có bao nhiêu cách chọn ra một đoàn thanh tra công tác ôn thi THPTQG gồm $3$ người có đủ $2$ môn Toán và Vật lý và phải có giáo viên nam và giáo viên nữ trong đoàn?
\loigiai{
\begin{enumerate}[\bf Trường hợp 1.]
\item $2$ giáo viên Toán và $1$ giáo viên Vật lý: $\mathrm{C}^1_3\cdot \mathrm{C}^1_5\cdot \mathrm{C}^1_4 +  \mathrm{C}^2_3\cdot \mathrm{C}^1_4 = 72$ (cách).
\item $1$ giáo viên Toán và $2$ giáo viên Vật lý: $\mathrm{C}^1_3 \cdot \mathrm{C}^2_4 =18$ (cách).
\end{enumerate}	
Vậy có tất cả $72+18=90$ cách.
}
\end{bt}

\begin{bt}%[0T8B3-1]%[Dự án đề kiểm tra HKII NH22-23- Nguyễn Ngọc Dũng]%[THPT Bà Điểm]
Khai triển theo công thức nhị thức Newton và rút gọn: $\left( 3x^2-2\right)^4$.
\loigiai{
Theo công thức nhị thức Newton, ta có:
\allowdisplaybreaks 
\begin{eqnarray*}
&&\left( 3x^2-2\right)^4 \\
&=& \mathrm{C}_4^0 \left(3x^2\right)^4 (-2)^0 + \mathrm{C}_4^1 \left(3x^2\right)^3 (-2)^1 + \mathrm{C}_4^2 \left(3x^2\right)^2 (-2)^2 + \mathrm{C}_4^3 \left(3x^2\right)^1 (-2)^3 + \mathrm{C}_4^4 \left(3x^2\right)^0 (-2)^4\\	
&=& \left(3x^2\right)^4 - 4 \cdot \left(3x^2\right)^3 \cdot 2 + 6 \cdot \left(3x^2\right)^2 \cdot 4 - 4 \cdot 3x^2 \cdot 8 + 16\\	
&=& 81 x^8 - 216 x^6 + 216 x^4 - 96 x^2 + 16
\end{eqnarray*}	
}
\end{bt}

\begin{bt}%[0T8K3-2]%[Dự án đề kiểm tra HKII NH22-23- Nguyễn Ngọc Dũng]%[THPT Bà Điểm]
Tìm số hạng không chứa $x$ trong khai triển của $P(x)=\left( 2x^3-\dfrac{3}{x^2} \right)^5$ $(x\neq 0)$.
\loigiai{
Số hạng tổng quát của khai triển là 
$$\mathrm{C}^k_5\cdot \left(2x^3\right)^{5-k} \cdot \left(-\dfrac{3}{x^2}\right)^k = \mathrm{C}^k_5\cdot 2^{5-k}\cdot (-3)^k\cdot \dfrac{x^{15-3k}}{x^{2k}} = \mathrm{C}^k_5\cdot 2^{5-k}\cdot (-3)^k\cdot x^{15-5k}, \text{ với } 0\leq k\leq 5.$$
Số hạng không chứa $x$ tương ứng với $15-5k=0\Leftrightarrow k=3$ (nhận).\\
Vậy số hạng không chứa $x$ trong khai triển là $\mathrm{C}^3_5\cdot 2^2\cdot (-3)^3 = -1080$.
}
\end{bt}

\begin{bt}%[0T5B4-6]%[Dự án đề kiểm tra HKII NH22-23- Nguyễn Ngọc Dũng]%[THPT Bà Điểm]
Trong mặt phẳng $Oxy$ cho $\triangle BMC$ với điểm $M(1;-2)$; $B(2;-4)$; $C(1;0)$. Tính góc $M$ của $\triangle BMC$.
\loigiai{
Ta có $\overrightarrow{MB} = (1;-2)$; $\overrightarrow{MC} = (0;2)$. Khi đó
$$\cos \widehat{BMC} = \cos \left( \overrightarrow{MB}, \overrightarrow{MC} \right) = \dfrac{\overrightarrow{MB}\cdot \overrightarrow{MC}}{MB\cdot MC} =\dfrac{1\cdot 0 + (-2)\cdot 2}{\sqrt{1^2+(-2)^2}\cdot \sqrt{0^+2^2}} = \dfrac{-2}{\sqrt{5}}$$
Suy ra $\widehat{BMC} \approx 153^\circ$.
}
\end{bt}
