\de{ĐỀ THI HỌC KỲ II NĂM HỌC 2022-2023}{THPT Trần Quang Khải}

%---Câu 1
\begin{bt}%[0T8Y1-2]%[0T8Y1-1] %[Dự án đề kiểm tra HKII NH22-23-Nguyen Huynh]%[THPT Trần Quang Khải]
	Lớp 10A có $20$ học sinh nam và $25$ học sinh nữ tham gia hội trại của trường.
	\begin{enumerate}
		\item[a)] Có bao nhiêu cách chọn một học sinh đi thi nhảy bao bố?
		\item[b)] Có bao nhiêu cách chọn một học sinh nam và một học sinh nữ đi thi cặp đôi thanh lịch?
	\end{enumerate}
	\loigiai{
		\begin{enumerate}
			\item[a)] Lớp có $45$ học sinh, chọn $1$ học sinh trong $45$ học sinh có $45$ cách chọn.
			\item[b)] Ta thực hiện 2 bước sau
			\begin{itemize}
				\item Chọn $1$ học sinh nam trong $20$ học sinh nam có $20$ cách.
				\item Chọn $1$ học sinh nữ trong $25$ học sinh nam có $25$ cách.
			\end{itemize}
			Vậy để chọn một học sinh nam và một học sinh nữ đi thi cặp đôi thanh lịch có $20 \cdot 25 = 500$ cách.
	\end{enumerate}}
\end{bt}


%---Câu 2
\begin{bt}%[0T7Y3-2]%[Dự án đề kiểm tra HKII NH22-23-Nguyen Huynh]%[THPT Trần Quang Khải]
	Giải phương trình:  $\sqrt{x^2+3x-3}= 3-2x$
	\loigiai{
		\allowdisplaybreaks
		\begin{eqnarray*}
			& &\sqrt{x^2+3x-3}= 3-2x \quad \hfill(1) \\ 
			&\Rightarrow& x^2 + 3x -3 = 9 -12x + 4x^2\\
			&\Rightarrow& 3x^2 - 15x + 12= 0\\
			&\Rightarrow& \hoac{&x=1\\&x=4.}
		\end{eqnarray*}
		Thử lại vào phương trình (1) ta thấy giá trị $x=1$ thỏa mãn.\\
		Vậy nghiệm của phương trình $\sqrt{x^2+3x-3}= 3-2x$ là $1$.
	}
\end{bt}

%---Câu 3
\begin{bt}%[0T8Y3-2]%[Dự án đề kiểm tra HKII NH22-23-Nguyen Huynh]%[THPT Trần Quang Khải]
	Sử dụng công thức nhị thức Newton, hãy khai triển biểu thức $(x-2)^4$ và cho biết hệ số của $x^3$ là bao nhiêu?
	\loigiai{	
		\begin{itemize}
			\item { Khai triển nhị thức Newton $(x-2)^4$
				\allowdisplaybreaks
				\begin{eqnarray*}
					& &(x-2)^4 \\
					&=& x^4 - 4x^3 \cdot 2 + 6x^2 \cdot 2^2 -4x \cdot 2^3 +2^4 \\
					&=& x^4 -8x^3 +24x^2-32x+16.
			\end{eqnarray*}}
			\item {
				Dựa vào khai triển 	$(x-2)^4 = x^4 -8x^3 +24x^2-32x+16$ hệ số của $x^3$ là $-8$.
			}
	\end{itemize}}
\end{bt}

%---Câu 4
\begin{bt}%[0T9B1-3]%[0T9Y2-2]%[Dự án đề kiểm tra HKII NH22-23-Nguyen Huynh]%[THPT Trần Quang Khải]
	Trong mặt phẳng, $Oxy$ cho $\triangle ABC$ với $A(1;2), B(3;0)$ và $C(-2;-1)$.
	\begin{enumerate}
		\item [a)] Tìm tọa độ trung điểm $M$ của cạnh $BC$ và tọa độ trọng tâm $G$ của $\triangle ABC$.
		\item [b)] Viết phương trình tham số của đường thẳng $d$ đi qua $A$ và $B$.
	\end{enumerate}
	\loigiai{
		\begin{enumerate}
			\item [a)] 
			{\begin{itemize}
					\item {
						Tọa độ trung điểm $M$ của $BC$ là nghiệm của hệ \\
						$\heva{&x_M=\dfrac{x_B+x_C}{2} \\ &y_M=\dfrac{y_B+y_C}{2}} 
						\Leftrightarrow \heva{&x_M=\dfrac{3-2}{2} \\ &y_M=\dfrac{0-1}{2}}
						\Leftrightarrow \heva{&x_M=\dfrac{1}{2} \\ &y_M=\dfrac{-1}{2}.} $\\
						Tọa độ trung điểm $M$ của $BC$ là $M\left( \dfrac{1}{2} ;\dfrac{-1}{2}\right)$.	
					}			
					\item {
						Tọa độ trọng tâm $G$ của $\triangle ABC$ là nghiệm của hệ \\
						$\heva{&x_G=\dfrac{x_A+x_B+x_C}{2} \\ &y_G=\dfrac{y_A+y_B+y_C}{2}} 
						\Leftrightarrow \heva{&x_G=\dfrac{1+3-2}{3} \\  &y_G=\dfrac{2+0-1}{3}}
						\Leftrightarrow \heva{&x_G=\dfrac{2}{3} \\ &y_G=\dfrac{1}{3}.} $\\
						Tọa độ trọng tâm $G$ của $\triangle ABC$ là $G\left( \dfrac{2}{3} ;\dfrac{1}{3}\right)$.
					}
			\end{itemize}}
			\item [b)]  
			{Đường thẳng $AB$ qua $A(1;2)$ và nhận véc-tơ $\overrightarrow{AB}=(2;-2)=2(1;-1)$ làm véc-tơ chỉ phương.\\
				Phương trình tham số của đường thẳng $AB$ là\\
				$$\heva{&x=1+t\\&y=2-t}(t\in\mathbb R)$$
			}
		\end{enumerate}
	}
\end{bt}


%---Câu 5
\begin{bt}%[0T9Y3-3]%[Dự án đề kiểm tra HKII NH22-23-Nguyen Huynh]%[THPT Trần Quang Khải]
	Một vận động viên ném đĩa đã vung đĩa theo một đường tròn $(C)$ có phương trình $x^2+y^2-2x-4y-20=0$. Khi người đó vung đĩa đến vị trí điểm $M(4;6)$ thì buông đĩa. Viết phương trình tiếp tuyến của đường tròn $(C)$ tại điểm $M$.
	\loigiai{
		Đường tròn $(C) \colon x^2+y^2-2x-4y-20=0$ có tâm $I(1;2)$.\\
		Véc-tơ $\overrightarrow{IM} = (3;4)$.\\
		Phương trình tiếp tuyến $\Delta$ của đường tròn $(C)$ qua điểm $M(4;6)$ và nhận véc-tơ $\overrightarrow{IM} = (3;4)$ làm véc-tơ pháp tuyến.
		$$\Delta \colon 3(x-4)+4(y-6)=0 
		\Leftrightarrow 3x+4y-36=0.
		$$
		Vậy phương trình tiếp tuyến của đường tròn $(C)$ tại điểm $M$ là $3x+4y-36=0$.
	}
\end{bt}

%---Câu 6
\begin{bt}%[0T9Y4-1]%[Dự án đề kiểm tra HKII NH22-23-Thy Nguyen Vo Diem]%[THPT Trần Quang Khải]
	Tìm tọa độ các tiêu điểm, độ dài trục lớn và trục nhỏ của elip $(E)\colon \dfrac{x^2}{36}+\dfrac{y^2}{16}=1$.
	\loigiai
	{
		Ta có $\heva{& a^2=36 \\ & b^2=16}\Rightarrow \heva{& a=6 \\ & b=4}$ (vì $a,b>0$).\\
		Ta có $a^2=b^2+c^2\Rightarrow c=5$ (vì $c>0$).\\
		Độ dài trục lớn $2a=12$.\\
		Độ dài trục nhỏ $2b=8$.\\
		Tọa độ các tiêu điểm $F_1(-5;0)$, $F_2(5;0)$.
	}
\end{bt}


%---Câu 7
\begin{bt}%[0T8B2-2] %[Dự án đề kiểm tra HKII NH22-23-Thy Nguyen Vo Diem]%[THPT Trần Quang Khải]
Lễ hội pháo hoa quốc tế Đà Nẵng (tên tiếng Anh: Danang International Fireworks Festival, viết tắt là DIFF) do Ủy ban nhân dân thành phố Đà Nẵng tổ chức lần đầu tiên vào năm 2008 với mục đích kích cầu du lịch, nhằm nâng cao trải nghiệm của du khách khi đến thăm miền Trung. Năm này qua năm khác, lễ hội pháo hoa quốc tế được đổi mới, ngày càng qui mô hơn, thu hút hàng triệu lượt khách du lịch. Từ năm 2020 đến năm 2022, lễ hội bị hủy do ảnh hưởng của dịch COVID-19. Năm 2023, lễ hội sẽ được tổ chức vào các tối thứ bảy từ ngày 3/6 đến 8/7 với sự tham gia của 8 đội: Anh, Ý, Ba Lan, Pháp, Úc, Canada, Phần Lan và Đà Nẵng (Việt Nam). Đêm khai mạc 3/6 sẽ diễn ra cuộc so tài của đội chủ nhà Đà Nẵng với đội Phần Lan (vô địch DIFF 2019). Các đội còn lại sẽ lần lượt trình diễn vào các đêm tiếp theo. Mỗi đội có 20 phút trình diễn các màn pháo hoa theo chủ đề từng đêm do ban tổ chức qui định. Hai đội có điểm cao nhất sẽ vào chung kết tranh ngôi vô địch trong đêm 8/7.
\begin{enumerate}
	\item Có bao nhiêu cách sắp xếp các cặp đấu cho 6 đội còn lại?
	\item Có bao nhiêu khả năng có thể xảy ra về kết quả đội vô địch và á quân?
\end{enumerate}
	\loigiai
	{
	\begin{enumerate}
		\item Chọn 2 đội tạo thành 1 cặp đấu trong 6 đội: $\mathrm{C}^2_6$ cách.\\
		Chọn 2 đội tạo thành 1 cặp đấu trong 4 đội: $\mathrm{C}^2_4$ cách.\\
		Chọn 2 đội tạo thành 1 cặp đấu trong 2 đội: $\mathrm{C}^2_2$ cách.\\
		Vậy có $\mathrm{C}^2_6 \cdot \mathrm{C}^2_4 \cdot \mathrm{C}^2_2=90$ cách sắp xếp các cặp đấu cho 6 đội còn lại.
		\item Có $\mathrm{A}_8^2=56$ khả năng có thể xảy ra về kết quả đội vô địch và á quân.
	\end{enumerate}	
	}
\end{bt}

%---Câu 8
\begin{bt}%[0T9K3-6]%[Dự án đề kiểm tra HKII NH22-23-Thy Nguyen Vo Diem]%[THPT Trần Quang Khải]
	Có một công viên nhỏ hình tam giác như hình 1. Người ta dự định đặt một cây đèn để chiếu sáng toàn bộ công viên. Để công việc tiến hành thuận lợi, người ta đo đạc và mô phỏng các kích thước công viên và thiết lập một hệ trục $O x y$ như hình 2, khi đó các đỉnh của công viên có tọa độ lần lượt là $A(0;3)$, $B(4;0)$, $C(4,7)$. Gọi $I$ là điểm đặt cây đèn sao cho đèn chiếu sáng toàn bộ công viên. Dùng kiến thức đã học, em hãy xác định vị trí chính xác của cây đèn.
	\begin{center}
		\begin{tikzpicture}[line join = round, line cap = round,>=stealth,font=\footnotesize,scale=1]
			\foreach \x/\y/\diem in {0/3/A,4/0/B,4/7/C} \coordinate (\diem) at (\x,\y);
			\draw[thick] (A)--(B)--(C)--cycle;
			\def \kc{0.5}
			\foreach \x in {0,1,2,3} \draw ({-\kc-0.1},\x)--++(0.2,0);
			\foreach \x in {0,1,2,3,4} \draw (\x,{-\kc-0.1})--++(0,0.2);
			\foreach \x in {0,...,7} \draw ({4+\kc-0.1},\x)--++(0.2,0);
			\draw (-\kc,0)--++(0,3) node[midway,left]{$3$ m};
			\draw (0,-\kc)--++(4,0) node[midway,below]{$4$ m};
			\draw ({4+\kc},0)--++(0,7) node[midway,right]{$7$ m};
		\end{tikzpicture}
		\qquad
		\begin{tikzpicture}[line join = round, line cap = round,>=stealth,font=\footnotesize,scale=1]
			\foreach \x/\y/\diem in {0/3/A,4/0/B,4/7/C} \coordinate (\diem) at (\x,\y);
			\def\xt{-1}\def\xp{5}\def\yd{-1}\def\yt{8}
			\draw[->] (\xt,0)--(\xp,0) node[below]{$x$};
			\draw[->] (0,\yd)--(0,\yt) node[left]{$y$};
			\fill (0,0) circle (1.5pt) node[below left]{$O$};
			\draw[thick] (A)--(B)--(C)--cycle;
			\foreach \x in {1,...,4} \draw[thin] (\x,1pt)--(\x,-1pt) node [below] {$\x$};
			\foreach \y in {1,...,7} \draw[thin] (1pt,\y)--(-1pt,\y) node [left] {$\y$};
			\foreach \diem/\goc in {A/0,B/45,C/0} \fill[black](\diem) circle (1pt) ($(\diem)+(\goc:3mm)$) node{$\diem$};
		\end{tikzpicture}
	\end{center}
	\loigiai
	{
		Từ giả thiết, ta có $I$ là tâm đường tròn ngoại tiếp tam giác $ABC$.\\
		Giả sử phương trình đường tròn ngoại tiếp tam giác $ABC$ là
		\[(C)\colon x^2+y^2-2ax-2by+c=0. \]
		với $I(a;b)$ là tâm đường tròn và $a^2+b^2-c>0$.\\
		Ta có $\heva{& A\in (C) \\ & B\in (C) \\ & C\in (C)}\Leftrightarrow \heva{& 0^2+3^2-2a\cdot 0-2b\cdot 3+c=0 \\ & 4^2+0^2-2a\cdot 4-2b\cdot 0+c=0 \\ & 4^2+7^2-2a\cdot 4-2b\cdot 7+c=0}\Leftrightarrow\heva{& a=\dfrac{7}{2} \\ & b=\dfrac{7}{2} \\ & c=12.}$\\
		Vậy $I\left(\dfrac{7}{2};\dfrac{7}{2}\right)$.
	}
\end{bt}

