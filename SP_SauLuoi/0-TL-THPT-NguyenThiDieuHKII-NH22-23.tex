\de{ĐỀ THI HỌC KỲ II NĂM HỌC 2022-2023}{THPT Nguyễn Thị Diệu}



%%%%%%%%%%%=======Câu 1
\begin{bt}%%[0D7B2-1]%[0D7B3-2]%[Dự án đề kiểm tra HKII NH22-23- Nguyễn Mộng Hùng]%[Nguyễn Thị Diệu]
	\begin{enumerate}
		\item Giải bất phương trình $x(x+4)>3(2x+1)$.
		\item Giải phương trình $\sqrt{-2x^2+x+7}=x-3$.
	\end{enumerate}
\loigiai{
	\begin{enumerate}
		\item Ta có
			\allowdisplaybreaks
			\begin{eqnarray*}
				&&x(x+4)>3(2x+1)\\
				&\Leftrightarrow&x^2+4x-6x-3>0\\
				&\Leftrightarrow&x^2-2x-3>0\\
				&\Leftrightarrow&\hoac{&x<-1\\&x>3.}
			\end{eqnarray*}
		 Vậy $S=(-\infty;-1)\cup(3;+\infty)$ là tập nghiệm của bất phương trình.
		\item Ta có
		\allowdisplaybreaks
		\begin{eqnarray*}
			&&\sqrt{-2x^2+x+7}=x-3\\
			&\Leftrightarrow&\heva{&x-3\ge 0\\&-2x^2+x+7=(x-3)^2}\\
			&\Leftrightarrow&\heva{&x\ge 3\\&3x^2-7x+2=0}\\
			&\Leftrightarrow&\heva{&x\ge 3\\&\hoac{&x=2\\&x=\dfrac{1}{3}}}\\
			&\Leftrightarrow&x\in\varnothing.
		\end{eqnarray*}
	Vậy phương trình vô nghiệm.
	\end{enumerate}
}
\end{bt}
%%%%%%%%%%%=======Câu 2
\begin{bt}%%[0D7K2-3]%[Dự án đề kiểm tra HKII NH22-23- Nguyễn Mộng Hùng]%[Nguyễn Thị Diệu]
	Tìm tất cả các giá trị của tham số $m$ để bất phương trình $x^2-2(m-1)x+4m-7>0$ đúng $\forall x\in\mathbb{R}$.
	\loigiai{
	 Để bất phương trình $x^2-2(m-1)x+4m-7>0$ đúng $\forall x\in\mathbb{R}$ thì
	 \allowdisplaybreaks
	 \begin{eqnarray*}
	 	&&\Delta'<0\,(\text{vì}\,a=1>0)\\
	 	&\Leftrightarrow&(m-1)^2-4m+7<0\\
	 	&\Leftrightarrow&m^2 -6m+8<0\\
	 	&\Leftrightarrow& 2<m<4.
	 \end{eqnarray*}
}
\end{bt}
%%%%%%%%%%%=======Câu 3
\begin{bt}%[0D8K2-2]%[Dự án đề kiểm tra HKII NH22-23- Nguyễn Mộng Hùng]%[Nguyễn Thị Diệu]
	Từ các chữ số $0$; $2$; $3$; $5$; $6$; $8$ có thể lập được bao nhiêu số tự nhiên gồm sáu chữ số đôi một khác nhau, trong đó có hai chữ số $0$ và $5$ không đứng cạnh nhau.
	\loigiai{
	 Gọi số có sáu chữ số là $\overline{abcdef}$ và tập hợp $A=\{0;2;3;5;6;8\}$.\\
	 $\bullet$ Chọn $a\in A\setminus\{0\}$ có $5$ cách.\\
	 $\bullet$ Xếp $5$ số còn lại thuộc  $A\setminus\{a\}$ vào $5$ vị trí còn lại $b$, $c$, $d$, $e$, $f$ có $5!$ cách.\\
	 Do đó theo quy tắc nhân có $5\cdot 5!=600$ số có sáu chữ số lấy từ tập $A$.\\
	 Ta đi tìm số có sáu chữ số có hai chữ số $0$ và $5$ đứng cạnh nhau.\\
	 $\bullet$ Xem số $0$ và $5$ như một phần tử $C$ cùng với bốn số còn lại của tập $A$ thành tập hợp\break $B=\{C;2;3;6;8\}$ có năm phần tử.\\
	 $\bullet$ Số có sáu chữ số trong đó có hai chữ số $0$ và $5$ đứng cạnh nhau là số hoán vị của năm phần tử lấy từ tập $B$ trên trừ đi số có dạng $\overline{05cdef}$ là $5!\cdot 2!-4!=216$ số.\\
	 Vậy số có sáu chữ số đôi một khác nhau, trong đó có hai chữ số $0$ và $5$ không đứng cạnh nhau là $600-216=384$ số.
}
\end{bt}
%%%%%%%%%%%=======Câu 4
\begin{bt}%%[0D8B3-1]%[Dự án đề kiểm tra HKII NH22-23- Nguyễn Mộng Hùng]%[Nguyễn Thị Diệu]
	Khai triển và rút gọn biểu thức sau $T=x\left(2-3x\right)^4$
	\loigiai{
	Ta có 
	\allowdisplaybreaks
	\begin{eqnarray*}
		T&=&x\left(2-3x\right)^4\\
		 &=&x\left[2+(-3x)\right]^4\\
		 &=&x\left[\mathrm{C}_4^0\cdot 2^4+\mathrm{C}_4^1\cdot 2^3\cdot(-3x)+\mathrm{C}_4^2\cdot 2^2\cdot(-3x)^2+\mathrm{C}_4^3\cdot 2\cdot(-3x)^3+\mathrm{C}_4^4\cdot(-3x)^4\right]\\
		 &=&x\left[16-96x+216x^2-216x^3+81x^4\right]\\
		 &=&16x-96x^2+216x^3-216x^4+81x^5.
	\end{eqnarray*}

}
\end{bt}

\begin{bt}%[0T0B2-5]%[Dự án đề kiểm tra HKII-NH22-23-TinDatTran]%[THPT Nguyễn Thị Diệu]
\begin{enumerate}
\item Tổ $1$ có $7$ học sinh nam và $5$ học sinh nữ. Giáo viên chọn ngẫu nhiên $3$ học sinh để trực nhật đầu tuần. Tính xác suất trong $3$ học sinh được chọn có cả học sinh nam và học sinh nữ.
\item Cho tập hợp $A=\left\{0;1;2;3;4;5\right\}$. Gọi $S$ là tập hợp các số có $3$ chữ số đôi một khác nhau được lập thành từ các số của tập $A$. Chọn ngẫu nhiên một số từ $S$, tính xác suất để số được chọn có chữ số hàng đơn vị gấp đôi chữ số hàng trăm.
\end{enumerate}
\loigiai{
\begin{enumerate}
\item Ta có $n(\Omega)=\mathrm{C}_{12}^{3}$.\\
Gọi $A$ là biến cố ``Trong ba học sinh được chọn có cả học sinh nam và học sinh nữ''. Ta có hai trường hợp:
\begin{itemize}
\item \textbf{Trường hợp 1:} Trong ba học sinh được chọn có $1$ học sinh nam và $2$ học sinh nữ có $\mathrm{C}_7^1\cdot \mathrm{C}_5^2$ cách chọn.
\item \textbf{Trường hợp 2:} Trong ba học sinh được chọn có $2$ học sinh nam và $1$ học sinh nữ có $\mathrm{C}_7^2\cdot \mathrm{C}_5^1$ cách chọn.
\end{itemize}
Theo quy tắc cộng, ta có $n\left(A\right)=\mathrm{C}_7^1\cdot \mathrm{C}_5^2+\mathrm{C}_7^2\cdot \mathrm{C}_5^1$.\\
Vậy xác suất cần tìm là $\mathrm{P}(A)=\dfrac{n(A)}{n(\Omega)}=\dfrac{\mathrm{C}_7^1\cdot \mathrm{C}_5^2+\mathrm{C}_7^2\cdot \mathrm{C}_5^1}{\mathrm{C}_{12}^{3}}=\dfrac{35}{44}$.
\item Ta có $n(\Omega)=5\cdot 5\cdot 4=100$.\\
Gọi $B$ là biến cố ``Chọn được số có chữ số ở hàng đơn vị gấp đôi chữ số ở hàng trăm''.\\
Gọi $\overline{abc}$ là số cần lập. Vì $c=2a$ nên ta có hai cặp số thỏa là $\heva{&a=1\\&c=2}$ và $\heva{&a=2\\&c=4}$.
\begin{itemize}
\item Chọn $1$ trong hai cặp số trên để đặt vào vị trí chữ số $a$ và $c$ có $2$ cách.
\item Chọn $1$ chữ số để đặt vào vị trí $b$ khác với hai chữ số $a$ và $c$ vừa chọn có $4$ cách.
\end{itemize}
Theo quy tắc nhân, ta có $n(B)=2\cdot 4=8$.\\
Vậy xác suất cần tìm là $\mathrm{P}(B)=\dfrac{n(B)}{n(\Omega)}=\dfrac{8}{100}=\dfrac{2}{25}$.
\end{enumerate}
}
\end{bt}
\begin{bt}%[0T9B2-2]%[0T9B2-5]%[0T9B3-2]%[Dự án đề kiểm tra HKII-NH22-23-TinDatTran]%[THPT Nguyễn Thị Diệu]
Trong mặt phẳng với hệ trục tọa độ $Oxy$, cho tam giác $ABC$ có $A(1;4)$, $B(3;-1)$, $C(6;2)$.
\begin{enumerate}
\item Tìm tọa độ trọng tâm $G$ của tam giác $ABC$ và tính độ dài đoạn thẳng $OG$.
\item Viết phương trình đường thẳng $d$ đi qua $A$ và vuông góc với $BC$.
\item Viết phương trình đường thẳng $\Delta$ đi qua $A$ và cách đều hai điểm $B$, $C$.
\item Viết phương trình đường tròn $(C)$ ngoại tiếp tam giác $ABC$. Xác định tâm và bán kính của $(C)$.
\end{enumerate}
\loigiai{
\begin{enumerate}
\item $G$ là trọng tâm tam giác $ABC$ nên $\heva{&x_G=\dfrac{x_A+x_B+x_C}{3}=\dfrac{1+3+6}{3}=\dfrac{10}{3}\\&y_G=\dfrac{y_A+y_B+y_C}{3}=\dfrac{4+(-1)+2}{3}=\dfrac{5}{3}}\Rightarrow G\left(\dfrac{10}{3};\dfrac{5}{3}\right)$.\\
Ta có $\vec{OG}=\left(\dfrac{10}{3};\dfrac{5}{3}\right)\Rightarrow OG=\left|\vec{OG}\right|=\sqrt{\left(\dfrac{10}{3}\right)^2+\left(\dfrac{5}{3}\right)^2}=\dfrac{5\sqrt{5}}{3}$.
\item Ta có $\vec{BC}=(6-3;2-(-1))=(3;3)$.\\
Đường thẳng $d$ vuông góc với $BC$ và đi qua $A=(1;4)$ nên $d$ có một vectơ pháp tuyến là $\vec{n}_d=\vec{BC}=(3;3)$. Suy ra phương trình đường thẳng $d$ là
\[(d)\colon 3(x-1)+3(y-4)=0\Leftrightarrow (d)\colon 3x+3y-15=0\Leftrightarrow (d)\colon x+y-5=0.\]
\item Gọi $\vec{n}=(a;b)$ là vectơ pháp tuyến của đường thẳng $\Delta$ đi qua $A(1;4)$ và cách đều hai điểm $B$, $C$. Khi đó ta có
\[\Delta \colon a(x-1)+b(y-4)=0\Leftrightarrow \Delta\colon ax+by-a-4b=0.\]
Vì $\Delta$ cách đều hai điểm hai điểm $B(3;-1)$ và $C(6;2)$ nên ta có
\begin{eqnarray*}
&\mathrm{d}\left(B,\Delta\right)=\mathrm{d}\left(C,\Delta\right)\Leftrightarrow\dfrac{\left|3\cdot a+(-1)\cdot b-a-4b\right|}{\sqrt{a^2+b^2}}=\dfrac{\left|6\cdot a+2\cdot b-a-4b\right|}{\sqrt{a^2+b^2}}\\
&\Leftrightarrow\left|2a-5b\right|=\left|5a-2b\right|\Leftrightarrow\hoac{&2a-5b=5a-2b\\&2a-5b=-5a+2b}\Leftrightarrow\hoac{&a=-b\\&a=b}.
\end{eqnarray*}
Chọn $b=1$, ta được $\heva{&a=-1\\&b=1}$ hoặc $a=b=1$.\\
Vậy phương trình đưởng thẳng $\Delta$ là $\Delta\colon -x+y-3=0$ hay $\Delta\colon x+y-5=0$.
\item Phương trình đường tròn $(C)$ có dạng
\[(C)\colon x^2+y^2-2ax-2by+c=0\text{ (với $a^2+b^2-c>0$)}\]

Ta có
\begin{eqnarray*}
	&\heva{&A\in (C)\\&B\in (C)\\&C\in (C)}\Leftrightarrow \heva{&1^2+4^2-2a\cdot 1-2b\cdot 4+c=0\\&3^2+(-1)^2-2a\cdot 3-2b\cdot (-1)+c=0\\&6^2+2^2-2a\cdot 6-2b\cdot 2+c=0}\\
	&\Leftrightarrow \heva{&-2a-8b+c=-17\\&-6a+2b+c=-10\\&-12a-4b+c=-40}\Leftrightarrow \heva{&a=\dfrac{43}{14}\\&b=\dfrac{27}{14}\\&c=\dfrac{32}{7}.}
\end{eqnarray*}
Suy ra phương trình đường tròn $(C)$ là $(C)\colon x^2+y^2-\dfrac{43}{7}x-\dfrac{27}{7}y+\dfrac{32}{7}=0$ có tâm $I\left(\dfrac{43}{14};\dfrac{27}{14}\right)$ và bán kính $R=\sqrt{\left(\dfrac{43}{14}\right)^2+\left(\dfrac{27}{14}\right)^2-\dfrac{32}{7}}=\dfrac{29\sqrt{2}}{14}$.
\end{enumerate}
}
\end{bt}
