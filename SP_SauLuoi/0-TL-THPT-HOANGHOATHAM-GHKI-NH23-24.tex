\de{ĐỀ THI GIỮA HỌC KỲ I NĂM HỌC 2023-2024}{THPT Hoàng Hoa Thám}
\begin{center}
	\textbf{PHẦN 1 - TRẮC NGHIỆM}
\end{center}
\Opensolutionfile{ans}[ans/ans]
%\setcounter{ex}{6}
\begin{ex}%%[0D2N1-1]%[Dự án đề kiểm tra Toán 10 GHKI NH23-24- Trương Đăng Khoa]%[THPT Hoàng Hoa Thám - Tp HCM]
	Bất phương trình nào sau đây tương đương với bất phương trình \break $3x-y>7(x-4y)+1$?
	\choice
	{\True $4x-27y< -1$}{$4x-27y+1\leq 0$}{$4x-27y+1\geq 0$}{$4x-27y+1>0$}
	\loigiai{
		Ta có \begin{align*}
			&\ 3x-y>7(x-4y)+1\\
			\Leftrightarrow&\ 3x-y-7x+28y-1>0\\
			\Leftrightarrow&\ -4x+27y-1>0\\
			\Leftrightarrow&\ 4x-27y<-1.
		\end{align*}
	}
\end{ex}

\begin{ex}%[0D3H1-2]%[Dự án đề kiểm tra Toán 10 GHKI NH23-24- Trương Đăng Khoa]%[THPT Hoàng Hoa Thám - Tp HCM]
	Tập xác định của hàm số $y=\sqrt{6-3x}$ là.
	\choice
	{	\True $\mathscr{D}=(-\infty;2]$}
	{$\mathscr{D}=[2;+\infty)$}
	{$\mathscr{D}=(-\infty;2)$}
	{$\mathscr{D}=\mathbb{R}\setminus\{-2\}$}
	\loigiai{
		Hàm số xác định khi và chỉ khi $$6-3x\geq 0\Leftrightarrow x\leq 2.$$
		Vậy tập xác định của hàm số là $\mathscr{D}=(-\infty;2]$.
	}
\end{ex}

\begin{ex}%[0D3H1-2]%[Dự án đề kiểm tra Toán 10 GHKI NH23-24- Trương Đăng Khoa]%[THPT Hoàng Hoa Thám - Tp HCM]
	Tập xác định của hàm số $f(x)=\dfrac{x+5}{x-1}+\dfrac{x-1}{x+5}$ là.
	\choice
	{$\mathscr{D}=\mathbb{R}\setminus \{1\}$}{\True $\mathscr{D}=\mathbb{R}\setminus\{-5;1\}$}
	{$\mathscr{D}=\mathbb{R}\setminus\{-5\}$}
	{$\mathscr{D}=\mathbb{R}$
	}
	\loigiai{
		Hàm số xác định khi và chỉ khi 
		$$\heva{&x-1\neq0\\ &x+5\neq 0}\Leftrightarrow\heva{&x\neq 1\\ &x\neq -5.}$$
		Vậy tập xác định của hàm số là $\mathscr{D}=\mathbb{R}\setminus\{-5;1\}$.
	}
\end{ex}

\begin{ex}%[0D1H1-3]%[Dự án đề kiểm tra Toán 10 GHKI NH23-24- Trương Đăng Khoa]%[THPT Hoàng Hoa Thám - Tp HCM]
	Mệnh đề phủ định $\overline{P}$ của mệnh đề $P:$\lq\lq$\forall x\in\mathbb{N},x^2-2=0$\rq\rq\, là.
	\choice
	{	\True $\overline{P}:$\lq\lq$\exists x\in\mathbb{N}, x^2-2\neq 0$\rq\rq}
	{$\overline{P}:$\lq\lq$\forall x\in\mathbb{N}, x^2-2\geq 0$\rq\rq}
	{$\overline{P}:$\lq\lq$\forall x\in\mathbb{N}, x^2-2> 0$\rq\rq	}
	{	$\overline{P}:$\lq\lq$\exists x\in\mathbb{N}, x^2-2< 0$\rq\rq	}
	\loigiai{
		Mệnh đề phủ định là $\overline{P}:$\lq\lq$\exists x\in\mathbb{N}, x^2-2\neq 0$\rq\rq.
	}
\end{ex}

\begin{ex}%[0D2N1-1]%[Dự án đề kiểm tra Toán 10 GHKI NH23-24- Trương Đăng Khoa]%[THPT Hoàng Hoa Thám - Tp HCM]
	Cặp số nào sau đây là nghiệm của bất phương trình $-2(x-y)+y>3$?
	\choice{\True $(-4;4)$}{$(4;-4)$}{$(-1;-2)$}{$(2;1)$}
	\loigiai{
		Thay $(-4;4)$ vào bất phương trình đã cho ta được
		$$-2(-4-4)+4>3\Leftrightarrow 20>3\, (\text{đúng}).$$
	}
\end{ex}

\begin{ex}%[0H5H1-2]%[Dự án đề kiểm tra Toán 10 GHKI NH23-24- Trương Đăng Khoa]%[THPT Hoàng Hoa Thám - Tp HCM]
	Cho tam giác $ABC$. Gọi $M$, $N$ lần lượt là trung điểm của $AB$, $AC$. Cặp vectơ nào sau đây cùng hướng?
	\choice
	{$\vv{AN}$ và $\vv{CA}$}
	{\True $\vv{AB}$ và $\vv{MB}$}
	{$\vv{MN}$ và $\vv{CB}$}
	{$\vv{MA}$ và $\vv{MB}$}
	\loigiai{
		\begin{center}
			\begin{tikzpicture}[line join=round, line cap=round,thick]
				\coordinate (A) at (-1,1);
				\coordinate (B) at (-2,0);
				\coordinate (C) at (1,0);
				\coordinate (M) at ($(A)!0.5!(B)$);
				\coordinate (N) at ($(A)!0.5!(C)$);
				\draw(A)--(B)--(C)--cycle (M)--(N);
				\foreach \i/\g in {A/90,B/-90,C/-90, M/170, N/20}{\draw[fill=white](\i) circle (1.5pt) ($(\i)+(\g:3mm)$) node[scale=1]{$\i$};}
			\end{tikzpicture}
		\end{center}
		Ta có $\vv{AB}$ và $\vv{MB}$ cùng hướng với nhau.
	}
\end{ex}
%Câu 7
\begin{ex}%[0D1B3-2]%[Dự án đề kiểm tra Toán 10 GHKI NH23-24- Bùi Quang Phú]%[THPT Hoàng Hoa Thám - Tp HCM]
	Cho hai tập hợp $A=\left\{2;4;6;9\right\}$ và $B=\left\{0;1;2;3;4\right\}$. Tìm tập hợp $A \setminus B$.
	\choice
	{$A \setminus B =\varnothing$}
	{$A \setminus B =\left\{0;1;3\right\}$}
	{$A \setminus B =\left\{6;9;1;3\right\}$}
	{\True $A \setminus B =\left\{6;9\right\}$}
	\loigiai{
		$A \setminus B$ là tập các phần tử thuộc $A$ và không thuộc $B$. Do đó $A \setminus B =\left\{6;9\right\}$.
	}
\end{ex}


%Câu 8
\begin{ex}%[0H4K3-2] %[Dự án đề kiểm tra Toán 10 GHKI NH23-24- Bùi Quang Phú]%[THPT Hoàng Hoa Thám - Tp HCM]
	Giả sử $CD=h$ là chiều cao của tháp, trong đó $C$ là chân tháp (\textit{tham khảo hình vẽ}). Chọn hai điểm $A$, $B$ trên mặt đất sao cho $A,B,C$ thẳng hàng. Ta đo được $AB=24~(\mathrm{m})$, $\widehat{CAD}=63^\circ$, $\widehat{CBD}=48^\circ$. Chiều cao $h$ của tháp \textbf{gần nhất} với giá trị nào sau đây?
	\begin{center}
		\begin{tikzpicture}[scale=0.7, font=\footnotesize, line join = round, line cap = round,>=stealth]
			\tkzDefPoints{0/7.7/D,0/0/C,4.5/0/A,7.5/0/B,9/0/x,-3/0/x'}
			\tkzDrawPoints[fill=black](A,B,C,D)
			\tkzDrawPolygon(A,B,D)
			\tkzDrawArc[R,fill=black](D,0.1cm)(0,360)
			\fill plot  coordinates{(0,0)(-1,0) (-0.63,0.77) (-0.4,0.77) (-0.4,5.05)(-0.63,5.05) (-0.63,5.53) (-0.45,5.53)(-0.22,6.24)(0,7.73)};
			\draw plot  coordinates{(0,0)(1,0) (0.63,0.77) (0.4,0.77) (0.4,5.05)(0.63,5.05) (0.63,5.53) (0.45,5.53)(0.22,6.24)(0,7.73)};
			\draw(0,5.05)--(0.45,5.05) (0,5.53)--(0.45,5.53)(0,0.77)--(0.4,0.77) (1.11,-0.21)--(3.17,-0.21);
			\draw plot[smooth]   coordinates{(1,0)(1.11,0)(1.11,-0.21)(1.3,-0.85) (1.96,-1.19) (2.75,-1.01) (3.17,-0.21)(3.7,0)(4.5,0)};
			\fill[pattern=north east lines,opacity=0.5] plot[smooth]   coordinates{(1.11,-0.21)(1.3,-0.85) (1.96,-1.19) (2.75,-1.01) (3.17,-0.21)};
			\tkzLabelPoints[above](D)  \tkzLabelPoints[below](C,A,B)
			\tkzDrawSegments(x,B x',C)
			\draw[dashed](0,7.7) --(-3,7.7) ; 
			\draw[<->,dashed](-2,0)--(-2,3.5)node[left] {$h$} --(-2,7.7) ; 
			\draw ($(A)!1/2!(B)$) node [below] {$24\mathrm{m}$}; 	 
			\tkzLabelAngles[left=-0.3cm,pos=1](D,A,C){$63^\circ$}
			\tkzLabelAngles[left=-0.3cm,pos=1](D,B,C){$48^\circ$}
			\draw pic[draw,angle radius=5.56mm] {angle = D--A--C};
			\draw pic[draw, angle radius=5.5mm, double]{angle=D--B--C};
		\end{tikzpicture}
	\end{center} 
	\choice
	{$18{,}5~(\mathrm{m})$}
	{$60~(\mathrm{m})$}
	{\True $61{,}4~(\mathrm{m})$}
	{$18~(\mathrm{m})$}
	\loigiai{
		$\widehat{CAD}=\widehat{ADB}+\widehat{CBD}$ (góc ngoài của $\triangle ADB$). Suy ra $\widehat{ADB}=63^\circ-48^\circ =15^\circ$.\\[0.15cm]
		Áp dụng định lí sin cho $\triangle ADB$ ta có $\dfrac{AD}{\sin \widehat{ABD}}=\dfrac{AB}{\sin \widehat{ADB}}$. Suy ra $AD=\dfrac{24 \sin 48^\circ}{\sin 15^\circ}~(\mathrm{m})$\\[0.15cm].
		Xét $\triangle CDA$ vuông tại $C$ có $\sin \widehat{CAD}=\dfrac{CD}{AD}$ (tỉ số lượng giác)\\[0.15cm]
		Suy ra $CD=AD \sin \widehat{CAD}=\dfrac{24 \sin 48^\circ}{\sin 15^\circ} \cdot \sin 63^\circ \approx 61{,}4$ (m).  
	}
\end{ex}

%Câu 9
\begin{ex}%[0H4K1-3]%[Dự án đề kiểm tra Toán 10 GHKI NH23-24- Bùi Quang Phú]%[THPT Hoàng Hoa Thám - Tp HCM]
	Cho $\tan \alpha=-2$. Tính giá trị của biểu thức $P=\dfrac{4\sin \alpha+3\cos \alpha}{\sin \alpha +\cos \alpha}$.
	\choice
	{\True $P=5$}
	{$P=-3$}
	{$P=3$}
	{$P=-5$}
	\loigiai{
		$P=\dfrac{4\sin \alpha+3\cos \alpha}{\sin \alpha +\cos \alpha}=\dfrac{4 \cdot \tfrac{\sin \alpha}{\cos \alpha}+3}{\tfrac{\sin \alpha}{\cos \alpha}+1}=\dfrac{4 \cdot (-2)+3}{-2+1}=5$.
	}
\end{ex}

%Câu 10
\begin{ex}%[0H4B3-1]%[Dự án đề kiểm tra Toán 10 GHKI NH23-24- Bùi Quang Phú]%[THPT Hoàng Hoa Thám - Tp HCM]
	Nếu tam giác $MNP$ có $MP=5$, $PN=8$, $\widehat{MPN}=120^\circ$ thì độ dài cạnh $MN$ bằng bao nhiêu? (\textit{kết quả làm tròn đến chữ số thập phân thứ nhất})
	\choice
	{\True $MN \approx 11{,}4$}
	{$MN \approx 12{,}0$}
	{$MN \approx 12{,}4$}
	{$MN \approx 7{,}0$}
	\loigiai{
		Áp dụng định lí cos cho $\triangle MNP$ ta có
		$$MN^2=PM^2+PN^2-2 \cdot PM \cdot PN \cdot \cos \widehat{MPN} \Rightarrow MN \approx 11{,}4$$.
	}
\end{ex}

%Câu 11
\begin{ex}%[0H4B1-3]%[Dự án đề kiểm tra Toán 10 GHKI NH23-24- Bùi Quang Phú]%[THPT Hoàng Hoa Thám - Tp HCM]
	Kết quả thu gọn của biểu thức $A=-\sin (180^\circ -x)+\cos (180^\circ-x)+\cos x$ là.
	\choice
	{$2\sin x$}
	{$\cos x$}
	{\True $-\sin x$}
	{$\sin x$}
	\loigiai{
		Ta có $\sin (180^\circ-x)=\sin x$ và $\cos (180^\circ-x)=-\cos x$\\[0.15cm]
		Do đó $A=-\sin x-\cos x+\cos x=-\sin x$.
	}
\end{ex}

%Câu 12
\begin{ex}%[0D1K3-2]%[Dự án đề kiểm tra Toán 10 GHKI NH23-24- Bùi Quang Phú]%[THPT Hoàng Hoa Thám - Tp HCM]
	Tập hợp $(-2;3) \setminus [1;5]$ bằng tập hợp nào sau đây?
	\choice
	{$(-2;5)$}
	{$(-2;1]$}
	{$(-3;-2)$}
	{\True $(-2;1)$}
	\loigiai{
		$(-2;3) \setminus [1;5]=(-2;1)$.
	}
\end{ex}

\Closesolutionfile{ans}
%\begin{center}
%	\textbf{ĐÁP ÁN}
%	\inputansbox{10}{ans/ans}	
%\end{center}



\begin{center}
	\textbf{PHẦN 2 - TỰ LUẬN}
\end{center}

%Câu 1...........................
\begin{bt}%[0D3Y1-2]%[Dự án đề kiểm tra Toán 10 GHKI NH23-24- Võ Xuân Cường Thịnh]%[THPT HOÀNG HOA THÁM - Tp HCM]
Tìm tập xác định của hàm số $y=\dfrac{2x-1}{(2x+1)(x-3)}$.
\loigiai{
ĐKXĐ: $\heva{&2x+1 \neq 0\\
 & x-3 \neq 0}  \Leftrightarrow \heva{&x\neq-\dfrac{1}{2} \\
 &x\neq3}$. \\
 Txđ: $\mathbb{D}=\mathbb{R}\setminus \left\lbrace \dfrac{1}{2} ; 3\right\rbrace $.
}
\end{bt}
\begin{bt}%[0D2Y1-2]%[Dự án đề kiểm tra Toán 10 GHKI NH23-24- Võ Xuân Cường Thịnh]%[THPT HOÀNG HOA THÁM - Tp HCM]
Xác định miền nghiệm của bất phương trình $3x-4y \leq 12$.
	\loigiai{
		Trong mặt phẳng tọa độ, vẽ đường thẳng $(d): 3x-4y=12$, ta có $(d)$ chia mặt phẳng thành hai nửa.\\

	\begin{center}
		\begin{tikzpicture}[scale=1,join=round,cap=round,font=\scriptsize]
		\fill [pattern=north west lines,pattern color=blue!50](-.5,-4)-- plot[domain=-.5:5](\x,{(12-3*\x)/-4})--(5,-4)--cycle;
		\draw[red,smooth,samples=100] plot[domain=-.5:5](\x,{(12-3*\x)/-4}) node [above right] {$d$};
	\draw[-stealth] (-.5,0)--(5,0) node[above]{$x$}; %Truc x
		\draw[-stealth] (0,-4)--(0,0)node[below right]{$O$}--(0,3) node[right]{$y$}; %Truc y
	\end{tikzpicture}
	\end{center}
	Chọn điểm $A(0,0)$ không thuộc vào $(d)$ ta được $0-0=0 \leq 12$ (đúng) nên điểm $A$ thuộc miền nghiệm của bất phương trình đã cho.\\
	Vậy miền nghiệm của bất phương trình là nửa mặt phẳng bờ $d$ chứa điểm $(0;0)$.


	}
\end{bt}
\begin{bt}%[0H1Y2-2]%[Dự án đề kiểm tra Toán 10 GHKI NH23-24- Võ Xuân Cường Thịnh]%[THPT HOÀNG HOA THÁM - Tp HCM]
	Cho tam giác ABC có AB = 4, AC = 5, và $\cos A = \dfrac{3}{5}$.\\
		\begin{enumerate}
			\item Tính độ dài cạnh BC
			\item Tính độ dài đường cao kẻ từ A của tam giác ABC 
		\end{enumerate}
	\loigiai{
	\begin{enumerate}
		\item 	Theo định lý cosin: 
		\begin{align*}
			BC^2 & = AB^2 + AC^2 - 2\cdot AB\cdot AC\cdot\cos A \\
			BC&  = \sqrt{4^2 + 5^2 - 2\cdot4\cdot5\cdot \dfrac{3}{5}} = \sqrt{17}.
			\end{align*} 
			\item  Ta có $\cos^2A + \sin^2A =1  \Rightarrow \sin A = \sqrt{1-\cos^2 A}=\dfrac{4}{5}$.\\
			$S_{\triangle ABC} = \dfrac{1}{2}AB.AC.\sin A = \dfrac{1}{2}\cdot4\cdot5\cdot\dfrac{4}{5} = 8$.\\
			$S_{\triangle ABC} = \dfrac{1}{2}.h.BC   \Rightarrow h = \dfrac{16}{\sqrt{17}} $.
			
	\end{enumerate}
		}
\end{bt}
