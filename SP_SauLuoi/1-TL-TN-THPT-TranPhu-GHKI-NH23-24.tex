
\de{ĐỀ THI GIỮA HỌC KỲ I NĂM HỌC 2023-2024}{THPT TRẦN PHÚ }
\begin{center}
	\textbf{PHẦN 1 - TRẮC NGHIỆM}
\end{center}
\Opensolutionfile{ans}[ans/ansTranPhu]
\setcounter{ex}{10}\setcounter{bt}{0}
\begin{ex}%[1D1N3-5]%[Dự án đề kiểm tra Toán 10-11 HKI NH23-24- Nguyễn Sĩ Đạt]%[THPT Trần Phú]
	Khẳng định nào sau đây \textbf{sai}?
	\choice
	{\True$\dfrac{1}{\sin^2 \alpha} = 1+ \cos^2 \alpha, \alpha \ne k \pi,(k \in \mathbb{Z})$}
	{$\tan \alpha \cdot \cot \alpha =1, \alpha \ne k\dfrac{\pi}{2}, (k \in \mathbb{Z})$}
	{$\sin^2 \alpha + \cos^2 \alpha =1, \forall \alpha \in \mathbb{R}$}
	{$\dfrac{1}{\cos^2 \alpha}=1+\tan^2 \alpha, \alpha \ne \dfrac{\pi}{2}+k \pi, (k \in \mathbb{Z})$}
	\loigiai{
		Ta có $\dfrac{1}{\sin^2 \alpha} = 1+ \cot^2 \alpha, \alpha \ne k \pi,(k \in \mathbb{Z})$.}
\end{ex}

\begin{ex}%[1D1N5-2]%[Dự án đề kiểm tra Toán 10-11 HKI NH23-24- Nguyễn Sĩ Đạt]%[THPT Trần Phú]
	Phương trình nào sau đây vô nghiệm?
	\choice
	{$3\sin{x}-2=0$}
	{$3\cot{x}-5=0$}
	{\True $\cos{x}=\dfrac{\pi}{3}$}
	{$\tan{x}-1=0$}
	\loigiai{
		Do $-1 \le \sin{x}$, $\cos{x} \le 1$ mà $\dfrac{\pi}{3}\ge 1$ nên phương trình $\cos{x}=\dfrac{\pi}{3}$ vô nghiệm.
	}
\end{ex}

\begin{ex}%[1D2H1-5]%[Dự án đề kiểm tra Toán 10-11 HKI NH23-24- Nguyễn Sĩ Đạt]%[THPT Trần Phú] 
	Cho dãy số $(u_n)$ với $u_n=1+\dfrac{1}{n}$. Khẳng định nào sau đây đúng?
	\choice
	{Dãy số $(u_n)$ tăng và chỉ bị chặn trên}
	{\True Dãy số $(u_n)$ giảm và bị chặn}
	{Dãy số $(u_n)$ giảm và chỉ bị chặn dưới}
	{Dãy số $(u_n)$ tăng và bị chặn}
	\loigiai{
		Ta có $u_{n+1} - u_n = 1+\dfrac{1}{n+1} - 1 - \dfrac{1}{n} = -\dfrac{1}{n(n+1)} <0 , \forall n \in \mathbb{N^*}$. \\ 
		Suy ra $u_{n+1} < u_n , \forall n \in \mathbb{N^*}$. Suy ra $(u_n)$ là dãy số giảm.\\
		Dễ thấy $1<u_n \le 2, \forall n \in \mathbb{N^*}$. Do đó dãy số $(u_n)$ bị chặn.\\
		Vậy dãy số $(u_n)$ giảm và bị chặn.
	}
\end{ex}

\begin{ex}%[1D1N2-2]%[Dự án đề kiểm tra Toán 10-11 HKI NH23-24- Nguyễn Sĩ Đạt]%[THPT Trần Phú] 
	Khẳng định nào sau đây đúng với mọi $\alpha \in (0;\pi)$?
	\choice
	{$\tan \alpha>0$}
	{$\cos \alpha>0$}
	{$\cot \alpha>0$}
	{\True $\sin \alpha>0$}
	\loigiai{
		Với mọi $\alpha \in (0;\pi)$, ta có $\sin \alpha>0$.
	}
\end{ex}

\begin{ex}%[1D1N5-4]%[Dự án đề kiểm tra Toán 10-11 HKI NH23-24- Nguyễn Sĩ Đạt]%[THPT Trần Phú]
	Khẳng định nào sau đây đúng?
	\choice
	{\True $\cos(x-30^\circ)=1 \Leftrightarrow x=30^\circ+k360^\circ (k \in \mathbb{Z})$}
	{$\cos(x-30^\circ)=1 \Leftrightarrow x=120^\circ+k360^\circ (k \in \mathbb{Z})$}
	{$\cos(x-30^\circ)=1 \Leftrightarrow x=\dfrac{\pi}{6}+k2\pi (k \in \mathbb{Z})$}
	{$\cos(x-30^\circ)=1 \Leftrightarrow x=120^\circ+k2\pi (k \in \mathbb{Z})$}
	\loigiai{ 
		$\cos(x-30^\circ)=1 \Leftrightarrow \cos(x-30^\circ)=\cos{0^\circ} \Leftrightarrow x-30^\circ=k360^\circ \Leftrightarrow x=30^\circ+k360^\circ (k \in \mathbb{Z})$.
	}
\end{ex}

\begin{ex}%[1D2H2-4]%[Dự án đề kiểm tra Toán 10-11 HKI NH23-24- Nguyễn Sĩ Đạt]%[THPT Trần Phú]
	Cho dãy số $(u_n)$ là một cấp số cộng, biết $u_9=25$ và $u_5=13$. Khẳng định nào sau đây đúng?
	\choice
	{$u_1=\dfrac{17}{5}$}
	{\True Công sai $d=3$}
	{$u_1=-2$}
	{Công sai $d=\dfrac{12}{5}$}
	\loigiai{
		Ta có: $u_9=u_1+8d=25$; $u_5=u_1=4d=13$.\\
		Từ đó, ta được $\heva{& u_9=25 \\ &   u_5=13 }$ $\Leftrightarrow \heva{& u_1+8d=25 \\ &   u_1+4d=13 } \Leftrightarrow \heva{& u_1=1 \\ &  d=3 .}$
	}		
\end{ex}

\begin{ex}%[1D1H2-2]%[Dự án đề kiểm tra Toán 10-11 HKI NH23-24- Nguyễn Sĩ Đạt]%[THPT Trần Phú]
	Cho $\sin{\alpha}=\dfrac{1}{3}$, khi đó giá trị của $\cos{2\alpha}$ bằng
	\choice
	{\True $\cos{2\alpha}=\dfrac{7}{9}$}
	{$\cos{2\alpha}=\dfrac{4\sqrt{2}}{9}$}
	{$\cos{2\alpha}=\dfrac{2\sqrt{2}}{3}$}
	{$\cos{2\alpha}=-\dfrac{2\sqrt{2}}{3}$}
	\loigiai{
		Ta có $\cos{2\alpha}=1-2\sin^2{\alpha}=1-2\cdot\left(\dfrac{1}{3}\right)^2=\dfrac{7}{9}$.}
\end{ex}

\begin{ex}%[1D2H2-2]%[Dự án đề kiểm tra Toán 10-11 HKI NH23-24- Nguyễn Sĩ Đạt]%[THPT Trần Phú]
	Trong các dãy số sau, dãy số nào không là cấp số cộng?
	\choice
	{$-5$; $-10$; $-15$; $-20$; $-25$}
	{$-3$; $2$; $7$; $12$; $17$}
	{\True $1$; $-4$; $-8$; $-13$; $-18$}
	{$5$; $8$; $11$; $14$; $17$}
	\loigiai{
		Xét dãy số $1$; $-4$; $-8$; $-13$; $-18$, ta thấy $u_2-u_1=-5$ và $u_3-u_2=-4$.\\
		Do đó $u_2-u_1 \ne u_3-u_2 $.\\
		Vậy dãy số $1$; $-4$; $-8$; $-13$; $-18$ không là cấp số cộng.
	}
\end{ex}

\begin{ex}%[1D1N2-2]%[Dự án đề kiểm tra Toán 10-11 HKI NH23-24- Nguyễn Sĩ Đạt]%[THPT Trần Phú]
	Cho hàm số $y = \sin 2023x$. Khẳng định nào sau đây là đúng?
	\choice
	{Giá trị nhỏ nhất của hàm số bằng $0$}
	{Giá trị nhỏ nhất của hàm số bằng $-2023$}
	{Giá trị lớn nhất của hàm số bằng $2023$}
	{\True Giá trị lớn nhất của hàm số bằng $1$}
	\loigiai{
		Ta có $-1 \leq \sin(2023x) \leq 1$. \\
		Nên giá trị lớn nhất của $y= \sin (2023x)$ bằng $1$.
	}
\end{ex}

\begin{ex}%[1D1N3-4]%[Dự án đề kiểm tra Toán 10-11 HKI NH23-24- Nguyễn Sĩ Đạt]%[THPT Trần Phú]
	Khẳng định nào sau đây đúng với mọi số thực $\alpha$ và $\beta$?
	\choice
	{\True $2\cos \alpha\cos \beta=\cos(\alpha+\beta)+\cos (\alpha-\beta)$}
	{$2\cos \alpha\cos \beta=\cos(\alpha-\beta)-\cos (\alpha+\beta)$}
	{$2\cos \alpha\cos \beta=\sin(\alpha+\beta)+\sin (\alpha-\beta)$}
	{$2\cos \alpha\cos \beta=\cos(\alpha+\beta)-\cos (\alpha-\beta)$}
	\loigiai{
		Ta có $\cos \alpha\cos \beta=\dfrac{1}{2}\left[ \cos(\alpha+\beta)+\cos (\alpha-\beta)\right] $ nên $2\cos \alpha\cos \beta=\cos(\alpha+\beta)+\cos (\alpha-\beta)$.
	}
\end{ex}
\begin{ex}%[Đề GHKI- THPT -Trần Phú- HCM]%[Nguyễn Tài Tuệ]%[1D1H2-4]
	Khẳng định nào sau đây \textbf{sai}?
	\choice
	{\True $\cos (\alpha)+\cos (-\alpha)=0, \forall \alpha \in \mathbb{R}$}
	{$\cos (\pi-\alpha)=-\cos \alpha, \forall a \in \mathbb{R}$}
	{$\sin (3 \pi-\alpha)=\sin \alpha, \forall \alpha \in \mathbb{R}$}
	{$\sin \left(\dfrac{\pi}{2}-\alpha\right)=\cos \alpha, \forall \alpha \in \mathbb{R}$}
	\loigiai{
	Theo giá trị lượng giác của hai góc đối nhau ta có \\
	   $\cos (-\alpha)=\cos\alpha \Rightarrow $ $\cos (\alpha)+\cos (-\alpha)=0, \forall \alpha \in \mathbb{R}$ là khẳng định sai. 
	}
\end{ex}

\begin{ex}%[Đề GHKI- THPT -Trần Phú- HCM]%[Nguyễn Tài Tuệ]%[1D2N1-3]
	Cho dãy số $\left(u_{n}\right)$ với $u_{n}=\dfrac{n}{3^{n}-1}$. Khi đó $u_2+u_3$ bằng 
	\choice
	{$1$}
	{$5$}
	{$\dfrac{23}{56}$}
	{\True $\dfrac{19}{52}$}
	\loigiai{
		Ta có $u_2=\dfrac{1}{4}$, $u_3=\dfrac{3}{26}$ nên $u_2+u_3=\dfrac{19}{52}$.
	}
\end{ex}

\begin{ex}%[Đề GHKI- THPT -Trần Phú- HCM]%[Nguyễn Tài Tuệ]%[1D2H2-5]
	Cho tam giác $ABC$ vuông, biết số đo ba góc $A, B, C$ tạo thành một cấp số cộng. Số đo của ba góc $A, B, C$ là
	\choice
	{$50^{\circ}; 70^{\circ}; 90^{\circ}$}
	{\True $30^{\circ}; 60^{\circ}; 90^{\circ}$}
	{$60^{\circ}; 90^{\circ}; 120^{\circ}$}
	{$45^{\circ}; 45^{\circ}; 90^{\circ}$}
	\loigiai{
		Giả sử với số đo ba góc $A, B, C$ tạo thành một cấp số cộng với công sai dương. Tam giác $ABC$ vuông nên $C=90^\circ$.\\
		Theo giả thiết $A+C=2B$ mà $A+B+C=180^\circ \Rightarrow 3B=180^\circ\Rightarrow B=60^\circ$.\\
		Do đó $A=180^\circ-90^\circ-60^\circ =30^\circ$.
		
	}
\end{ex}

\begin{ex}%[Đề GHKI- THPT -Trần Phú- HCM]%[Nguyễn Tài Tuệ]%[1D1N3-3]
	Biết $\sin \alpha=\dfrac{3}{5}$ và $\cos \alpha=\dfrac{-4}{5}$, khi đó giá trị $\sin 2 \alpha$ bằng  
	\choice
	{$\sin 2 \alpha=\dfrac{-8}{5}$}
	{$\sin 2 \alpha=\dfrac{6}{5}$}
	{\True $\sin 2 \alpha=\dfrac{-24}{25}$}
	{$\sin 2 \alpha=\dfrac{-12}{5}$}
	\loigiai{
		Ta có $\sin 2\alpha =2\sin \alpha\cos\alpha =2\cdot \dfrac{3}{5}\cdot \dfrac{-4}{5}=-\dfrac{24}{25}$.
	}
\end{ex}

\begin{ex}%[Đề GHKI- THPT -Trần Phú- HCM]%[Nguyễn Tài Tuệ]%[1D1N1-3]
\immini{Biết góc $\widehat{MON}=60^{\circ}$, góc lượng giác $(OM, ON)$ như hình vẽ có số đo bằng  
	\choice
	{$-420^{\circ}$}
	{$420^{\circ}$}
	{\True $780^{\circ}$}
	{$-780^{\circ}$}
}{	\begin{tikzpicture}[>=stealth,line join=round,line cap=round,font=\footnotesize,scale=1]
	\def\r{2.5}
	\path 
	(0,0) coordinate (O)
	(\r,0) coordinate (M)
	(60:\r) coordinate (N)
	;
	\pic[draw,angle radius=2mm,angle eccentricity=1.5] {angle = M--O--N};
	\draw (M)--(O)--(N);
	\draw[->,domain=0:780,variable=\t,samples=200,>=latex]
	plot ({0.6*(\t+2*780)*cos(\t)/(2*780)},
	{0.6*(\t+ 2*780)*sin(\t)/(2*780)})
	;
	\fill[black] (O) circle (1pt);
	\foreach \l/\g in {M/-90,N/180,O/-135}
	\draw[fill=black] (\l) +(\g:.3) node{$\l$};
\end{tikzpicture}}
	\loigiai{
		Góc lượng giác $(OM, ON)$ như hình vẽ có số đo bằng  $60^\circ +360^\circ +360^\circ =780^\circ$.
	}
\end{ex}

\begin{ex}%[Đề GHKI- THPT -Trần Phú- HCM]%[Nguyễn Tài Tuệ]%[1D1H5-3]
	Khằng định nào sau đây \textbf{sai}?
	\choice
	{$\cos x=-1 \Leftrightarrow x=\pi+k 2 \pi,(k \in \mathbb{Z})$}
	{\True $\sin x=1 \Leftrightarrow x=\dfrac{\pi}{2}+k \pi,(k \in \mathbb{Z})$}
	{$\sin x=-1 \Leftrightarrow x=\dfrac{3 \pi}{2}+k 2 \pi,(k \in \mathbb{Z})$}
	{$\cos x=0 \Leftrightarrow x=\dfrac{\pi}{2}+k \pi,(k \in \mathbb{Z})$}
	\loigiai{
		Theo phương trình lượng giác cơ bản ta có $\sin x=1 \Leftrightarrow x=\dfrac{\pi}{2}+k 2\pi,(k \in \mathbb{Z})$.
	}
\end{ex}
\begin{ex}%[1D2H2-3]
dây số $\left(u_{n}\right)$ là một cấp số cộng có số hạng đầu tiên $u_1=1$ và công sai $d=0$. Khi đó số hạng $u_{2023}$ bằng
\choice
{\True $1$}
{$2022$}
{$2024$}
{$2023$}
\loigiai{
	Ta có $u_{2023}=u_1+2022\cdot d=1$.
}
\end{ex}

\begin{ex}%[Đề GHKI- THPT -Trần Phú- HCM]%[Nguyễn Tài Tuệ]%[1D1N3-3]
	Khẳng định nào sau đây đúng với mọi số thực $\alpha$?
	\choice
	{$\cos 2 \alpha=2 \cos ^2(\alpha-1)$}
	{$\cos 2 \alpha=\cos ^2 \alpha+\sin ^2 \alpha$}
	{$\cos 2 \alpha=1+2 \sin ^2 \alpha$}
	{\True $\cos 2 \alpha=2 \cos ^2 \alpha-1$}
	\loigiai{
		Theo công thức nhân đôi ta có $\cos 2\alpha =\cos^2\alpha -\sin \alpha =2\cos^2 \alpha -1=1-2\sin^2\alpha $.
	}
	\end{ex}
\begin{ex}%[Đề GHKI- THPT -Trần Phú- HCM]%[Nguyễn Tài Tuệ]%[1D1H5-3]
Số nghiệm của phương trình  $\tan x-1=0$ trên khoảng $\left(0 ; \dfrac{3 \pi}{2}\right)$ là 
		\choice
		{$0$}
		{\True $2$}
		{$3$}
		{$1$}
		\loigiai{
			Ta có $\tan x=1\Leftrightarrow x=\dfrac{\pi}{4}+k\pi $.\\
			Vì $x\in \left(0 ; \dfrac{3 \pi}{2}\right)\Rightarrow x=\dfrac{\pi}{4}, x=\dfrac{5\pi}{4}$.
		}
	\end{ex}
	
	\begin{ex}%[Đề GHKI- THPT -Trần Phú- HCM]%[Nguyễn Tài Tuệ]%[1D1H4-2]
		Hàm số nào sau đây có tập xác định là tập số thực $\mathbb{R}$?
		\choice
		{$y=\tan x$}
		{\True $y=\dfrac{\cos 2 x}{\cos x-2}$}
		{$y=\dfrac{\sin 2 x}{\cos x}$}
		{$y=\cot x$}
		\loigiai{
			Với mọi $x\in\mathbb{R}$, ta có $\cos x-2\ne0$ do đó $y=\dfrac{\cos 2 x}{\cos x-2}$ xác định trên tập số thực $\mathbb{R}$.
		}
	\end{ex}
\Closesolutionfile{ans}
%\begin{center}
%	\textbf{ĐÁP ÁN}
%	\inputansbox{10}{ans/ans}	
%\end{center}
\begin{center}
	\textbf{PHẦN 2 - TỰ LUẬN}
\end{center}
\begin{bt}%[Đề GHKI- THPT -Trần Phú- HCM]%[Lê Văn Toàn]%[1D1V5-6]
	Số giờ có ánh sáng mặt trời của một thành phố A trong ngày thứ $t$ của một năm không nhuận được cho bởi hàm số $d(t)=2 \sin \left[\dfrac{\pi(t-107)}{186}\right]+12$ với $t \in \mathbb{N}^*$ và $t \leq 365$. Khi đó ngày tháng nào trong năm thì thành phố A có 11 giờ ánh sáng?
	\loigiai{
		Thành phố A có đúng $11$ giờ có ánh sáng mặt trời thì $d(t)=11$.
		Khi đó
		$$\begin{aligned}
			& 11=2 \sin \left[\dfrac{\pi}{186}(t-107)\right]+12 \\
			& \Leftrightarrow \sin \left[\dfrac{\pi}{186}(t-107)\right]=-\dfrac{1}{2} \\
			& \Leftrightarrow \sin \left[\dfrac{\pi}{186}(t-107)\right]=\sin \left(-\dfrac{\pi}{6}\right) \\
			& \Leftrightarrow \hoac{&\dfrac{\pi}{186}(t-107)=-\dfrac{\pi}{6}+k 2 \pi\\ & \dfrac{\pi}{186}(t-107)=\pi + \dfrac{\pi}{6}+k 2 \pi} \\
			& \Leftrightarrow \hoac{&t=76+372k \\ &t = 262+372k} k\in \mathbb{Z}.
		\end{aligned}$$
		Mà $0<t \leq 365$ nên
		$\hoac{&0<76+372k \leq 365\\& 0<262+372k \leq 365} \Leftrightarrow \hoac{&-0{,}2<k \leq 0{,}98\\&-0{,}7<k \leq 0{,}27.}$\\
		Mà $k \in \mathbb{Z}$ nên suy ra $k=0$ khi đó $t=76+372 \cdot 0=76$ và $t=262+372 \cdot 0=262$.\\
		Vậy Thành phố A có đúng $11$ giờ có ánh sáng mặt trời vào ngày thứ $76$ và $262$ trong năm.
	}
\end{bt}

\begin{bt}%[Đề GHKI- THPT -Trần Phú- HCM]%[Lê Văn Toàn]%[1D2H2-1]
	Cho dãy số $(u_n)$ là một cấp số cộng và thỏa mãn điều kiện $\heva{&u_3+u_5=36\\&u_6+u_8=66}$. Tìm số hạng đầu $u_1$, công sai $d$ và tính tổng của $2024$ số hạng đầu tiên của cấp số cộng $\left(u_n\right)$.
	
	\loigiai{	
		Ta có $\heva{&u_3+u_5=36\\&u_6+u_8=66} \Leftrightarrow \heva{&2u_1+6d=36\\&2u_1+12d=66}\Leftrightarrow \heva{&u_1=3\\&d=5.}$\\
		Do đó $S_{2024}=\dfrac{\left(2\cdot3+2023\cdot5\right)\cdot2024}{2}=122452$.
	}	
\end{bt}
\begin{bt}%[Đề GHKI- THPT -Trần Phú- HCM]%[Lê Văn Toàn]%[1H4V3-4]
	Cho hình chóp $S.ABCD$ có đáy $ABCD$ là hình bình hành tâm $O$. Gọi $M$ là trung điểm cạnh $SB$ và $N$ là điểm thuộc cạnh $SC$ sao cho $NS=2NC$.
	\begin{enumerate}[a)]
		\item Tìm giao tuyến của hai mặt phẳng $(SAC)$  và $(SBD)$.
		\item Chứng minh $OM$ song song với mặt phẳng $\left(SAD\right)$.
		\item Tìm giao điểm $K$ của đường thẳng $ND$ và mặt phẳng $\left(AMC\right)$.
	\end{enumerate}
	\loigiai{
		\immini{
			\begin{enumerate}[a)]
				\item Tìm giao tuyến $\left(SAC\right)$ và $\left(SBD\right)$.\\
				Ta có $\heva{S \in \left(SAC\right)\\S \in \left(SBD\right)} \Rightarrow S$ là điểm chung thứ nhất.\\
				Gọi $O = AC \cap BD$.\\
				Ta có $\heva{O \in AC \subset \left(SAC\right)\\ O \in BD \subset \left(SBD\right)} \Rightarrow O$ là điểm chung thứ hai.\\
				Vậy $\left(SAC\right) \cap \left(SBD\right) = SO$.
				\item Chứng minh $OM \parallel \left(SAD\right)$.\\
				Ta có $OM$ là đường trung bình của tam giác $BSD \Rightarrow OM \parallel SD$.\\
				Mặt khác $SD \subset \left(SAD\right)$.\\
				Do đó $OM \parallel \left(SAD\right)$.
				\item Tìm giao điểm $K = ND \cap \left(AMC\right)$.\\
				Chọn mặt phẳng phụ $\left(DBN\right) \supset DN$.\\
				Tìm giao tuyến của $\left(DBN\right)$ và $\left(AMC\right)$.\\
				Ta có O là điểm chung thứ nhất của $\left(DBN\right)$ và $\left(AMC\right)$.\\
				Gọi $H = CM \cap BN$.\\
				$\heva{H \in CM \subset \left(ACM\right)\\ H \in BN \subset \left(DBN\right)} \Rightarrow H$ là điểm chung thứ hai.\\
				Do đó $\left(DBN\right) \cap \left(AMC\right) = OH$.\\
				Gọi $K = OH \cap DN$.
				Suy ra $K = ND \cap \left(AMC\right)$. 
			\end{enumerate}
		}{
			\begin{tikzpicture}[line join=round, line cap=round,thick, scale=1]
				\coordinate (A) at (1,1);
				\coordinate (B) at (6,1);
				\coordinate (D) at (0,0);
				\coordinate (S) at (2,5);
				\coordinate (C) at ($(B)+(D)-(A)$);
				\coordinate (M) at ($(S)!1/2!(B)$);
				\coordinate (N) at ($(S)!2/3!(C)$);
				
				\draw(B)--(C)--(D);
				\draw(S)--(D) (S)--(C) (S)--(B) (D)--(N)--(B);
				\draw[dashed] (S)--(A) (D)--(A)--(B) (A)--(C) (B)--(D) (A)--(M);
				
				\coordinate (O) at (intersection of A--C and B--D);
				\coordinate (H) at (intersection of B--N and C--M);
				\coordinate (K) at (intersection of D--N and O--H);
				\coordinate (L) at (intersection of C--M and N--K);
				\draw (N)--(K)--(H)--(C) (L)--(M);
				\draw[dashed] (O)--(H)--(L);
				\foreach \i/\g in {A/170,B/90,C/-90,D/-90,S/90,M/80,O/-70,K/90}{\draw[fill=black](\i) circle (1.5pt) ($(\i)+(\g:3mm)$) node[scale=1]{$\i$};}
				\foreach \i/\g in {N/170,H/-45}{\draw[fill=black](\i) circle (1.5pt) ($(\i)+(\g:4mm)$) node[scale=1]{$\i$};}
			\end{tikzpicture}
		}
		
	}
\end{bt}

