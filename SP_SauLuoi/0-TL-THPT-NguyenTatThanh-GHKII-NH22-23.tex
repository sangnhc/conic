
\de{ĐỀ THI GIỮA HỌC KỲ II NĂM HỌC 2022-2023}{THPT Nguyễn Tất Thành}


\begin{bt}%[0T7Y2-1]%[Dự án Đề Kiểm tra GHKII NH22-23-TinDatTran]%[THPT Nguyễn Tất Thành]
Giải bất phương trình $3x^2-2x-8>0$.
\loigiai{
Tam thức bậc hai $f(x)=3x^2-2x-8$ có hai nghiệm phân biệt là $x_1=\dfrac{-4}{3}$; $x_2=2$.\\
Vì $a=3>0$ nên $f(x)$ dương trên các khoảng $\left(-\infty;\dfrac{-4}{3}\right)$, $\left(2;+\infty\right)$.\\
Vậy bất phương trình $3x^2-2x-8>0$ có tập nghiệm là $\left(-\infty;\dfrac{-4}{3}\right)\cup\left(2;+\infty\right)$.
}
\end{bt}
\begin{bt}%[0T7Y3-2]%[Dự án Đề Kiểm tra GHKII NH22-23-TinDatTran]%[THPT Nguyễn Tất Thành]
Giải các phương trình sau
\begin{listEX}[2]
	\item $\sqrt{3x^2-4x+5}=\sqrt{5x^2+3x-17}$;
	\item $\sqrt{4+3x-x^2}=5x-12$.
\end{listEX}
\loigiai{
\begin{listEX}
	\item Bình phương hai vế của phương trình đã cho, ta được	
	\allowdisplaybreaks
	\begin{eqnarray*}
	&&3x^2-4x+5=5x^2+3x-17\\
	&\Rightarrow&2x^2+7x-22=0\\
	&\Rightarrow&x=2\text{~hoặc~}x=\dfrac{-11}{2}.
	\end{eqnarray*}
	Thay lần lượt các giá trị trên vào phương trình đã cho, ta thấy cả hai giá trị $x=2$ và $x=\dfrac{-11}{2}$ đều thỏa mãn.\\
	Vậy tập nghiệm của phương trình là $S=\left\{\dfrac{-11}{2};2\right\}$.
	\item Bình phương hai vế của phương trình đã cho, ta được
	\allowdisplaybreaks
	\begin{eqnarray*}
	&&4+3x-x^2=\left(5x-12\right)^2\\
	&\Rightarrow&4+3x-x^2=25x^2-120x+144\\
	&\Rightarrow&26x^2-123x+140=0\\
	&\Rightarrow &x=\dfrac{123+\sqrt{569}}{52}\text{~hoặc~}x=\dfrac{123-\sqrt{569}}{52}.
	\end{eqnarray*}
	Thay lần lượt các giá trị trên vào phương trình đã cho, ta thấy chỉ có $x=\dfrac{123-\sqrt{569}}{52}$ thỏa mãn.\\
	Vậy tập nghiệm của phương trình đã cho là $S=\left\{\dfrac{123-\sqrt{569}}{52}\right\}$.
\end{listEX}
}
\end{bt}
\begin{bt}%[0T9Y1-1]%[0T9B2-2]%[0T9K2-6]%[Dự án Đề Kiểm tra GHKII NH22-23-TinDatTran]%[THPT Nguyễn Tất Thành]
Trong mặt phẳng $Oxy$, cho tam giác $ABC$ có $A(3;0)$; $B(0;4)$; $C(1;3)$, đường cao $BH$, trung tuyến $AM$.
\begin{enumerate}
\item Tính chu vi của tam giác $ABC$.
\item Viết phương trình tham số của đường thẳng $AB$; phương trình tổng quát của đường thẳng $BH$ và phương trình tổng quát của đường thẳng $AM$.
\item Tìm tọa độ điểm $K$ trên đường thẳng $BC$ sao cho khoảng cách $AK$ ngắn nhất.
\end{enumerate}
\loigiai{
\begin{enumerate}
	\item Ta có $\vec{AB}=\left(-3;4\right)$, $\vec{AC}=\left(-2;3\right)$, $\vec{BC}=\left(-1;-1\right)$.\\
	Chu vi tam giác $ABC$ là 
	\begin{eqnarray*}
		AB+AC+BC&=&\left|\vec{AB}\right|+\left|\vec{AC}\right|+\left|\vec{BC}\right|\\
	&=&\sqrt{(-3)^2+4^2}+\sqrt{(-2)^2+3^2}+\sqrt{(-1)^2+(-1)^2}\\
	&=&5+\sqrt{13}+\sqrt{2}.
	\end{eqnarray*}
	\item Đường thẳng $AB$ đi qua $A(3;0)$ và có véc-tơ chỉ phương $\vec{AB}=(-3;4)$ có phương trình tham số
	\[\heva{&x=3-3t\\&y=4t}(t\in \mathbb{R}).\]
	Đường thẳng $BH$ đi qua $B(0;4)$ và có véc-tơ pháp tuyến $\vec{AC}=(-2;3)$ có phương trình tổng quát
	\[-2x+3(y-4)=0 \Leftrightarrow -2x+3y-12=0.\]
	Do $M$ là trung điểm của $BC$ nên $M\left(\dfrac{1}{2};\dfrac{7}{2}\right)$.\\
	Đường thẳng $AM$ đi qua $A(3;0)$ và có véc-tơ chỉ phương $\vec{AM}=\left(\dfrac{-5}{2};\dfrac{7}{2}\right) \Rightarrow $ véc-tơ pháp tuyến $\vec{n}_1=(7;5)$ có phương trình tổng quát
	\[7(x-3)+5y=0 \Leftrightarrow 7x+5y-21=0.\]
	\item Để khoảng cách $AK$ ngắn nhất thì $K$ là chân đường vuông góc kẻ từ $A$ đến $BC$.\\
	Đường thẳng $BC$ đi qua $B(0;4)$ và có véc-tơ chỉ phương $\vec{BC}=(-1;-1)$ có phương trình tham số
	\[\heva{&x=-t\\&y=4-t}(t\in \mathbb{R}).\]
	Gọi $K(-t;4-t)\in(BC)$. Ta có $\vec{AK}=(-t-3;4-t)$.\\
	Vì $AK\perp BC$ nên $\vec{AK}\cdot \vec{BC}=0\Leftrightarrow -(-t-3)-(4-t)=0\Leftrightarrow 2t+1=0\Leftrightarrow t=\dfrac{-1}{2}$.\\
	Vậy $K\left(\dfrac{1}{2};\dfrac{9}{2}\right)$.
\end{enumerate}
}
\end{bt}


\begin{bt}%[0T8B1-1]%[0T8K1-2]%[Dự án đề kiểm tra GHKII NH22-23- Võ Thị Thùy Trang]%[Nguyễn Tất Thành]
	\begin{enumerate}
		\item Một bộ cờ vua có $16$ quân cờ trắng và $16$ quân cờ đen; mỗi bên có $1$ quân vua, $1$ quân hậu, $2$ quân xe, $2$ quân mã, $2$ quân tượng và $8$ quân tốt. Bạn Nam lấy ra tất cả quân cờ trắng, tất cả quân xe và tất cả quân tốt. Hãy đếm số quân cờ Nam đã lấy ra.
		\item Có bao nhiêu số tự nhiên có ba chữ số khác nhau thuộc khoảng $(100;547)$ ?
	\end{enumerate}
\loigiai{
	\begin{enumerate}
		\item 
		\begin{itemize}
			\item Có $16$ quân cờ trắng.
			\item Có $2$ quân xe màu đen.
			\item Có $8$ quân tốt màu đen.
			\item Do đó có $26$ quân cờ Nam đã lấy ra.
		\end{itemize}
		\item Gọi số tự nhiên có ba chữ số khác nhau thuộc khoảng $(100;547)$ là $\overline{abc}$.
		\begin{itemize}
			\item Trường hợp 1
				\begin{itemize}
				\item Số $a\in \{1;2;3;4\}$ nên $a$ có $4$ cách chọn.
				\item Số $b\in \{0;1;2;3;4;5;6;7;8;9\}\setminus \{a\}$ nên $b$ có  $9$ cách chọn.
				\item Số $c\in \{0;1;2;3;4;5;6;7;8;9\}\setminus \{a,b\}$ nên $b$ có  $8$ cách chọn.
				\item Trường hợp này có $4\cdot 9\cdot 8=288$ số.
			\end{itemize} 
			\item Trường hợp 2
				\begin{itemize}
				\item Số $a\in \{5\}$ nên $a$ có $1$ cách chọn.
				\item Số $b\in \{0;1;2;3\}$ nên $b$ có  $4$ cách chọn.
				\item Số $c\in \{0;1;2;3;4;5;6;7;8;9\}\setminus \{a,b\}$ nên $b$ có  $8$ cách chọn.
				\item Trường hợp này có $1\cdot 4\cdot 8=32$ số.
			\end{itemize} 
			\item Trường hợp 3
				\begin{itemize}
				\item Số $a\in \{5\}$ nên $a$ có $1$ cách chọn.
				\item Số $b\in \{4\}\setminus \{a\}$ nên $b$ có  $1$ cách chọn.
				\item Số $c\in \{0;1;2;3;4;5;6\}\setminus \{a,b\}$ nên $4$ có  $8$ cách chọn.
				\item Trường hợp này có $4$ số.
			\end{itemize} 
		\end{itemize}	
		Do đó có $288+32+4=324$ số.
	\end{enumerate}
}
\end{bt}
\begin{bt}%[0T7K2-1]%[Dự án đề kiểm tra GHKII NH22-23- Võ Thị Thùy Trang]%[Nguyễn Tất Thành]
	Một viên đá nhỏ rơi từ độ cao $320$ m xuống đất, biết độ cao của viên đá so với mặt đất được tính theo công thức $h=320+20t-5t^2$ (trong đó $h$ tính bằng mét, $t$ tính bằng giây). Sau bao lâu kể từ lúc bắt đầu rơi, viên đá còn cách mặt đất không quá $100$ m?
	\loigiai{Ta có $320+20 t-5 t^2\le 100 \Leftrightarrow 5t^2-20t-220\ge 0 \Leftrightarrow t\ge 2+4\sqrt{3}$ do $t\ge 0$.\\
		Do đó sau $8{,}93$ giây kể từ lúc bắt đầu rơi, viên đá còn cách mặt đất không quá $100$ m.
	}
\end{bt}


