
\de{ĐỀ THI GIỮA HỌC KỲ I NĂM HỌC 2023-2024}{THPT Võ THỊ SÁU }

\begin{bt}%[1D1N2-2]%[Dự án đề kiểm tra Toán 11 GHKI NH23-24- Đạt Lê]%[THPT Võ Thị Sáu - Tp HCM]
	Cho $\sin x=- \dfrac{\sqrt{3}}{4}$ và  $x \in\left(\pi ; \dfrac{3 \pi}{2}\right)$. Tính giá trị của $\cos x$, $\tan x$.
	\loigiai{
	Ta có  $\sin ^2 x+\cos^ 2 x=1 \Rightarrow \cos^2x=1- \sin^2 x=\dfrac{13}{16}\Rightarrow \cos x=\pm\dfrac{\sqrt{13}}{4}$.\\
	Do $x \in\left(\pi ; \dfrac{3 \pi}{2}\right)$ nên $\cos x<0$, do đó $\cos x=-\dfrac{\sqrt{13}}{4}.$\\
	Từ đó suy ra $\tan x=\dfrac{\sin x}{\cos x} = \dfrac{\sqrt{39}}{13}. $
}
\end{bt}

\begin{bt}%[1D1N2-2]%[Dự án đề kiểm tra Toán 11 GHKI NH23-24- Đạt Lê]%[THPT Võ Thị Sáu - Tp HCM]
	Cho $\tan x=\sqrt{2}$. Tính $\cos 2 x$.
	\loigiai{
		Ta có $\dfrac{1}{\cos^2 x}=1+\tan^2 x=1+(\sqrt{2})^2=3 \Rightarrow \cos^2 x =\dfrac{1}{3}$\\
		Khi đó $ \cos2 x=2 \cos ^2 x -1 = 2\cdot\dfrac{1}{3}-1=-\dfrac{1}{3}$.
	}
\end{bt}

\begin{bt}%[1D1H3-2]%[Dự án đề kiểm tra Toán 11 GHKI NH23-24- Đạt Lê]%[THPT Võ Thị Sáu - Tp HCM]
Rút gọn biểu thức $T=\dfrac{(\sin 4 x-\sin 2 x) \cdot \cos 5 x}{2 \cos ^2 4 x+\cos 2 x-1}$
\loigiai{
\begin{align*}
	T&=\dfrac{2\cos 3x\cdot \sin x\cdot \cos 5 x}{\cos 8 x+\cos 2 x}\\
	&=\dfrac{2\cos 3x\cdot \sin x\cdot \cos 5 x}{2\cos 5x\cdot\cos 3 x}\\
	&=\sin x.
\end{align*}
}
\end{bt}

\begin{bt}%[1D1H5-3]%[Dự án đề kiểm tra Toán 11 GHKI NH23-24- Võ Thị Thùy Trang]%[THPT Võ Thị Sáu - Tp HCM]
	Giải phương trình sau $\cos 2 x+\cos \dfrac{3 \pi}{8}=0$
	\loigiai{
		\allowdisplaybreaks
		\begin{eqnarray*}
			\cos 2 x+\cos \dfrac{3 \pi}{8}=0 &\Leftrightarrow &\cos 2 x=\cos \dfrac{5 \pi}{8}\\ 
			&\Leftrightarrow &
			\hoac{&2x=\dfrac{5 \pi}{8}+k2\pi\\&2x=-\dfrac{5 \pi}{8}+k2\pi}\\
			&\Leftrightarrow & \hoac{&x=\dfrac{5 \pi}{16}+k\pi\\&x=-\dfrac{5 \pi}{16}+k\pi} (k\in \mathbb{Z}).
		\end{eqnarray*}
		Vậy nghiệm của phương trình là $x=\dfrac{5 \pi}{16}+k\pi$ $(k\in \mathbb{Z}) $ và $x=-\dfrac{5 \pi}{16}+k\pi$ $(k\in\mathbb{Z}) $.
	}
\end{bt}

\begin{bt}%[1D2H2-4]%[Dự án đề kiểm tra Toán 11 GHKI NH23-24- Võ Thị Thùy Trang]%[THPT Võ Thị Sáu - Tp HCM]
	Tìm số hạng đầu tiên và công sai của cấp số cộng $\left(u_n\right)$ biết $\heva{&u_1+2 u_5=0 \\& S_4=14.}$
	\loigiai{
		\allowdisplaybreaks
		\begin{eqnarray*}
			\heva{&u_1+2 u_5=0 \\& S_4=14} &\Leftrightarrow & \heva{&u_1+2(u_1+4d)=0\\& 2(2u_1+3d)=14}\\
			&\Leftrightarrow & \heva{&3u_1+8d=0\\&2u_1+3d=7}\\
			&\Leftrightarrow & \heva{&u_1=8\\&d=-3.}
		\end{eqnarray*}
	}
\end{bt}

\begin{bt}%[1D2V2-7]%[Dự án đề kiểm tra Toán 11 GHKI NH23-24- Võ Thị Thùy Trang]%[THPT Võ Thị Sáu - Tp HCM]
	Một công ty $A$ tuyển nhân viên bằng hai phương án khi kí hợp đồng lao động.
	\begin{itemize}
		\item Phương án $1\colon $ Năm đầu tiên nhận mức lương $100$ triệu đồng, mỗi năm tiếp theo tăng thêm $12$ triệu đồng.
		\item Phương án $2\colon$ Quý đầu tiên nhận mức lương $15$ triệu đồng, mỗi quý tiếp theo tăng thêm $2{,}5$ triệu đồng.
	\end{itemize}
	Một người lao động quyết định kí hợp đồng lao động với công ty $A$ trong $10$ năm. Để tổng tiền lương nhận được trong $10$ năm tốt hơn, người lao động nên chọn phương án nào?
	\loigiai{
		\begin{itemize}
			\item Tổng tiền lương theo phương án $1$ là $S_{10}=5(2\cdot 100+9\cdot 12)=1540$ (triệu đồng).
			\item Tổng tiền lương theo phương án $2$ là $S_{40}=20(2\cdot 15+39\cdot 2{,}5 )=2550$ (triệu đồng).
		\end{itemize}
		Để tổng tiền lương nhận được trong $10$ năm tốt hơn, người lao động nên chọn phương án $2$.	
	}
\end{bt}
\begin{bt}%[1H4N1-3]%[1H4H2-2]%[1H4V2-4]%[Dự án đề kiểm tra Toán 11 GHKI NH23-24- Dương Phước Sang]%[THPT Võ Thị Sáu - Tp HCM]
	Cho hình chóp $S.ABCD$ có đáy $ABCD$ là hình bình hành tâm $O$, $G$ là trọng tâm tam giác $SCD$, $H$ là trọng tâm tam giác $ACD$.
	\begin{listEX}
		\item Tìm giao tuyến của hai mặt phẳng $(SAC)$ và $(SBD)$.
		\item Chứng minh hai đường thẳng $HG$ và $SA$ song song.
		\item Tìm giao điểm của đường thẳng $OG$ và $(SAD)$.
	\end{listEX}
	\loigiai{
		\begin{listEX}
			\item Ta có $O=AC \cap BD \Rightarrow \heva{&O \in AC,\,AC \subset (SAC)\\&O \in BD,\,BD \subset (SBD).}$
			\immini{
				$$\Rightarrow O \in (SAC) \cap (SBD).$$
				Và $S \in (SAC) \cap (SBD)$ nên 
				$SO=(SAC) \cap (SBD)$.
				\item Gọi $I$ là trung điểm cạnh $CD$.\\
				Do $G,H$ lần lượt là trọng tâm $\triangle SCD$ và $\triangle ACD$ nên
				$$\dfrac{IH}{IA}=\dfrac{1}{3}=\dfrac{IG}{IS} \Rightarrow HG \parallel SA.$$
				\item Xét 2 mặt phẳng $(OGD)$ và $(SAD)$ có $D$ là 1 điểm chung và lần lượt chứa 2 đường thẳng song song $GH$ và $SA$, do đó $(OGD)$ và $(SAD)$ cắt nhau theo giao tuyến là đường thẳng $d$ qua $D$ và song song với $SA$.
			}
			{\begin{tikzpicture}[scale=1, font=\footnotesize, line join=round, line cap=round]
					\foreach \x\y\t in {0/0/A,-1.7/-1.6/B,2.5/-1.6/C,-0.1/2.8/S}
					\coordinate (\t) at (\x,\y);
					\coordinate (D) at ($(A)+(C)-(B)$);
					\coordinate (O) at ($(A)!1/2!(C)$);
					\coordinate (I) at ($(D)!1/2!(C)$);
					\coordinate (H) at ($(A)!2/3!(I)$);
					\coordinate (G) at ($(S)!2/3!(I)$);
					\coordinate (K) at ($(S)+(D)-(A)$);
					\coordinate (x) at (intersection of S--D and G--K);
					\draw (I)--(S)--(B)--(C)--(S)--(x) (G)--(D)--(C) (D)--(K)--(G);
					\draw[dashed](B)--(A)--(D)--(B) (H)--(G)--(O)--(S)--(A)--(C) (A)--(I)  (x)--(D);
					\foreach \t/\g in {S/90,A/160,B/-140,C/-45,D/0,O/-95,I/-20,H/-90,G/90,K/80}
					\draw[fill=black] (\t) circle(1pt)
					node[shift={(\g:7pt)}]{$\t$};
			\end{tikzpicture}}\noindent
			Trong $(OGD)$, gọi $K=d \cap OG$ ta có $\heva{&K \in OG\\&K\in d,\,d \subset (SAD)}$ nên $K=OG \cap (SAD)$.
		\end{listEX}
	}
\end{bt}