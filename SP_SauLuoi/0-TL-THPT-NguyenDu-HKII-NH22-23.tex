\de{ĐỀ THI HỌC KỲ II NĂM HỌC 2022-2023}{THPT Nguyễn Du}


\begin{bt}%[0T7B2-1]%[Dự án đề kiểm tra HKII NH22-23- Tên Ngô Quang Anh]%[Nguyễn Du]
	Giải bất phương trình  $(2x-1)(x+1)\geq x^2-5 x-6$.
	\loigiai
	{ Ta có
			$$(2x-1)(x+1)\geq x^2-5 x-6\Leftrightarrow x^2+6x+5\geq 0\Leftrightarrow x\le -5 \vee x\geq -1.$$
			Vậy tập nghiệm của bất phương trình đã cho là $S=\left( -\infty ;-5 \right]\cup \left[ -1;+\infty  \right)$.
	}
\end{bt}
\begin{bt}%[0T8Y2-1]%[Dự án đề kiểm tra HKII NH22-23- Tên Ngô Quang Anh]%[Nguyễn Du]
	Lớp $10A$ có $35$ học sinh, trong đó có $18$ học sinh nữ. Hỏi có bao nhiêu cách chọn $4$ học sinh làm ban cán sự lớp gồm $ 2 $ nam và $ 2 $ nữ? 
	\loigiai{
		Số cách chọn $4$ học sinh làm ban cán sự lớp gồm $ 2 $ nam và $ 2 $ nữ là $\mathrm{C}_{17}^2\cdot\mathrm{C}_{18}^2=20808$ cách.}
\end{bt}
\begin{bt}%[0T9B2-2]%[Dự án đề kiểm tra HKII NH22-23- Tên Ngô Quang Anh]%[Nguyễn Du]
	Trong mặt phẳng tọa độ $Oxy$, cho hai điểm $A(-1; 3)$ và $B(9; 5)$. Viết phương trình đường trung trực của đoạn thẳng $AB$.
	\loigiai{Ta có $ \vec{AB}(10; 2)\parallel (5;1)$. Gọi $ I $ là trung điểm của đoạn thẳng $AB\Rightarrow I(4; 4)$. Khi đó phương trình đường trung trực của đoạn thẳng $AB$ đi qua $I$ và nhận $ \vec{n}(5;1)$ làm véc-tơ pháp tuyến có dạng $ 5(x-4)+1(y-4)=0 \Leftrightarrow 5x+y-24=0$.
			}
\end{bt}
\begin{bt}%[0T0B2-2]%[Dự án đề kiểm tra HKII NH22-23- Tên Ngô Quang Anh]%[Nguyễn Du]
Trên giá sách có $ 8 $ quyển sách Toán khác nhau; $ 5 $ quyển sách Văn khác nhau và $ 10 $ quyển Anh khác nhau, chọn ngẫu nhiên $ 3 $ quyển sách. Tính xác suất để chọn được $ 3 $ quyển thuộc ba môn học khác nhau.
	\loigiai{
	Số phần tử của không gian mẫu là $n\left( \Omega \right)=\mathrm{C}_{23}^3=1771$.\\
	Gọi $A$ là biến cố  \lq\lq chọn được $ 3 $ quyển thuộc ba môn học khác nhau\rq\rq.\\
	Suy ra $n(A)=\mathrm{C}_{8}^1\cdot\mathrm{C}_{5}^1\cdot\mathrm{C}_{10}^1=400$.\\
	Vậy xác suất cần tìm là $\mathrm{P}(A)=\dfrac{n(A)}{n\left( \Omega \right)}=\dfrac{400}{1771}$.
	}
\end{bt}
\begin{bt}%[0T8Y3-1]%[Dự án đề kiểm tra HKII NH22-23- Tên Ngô Quang Anh]%[Nguyễn Du]
	Sử dụng nhị thức Newton, khai triển biểu thức $(x+3)^4$ và tìm hệ số của số hạng chứa $x^3$ trong khai triển đó.
	\loigiai{
		\allowdisplaybreaks
		\begin{eqnarray*}
			(x+3)^4&=&\mathrm{C}_4^0x^4+\mathrm{C}_4^1x^3\cdot3^1+\mathrm{C}_4^2x^2\cdot3^2+\mathrm{C}_4^3x\cdot3^3+\mathrm{C}_4^4 3^4\\
			&=&x^{4}+12x^3+54x^2+108x+81.
		\end{eqnarray*}
		Do đó hệ số của $x^3$ trong khai triển trên là $12$.
	}
\end{bt}

\begin{bt}%[0T9B4-5]%[Dự án đề kiểm tra HKII NH22-23- Phạm Duy Phương]%[THPT Nguyễn Du - Sở TP Hồ Chí Minh]
	Viết phương trình chính tắc của hypebol $(H)$, biết $(H)$ cắt trục hoành tại điểm $A(4;0)$ và có một tiêu điểm $F_1(-5;0)$.
\loigiai{
	Gọi $(H)$ có phương trình chính tắc dạng $\dfrac{x^2}{a^2}-\dfrac{y^2}{b^2}=1$.
	\begin{itemize}
		\item $(H)$ cắt trục hoành tại điểm $A(4;0)$ suy ra $a=4$.
		\item $(H)$ có một tiêu điểm $F_1(-5;0)$ suy ra $c=5$.
	\end{itemize}
	$\Rightarrow b^2=c^2-a^2=5^2-4^2=9 \Rightarrow b=3$.\\
	Với $a=4$, $b=3$ $\Rightarrow (H)\colon \dfrac{x^2}{16}-\dfrac{y^2}{9}=1$.
	}
\end{bt}
\begin{bt}%[0T0K2-2]%[Dự án đề kiểm tra HKII NH22-23- Phạm Duy Phương]%[THPT Nguyễn Du - Sở TP Hồ Chí Minh]
	Lớp $10A$ có $10$ học sinh nữ và $25$ học sinh nam. Giáo viên chủ nhiệm muốn lập một ban cán sự lớp gồm 6 học sinh. Tính xác suất để ban cán sự lớp được chọn có ít nhất $1$ học sinh nữ.
	\loigiai{
	Chọn $6$ học sinh bất kỳ từ $35$ học sinh có $\mathrm{C}_{35}^6$ cách hay $n(\Omega)=\mathrm{C}_{35}^6$.\\
	Gọi $A$ là biến cố \lq\lq Ban cán sự lớp được chọn có ít nhất $1$ học sinh nữ.\rq\rq\ Khi đó biến cố đối của $A$ là $\overline{A}:$\lq\lq Ban cán sự lớp được chọn không có học sinh nữ.\rq\rq \\
	Khi đó ta có $n\left(\overline{A}\right)=\mathrm{C}_{25}^6$.
	Vậy xác suất của biến cố $A$ là 
	$$\mathrm{P}(A)=1-\mathrm{P}\left(\overline{A}\right)=1-\dfrac{n\left(\overline{A}\right)}{n\left(\Omega\right)}=1-\dfrac{\mathrm{C}_{25}^6}{\mathrm{C}_{35}^6}=\dfrac{939}{1054}.$$

	}
\end{bt}
\begin{bt}%[0T9K4-1]%[Dự án đề kiểm tra HKII NH22-23- Phạm Duy Phương]%[THPT Nguyễn Du - Sở TP Hồ Chí Minh]
	\immini{
	Bạn An muốn làm một cái bìa giấy hình elip, bạn ấy làm như sau: Bạn An lấy một bìa các tông hình chữ nhật có kích thước chiều dài $12$ cm, chiều rộng $8$ cm, trên tấm bìa đó bạn ấy vẽ một hình elip nội tiếp trong hình chữ nhật (tham khảo hình vẽ bên). Sau đó bạn ấy lấy kéo cắt theo đường elip mà bạn đã vẽ. Tính diện tích mà bạn ấy đã cắt bỏ đi (phần gạch chéo trên hình, lấy 1 số thập phân), cho biết diện tích elip được tính theo công thức $S=\pi ab$ với $a$, $b$ lần lượt là nửa độ dài trục lớn, nửa độ dài trục nhỏ của elip.
}{
\begin{tikzpicture}[scale=1,>=stealth, font=\footnotesize, line join=round, line cap=round]
	\foreach \i/\j/\k in{0/0/A,4/0/B}
	\coordinate (\k) at(\i,\j);
	\coordinate (D) at($(A)+(0,3)$);
	\coordinate (C) at($(B)+(D)-(A)$);
	\coordinate (M) at($(C)!0.5!(A)$);
	\fill[pattern=north west lines] (A) rectangle (C);
	\fill[color=white] (M) ellipse (2 cm and 1.5 cm);
	\draw (M) ellipse (2 cm and 1.5 cm);
	\draw (A) rectangle (C);
	\draw[<->] ($(A)+(-0.3,0)$)--($(D)+(-0.3,0)$)node[above,rotate=90,midway]{$8$ cm};
	\draw[<->] ($(C)+(0,0.3)$)--($(D)+(0,0.3)$)node[above,midway]{$12$ cm};
\end{tikzpicture}
}
	\loigiai{
	Từ hình vẽ ta có
	\begin{itemize}
		\item Nửa độ dài trục lớn của elip là $a=\dfrac{12}{2}=6$ cm.
		\item Nửa độ dài trục nhỏ của elip là $b=\dfrac{8}{2}=4$ cm.
	\end{itemize}
	Diện tích bìa các tông là $S_1=12\cdot 8=96$ cm$^2$.\\
	Diện tích phần hình elip là $S_2=\pi ab=\pi\cdot 6\cdot 4=24\pi$ cm$^2$.\\
	Vậy diện tích phần giấy bị cắt bỏ đi là 
	$$S=S_1-S_2=96-24\pi \approx 20{,}6 \,\mathrm{cm}^2.$$
	}
\end{bt}
\begin{bt}%[0T7K2-1]%[Dự án đề kiểm tra HKII NH22-23- Phạm Duy Phương]%[THPT Nguyễn Du - Sở TP Hồ Chí Minh]
	Tìm tham số $m$ để hàm số $y=\dfrac{2x+1}{\sqrt{x^2-2(m+1)x+5m+1}}$
	xác định với mọi $x \in \mathbb{R}$.
	\loigiai{
	Hàm số đã cho xác định với mọi $x \in \mathbb{R}$ khi $x^2-2(m+1)x+5m+1>0$ với mọi $x \in \mathbb{R}$. Khi đó ta có
	\allowdisplaybreaks\begin{eqnarray*}
		& & \heva{&a>0\\&\Delta'<0} \Leftrightarrow \heva{&1>0\\&(m+1)^2-1(5m+1)<0}\\
		&\Leftrightarrow & m^2-3m<0 \\
		&\Leftrightarrow & 0<m<3.
	\end{eqnarray*}
	}
\end{bt}
\begin{bt}%[0T9K3-5]%[Dự án đề kiểm tra HKII NH22-23- Phạm Duy Phương]%[THPT Nguyễn Du - Sở TP Hồ Chí Minh]
	Trong mặt phẳng với hệ trục tọa độ $Oxy$, viết phương trình đường tròn tâm $O(0;0)$ cắt đường thẳng $(\Delta)\colon x+2y-5=0$ tại hai điểm $M$, $N$ sao cho $MN=4$.
	\loigiai{
	\begin{center}
	\begin{tikzpicture}[scale=1,>=stealth, font=\footnotesize, line join=round, line cap=round]
		\def\r{2}
		\foreach \i/\j/\k in{0/0/O}
		\coordinate (\k) at(\i,\j);
		\draw (O) circle (\r cm);
		\coordinate (N) at ($(O)+({\r*cos(-50)},{\r*sin(-50)})$);
		\coordinate (M) at ($(O)+({\r*cos(-130)},{\r*sin(-130)})$);
		\coordinate (H) at ($(M)!0.5!(N)$);
		\draw ($(M)!-0.3!(N)$)--($(M)!1.3!(N)$)
		(O)--(H)(M)--(O)--(N);
		\foreach \p/\r/\t in{O/H/N}
		\draw pic[draw=black, angle eccentricity=0.75, angle radius=0.2cm]{right angle=\p--\r--\t};
		\foreach \p/\r in {O/135,M/-135,N/-45,H/-90}
		\fill (\p) circle (1pt) node[shift={(\r:3mm)}]{$\p$};
	\end{tikzpicture}
	\end{center}
	Gọi $H$ là trung điểm của $MN$. Khoảng cách từ $O$ đến $(\Delta)$ là
	$$OH=d\left(O,\Delta\right)=\dfrac{\left|1\cdot 0+2\cdot 0+(-5)\right|}{\sqrt{1^2+2^2}}=\sqrt{5}.$$
	Khi đó bán kính của đường tròn là
	$$R=\sqrt{OH^2+HM^2}=\sqrt{OH^2+\left(\dfrac{MN}{2}\right)^2}=\sqrt{\left(\sqrt{5}\right)^2+2^2}=3.$$
	Vậy phương trình đường tròn thỏa yêu cầu đề bài là
	$$x^2+y^2=9.$$
	}
\end{bt}

