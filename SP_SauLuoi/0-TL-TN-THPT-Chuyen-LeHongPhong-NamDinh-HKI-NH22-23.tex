
\de{ĐỀ THI HỌC KỲ I NĂM HỌC 2022-2023}{THPT Chuyên Lê Hồng Phong - Nam Định}
\begin{center}
	\textbf{PHẦN 1 - TRẮC NGHIỆM}
\end{center}
\Opensolutionfile{ans}[ans/ans]

% Câu 1
\begin{ex}%[0H3Y1-2]%[Dự án đề kiểm tra HKI NH22-23 - Quan Ón]%[Chuyên Lê Hồng Phong - Nam Định]
	Trong mặt phẳng tọa độ $Oxy$, cho $A(1;-3)$, $B(2;-5)$. Tìm tọa độ của véc-tơ $\overrightarrow{AB}$.
	\choice
	{\True $\overrightarrow{AB} = (1;-2)$}
	{$\overrightarrow{AB} = (-1;-2)$}
	{$\overrightarrow{AB} = (1;2)$}
	{$\overrightarrow{AB} = (-1;-2)$}
	\loigiai{
		Tọa độ của véc-tơ $\overrightarrow{AB} = (1;-2)$.
	}
\end{ex}

% Câu 2
\begin{ex}%[0X2B2-1]%[Dự án đề kiểm tra HKI NH22-23 - Quan Ón]%[Chuyên Lê Hồng Phong - Nam Định]
	Số cách xếp $4$ học sinh nam và $3$ học sinh nữ vào một dãy ghế hàng ngang có $7$ chỗ ngồi là
	\choice
	{$12!$}
	{\True $7!$}
	{$4!3!$}
	{$4!+3!$}
	\loigiai{
		Số cách xếp $4$ học sinh nam và $3$ học sinh nữ vào một dãy ghế hàng ngang có $7$ chỗ ngồi là số cách xếp $4+3 = 7$ học sinh vào một dãy ghế hàng ngang có $7$ chỗ ngồi, do đó có $7!$ cách xếp.
	}
\end{ex}

% Câu 3
\begin{ex}%[0H3B1-1]%[Dự án đề kiểm tra HKI NH22-23 - Quan Ón]%[Chuyên Lê Hồng Phong - Nam Định]
	Trong mặt phẳng tọa độ $Oxy$, cho hai điểm $A(-2;-2)$ và $B(5;-4)$. Tìm tọa độ trọng tâm $G$ của tam giác $OAB$.
	\choice
	{$G\left( -\dfrac{7}{2};1 \right)$}
	{$G\left( \dfrac{7}{3};\dfrac{2}{3} \right)$}
	{\True $G\left( 1;-2 \right)$}
	{$G\left( -\dfrac{3}{2};-3 \right)$}
	\loigiai{
		Tọa độ trọng tâm $G$ của tam giác $OAB$ là
		$$ \heva{&x_{G} = \dfrac{0+(-2)+5}{3} = 1\\&y_{G} = \dfrac{0+(-2)+(-4)}{3} = -2} \Rightarrow G\left( 1;-2 \right). $$
	}
\end{ex}

% Câu 4
\begin{ex}%[0X2B2-3]%[Dự án đề kiểm tra HKI NH22-23 - Quan Ón]%[Chuyên Lê Hồng Phong - Nam Định]
	Trên giá sách có $5$ quyển sách Tiếng Anh khác nhau, $6$ quyển sách Toán khác nhau và $8$ quyển sách Tiếng Việt khác nhau. Số cách chọn $1$ quyển sách bất kì trên giá sách là
	\choice
	{$18$}
	{$240$}
	{$210$}
	{\True $19$}
	\loigiai{
		Số cách chọn $1$ quyển sách bất kì trên giá sách là số cách chọn $1$ trong $5 + 6 + 8 = 19$ quyển sách, do đó có $\mathrm{C}^{1}_{19} = 19$ cách chọn.
	}
\end{ex}

% Câu 5
\begin{ex}%[0X2B2-3]%[Dự án đề kiểm tra HKI NH22-23 - Quan Ón]%[Chuyên Lê Hồng Phong - Nam Định]
	Trong giải thi đấu bóng đá World Cup, vòng bảng có $32$ đội tham gia được chia làm $8$ bảng, mỗi bảng có $4$ đội đấu vòng tròn một lượt. Số trận được thi đấu trong vòng bảng theo thể thức trên là
	\choice
	{$36$ trận}
	{$32$ trận}
	{$64$ trận}
	{\True $48$ trận}
	\loigiai{
		Số trận đấu trong mỗi bảng là $\mathrm{C}^{2}_{4} = 6$ trận.\\
		Vì vòng bảng có $32$ chia thành $8$ bảng nên số trận được thi đấu trong vòng bảng theo thể thức trên là $8\cdot 6 = 48$ trận. 
	}
\end{ex}

% Câu 6
\begin{ex}%[0H2B4-1]%[Dự án đề kiểm tra HKI NH22-23 - Quan Ón]%[Chuyên Lê Hồng Phong - Nam Định]
	Cho tam giác $ABC$ cân tại $A$, $\widehat{A} = 120^\circ$ và $AB = a$. Tính $\overrightarrow{AB}\cdot \overrightarrow{AC}$.
	\choice
	{$-\dfrac{a^2\sqrt{3}}{2}$}
	{$\dfrac{a^2}{2}$}
	{$\dfrac{a^2\sqrt{3}}{2}$}
	{\True $-\dfrac{a^2}{2}$}
	\loigiai{
		Vì tam giác $ABC$ cân tại $A$ và $AB = a$ nên $\left| \overrightarrow{AB} \right| = \left| \overrightarrow{AC} \right| = a$.\\
		Ta có
		$$\overrightarrow{AB}\cdot \overrightarrow{AC} = \left| \overrightarrow{AB} \right|\cdot \left| \overrightarrow{AC} \right|\cdot \cos\widehat{BAC} = a\cdot a\cdot \cos 120^\circ = a\cdot a\cdot \left( -\dfrac{1}{2} \right) = -\dfrac{a^2}{2}.$$
	}
\end{ex}

% Câu 7
\begin{ex}%[0X2B2-3]%[Dự án đề kiểm tra HKI NH22-23 - Quan Ón]%[Chuyên Lê Hồng Phong - Nam Định]
	Cho tập $A = \left\lbrace 1;2;3;4;5;6;7 \right\rbrace $. Số tập con gồm $3$ phần tử của tập hợp $A$ là
	\choice
	{$\mathrm{A}^{3}_{7}$}
	{$\mathrm{P}_{3}$}
	{$\mathrm{P}_{7}$}
	{\True $\mathrm{C}^{3}_{7}$}
	\loigiai{
		Số tập con gồm $3$ phần tử của tập hợp $A$ là số cách lấy $3$ phần tử trong tập hợp có $7$ phần tử nên có $\mathrm{C}^{3}_{7}$ tập con.
	}
\end{ex}

% Câu 8
\begin{ex}%[0X2B2-5]%[Dự án đề kiểm tra HKI NH22-23 - Quan Ón]%[Chuyên Lê Hồng Phong - Nam Định]
	Từ các số $1$; $2$; $3$; $4$; $5$; $6$; $7$; $8$ lập được bao nhiêu số tự nhiên có $4$ chữ số đôi một khác nhau?
	\choice
	{$40320$}
	{$4096$}
	{$65536$}
	{\True $1680$}
	\loigiai{
		Số tự nhiên có $4$ chữ số đôi một khác nhau có thể được lập là $\mathrm{A}^{4}_{8} = 1680$ số.
	}
\end{ex}

% Câu 9
\begin{ex}%[0X2B2-5]%[Dự án đề kiểm tra HKI NH22-23 - Quan Ón]%[Chuyên Lê Hồng Phong - Nam Định]
	Từ các số $0$; $1$; $2$; $3$; $4$; $5$ lập được bao nhiêu số tự nhiên có $6$ chữ số đôi một khác nhau và chia hết cho $10$?
	\choice
	{$720$}
	{$120$}
	{$100$}
	{$600$}
	\loigiai{
		Số chia hết cho $10$ là số có tận cùng là $0$.\\
		Gọi $\overline{a_1a_2a_3a_4a_5a_6}$ là số cần tìm $(a_1 \neq 0)$.
		\begin{itemize}
			\item Vì số cần tìm chia hết cho $10$ nên $a_6 = 0$ có $1$ cách chọn.
			\item $5$ số còn lại $a_1$, $a_2$, $a_3$, $a_4$, $a_5$ có $5! = 120$ cách chọn.
		\end{itemize}
	    Theo quy tắc nhân, ta có $1\cdot 120 = 120$ số có thể được lập thỏa mãn yêu cầu bài toán.
	}
\end{ex}

% Câu 10
\begin{ex}%[0X2B3-1]%[Dự án đề kiểm tra HKI NH22-23 - Quan Ón]%[Chuyên Lê Hồng Phong - Nam Định]
	Khai triển $(x+1)^4$ ta được
	\choice
	{$x^4 - 4x^3 + 6x^2 - 4x + 1$}
	{$x^4 + 6x^3 + 8x^2 + 6x + 1$}
	{\True $x^4 + 4x^3 + 6x^2 + 4x + 1$}
	{$x^4 + 4x^3 + 8x^2 + 4x + 1$}
	\loigiai{
		Ta có
		$$ (x+1)^4 = \mathrm{C}^{0}_{4}\cdot x^4 + \mathrm{C}^{1}_{4}\cdot x^3\cdot 1 + \mathrm{C}^{2}_{4}\cdot x^2\cdot 1^2 + \mathrm{C}^{3}_{4}\cdot x\cdot 1^3 + \mathrm{C}^{4}_{4}\cdot 1^4 = x^4 + 4x^3 + 6x^2 + 4x + 1.$$
	}
\end{ex}

% Câu 11
\begin{ex}%[0H3B2-4] %[Dự án đề kiểm tra HKI NH22-23 - Quan Ón]%[Chuyên Lê Hồng Phong - Nam Định]
	Trong mặt phẳng tọa độ $Oxy$, cho $\overrightarrow{u} = (1;-2)$, $\overrightarrow{v} = (-2;1)$. Khẳng định nào sau đây \textbf{sai}?
	\choice
	{\True $\overrightarrow{u}\perp \overrightarrow{v}$}
	{$\left|\overrightarrow{u} \right| = \left|\overrightarrow{v} \right|$}
	{$\left|\overrightarrow{u} \right| = \sqrt{5}$}
	{\True $\overrightarrow{u}\cdot \overrightarrow{v} = -4$}
	\loigiai{
		Ta có $\overrightarrow{u}\cdot \overrightarrow{v} = 1\cdot (-2) + (-2)\cdot 1 = -2 - 2 = -4 \neq 0$.\\
		Do đó khẳng định $\overrightarrow{u}\perp \overrightarrow{v}$ là sai.
	}
\end{ex}

% Câu 12
\begin{ex}%[0X2B2-4]%[Dự án đề kiểm tra HKI NH22-23 - Quan Ón]%[Chuyên Lê Hồng Phong - Nam Định]
	Cho $10$ điểm phân biệt và không có ba điểm nào thẳng hàng. Hỏi có bao nhiêu véc-tơ khác $\overrightarrow{0}$ có điểm đầu và điểm cuối là hai trong số $10$ điểm đã cho?
	\choice
	{$45$}
	{\True $90$}
	{$20$}
	{$10$}
	\loigiai{
		Vì trong $10$ điểm phẩn biệt không có ba điểm nào thẳng hàng nên số véc-tơ khác $\overrightarrow{0}$ có điểm đầu và điểm cuối là hai trong số $10$ điểm đã cho là $\mathrm{A}^{2}_{10} = 90$ véc-tơ.
	}
\end{ex}

% Câu 13
\begin{ex}%[0H3B2-2]%[Dự án đề kiểm tra HKI NH22-23 - Quan Ón]%[Chuyên Lê Hồng Phong - Nam Định]
	Trong mặt phẳng tọa độ $Oxy$, cho tam giác $ABC$ có $A(3;1)$, $B(6;0)$ và $C(-1;-1)$. Tính số đo góc $A$ của tam giác $ABC$.
	\choice
	{$15^\circ$}
	{$120^\circ$}
	{$60^\circ$}
	{\True $135^\circ$}
	\loigiai{
		Ta có $\overrightarrow{AB} = (3;-1)$, $\overrightarrow{AC} = (-4;-2)$ và $AB = \sqrt{10}$, $AC = 2\sqrt{5}$.\\
		Do đó
		$$ \cos\left(\overrightarrow{AB},\overrightarrow{AC}  \right) = \dfrac{\overrightarrow{AB}\cdot \overrightarrow{AC}}{\left|\overrightarrow{AB} \right|\cdot \left|\overrightarrow{AC} \right| } = \dfrac{3\cdot (-4) + (-1)\cdot (-2)}{\sqrt{10}\cdot 2\sqrt{5}} = -\dfrac{\sqrt{2}}{2}. $$
		Suy ra $\widehat{BAC} = 135^\circ$.
	}
\end{ex}

% Câu 14
\begin{ex}%[0X2B1-2]%[Dự án đề kiểm tra HKI NH22-23 - Quan Ón]%[Chuyên Lê Hồng Phong - Nam Định]
	Một nhóm gồm $10$ học sinh, trong đó có $4$ học sinh nam và $6$ học sinh nữ. Hỏi có bao nhiêu cách chọn ra $2$ học sinh từ nhóm sao cho trong $2$ học sinh được chọn có cả nam và nữ?
	\choice
	{$10$}
	{$36$}
	{$20$}
	{\True $24$}
	\loigiai{
		Ta có
		\begin{itemize}
			\item Số cách chọn $1$ trong $4$ học sinh nam là $4$ cách.
			\item Số cách chọn $1$ trong $6$ học sinh nữ là $6$ cách.
		\end{itemize}
	    Như vậy, có $4\cdot 6 = 24$ cách chọn $2$ học sinh từ nhóm sao cho trong $2$ học sinh được chọn có cả nam và nữ.
	}
\end{ex}

% Câu 15
\begin{ex}%[0X2B3-2]%[Dự án đề kiểm tra HKI NH22-23 - Quan Ón]%[Chuyên Lê Hồng Phong - Nam Định]
	Tìm hệ số của số hạng chứa $xy^3$ trong khai triển $(3x-2y)^4$.
	\choice
	{\True $-96$}
	{$16$}
	{$81$}
	{$-216$}
	\loigiai{
		Ta có
		\begin{eqnarray*}
			(3x-2y)^4 &=& \mathrm{C}^{0}_{4}\cdot (3x)^4 + \mathrm{C}^{1}_{4}\cdot (3x)^3\cdot (-2y) + \mathrm{C}^{2}_{4}\cdot (3x)^2\cdot (-2y)^2 + \mathrm{C}^{3}_{4}\cdot (3x)\cdot (-2y)^3 + \mathrm{C}^{4}_{4}\cdot (-2y)^4 \\
			&=& 81x^4 - 216x^3y + 216x^2y^2 - 96xy^3 + 16y^4.
		\end{eqnarray*}
		Vậy hệ số của số hạng chứa $xy^3$ trong khai triển $(3x-2y)^4$ là $-96$.
	}
\end{ex}

% Câu 16
\begin{ex}%[0X2B3-3]%[Dự án đề kiểm tra HKI NH22-23 - Quan Ón]%[Chuyên Lê Hồng Phong - Nam Định]
	Mệnh đề nào sau đây \textbf{sai}?
	\choice
	{$\mathrm{C}^{0}_{5} + \mathrm{C}^{1}_{5} + \mathrm{C}^{2}_{5} + \mathrm{C}^{3}_{5} + \mathrm{C}^{4}_{5} + \mathrm{C}^{5}_{5} = 32$}
	{$\mathrm{C}^{0}_{5} + \mathrm{C}^{1}_{5} + \mathrm{C}^{2}_{5} = 16$}
	{$\mathrm{C}^{0}_{5} - \mathrm{C}^{1}_{5} + \mathrm{C}^{2}_{5} - \mathrm{C}^{3}_{5} + \mathrm{C}^{4}_{5} - \mathrm{C}^{5}_{5} = 0$}
	{\True $\mathrm{C}^{0}_{5} - \mathrm{C}^{1}_{5} + \mathrm{C}^{2}_{5} = 0$}
	\loigiai{
		Ta có $\mathrm{C}^{0}_{5} - \mathrm{C}^{1}_{5} + \mathrm{C}^{2}_{5} = 6$.\\
		Do đó mệnh đề $\mathrm{C}^{0}_{5} - \mathrm{C}^{1}_{5} + \mathrm{C}^{2}_{5} = 0$ là mệnh đề sai.
	}
\end{ex}

% Câu 17
\begin{ex}%[0X2Y2-8]%[Dự án đề kiểm tra HKI NH22-23 - Quan Ón]%[Chuyên Lê Hồng Phong - Nam Định]
	Cho $k$, $n$ là các số nguyên dương, $k \leq n$. Trong các phát biểu sau, phát biểu nào \textbf{sai}?
	\choice
	{$\mathrm{C}^{k}_{n} = \mathrm{C}^{n-k}_{n}$}
	{$\mathrm{A}^{k}_{n} = \dfrac{n!}{(n-k)!}$}
	{$\mathrm{C}^{k}_{n} = \dfrac{n!}{k!(n-k)!}$}
	{\True $\mathrm{C}^{k}_{n} = \mathrm{A}^{k}_{n}\cdot k!$}
	\loigiai{
		Ta có $\mathrm{A}^{k}_{n} = \mathrm{C}^{k}_{n}\cdot k!$.\\
		Do đó, phát biểu $\mathrm{C}^{k}_{n} = \mathrm{A}^{k}_{n}\cdot k!$ là sai.
	}
\end{ex}

% Câu 18
\begin{ex}%[0X2B2-1]%[Dự án đề kiểm tra HKI NH22-23 - Quan Ón]%[Chuyên Lê Hồng Phong - Nam Định]
	Từ các chữ số $1$; $2$; $3$; $4$; $5$; $6$; $7$ lập được bao nhiêu số tự nhiên có $3$ chữ số đôi một khác nhau, trong đó phải có mặt chữ số $2$?
	\choice
	{$98$}
	{$80$}
	{\True $90$}
	{$120$}
	\loigiai{
		Nhận xét: Số cách lập số tự nhiên có $3$ chữ số đôi một khác nhau, trong đó phải có mặt chữ số $2$ là số cách xếp các chữ số $1$; $2$; $3$; $4$; $5$; $6$; $7$ vào $3$ vị trí sao cho luôn có chữ số $2$.\\
		Vậy có $3\cdot \mathrm{A}^{2}_{6} = 90$ số.
	}
\end{ex}

% Câu 19
\begin{ex}%[0X2B3-3]%[Dự án đề kiểm tra HKI NH22-23 - Quan Ón]%[Chuyên Lê Hồng Phong - Nam Định]
	Biểu diễn $\left( 3 + \sqrt{2} \right)^5$ dưới dạng $a + b\sqrt{2}$ với $a$, $b \in \mathbb{N}^{*}$. Giá trị của biểu thức $M = a + b$ là
	\choice
	{$1431$}
	{\True $1432$}
	{$1433$}
	{$1434$}
	\loigiai{
		Ta có
		\begin{eqnarray*}
			\left( 3 + \sqrt{2} \right)^5 &=& \mathrm{C}^{0}_{5}\cdot 3^5 + \mathrm{C}^{1}_{5}\cdot 3^4\cdot \sqrt{2} + \mathrm{C}^{2}_{5}\cdot 3^3\cdot \left(\sqrt{2}\right)^2 + \cdots + \mathrm{C}^{5}_{5}\cdot \left(\sqrt{2}\right)^5\\
			&=& 243 + 405\sqrt{2} + 540 + 180\sqrt{2} + 60 + 4\sqrt{5}\\
			&=& 843 + 589\sqrt{2}.
		\end{eqnarray*}
	    Suy ra $a = 843$, $b = 589$.\\
	    Do đó $M= a + b = 843 + 589 = 1432$.
	}
\end{ex}

% Câu 20
\begin{ex}%[0H2B4-1]%[Dự án đề kiểm tra HKI NH22-23 - Quan Ón]%[Chuyên Lê Hồng Phong - Nam Định]
	Cho tam giác $ABC$ có $AB = 4$, $AC = 6$. $M$ là trung điểm của $BC$. Tính tích vô hướng $\overrightarrow{AM}\cdot \overrightarrow{BC}$.
	\choice
	{\True $10$}
	{$12$}
	{$8$}
	{$20$}
	\loigiai{
		Vì $M$ là trung điểm của $BC$ nên $\overrightarrow{AM} = \dfrac{1}{2}\left( \overrightarrow{AB} + \overrightarrow{AC} \right)$.\\
		Do đó
		$$ \overrightarrow{AM}\cdot \overrightarrow{BC} = \dfrac{1}{2}\left( \overrightarrow{AB} + \overrightarrow{AC} \right)\left( \overrightarrow{AC}  - \overrightarrow{AB} \right) = \dfrac{1}{2}\left( AC^2 - AB^2 \right) = \dfrac{1}{2}\left( 6^2 - 4^2 \right) = 10. $$
	}
\end{ex}

% Câu 21
\begin{ex}%[0H3B1-1]%[Dự án đề kiểm tra HKI NH22-23 - Quan Ón]%[Chuyên Lê Hồng Phong - Nam Định]
	Trong mặt phẳng tọa độ $Oxy$, cho $A(-1;2)$, $B(2;3)$. Tọa độ điểm $C$ nằm trên trục tung sao cho $A$, $B$, $C$ thẳng hàng là
	\choice
	{$C\left( 0; -\dfrac{1}{3} \right)$}
	{$C(3;0)$}
	{\True $C\left( 0;\dfrac{7}{3} \right)$}
	{$C\left( 0;\dfrac{4}{3} \right)$}
	\loigiai{
		Vì $C$ nằm trên trục tung nên $C(0;y_C)$.\\
		Ta có $\overrightarrow{AB} = (3;1)$; $\overrightarrow{AC} = (1;y_C - 2)$.\\
		Vì $A$, $B$, $C$ thẳng hàng nên $\overrightarrow{AB}$ và $\overrightarrow{AC}$ cùng phương hay
		$$ 3(y_C - 2) = 1\cdot 1 \Leftrightarrow y_C = \dfrac{7}{3} \Rightarrow C\left( 0;\dfrac{7}{3} \right). $$
	}
\end{ex}

% Câu 22
\begin{ex}%[0H3B1-2]%[Dự án đề kiểm tra HKI NH22-23 - Quan Ón]%[Chuyên Lê Hồng Phong - Nam Định]
	Trong mặt phẳng tọa độ $Oxy$, cho $\overrightarrow{a} = (-1;2)$, $\overrightarrow{b} = (0;-2)$, $\overrightarrow{c} = (3;1)$. Tìm tọa độ $\overrightarrow{x}$ sao cho $\overrightarrow{x} + 3\overrightarrow{c} = \overrightarrow{a} - \overrightarrow{b}$.
	\choice
	{$\overrightarrow{x} = (1;10)$}
	{$\overrightarrow{x} = (8;-2)$}
	{\True $\overrightarrow{x} = (-10;1)$}
	{$\overrightarrow{x} = (8;7)$}
	\loigiai{
		Ta có $\overrightarrow{x} + 3\overrightarrow{c} = \overrightarrow{a} - \overrightarrow{b} \Leftrightarrow \overrightarrow{x} = \overrightarrow{a} - \overrightarrow{b} - 3\overrightarrow{c} = (-10;1)$.
	}
\end{ex}

% Câu 23
\begin{ex}%[0X2B1-3]%[Dự án đề kiểm tra HKI NH22-23 - Quan Ón]%[Chuyên Lê Hồng Phong - Nam Định]
	Trong mặt phẳng, cho hai đường thẳng song song $a$ và $b$. Trên đường thẳng $a$ cho $6$ điểm phân biệt, trên đường thẳng $b$ cho $5$ điểm phân biệt. Hỏi có bao nhiêu tam giác có $3$ đỉnh là $3$ điểm trong $11$ điểm nói trên?
	\choice
	{$140$}
	{\True $135$}
	{$120$}
	{$130$}
	\loigiai{
		Trường hợp 1: Tam giác có $1$ đỉnh nằm trên đường thẳng $a$ và $2$ đỉnh còn lại nằm trên đường thẳng $b$
		\begin{itemize}
			\item Chọn $1$ điểm trong $6$ điểm trên đường thẳng $a$ có $6$ cách.
			\item Chọn $2$ điểm trong $5$ điểm trên đường thẳng $b$ có $\mathrm{C}^{2}_{5} = 10$ cách.
		\end{itemize}
	    Có $6\cdot 10 = 60$ tam giác.\\
	    Trường hợp 2: Tam giác có $1$ đỉnh nằm trên đường thẳng $b$ và $2$ đỉnh còn lại nằm trên đường thẳng $a$
	    \begin{itemize}
	    	\item Chọn $1$ điểm trong $5$ điểm trên đường thẳng $b$ có $5$ cách.
	    	\item Chọn $2$ điểm trong $6$ điểm trên đường thẳng $a$ có $\mathrm{C}^{2}_{6} = 15$ cách.
	    \end{itemize}
	    Có $5\cdot 15 = 75$ tam giác.\\
	    Vậy có $60 + 75 = 135$ tam giác được lập.
	}
\end{ex}

% Câu 24
\begin{ex}%[0X2B1-3] %[Dự án đề kiểm tra HKI NH22-23 - Quan Ón]%[Chuyên Lê Hồng Phong - Nam Định]
	Từ các chữ số $1$; $2$; $3$; $4$; $5$ có thể lập được bao nhiêu số tự nhiên nhỏ hơn $2022$?
	\choice
	{\True $280$}
	{$300$}
	{$250$}
	{$125$}
	\loigiai{
		\begin{itemize}
			\item Trường hợp 1: Số có $1$ chữ số và nhỏ hơn $2022$ có $5$ số.
			\item Trường hợp 2: Số có $2$ chữ số và nhỏ hơn $2022$ có $5\cdot5 = 25$ số.
			\item Trường hợp 3: Số có $3$ chữ số và nhỏ hơn $2022$ có $5\cdot 5\cdot 5 = 125$ số.
			\item Trường hợp 4: Số có $4$ chữ số và nhỏ hơn $2022$ và lập từ $1$; $2$; $3$; $4$; $5$ nên chữ số hàng nghìn phải là $1$ nên có $1\cdot5\cdot 5\cdot5 = 125$ số.
		\end{itemize}
		Vậy có tất cả $5 + 25 + 125 + 125 = 280$ số.
	}
\end{ex}

% Câu 25
\begin{ex}%[0H3B1-3] %[Dự án đề kiểm tra HKI NH22-23 - Quan Ón]%[Chuyên Lê Hồng Phong - Nam Định]
	Trong mặt phẳng tọa độ $Oxy$, cho tam giác $ABC$ với $A(1;5)$, $B(3;-1)$ và $C(6;0)$. Tìm tọa độ điểm $B'$ là chân đường cao kẻ từ điểm $B$ lên $CA$.
	\choice
	{\True $B'(5;1)$}
	{$B'(1;-5)$}
	{$B'(-5;1)$}
	{$B'(-5;-1)$}
	\loigiai{
		Gọi $B'(x_{B'};y_{B'})$ là điểm cần tìm.\\
		Ta có $\overrightarrow{BB'} = (x_{B'}-3;y_{B'}+1)$; $\overrightarrow{AB'} = (x_{B'}-1;y_{B'}-5)$; $\overrightarrow{AC} = (5;-5)$.\\
		Vì $B'$ là chân đường cao kẻ từ điểm $B$ lên $CA$ nên 
		$$ \heva{&\overrightarrow{BB'}\cdot \overrightarrow{AC} = 0\\&\overrightarrow{AB'} \textrm{ cùng phương } \overrightarrow{AC}} \Leftrightarrow \heva{&x_{B'}-y_{B'} = 4\\&x_{B'}+y_{B'} = 6} \Leftrightarrow \heva{&x_{B'} = 5\\&y_{B'} = 1} \Rightarrow B'(5;1). $$
	}
\end{ex}
\Closesolutionfile{ans}
%\begin{center}
%	\textbf{ĐÁP ÁN}
%	\inputansbox{10}{ans/ans}	
%\end{center}


\begin{center}
	\textbf{PHẦN 2 - TỰ LUẬN}
\end{center}


\begin{bt}%[0D2B1-2]%[0D2K1-3]%[Dự án đề kiểm tra HKI NH22-23 - Dương Phước Sang]%[Chuyên Lê Hồng Phong - Nam Định]
		Từ các chữ số $0$, $1$, $2$, $3$, $4$, $5$, $6$, $7$, $8$ có thể lập được bao nhiêu số tự nhiên
		\begin{listEX}[2]
			\item có $3$ chữ số?
			\item là số chẵn có $4$ chữ số khác nhau?
		\end{listEX}
		\dapso{
			a) $648$ số. \hspace{1cm}
			b) $1512$ số.
		}
		\loigiai{
			Đặt $X=\big\{0;1;2;3;4;5;6;7;8\big\}$ ta có $n(X)=9$.
			\begin{listEX}
				\item Xét số $x=\overline{abc}$, với $a \neq 0$ và $a,b,c \in X$.\\
				Số cách chọn $a$ từ $X \setminus \big\{0\big\}$ là $8$.\\
				Số cách chọn mỗi số $b$, $c$ từ $X$ là $9$.\\
				Theo quy tắc nhân, số cách chọn $a,b,c$ để được số $x$ là $8\cdot 9\cdot 9=648$ số.
				\item Xét số $y=\overline{abcd}$ chẵn, với $a \neq 0$, $a,b,c,d \in X$ và chúng khác nhau.\\
				Do $y$ chẵn nên $d \in \big\{0;2;4;6;8\big\}$.
				\begin{itemize}
					\item Trường hợp $d=0$ thì số cách chọn $d$ là $1$.\\
						Số cách chọn $a$ từ $X \setminus \big\{0\big\}$ là $8$.\\
						Số cách chọn $b$ từ $X \setminus \big\{0;a\big\}$ là $7$.\\
						Số cách chọn $c$ từ $X \setminus \big\{0;a;b\big\}$ là $6$.\\
						Trường hợp này có $1\cdot 8\cdot 7\cdot 6=336$ số.
						\item Trường hợp $d \in \big\{2;4;6;8\big\}$ thì số cách chọn $d$ là $4$.\\
						Số cách chọn $a$ từ $X \setminus \big\{d;0\big\}$ là $7$.\\
						Số cách chọn $b$ từ $X \setminus \big\{d;a\big\}$ là $7$.\\
						Số cách chọn $c$ từ $X \setminus \big\{d;a;b\big\}$ là $6$.\\
						Trường hợp này có $4\cdot 7\cdot 7\cdot 6=1176$ số.
				\end{itemize}
				Như vậy, có $336+1176=1512$ số $y$. 
			\end{listEX}
		}
	\end{bt}
	
	\begin{bt}%[0D2B2-3]%[0D2K2-3]%[Dự án đề kiểm tra HKI NH22-23 - Dương Phước Sang]%[Chuyên Lê Hồng Phong - Nam Định]
		Một hộp đựng $15$ chiếc thẻ được đánh số từ $1$ đến $15$. Hỏi có bao nhiêu cách chọn $3$ trong số $15$ thẻ từ hộp sao cho
		\begin{listEX}
			\item cả $3$ thẻ đều mang số lẻ?
			\item tổng $3$ số ghi trên các thẻ là một số chẵn?
		\end{listEX}
		\dapso{
			a) $56$. \hspace{1cm}
			b) $231$.
		}
		\loigiai{
			Trong hộp có $8$ thẻ được đánh số lẻ và $7$ thẻ được đánh số chẵn.
			\begin{listEX}
				\item Mỗi cách chọn được $3$ thẻ đều mang số lẻ là một tổ hợp chập $3$ của $8$ phần tử mà mỗi phần tử là một thẻ ghi số lẻ, do đó số cách chọn được $3$ thẻ đều ghi số lẻ là $\mathrm{C}_8^3=56$.
				\item Để chọn được $3$ thẻ có tổng các số ghi trên đó là một số chẵn, ta cần chọn được đúng $1$ thẻ chẵn hoặc chọn cả $3$ thẻ đều chẵn.\\
				Số cách chọn được $3$ thẻ ghi số chẵn là $\mathrm{C}_7^3$.\\
				Số cách chọn được $3$ thẻ có đúng $1$ thẻ chẵn là $\mathrm{C}_7^1\cdot\mathrm{C}_8^2$.\\
				Vậy số cách chọn $3$ thẻ có tổng các số ghi trên đó là một số chẵn là $\mathrm{C}_7^3+\mathrm{C}_7^1\cdot\mathrm{C}_8^2=231$.
			\end{listEX}
		}
	\end{bt}
	
	\begin{bt}%[0D2K3-2]%[Dự án đề kiểm tra HKI NH22-23 - Dương Phước Sang]%[Chuyên Lê Hồng Phong - Nam Định]
		Tìm hệ số của số hạng chứa $x^3$ trong khai triển của biểu thức $x(3x-5)^5$.
		\dapso{$-11250$}
		\loigiai{
			Mỗi số hạng của khai triển nhị thức Newton $(3x-5)^5$ đều có dạng
			$$\mathrm{C}_5^k\,(3x)^{5-k}(-5)^k=\mathrm{C}_5^k\,3^{5-k}(-5)^k\cdot x^{5-k}, \text{ với } k \in \mathbb{N},k \leq 5.$$
			Do đó mỗi số hạng của khai triển biểu thức $x(3x-5)^5$ có dạng
			$$\mathrm{C}_5^k\,3^{5-k}(-5)^k\cdot x^{6-k}.$$
			Số hạng của khai triển $x(3x-5)^5$ chứa $x^3$ ứng với $6-k=3 \Leftrightarrow k=3$.\\
			Số hạng đó có hệ số là $\mathrm{C}_5^3\,3^2(-5)^3=-11250$.
		}
	\end{bt}
	
	\begin{bt}%[0H3B2-1]%[0H3B1-4]%[0H4K2-1]%[Dự án đề kiểm tra HKI NH22-23 - Dương Phước Sang]%[Chuyên Lê Hồng Phong - Nam Định]
		Trong mặt phẳng tọa độ $Oxy$, cho tam giác $ABC$ có $A(1;-1)$, $B(2;2)$, $C(8;0)$.
		\begin{listEX}
			\item Tính chu vi tam giác $ABC$.
			\item Tìm tọa độ điểm $D$ sao cho tứ giác $ABCD$ là hình thang có $AB \parallel CD$ và $CD=3AB$.
			\item Tìm tọa độ tâm $I$ của đường tròn ngoại tiếp tam giác $ABC$.
		\end{listEX}
		\dapso{
			a) $2p=3\sqrt{10}+5\sqrt{2}$.\hspace{1cm}
			b) $D(5;-9)$. \hspace{1cm}
			c) $I\left(\dfrac{9}{2};-\dfrac{1}{2}\right)$.
		}
		\loigiai{
			Tam giác $ABC$ có $A(1;-1)$, $B(2;2)$, $C(8;0)$.
			\begin{listEX}
				\item Ta có $\heva{&\vec{AB}=(1;3)\\&\vec{AC}=(7;1)\\&\vec{BC}=(6;-2)} 
				\Rightarrow \heva{&AB=\sqrt{1^2+3^2}=\sqrt{10}\\&AC=\sqrt{7^2+1^2}=5\sqrt{2}\\&BC=\sqrt{6^2+(-2)^2}=2\sqrt{10}.}$\\
				Chu vi tam giác $ABC$ là $2p=AB+AC+BC=\sqrt{10}+5\sqrt{2}+2\sqrt{10}=3\sqrt{10}+5\sqrt{2}$.
				\item Xét điểm $D\left(x_D;y_D\right)$, ta có $\heva{&\vec{CD}=\left(x_D-8;y_D\right)\\&\vec{AB}=(1;3).}$\\
				Do $ABCD$ là hình thang, $AB \parallel CD$ và $CD=3AB$ nên 
				$$\vec{CD}=-3\vec{AB} \Leftrightarrow \heva{x_D-8&=-3\cdot 1\\y_D&=-3\cdot 3} \Leftrightarrow \heva{&x_D=5\\&y_D=-9.}$$
				Vậy $D(5;-9)$.
				\item Giả sử $(C)\colon x^2+y^2-2ax-2by+c=0$ là đường tròn ngoại tiếp tam giác $ABC$.\\
				Khi đó $A,B,C \in (C)$ nên 
				$\heva{&2-2a+2b+c=0\\&8-4a-4b+c=0\\&64-16a+c=0} \Leftrightarrow \heva{&-2a+2b+c=-2\\&-4a-4b+c=-8\\&-16a+c=-64} \Leftrightarrow \heva{&a=\dfrac{9}{2}\\&b=-\dfrac{1}{2}\\&c=8.}$\\
				Vậy tâm đường tròn ngoại tiếp tam giác $ABC$ là điểm $I\left(\dfrac{9}{2};-\dfrac{1}{2}\right)$.
			\end{listEX}
		}
	\end{bt}
	
	\begin{bt}%[0H2B4-1]%[0H3G2-7]%[Dự án đề kiểm tra HKI NH22-23 - Dương Phước Sang]%[Chuyên Lê Hồng Phong - Nam Định]
		Cho hình thang $ABCD$ vuông  tại $A$ và $B$ có $AD \parallel BC$, $AB=AD=2$, $BC=6$.
		\begin{listEX}
			\item Tính tích vô hướng $\vec{BA}\cdot\vec{BD}$.
			\item Gọi $E$ là giao điểm của $AC$ và $BD$, $F$ là điểm trên cạnh $BC$ sao cho $BF=\dfrac{1}{3} BC$, $G$ là trung điểm của $AB$. Chứng minh rằng $\widehat{GEF}=90^{\circ}$.
		\end{listEX}
		\loigiai{
				\immini{
					\begin{listEX}
						\item Do $\vec{BA} \perp \vec{AD}$ nên $\vec{BA}\cdot \vec{AD}=\vec{0}$. Từ đó
						\allowdisplaybreaks
						$$\begin{aligned}
							\vec{BA}\cdot\vec{BD}
							&=\vec{BA}\cdot\left(\vec{BA}+\vec{AD}\right)\\
							&=\vec{BA}^2+\vec{BA}\cdot\vec{AD}=BA^2+0=4.
						\end{aligned}$$
					\item Gắn hệ tọa độ $Oxy$ sao cho 
						$$B \equiv O(0;0),\; A(0;2),\; C(6;0),\; D(2;2).$$
						Khi đó, $G(0;1)$ và $F(2;0)$.
					\end{listEX}}
				{\begin{tikzpicture}[scale=1, font=\footnotesize, line join=round, line cap=round]
						%---------------------------
						\foreach \x\y\t in {0/2/A,0/0/B, 6/0/C, 2/2/D}
						\coordinate (\t) at (\x,\y);
						\coordinate (E) at ($(A)!1/4!(C)$);
						\coordinate (F) at ($(B)!1/3!(C)$);
						\coordinate (G) at ($(B)!1/2!(A)$);
						\draw (C)--(A)--(B)--(C)--(D)--(A) (B)--(D);
						\foreach \t\g in {A/140, B/-140, C/-70,D/70,E/95,F/-90,G/180}
						\draw[fill=black] (\t)circle(1pt) +(\g:8pt)node{$\t$};
						\path pic[draw,angle radius=4]{right angle=A--B--C}
						pic[draw,angle radius=4]{right angle=B--A--D};
						\draw[->] (C)--(7.5,0) node[below]{$x$};
						\draw[->] (A)--(0,3) node[left]{$y$};
						\node at (0,0)[below right=-2pt]{$O$};
				\end{tikzpicture}}
				\begin{listEX}
					\item[] 
					Ngoài ra do $AD \parallel BC$ nên $\dfrac{EB}{ED}=\dfrac{BC}{AD}=\dfrac{6}{2}$. Suy ra $BE=\dfrac{3}{4}\,BD$ và $\vec{OE}=\dfrac{3}{4}\,\vec{OD}$. Từ đó
					$$\vec{OE}=\left(\dfrac{3}{2};\dfrac{3}{2}\right) \text{ hay } E\left(\dfrac{3}{2};\dfrac{3}{2}\right).$$
					Như vậy $\heva{&\vec{EG}=\left(-\dfrac{3}{2};-\dfrac{1}{2}\right)\\&\vec{EF}=\left(\dfrac{1}{2};-\dfrac{3}{2}\right)} 
					\Rightarrow \vec{EG}\cdot\vec{EF}=0$, tức là $EG \perp EF$ hay $\widehat{GEF}=90^{\circ}$.
				\end{listEX}
		}
	\end{bt}