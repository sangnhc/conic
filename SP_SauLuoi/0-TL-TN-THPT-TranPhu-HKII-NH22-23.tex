\de{ĐỀ THI HỌC KỲ II NĂM HỌC 2022-2023}{THPT Trần Phú}


\begin{center}
	\textbf{PHẦN 1 - TRẮC NGHIỆM}
\end{center}
\Opensolutionfile{ans}[ans/ans]
%Câu 1...........................
\begin{ex}%[0T8B2-1]%[Dự án đề kiểm tra HKII NH22-23- Võ Thị Thùy Trang]%[THPT Trần Phú]
	Cho hai đường thẳng song song $d_1$ và $d_2$. Trên $d_1$ lấy $17$ điểm phân biệt, trên $d_2$ lấy $20$ điểm phân biệt. Tính số tam giác mà có các đỉnh được chọn từ $37$ điểm này.
	\choice
	{\True $5950$}
	{$4690$}
	{$5590$}
	{$5960 $}
	\loigiai{
		Số tam giác mà có các đỉnh được chọn từ $37$ điểm là $\mathrm{C}_{17}^1\cdot \mathrm{C}_{20}^2+\mathrm{C}_{17}^2\cdot \mathrm{C}_{20}^1=5950$.
	}
\end{ex}

%%%%%%%%=====Câu 2
\begin{ex}%[0T7B1-1]%[Dự án đề kiểm tra HKII NH22-23- Võ Thị Thùy Trang]%[THPT Trần Phú]
	Tam thức bậc hai $f(x)=2x^2+2x+5$ nhận giá trị dương khi và chỉ khi
	\choice
	{\True $x \in \mathbb{R}$}
	{$x \in(0 ;+\infty)$}
	{$x \in(-\infty ; 2)$}
	{$x \in(-2 ;+\infty)$}
	\loigiai{
		Do tam thức bậc hai $f(x)=2x^2+2x+5$ có $a=2>0$ và $\Delta'=-9<0$ nên
		tam thức bậc hai $f(x)=2x^2+2x+5$ nhận giá trị dương khi và chỉ khi $x \in \mathbb{R}$.
	}
\end{ex}

%%%%%%%%=====Câu 3
\begin{ex}%[0T8Y1-1]%[Dự án đề kiểm tra HKII NH22-23- Võ Thị Thùy Trang]%[THPT Trần Phú]
	 Một người có $5$ cái quần khác nhau, $6$ cái áo khác nhau, $3$ chiếc cà vạt khác nhau. Để chọn một cái quần hoặc một cái áo hoặc một cái cà vạt thì số cách chọn khác nhau là
	\choice
	{$90$}
	{$12$}
	{\True $14$}
	{$13$}
	\loigiai{Theo quy tắc cộng ta có $5+6+3=14$ cách.
	}
\end{ex}

%%%%%%%%=====Câu 4
\begin{ex}%[0T9B2-2]%[Dự án đề kiểm tra HKII NH22-23- Võ Thị Thùy Trang]%[THPT Trần Phú]
   Phương trình tổng quát của đường thẳng đi qua hai điểm $A(3 ;-7)$ và $B(1 ;-7)$ là
	\choice
	{$x+y+4=0$}
	{$y-7=0$}
	{\True $y+7=0$}
	{$x+y+6=0$}
	\loigiai{Ta có $\overrightarrow{AB}=(-2;0)$ là véc-tơ chỉ phương của đường thẳng $AB$ nên $\overrightarrow{n}=(0;2)$ là véc-tơ pháp tuyến của đường thẳng $AB$.\\
		Khi đó
		$AB\colon 2(y+7)=0$ hay $AB\colon y+7=0$.\\
		Vậy Phương trình tổng quát của đường thẳng đi qua hai điểm $A(3 ;-7)$ và $B(1 ;-7)$ là $y+7=0$.
	}
\end{ex}

%%%%%%%%=====Câu 5
\begin{ex}%[0T0B2-2]%[Dự án đề kiểm tra HKII NH22-23- Võ Thị Thùy Trang]%[THPT Trần Phú]
	Gieo một con xúc xắc cân đối đồng chất $2$ lần. Tính xác suất để biến cố có tổng hai mặt bằng $8$.
	\choice
	{$\dfrac{1}{6}$}
	{$\dfrac{1}{2}$}
	{$\dfrac{1}{9}$}
	{\True $\dfrac{5}{36}$}
	\loigiai{
			Ta có $n\left(\Omega\right)=36$.\\
		Gọi $A$ là biến cố \lq\lq có tổng hai mặt bằng $8$\rq\rq.\\
		$2+6=8$, $3+5=8$, $4+4=8$.
		Suy ra $n\left(A\right)=5$.\\
		Do đó $\mathrm{P}(A)=\dfrac{n(A)}{n(\Omega)}=\dfrac{5}{36}$.	
	}
\end{ex}

%%%%%%%%=====Câu 6
\begin{ex}%[0T0B2-2]%[Dự án đề kiểm tra HKII NH22-23- Võ Thị Thùy Trang]%[THPT Trần Phú]
	Trong một chiếc hộp có $20$ viên bi, trong đó có $8$ viên bi màu đỏ, $7$ viên bi màu xanh và $5$ viên bi màu vàng. Lấy ngẫu nhiên ra $3$ viên bi. Xác suất để $3$ viên bi lấy ra đều màu đỏ là
	\choice
	{$\dfrac{15}{248}$}
	{$\dfrac{45}{128}$}
	{\True $\dfrac{14}{285}$}
	{$\dfrac{28}{145}$}
	\loigiai{
		Ta có $n\left(\Omega\right)=\mathrm{C}_{20}^3=1140$.\\
		Gọi $A$ là biến cố \lq\lq $3$ viên bi lấy ra đều màu đỏ\rq\rq.\\
		Suy ra $n\left(A\right)=\mathrm{C}_8^3=56$.\\
		Do đó $\mathrm{P}(A)=\dfrac{n(A)}{n(\Omega)}=\dfrac{56}{1140}=\dfrac{14}{285}$.	
	}
\end{ex}

%%%%%%%%=====Câu 7
\begin{ex}%[0T0Y1-1]%[Dự án đề kiểm tra HKII NH22-23- Võ Thị Thùy Trang]%[THPT Trần Phú]
	Gieo một đồng xu liên tiếp $2$ lần. Số phần tử của không gian mẫu $n(\Omega)$ là?
	\choice
	{$2$}
	{$1$}
	{\True $4$}
	{$8$}
	\loigiai{Ta có $n(\Omega)=4$.
	}
\end{ex}


%%%%%%%%=====Câu 8
\begin{ex}%[0T9B2-4]%[Dự án đề kiểm tra HKII NH22-23- Võ Thị Thùy Trang]%[THPT Trần Phú]
	Tính góc tạo bởi hai đường thẳng $d_1\colon 2x-y-10=0$ và $d_2\colon x-3y+9=0$
	\choice
	{$135^\circ$}
	{\True $45^\circ$}
	{$30^\circ$}
	{$60^\circ$}
	\loigiai{Ta có $d_1\colon 2x-y-10=0$ có véc-tơ pháp tuyến là $\overrightarrow{n}_1=(2;-1)$.\\
		$d_2\colon x-3y+9=0$ có véc-tơ pháp tuyến là $\overrightarrow{n}_2=(1;-3)$.\\
		$\cos (d_1,d_2)=\dfrac{\left|\overrightarrow{n}_1\cdot \overrightarrow{n}_2\right|}{\left|\overrightarrow{n}_1\right|\cdot \left|\overrightarrow{n}_2\right|}=\dfrac{\left|2+3\right|}{\sqrt{4+1}\cdot \sqrt{1+9}}=\dfrac{1}{\sqrt{2}}$.\\
		Suy ra $\left(d_1,d_2\right)=45^\circ$.
	}
\end{ex}

%%%%%%%%=====Câu 9
\begin{ex}%[0T8Y2-1]%[Dự án đề kiểm tra HKII NH22-23- Võ Thị Thùy Trang]%[THPT Trần Phú]
	Một lớp học có $40$ học sinh gồm $25$ nam và $15$ nữ. Chọn $3$ học sinh để tham gia vệ sinh công cộng toàn trường, hỏi có bao nhiêu cách chọn như trên?
	\choice
	{$59280$}
	{\True $9880$}
	{$455$}
	{$2300$}
	\loigiai{Số cách chọn $3$ học sinh để tham gia vệ sinh công cộng toàn trường là $\mathrm{C}_{40}^3=9880$ cách.
	}
\end{ex}

%%%%%%%%=====Câu 10
\begin{ex}%[0T7Y2-1]%[Dự án đề kiểm tra HKII NH22-23- Võ Thị Thùy Trang]%[THPT Trần Phú]
	Tập nghiệm của bất phương trình $-x^2+5 x-4<0$ là
	\choice
	{\True $(-\infty ; 1)\cup(4 ;+\infty)$}
	{$(1 ; 4)$}
	{$[1 ; 4]$}
	{$(-\infty ; 1]\cup[4 ;+\infty)$}
	\loigiai{
		Tập nghiệm của bất phương trình $-x^2+5 x-4<0$ là $(-\infty ; 1)\cup(4 ;+\infty)$.
	}
\end{ex}

%%%%%%%%=====Câu 11
\begin{ex}%[0T9K3-2]%[Dự án đề kiểm tra HKII NH22-23- Võ Thị Thùy Trang]%[THPT Trần Phú]
	Đường tròn $(C)$ có tâm $I$ thuộc đường thẳng $d\colon x+3 y+8=0$, đi qua điểm $A(-2 ; 1)$ và tiếp xúc với đường thẳng $\Delta\colon 3x-4y+10=0$. Phương trình của đường tròn $(C)$ là
	\choice
	{$(x+2)^2+(y+2)^2=9$}
	{$(x+5)^2+(y+1)^2=16$}
	{\True $(x-1)^2+(y+3)^2=25$}
	{$(x-2)^2+(y+2)^2=25$}
	\loigiai{Do $I$ thuộc đường thẳng $d\colon x+3 y+8=0$ nên $I=(-3m-8;m)$.\\
		Ta có 
		\allowdisplaybreaks
		\begin{eqnarray*}
		\mathrm{d}\left(I,\Delta\right)=IA &\Leftrightarrow& \dfrac{\left|3(-3m-8)-4m+10\right|}{\sqrt{9+16}}=\sqrt{\left(3m+6\right)^2+\left(1-m\right)^2}\\
		& \Leftrightarrow & \left(-13m-14\right)^2=25\left(10m^2+34m+37\right)\\
		& \Leftrightarrow & 81m^2+486m+729=0 \Leftrightarrow m=-3.
		\end{eqnarray*}
		Suy ra $I=(1;-3)$, $R=IA=5$.\\
		Phương trình của đường tròn $(C)$ là $(x-1)^2+(y+3)^2=25$.
	}
\end{ex}

%%%%%%%%=====Câu 12
\begin{ex}%[0T0B2-2]%[Dự án đề kiểm tra HKII NH22-23- Võ Thị Thùy Trang]%[THPT Trần Phú]
	Từ một hộp chứa $3$ quả cầu trắng và $2$ quả cầu đen, lấy ngẫu nhiên đồng thời $2$ quả. Xác suất để lấy được cả hai quả trắng là
	\choice
	{$\dfrac{12}{30}$}
	{\True $\dfrac{9}{30}$}
	{$\dfrac{6}{30}$}
	{$\dfrac{10}{30}$}
	\loigiai{Ta có $n\left(\Omega\right)=\mathrm{C}_5^2=10$.\\
		Gọi $A$ là biến cố \lq\lq Lấy được cả hai quả trắng\rq\rq.\\
		Suy ra $n\left(A\right)=\mathrm{C}_3^2=3$.\\
		Do đó $\mathrm{P}(A)=\dfrac{n(A)}{n(\Omega)}=\dfrac{3}{10}=\dfrac{9}{30}$.		
	}
\end{ex}

%%%%%%%%=====Câu 13
\begin{ex}%[0T8Y2-1]%[Dự án đề kiểm tra HKII NH22-23- Võ Thị Thùy Trang]%[THPT Trần Phú]
	Từ các số tự nhiên $1$, $2$, $3$, $4$ có thể lập được bao nhiêu số tự nhiên có $4$ chữ số khác nhau?
	\choice
	{$1$}
	{$42$}
	{$4^4$}
	{\True $24$}
	\loigiai{
		Từ các số tự nhiên $1$, $2$, $3$, $4$ có thể lập được $4!=24$ số tự nhiên có $4$ chữ số khác nhau.
	}
\end{ex}
	


%%%%%%%%=====Câu 14
\begin{ex}%[0T9B4-1]%[Dự án đề kiểm tra HKII NH22-23- Tin Đạt Trần]%[THPT Trần Phú]
	Một elip $(E)$ có phương trình $\dfrac{x^2}{a^2}+\dfrac{y^2}{b^2}=1$, trong đó $a>b>0$. Biết $(E)$ đi qua điểm $A(2 ; \sqrt{2})$ và $B(2 \sqrt{2} ; 0)$ thì $(E)$ có độ dài trục bé là
	\choice
	{\True $4$}
	{$2$}
	{$6$}
	{$2 \sqrt{2}$}
	\loigiai{
		Vì $(E)$ đi qua điểm $B(2 \sqrt{2} ; 0)$ nên $B(2 \sqrt{2} ; 0)$ là đỉnh của $(E)\Rightarrow a=2\sqrt{2}$.\\
		$A(2 ; \sqrt{2})\in (E)\Leftrightarrow \dfrac{2^2}{\left(2\sqrt{2}\right)^2}+\dfrac{\left(\sqrt{2}\right)^2}{b^2}=1\Leftrightarrow b=2$ (vì $b>0$).\\
		Vậy độ dài trục bé của $(E)$ là $2\cdot 2=4$.
	}
\end{ex}
%%%%%%%%=====Câu 15
\begin{ex}%[0T8B1-2]%[Dự án đề kiểm tra HKII NH22-23- Tin Đạt Trần]%[THPT Trần Phú]
	Sắp xếp năm bạn học sinh An, Bình, Chi, Dũng, Lệ vào một chiếc ghế dài có $5$ chỗ ngồi. Hỏi có bao nhiêu cách sắp xếp sao cho bạn An và bạn Dũng luôn ngồi ở hai đầu ghế?
	\choice
	{$120$}
	{$24$}
	{\True $12$}
	{$16$}
	\loigiai{
		\begin{itemize}
			\item Xếp $2$ bạn An và Dũng ngồi ở hai đầu ghế có $2!$ cách.
			\item Xếp $3$ bạn còn lại ngồi ở giữa An và Dũng có $3!$ cách.
		\end{itemize}
		Theo quy tắc nhân, có $2!\cdot 3!=12$ cách để xếp sao cho bạn An và bạn Dũng luôn ngồi ở hai đầu ghế.
	}
\end{ex}
%%%%%%%%=====Câu 16
\begin{ex}%[0T7B3-2]%[Dự án đề kiểm tra HKII NH22-23- Tin Đạt Trần]%[THPT Trần Phú]
	Tổng tất cả các nghiệm của phương trình: $\sqrt{x^2+5 x-5}=2 x-1$ là
	\choice
	{$-3$}
	{$1$}
	{$-2$}
	{\True $3$}
	\loigiai{
		\begin{eqnarray*}
			&\sqrt{x^2+5 x-5}&=2 x-1\\
			\Rightarrow&x^2+5x-5&=(2x-1)^2\\
			\Rightarrow&3x^2-9x+6&=0\\
			&\Rightarrow\hoac{&x=1\\&x=2.}
		\end{eqnarray*}
		Thử lại, phương trình đã cho có tập nghiệm là $S=\left\{1;2\right\}$.\\
		Tổng các nghiệm của phương trình là $1+2=3$.
	}
\end{ex}
%%%%%%%%=====Câu 17
\begin{ex}%[0T8Y1-1]%[Dự án đề kiểm tra HKII NH22-23- Tin Đạt Trần]%[THPT Trần Phú]
	Giả sử bạn muốn mua một áo sơ mi cỡ $39$ hoặc $40$. Áo cỡ $39$ có $5$ màu khác nhau, áo cỡ $40$ có $4$ màu khác nhau. Hỏi có bao nhiêu sự lựa chọn?
	\choice
	{\True $9$}
	{$1$}
	{$4$}
	{$5$}
	\loigiai{
		\begin{itemize}
			\item \textbf{Trường hợp 1:} Mua một áo sơ mi cỡ $39$ có $5$ cách chọn.
			\item \textbf{Trường hợp 2:} Mua một áo sơ mi cỡ $40$ có $4$ cách chọn.
		\end{itemize}
		Theo quy tắc cộng, để mua một chiếc áo sơ mi cỡ $39$ hoặc $40$ có $5+4=9$ cách chọn.
	}
\end{ex}
%%%%%%%%=====Câu 18
\begin{ex}%[0T9B3-2]%[Dự án đề kiểm tra HKII NH22-23- Tin Đạt Trần]%[THPT Trần Phú]
	Trong mặt phẳng $Oxy$, đường tròn $(C)$ đi qua 3 điểm $O(0 ; 0)$, $A(8 ; 0)$, $B(0 ; 6)$ có phương trình là:
	\choice
	{$(x-4)^2+(y-3)^2=5$}
	{$(x+4)^2+(y+3)^2=25$}
	{\True $(x-4)^2+(y-3)^2=25$}
	{$(x+4)^2+(y+3)^2=5$}
	\loigiai{
		Phương trình đường tròn $(C)$ có dạng $x^2+y^2-2ax-2by+c=0$ (với $a^2+b^2-c>0$).\\
		Ta có $\heva{&O(0;0)\in (C)\\&A(8;0)\in(C)\\&B(0;6)\in (C)}\Leftrightarrow\heva{&c=0\\&64+0-16a+c=0\\&0+36-12b+c=0}\Leftrightarrow \heva{&a=4\\&b=3\\&c=0.}$\\
		$\Rightarrow R=\sqrt{a^2+b^2-c}=\sqrt{4^2+3^2}=5$.\\
		Vậy phương trình đường tròn cần tìm là $(x-4)^2+(y-3)^2=25$.
	}
\end{ex}
%%%%%%%%=====Câu 19
\begin{ex}%[0T9B4-2]%[Dự án đề kiểm tra HKII NH22-23- Tin Đạt Trần]%[THPT Trần Phú]
	Lập phương trình chính tắc của Elip có trục lớn bằng $6$ và tỉ số tiêu cự với độ dài trục lớn bằng $\dfrac{1}{3}$
	\choice
	{\True $\dfrac{x^2}{9}+\dfrac{y^2}{8}=1$}
	{$\dfrac{x^2}{9}+\dfrac{y^2}{3}=1$}
	{$\dfrac{x^2}{6}+\dfrac{y^2}{5}=1$}
	{$\dfrac{x^2}{9}+\dfrac{y^2}{5}=1$}
	\loigiai{
		Elip có trục lớn bằng $6$ suy ra $a=\dfrac{6}{2}=3$.\\
		Tỉ số tiêu cự với độ dài trục lớn bằng $\dfrac{1}{3}\Rightarrow \dfrac{2c}{2a}=\dfrac{1}{3}\Rightarrow c=1$.\\
		Khi đó ta có $b=\sqrt{a^2-c^2}=\sqrt{3^2-1^2}=2\sqrt{2}$.\\
		Phương trình chính tắc của Elip cần tìm là $\dfrac{x^2}{3^2}+\dfrac{y^2}{\left(2\sqrt{2}\right)^2}=1\Leftrightarrow \dfrac{x^2}{9}+\dfrac{y^2}{8}=1$.
	}
\end{ex}
%%%%%%%%=====Câu 20
\begin{ex}%[0T0Y2-2]%[Dự án đề kiểm tra HKII NH22-23- Tin Đạt Trần]%[THPT Trần Phú]
	Cho tập hợp $A=\left\{1 ; 2 ; 3 ; 4 ; 5 ; 6 ; 7\right\}$. Chọn ngẫu nhiên một số từ tập $A$, tính xác suất để số được chọn chia hết cho $3$.
	\choice
	{$\dfrac{2}{3}$}
	{$\dfrac{2}{7}$}
	{$\dfrac{3}{7}$}
	{$\dfrac{5}{7}$}
	\loigiai{
		Số phần tử của không gian mẫu là $7$.\\
		Chọn $1$ số chia hết cho $3$ từ tập $A$ có $2$ cách (số được chọn là $3$ hoặc $6$).\\
		Suy ra xác suất cần tìm là $\dfrac{2}{7}$.
	}
\end{ex}
%%%%%%%%=====Câu 21
\begin{ex}%[0T9K2-5]%[Dự án đề kiểm tra HKII NH22-23- Tin Đạt Trần]%[THPT Trần Phú]
	Trong mặt phẳng với hệ tọa độ $Oxy$, cho hai đường thẳng $\Delta_1\colon3 x-2 y-6=0$ và $\Delta_2\colon 3 x-2 y+3=0$. Tìm điểm $M$ thuộc trục hoành sao cho $M$ cách đều hai đường thẳng đã cho.
	\choice
	{$M(\sqrt{2} ; 0)$}
	{$M\left(0 ; \dfrac{1}{2}\right)$}
	{$M\left(-\dfrac{1}{2} ; 0\right)$}
	{\True $M\left(\dfrac{1}{2} ; 0\right)$}
	\loigiai{
		Gọi $M(m;0)\in Ox$ là điểm thuộc trục hoành cần tìm.\\
		Ta có:
		\allowdisplaybreaks
		\[\begin{aligned}
			&\mathrm{d}(M,\Delta_1)=\mathrm{d}(M,\Delta_2)\\
			&\Leftrightarrow\dfrac{\left|3\cdot m-2\cdot 0-6\right|}{\sqrt{3^2+(-2)^2}}=\dfrac{\left|3\cdot m-2\cdot 0+3\right|}{\sqrt{3^2+(-2)^2}}\\
			&\Leftrightarrow\left|3m-6\right|=\left|3m+3\right|\\
			&\Leftrightarrow\hoac{&3m-6=3m+3\\&3m-6=-3m-3}\\
			&\Leftrightarrow\hoac{&-6=3\text{ (Vô lí)}\\&m=\dfrac{1}{2}}\\
			&\Leftrightarrow m=\dfrac{1}{2}.
		\end{aligned}\]
		Vậy $M\left(\dfrac{1}{2};0\right)$ là điểm cần tìm.
	}
\end{ex}
%%%%%%%%=====Câu 22
\begin{ex}%[0T7B2-1]%[Dự án đề kiểm tra HKII NH22-23- Tin Đạt Trần]%[THPT Trần Phú]
	Số thực dương lớn nhất thỏa mãn $x^2-x-12 \leq 0$ là?
	\choice
	{$2$}
	{$1$}
	{\True $4$}
	{$3$}
	\loigiai{
		Ta có $x^2-x-12 \leq 0\Leftrightarrow -3\leq x\leq 4$.\\
		Suy ra số thực dương lớn nhất thỏa mãn $x^2-x-12 \leq 0$ là $4$.
	}
\end{ex}
%%%%%%%%=====Câu 23
\begin{ex}%[0T8B2-3]%[Dự án đề kiểm tra HKII NH22-23- Tin Đạt Trần]%[THPT Trần Phú]
	Trong mặt phẳng cho một tập hợp gồm $6$ điểm phân biệt. Có bao nhiêu vectơ khác $\overrightarrow{0}$ có điểm đầu và điểm cuối thuộc tập hợp điểm này?
	\choice
	{$1440$}
	{$15$}
	{$12$}
	{\True $30$}
	\loigiai{
		Có tổng cộng $\mathrm{A}_6^2=30$ vectơ khác không có điểm đầu và điểm cuối thuộc tập hợp $6$ điểm trên.
	}
\end{ex}
%%%%%%%%=====Câu 24
\begin{ex}%[0T9Y4-1]%[Dự án đề kiểm tra HKII NH22-23- Tin Đạt Trần]%[THPT Trần Phú]
	Trong mặt phẳng $(Oxy)$, cho elip $(E)$ có phương trình $\dfrac{x^2}{36}+\dfrac{y^2}{16}=1$. Tìm tiêu cự của $(E)$.
	\choice
	{$F_1 F_2=12$}
	{$F_1 F_2=8$}
	{$F_1 F_2=2 \sqrt{5}$}
	{\True $F_1 F_2=4 \sqrt{5}$}
	\loigiai{
		Ta có $(E)\colon \dfrac{x^2}{36}+\dfrac{y^2}{16}=1\Rightarrow \heva{&a=6\\&b=4}\Rightarrow c=\sqrt{a^2-b^2}=\sqrt{6^2-4^2}=2\sqrt{5}$.\\
		Tiêu cự của $(E)$ là $F_1 F_2=2c=4\sqrt{5}$.
	}
\end{ex}
%%%%%%%%=====Câu 25
\begin{ex}%[0T9K3-2]%[Dự án đề kiểm tra HKII NH22-23- Tin Đạt Trần]%[THPT Trần Phú]
Đường tròn $(C)$ đi qua hai điểm $A(1 ; 1)$, $B(5 ; 3)$ và có tâm $I$ thuộc trục hoành có phương trình là
\choice
{\True $(x-4)^2+y^2=10$}
{$(x+4)^2+y^2=10$}
{$(x+4)^2+y^2=\sqrt{10}$}
{$(x-4)^2+y^2=\sqrt{10}$}
\loigiai{
Ta có $I(a;0)\in Ox$ là tâm của đường tròn $(C)$.\\
$IA=IB=R\Leftrightarrow \sqrt{(a-1)^2+1^2}=\sqrt{(a-5)^2+3^2}\Leftrightarrow a=4$.\\
Suy ra $I(4;0)$ và $R=IA=IB=\sqrt{10}$.\\
Phương trình đường tròn $(C)$ là $(C)\colon (x-4)^2+y^2=10$.
}
\end{ex}







\Closesolutionfile{ans}
%\begin{center}
%	\textbf{ĐÁP ÁN}
%	\inputansbox{10}{ans/ans}	
%\end{center}
\begin{center}
	\textbf{PHẦN 2 - TỰ LUẬN}
\end{center}
\begin{bt}%[0T7B3-4]
	Tìm tất cả các giá trị của tham số $ m $ để phương trình $ mx^2-2(m-1)x+m-3=0 $ có nghiệm duy nhất.
	\loigiai{
	\begin{itemize}
		\item \textbf{Trường hợp 1.} Nếu $ m=0 $, phương trình đã cho trở thành $ 2x-3=0\Leftrightarrow x=\dfrac{3}{2} $. Do đó  $ m=0 $ thỏa mãn.
		\item  \textbf{Trường hợp 2.} Nếu $ m\ne 0 $, phương trình đã cho là phương trình bậc hai có $$ \Delta'=(m-1)^2-m(m-3)=m+1. $$
		Phương trình có nghiệm duy nhất khi và chỉ khi $ \Delta'=0\Leftrightarrow m=-1 $.
	\end{itemize}
Vậy với $ m\in \{-1,0\} $ thì phương trình đã cho có nghiệm duy nhất. 
	}
\end{bt}
\begin{bt}%[0T8B3-1]%[Dự án đề kiểm tra HKII NH22-23- Nguyễn Tài Tuệ]%[THPT Trần Phú]
	Sử dụng công thức nhị thức Newton, hãy khai triển biểu thức sau $ (x+3)^4 $.
	\loigiai{
		Theo công thức khai triển nhị thức Newton ta có 
		\begin{eqnarray*}
			(x+3)^4&=& \mathrm{C}^{0}_4 \cdot x^4 \cdot 3^0+\mathrm{C}^{1}_4\cdot x^3 \cdot 3^1+\mathrm{C}^{2}_4\cdot x^2 \cdot 3^2+\mathrm{C}^{3}_4\cdot x^1 \cdot 3^3+\mathrm{C}^{4}_4\cdot x^0\cdot 3^4\\
			&=&x^4+12 x^3+54 x^2+108 x+81.		 
		\end{eqnarray*}
	}
\end{bt}
\begin{bt}%[0T7B3-2]%[Dự án đề kiểm tra HKII NH22-23- Nguyễn Tài Tuệ]%[THPT Trần Phú]
	Giải phương trình $ \sqrt{3x^2+5x-13}=x+1 $.
	\loigiai{
		Ta có 
		\begin{eqnarray*}
			&&\sqrt{3x^2+5x-13}=x+1 \\
			&\Rightarrow& 3x^2+5x-13 =(x+1)^2 \\
			&\Rightarrow&2x^2+3x-14=0\\
			&\Rightarrow&\hoac{&x=-\dfrac{7}{2}\\&x=2.}
		\end{eqnarray*}
	Thử lại, ta thấy $ x=2 $ thỏa mãn.\\
	Vậy phương trình đã cho có nghiệm là $ x=2 $.
	}
\end{bt}
\begin{bt}%[0T9B3-3]%[Dự án đề kiểm tra HKII NH22-23- Nguyễn Tài Tuệ]%[THPT Trần Phú]
	Viết phương trình tiếp tuyến $ d $ của đường tròn $ (C)\colon x^2+y^2=5 $ tại điểm $ M(1;2) $.
	\loigiai{
		Đường tròn $ (C)\colon x^2+y^2=5 $ có tâm $ O(0;0) $ và bán kính $ R=\sqrt{5} $.\\
		Ta có $ \overrightarrow{OM}=(1;2) $.\\
		Phương trình tiếp tuyến tại $ M $ đi qua $ M $ nhận véc-tơ  $ \overrightarrow{OM} $ làm véc-tơ pháp tuyến có phương trình $ 1\cdot (x-1)+2\cdot  (y-2)=0\Leftrightarrow x+2y-5=0 $.
	}
\end{bt}

\begin{bt}%[0H9B2-2]%[0H9B3-2]%[Dự án đề kiểm tra HKII NH22-23- Sauluoi3105]%[THPT Trần Phú]
	Cho $A(2;-1)$; $B(4 ; 3)$; $C(-5 ; 0)$
	\begin{listEX}
		\item Viết phương trình tổng quát đường thẳng đi qua $2$ điểm $A$ và $C$.
		\item Viết phương trình đường tròn tâm $B$ và tiếp xúc với đường thẳng $AC$.
	\end{listEX}
	\loigiai{
		\begin{listEX}
			\item Đường thẳng $AC$ có véc-tơ chỉ phương $\overrightarrow{A C}=(-7 ; 1)$.\\
			Suy ra đường thẳng $AC$ qua $A$, có véc-tơ pháp tuyến $\vec{n}=(1 ; 7)$ có phương trình
			\[1\cdot (x-2)+7\cdot (y+1)=0 \Leftrightarrow x+7y+5=0 \]
			\item Ta có đường tròn tâm $B$ và tiếp xúc với đường thẳng $AC$ nên
			\[R=\mathrm{d}\,(B,AC)=\dfrac{|4+7 \cdot 3+5|}{\sqrt{1^2+7^2}}=\dfrac{30}{\sqrt{50}}=3 \sqrt{2}.\]
			Suy ra đường tròn $(C)$ có tâm $B$, bán kính $R=3 \sqrt{2}$ có phương trình
			\[(C)\colon (x-4)^2+(y-3)^2=18.\]
		\end{listEX}
	}
\end{bt}

\begin{bt}%[0D7K3-2]%[Dự án đề kiểm tra HKII NH22-23- Sauluoi3105]%[THPT Trần Phú]
	Cho tam giác $ABC$ vuông tại $A$ có $AB$ ngắn hơn $AC$ là $2\,\mathrm{cm}$, đặt $A B=x\, \mathrm{cm} $.
	\begin{listEX}
		\item Biểu diễn độ dài cạnh huyền $BC$ theo $x$.
		\item Biết chu vi của tam giác $ABC$ là $24\,\mathrm{cm}$. Tìm độ dài $3$ cạnh của tam giác đó.
	\end{listEX}
	\loigiai{
		\begin{listEX}
			\item Gọi $A B=x \Rightarrow A C=x+2$.\\
			Ta có tam giác $ABC$ vuông tại $A$ nên 
			\allowdisplaybreaks{
				\begin{eqnarray*}
				&&	BC^2 =  AB^2+AC^2 \\
				&\Leftrightarrow&	BC^2 =  x^2+(x+2)^2 \\
				&\Rightarrow&	BC  =  \sqrt{2x^2+4x+4}.
				\end{eqnarray*}
			}
			\item Ta có chu vi tam giác $ABC$ bằng $24$ nên
			\allowdisplaybreaks{
				\begin{eqnarray*}
					&& A B+A C+B C=24\\
					&\Leftrightarrow& x+x+2+\sqrt{2x^2+4x+4}=24\\
					&\Leftrightarrow& \sqrt{2x^2+4x+4}=22-2x\\
					&\Leftrightarrow& -2 x^2+92 x-480=0\\
					&\Leftrightarrow& \hoac{&x=6 \\&x=40.}
				\end{eqnarray*}
			}
			Thử lại, nhận $x=6$, loại $x=40$.\\
			Vậy độ dài ba cạnh là $AB=6$ cm, $AC=8$ cm và $B C=10$ cm.
		\end{listEX}
	}
\end{bt}

