\de{ĐỀ THI HỌC KỲ II NĂM HỌC 2022-2023}{THPT Phạm Phú Thứ}


\begin{bt}%[0D7B3-2]%[Dự án đề kiểm tra HKII NH22-23- Nguyễn Vương Hiển]%[Phạm Phú Thứ]
Giải các phương trình sau
	\begin{enumEX}[a)]{2}
		\item $\sqrt{5-3x+x^2}=\sqrt{1-4x+4x^2}$.
		\item $-3x+1=\sqrt{4x^2+4x}$.
	\end{enumEX}
\loigiai{
\begin{enumEX}[a)]{1}
	\item $\sqrt{5-3x+x^2}=\sqrt{1-4x+4x^2}$\\
	Bình phương hai vế của phương trình, ta được
	\allowdisplaybreaks
	$$\begin{aligned}[t]
	&5-3x+x^2=1-4x+4x^2\Rightarrow&3x^2-x-4=0\Rightarrow&\hoac{&x=-1\\&x=\dfrac{4}{3}.}
	\end{aligned}$$
Thử lại:
\begin{itemize}
	\item $x=-1$ thỏa mãn phương trình.
	\item $x=\dfrac{4}{3}$ thỏa mãn phương trình.
\end{itemize}
Vậy tập nghiệm của phương trình là $S=\left\{-1;-\dfrac{4}{3}\right\}$.
	\item $-3x+1=\sqrt{4x^2+4x}$\\
	Bình phương hai vế của phương trình, ta được
	\allowdisplaybreaks
	$$\begin{aligned}[t]
		&9x^2-6x+1=4x^2+4x\Rightarrow&5x^2-10x+1=0\Rightarrow&\hoac{&x=\dfrac{5+2\sqrt{5}}{2}\\&x=\dfrac{5-2\sqrt{5}}{2}.}
	\end{aligned}$$
	Thử lại:
	\begin{itemize}
		\item $x=\dfrac{5+2\sqrt{5}}{2}$ không thỏa mãn phương trình.
		\item $x=\dfrac{5-2\sqrt{5}}{2}$ thỏa mãn phương trình.
	\end{itemize}
	Vậy tập nghiệm của phương trình là $S=\left\{\dfrac{5-2\sqrt{5}}{2}\right\}$.
\end{enumEX}
}
\end{bt}
\begin{bt}%[0D8Y1-1]%[Dự án đề kiểm tra HKII NH22-23- Nguyễn Vương Hiển]%[Phạm Phú Thứ]
Một nhóm tình nguyện viên gồm $4$ học sinh lớp $10$A, $5$ học sinh lớp $10$B và $6$ học sinh lớp $10$C. Để tham gia một công việc tình nguyện, nhóm có bao nhiêu cách cử ra $1$ thành viên của nhóm?
\loigiai{
\begin{itemize}
	\item Số cách cử $1$ thành viên trong $4$ học sinh lớp $10$A là $4$ cách.
	\item Số cách cử $1$ thành viên trong $5$ học sinh lớp $10$B là $5$ cách.
	\item Số cách cử $1$ thành viên trong $6$ học sinh lớp $10$C là $6$ cách.
\end{itemize}
Theo quy tắc cộng, ta có $4+5+6=15$ cách.
}
\end{bt}
\begin{bt}%[0D8B1-3]%[Dự án đề kiểm tra HKII NH22-23- Nguyễn Vương Hiển]%[Phạm Phú Thứ]
Từ các chữ số $0$, $1$, $2$, $3$, $4$, $5$, $6$, có thể lập được bao nhiêu số tự nhiên chẵn có $3$ chữ số đôi một khác nhau?
\loigiai{
Gọi số tự nhiên chẵn có $3$ chữ số đôi một khác nhau là $n=\overline{abc}\,(a\ne b\ne c)$.
\begin{itemize}
\item \textbf{Trường hợp 1:} $c=0$.\\
Số cách chọn $a$, $b$ là $\mathrm{A}_6^2=30$ cách.\\
Do đó, ta có $30$ số.
\item \textbf{Trường hợp 2:} $c\in\{2;4;6\}$.\\
Chọn $c$ có $3$ cách.\\
Chọn $a$ có $5$ cách ($a\ne0\ne c$).\\
Chọn $b$ có $5$ cách ($b\ne a\ne c$).\\
Theo quy tắc nhân, ta có $3\cdot5\cdot5=75$ số. 
\end{itemize}
Vậy số tự nhiên chẵn có $3$ chữ số đôi một khác nhau là $n=30+75=105$ số.
}
\end{bt}
\begin{bt}%[0H9Y2-4]%[0H9Y2-5]%[Dự án đề kiểm tra HKII NH22-23- Nguyễn Vương Hiển]%[Phạm Phú Thứ]
Trong mặt phẳng tọa độ $Oxy$, cho điểm $A(3 ;-1)$, đường thẳng $\Delta_1\colon x-3y+15=0$ và đường thẳng $\Delta_2\colon 2x-y-100=0$.
\begin{enumerate}
\item Tính góc giữa đường thẳng $\Delta_1$ và $\Delta_2$.
\item Tính khoảng cách từ điểm $A$ đến đường thẳng $\Delta_2$.
\end{enumerate}
\loigiai{
\begin{enumerate}
	\item $\Delta_1$ có véc-tơ pháp tuyến $\overrightarrow{n}_1=(1;-3)$ và $\Delta_2$ có véc-tơ pháp tuyến $\overrightarrow{n}_2=(2;-1)$.\\
	$\cos(\Delta_1;\Delta_2)=\left|\cos(\overrightarrow{n}_1;\overrightarrow{n}_2)\right|=\dfrac{|1\cdot2+(-3)\cdot(-1)|}{\sqrt{1^2+(-3)^2}\cdot\sqrt{2^2+(-1)^2}}=\dfrac{\sqrt{2}}{2}$.\\
	Suy ra $(\Delta_1;\Delta_2)=45^\circ$.
	\item $\mathrm{d}(A,\Delta_2)=\dfrac{|2\cdot3+1-100|}{\sqrt{2^2+(-1)^2}}=\dfrac{93\sqrt{5}}{5}$.
\end{enumerate}	
}
\end{bt}
\begin{bt}%[0H9B2-2]%[Dự án đề kiểm tra HKII NH22-23- Nguyễn Vương Hiển]%[Phạm Phú Thứ]
Trong mặt phẳng tọa độ $Oxy$, cho điểm $N(-2 ;-3)$ và đường thẳng $d\colon-3x+5y+9=0$. Viết phương trình của đường thẳng $\Delta$ đi qua $N$ và $\Delta$ vuông góc với đường thẳng $d$.
\loigiai{
\begin{itemize}
	\item $d$ có véc-tơ pháp tuyến $\overrightarrow{n}=(-3;5)$.
	\item Vì $\Delta\perp d$ nên $\Delta$ có véc-tơ pháp tuyến $\overrightarrow{m}=(5;3)$.
	\item Phương trình của đường thẳng $\Delta$ là
	$$5(x+2)+3(y+3)=0\Leftrightarrow5x+3y+19=0.$$
\end{itemize}
}
\end{bt}

\begin{bt}%[0H9K3-3]%[Dự án đề kiểm tra HKII NH22-23- Hieu Phan]%[PHẠM PHÚ THỨ - HCM]
    \begin{enumerate}
        \item[1)] Trong mặt phẳng tọa độ $ Oxy $, cho tam giác $ ABC $ với $ A(1;0) $, $ B(3;-1) $, $ C(4;1) $. Viết phương trình đường tròn $ (C) $ ngoại tiếp tam giác $ ABC $.
        \item[2)] Trong mặt phẳng tọa độ $ Oxy $, cho đường tròn $ (C_1) \colon x^2+y^2-x-3=0$. Viết phương trình tiếp tuyến $ \Delta $ của $ (C_1) $, biết $ \Delta $ song song với đường thẳng $ \Delta' \colon 3x+2y+5=0 $.
    \end{enumerate}
    \loigiai{
    \begin{enumerate}
        \item[1)] Phương trình đường tròn $ (C) $ có dạng 
        $ x^2+y^2-2ax-2by+c=0 $. \\
        Đường tròn $ (C) $ đi qua $ 3 $ điểm $ A(1;0) $, $ B(3;-1) $, $ C(4;1) $ nên ta có hệ phương trình
        $$ \heva{&1^2+0^2-2\cdot a\cdot 1-2\cdot b\cdot 0 +c =0  \\ & 3^2+(-1)^2-6a-2\cdot b\cdot (-1) +c =0\\ & 4^2+1^2-2\cdot a\cdot 4-2\cdot b\cdot 1 +c =0} \Leftrightarrow \heva{& -2a+c=-1 \\ & -6a+2b+c=-10\\& -8a-2b+c=-17}\Leftrightarrow \heva{& a=\dfrac{5}{2} \\ &b= \dfrac{1}{2}\\&c=4 .}$$
        Vậy $ (C)\colon x^2+y^2-6x-2y+4=0 $.
        \item[2)] Vì $ \Delta \parallel \Delta'\colon 3x+2y+5=0$ nên $ \Delta\colon  3x+2y+c=0~(c \ne 5)$.\\
        Ta có $ (C_1) \colon x^2+y^2-x-3=0 \Rightarrow a=\dfrac{1}{2},b=0,c=-3$.\\
        Do đó tâm $ I\left(\dfrac{1}{2};0\right) $ và $ R=\sqrt{\dfrac{1}{4}+3}=\dfrac{\sqrt{13}}{2} $.\\
        Mặt khác \begin{eqnarray*}
            \mathrm{d}(I,\Delta)=R&\Leftrightarrow& \dfrac{\bigg|3\cdot \dfrac{1}{2}+2\cdot 0 +c\bigg|}{\sqrt{3^2+2^2}}= \dfrac{\sqrt{13}}{2}\\
            &\Leftrightarrow &\bigg|\dfrac{3}{2}+c\bigg|=\dfrac{13}{2}\Leftrightarrow \hoac{& c=5 \text{ (loại)}\\ &c=-8. }
        \end{eqnarray*}
    Vậy $ \Delta\colon 3x+2y-8=0 $.
    \end{enumerate}
}
\end{bt}
\begin{bt}%[0H9B4-1]%[Dự án đề kiểm tra HKII NH22-23- Hieu Phan]%[PHẠM PHÚ THỨ - HCM]
    Trong mặt phẳng tọa độ $ Oxy $, cho elip có phương trình $ \dfrac{x^2}{169}+\dfrac{y^2}{25} =1$. Tìm tọa độ các tiêu điểm, độ dài trục lớn và độ dài trục nhỏ của elip.

    \loigiai{
    Ta có $ \dfrac{x^2}{169}+\dfrac{y^2}{25} =1 \Rightarrow a=13, b=5 \Rightarrow c=\sqrt{169 -25}=12$.\\
    Suy ra $ (E) $ có các tiêu điểm $ F_1(-12;0) $ và $ F_2(12;0) $.\\
    Độ dài trục lớn $ 2a=26 $ và độ dài trục nhỏ $ 2b=10 $.
}
\end{bt}
\begin{bt}%[0D7K3-2]%[Dự án đề kiểm tra HKII NH22-23- Hieu Phan]%[PHẠM PHÚ THỨ - HCM]
   \immini
   {
        Mặt cắt đứng của cột cây số trên quốc lộ có dạng nửa hình tròn ở phía trên và phía dưới có dạng hình chữ nhật (xem hình bên). Biết rằng đường kính của nửa hình tròn cũng là cạnh trên của hình chữ nhật và đường chéo của hình chữ nhật có độ dài là $ 66 $ cm. Tính kích thước của hình chữ nhật, biết diện tích của phần hình chữ nhật bằng $\dfrac{10}{3}$ diện tích của phần nửa hình tròn. Lấy $ \pi =3{,}14$ và làm tròn kết quả đến số thập phân thứ $ 2 $.
   }
   {
        \begin{tikzpicture}[line join = round, line cap = round,>=stealth,thick,scale=1]
           \coordinate (A) at (0,0) ;
           \coordinate (B) at (-3.5,0);     
           \coordinate (C) at (-3.5,-4.5);    
           \coordinate (D) at (0,-4.5);    
           \draw (A) arc (0:180:1.75cm);
           \draw (A)--(B)--(C)--(D)--cycle (A)--(C);
           \path (0.5,0)--(-3,-4.5)node[midway,align=center,sloped]{$ 66 $ cm};
           \draw (-2,-5) ;node{\text{mặt cắt của cột cây số}};
       \end{tikzpicture}
   }
    \loigiai{
    Gọi $ x ~\mathrm{cm}~(x> 0)$ là đường kính của đường tròn.\\
    Ta có diện tích của phần hình chữ nhật là $ x\cdot\sqrt{66^2-(x)^2} ~ \mathrm{cm^2} $.\\
    Diện tích của phần nửa hình tròn là $\dfrac{1}{2}\cdot \pi\cdot \left(\dfrac{x}{2}\right)^2=\dfrac{1}{8}\cdot \pi x^2~  \mathrm{cm^2}$.\\
    Theo đề bài ta có  \begin{eqnarray*}
       x\cdot\sqrt{66^2-(x)^2} =\dfrac{10}{3}\cdot\dfrac{1}{8}\cdot \pi x^2&\Leftrightarrow &\sqrt{66^2-(x)^2} =\dfrac{5}{12}\cdot 3{,}14 \cdot x\\
         &\Leftrightarrow&  \left(\dfrac{25}{144}\cdot 3{,}14^2+1\right)x^2=4356\\
         & \Leftrightarrow& x \approx 40{,}08 .
    \end{eqnarray*}
Vậy kích thước của hình chữ nhật là $ 2\left(40{,}08+ \sqrt{66^2-(40{,}08 )^2}\right) \approx 185{,}03 ~\mathrm{cm} $.
}
\end{bt}