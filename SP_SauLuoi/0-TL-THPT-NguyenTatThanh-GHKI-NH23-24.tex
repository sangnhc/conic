
\de{ĐỀ THI GIỮA HỌC KỲ I NĂM HỌC 2023-2024}{THPT NGUYỄN TẤT THÀNH }

\begin{bt}%[0D1B1-5]%[Dự án đề kiểm tra Toán 10 GHKI NH23-24- Tacgia]%[THPT NGUYỄN TẤT THÀNH - Tp HCM]
	Bạn An phát biểu hai mệnh đề sau %\lq\lq xxx\rq\rq~
	\begin{enumEX}{2}
		\item  $A:$ \lq\lq $\forall x \in \mathbb{Q}, x \in \mathbb{Z} $\rq\rq~.% thầy xem lại phần này 
		\item $B:$ \lq\lq$\exists x \in \mathbb{R}, x^2-x+1 < 0$\rq\rq~.
	\end{enumEX}
	Bạn Bình nhận xét đây là các mệnh đề sai và sửa lại thành các mệnh đề đúng. Bình đã viết lại như thế nào?
	\loigiai{
		\begin{enumEX}{2}
			\item $A:$ \lq\lq $\forall x \in \mathbb{Z}, x \in \mathbb{Q} $\rq\rq~.
			\item $B:$ \lq\lq $ \forall x \in \mathbb{R}, x^2-x+1 \geq 0 $\rq\rq~.
		\end{enumEX}
	}
\end{bt}

%Câu6
\begin{bt}%[0D1B2-1]%[Dự án đề kiểm tra Toán 10 GHKI NH23-24- Tacgia]%[THPT  NGUYỄN TẤT THÀNH - Tp HCM]
	Cho hai tập hợp $C=\left\{x \in \mathbb{R} \mid 3 x^2-2 x-1=0\right\}$, $D=\{x \in \mathbb{Z} \mid-2<x \leq 2\}$.\\
	Liệt kê các phần tử của các tập hợp $C$, $D$, $C \cap D$, $C \cup D$, $C \backslash D$, $D \backslash C$.        
	\loigiai{    % lời giải sai D
		$C =\left\{-\dfrac{1 }{3};1\right\}$, $D =\left\{-1;0;1;2\right\}$.\\
		$ C \cap D = \left\{1\right\}$, $ C \cup D = \left\{-1;-\dfrac{1 }{3};0;1;2 \right\} $.\\
		$ C \backslash D = \left\{-\dfrac{1 }{3}\right\}$, $D \backslash C = \left\{-1;0;2\right\} $.
	}
\end{bt}

%Câu 7
\begin{bt}%[0D1B2-2]%[Dự án đề kiểm tra Toán 10 GHKI NH23-24- Tacgia]%[THPT NGUYỄN TẤT THÀNH - Tp HCM]
	Tìm các tập hợp sau 
	\begin{enumEX}{2}
		\item $(-3 ;+\infty) \cap[-4 ; 8)$.
		\item $(-\infty ;-1) \cup(-2 ; 3]$.
		\item $(-\infty ;-2) \backslash[-4 ;+\infty)$.
		\item $C_{\mathbb{R}}(-1 ; 2]$.
	\end{enumEX}
	\loigiai{
		\begin{enumEX}{2}
			\item $(-3 ;+\infty) \cap[-4 ; 8) =(-3 ;8)$. 
			\item $(-\infty ;-1) \cup(-2 ; 3] = (-\infty;3]$.
			\item $(-\infty ;-2) \backslash[-4 ;+\infty)=(-\infty;-4)$.
			\item $C_{\mathbb{R}}(-1 ; 2] = (-\infty ;-1] \cup (2;+\infty)$.
		\end{enumEX}
	}
\end{bt}

% Câu 4

\begin{bt}%[0D2K2-2]%[Dự án đề kiểm tra Toán 10 GHKI NH23-24- Tacgia]%[THPT NGUYỄN TẤT THÀNH - Tp HCM]
	Hình vẽ bên dưới biểu diễn miền nghiệm của bất phương trình nào ?
	\begin{center}
		\begin{tikzpicture}[scale=1.5, line join=round, line cap=round, >=stealth]
			\draw[->] (-1.5,0) -- (2.5,0)node[below]{$x$};
			\draw[->] (0,-2.5) -- (0,2)node[left]{$y$};
			\foreach \x in {-1,0,1,2} \draw[shift={(\x,0)},color=black] (0pt,2pt) -- (0pt,-2pt) node[below] {\footnotesize $\x$};
			\foreach \y in {-2,-1,1} \draw[shift={(0,\y)},color=black] (2pt,0pt) -- (-2pt,0pt) node[left] {\footnotesize $\y$};
			\draw[line width=1.0pt,dashed,domain=-0.25:2] plot(\x,{2*\x -2});
			\fill[pattern=north east lines] (-0.25,-2.5)--(2,2)--(2.5,2)--(2.5,-2.5)--cycle;
			
		\end{tikzpicture}
		
	\end{center}
	\loigiai{
		Đồ thị hàm số đi qua 2 điểm có toạ độ là  $A(1,0)$ và $B(0,-2)$.\\
		Từ đó ta viết được phương trình đường thẳng $(d)$  đi qua 2 điểm $A$ và $B$ là $y =2x-2$.\\
		Miền nghiệm theo hình vẽ nằm dưới đường thẳng $(d)$.\\
		Suy ra miền nghiệm biểu diễn của bất phương trình $y-2x+2>0$ (không kể bờ chứa đường thẳng $(d)$). 
		
	}
\end{bt}

%Câu 1...........................
\begin{bt}%0H1K2-2%[Dự án đề kiểm tra Toán 11 GHKI NH23-24- NguyenHuuDuc]%[THPT - Tp HCM]
($2$ điểm) Cho $\triangle ABC$ có $ AB=26 $, $ AC=28 $, $ BC=30 $. Tính diện tích $\triangle ABC$, chiều cao $ CH $, bán kính đường tròn ngoại tiếp và bán kính đường tròn nội tiếp $\triangle ABC$.
\loigiai{
Ta có: $\triangle ABC$ có $ AB=26=c $, $ AC=28=b $, $ BC=30=a $.%Cho tam giác $ ABC $ không phải là tam giác vuông.
\begin{itemize}
	\item Tính diện tích $\triangle ABC$.\\
	Áp dụng công thức He-ron trong $\triangle ABC$, ta có $S_{\triangle ABC}=\sqrt{p(p-a)(p-b)(p-c)}$.\\
	Với  $a$, $b$, $c$ lần lượt ba cạnh $\triangle ABC$ và $ p$ là nữa chu vi $ p=\dfrac{a+b+c}{2}=\dfrac{26+28+30}{2}=42 $.\\
	Vậy $S_{\triangle ABC}=\sqrt{42(42-26)(42-28)(42-30)}=336$(đvdt).
	\item Bán kính đường tròn ngoại tiếp $\triangle ABC$.\\
	Áp dụng công thức $S_{\triangle ABC}=\dfrac{abc}{4R} \Rightarrow R=\dfrac{abc}{4S_{\triangle ABC}}=\dfrac{26\cdot28\cdot30}{4\cdot336}=\dfrac{65}{4}$.
	\item Bán kính đường tròn nội tiếp $\triangle ABC$.\\
	Áp dụng công thức $S_{\triangle ABC}=pr\Rightarrow r=\dfrac{S_{\triangle ABC}}{p}=\dfrac{336}{42}=8$.
\end{itemize}}
\end{bt}
%Câu 2...........................
\begin{bt}%0H1K2-2%[Dự án đề kiểm tra Toán 11 GHKI NH23-24- NguyenHuuDuc]%[THPT - Tp HCM]
	($1$ điểm) Tính diện tích tứ giác $ABCD$, biết $ AB=50$m, $AD=70$m, $\widehat{ABD}=60^{\circ}$, $ BC=60 $m, $\widehat{CBD}=45^{\circ}$.
	\loigiai{
	\immini
	{Xét $\triangle ABD$ và $\triangle BCD$, hiển nhiên\\
		 $S_{ACBD}=S_{\triangle ABD}+S_{\triangle BCD}$.
		\begin{itemize}
		\item Áp dụng định lí côsin cho $\triangle ABD$ đối với góc $\widehat{B}$, ta có\\
		$ AD^2=BA^2+BD^2-2BA.BD\cos \widehat{ABD} $ hay\\
		$70^2=50^2+BD^2-2.50.BD\cos \widehat{60^{\circ}} $.\\
		$\Rightarrow BD^2-50BD-2400=0$\\
		$\Leftrightarrow BD=80$m (nhận), $ BD=-30 $m (loại).
		\end{itemize}
	}
	{
	\begin{tikzpicture}
		\pgfmathsetmacro \a{3*sqrt(3)};
		\def \b{3}
		\path (0,0) coordinate (A) (60:\b) coordinate (D);
		\coordinate (B) at ($(D)+(-30:\a)$);
		\coordinate (C) at ($(D)+(20:\b)$);
		\draw (A)--(B)--(C)--(D)--cycle;
		\draw (D)--(B);
		%\path ($(A)!(C)!(B)$) coordinate (H);% ($(H)!1!90:(A)$) coordinate (M);
%		\draw (C)--(A)--(H)--(C);
		\foreach \t/\g in {A/180,C/90,B/-10,D/100} \draw[fill=white] (\t) circle (1.5pt) ($(\t)+(\g:3mm)$)node{$\t$};
	%	\path (A)--(H) node[midway,above]{$x$}
	%	(A)--(B) node[midway,below]{$c$}
	%	(A)--(C) node[midway,above]{$b$}
	%	(C)--(B) node[midway,above]{$a$}
	%	(C)--(H) node[midway,right]{$d$}
	%	(B)--(H) node[midway,above]{$c-x$};
	%	\foreach \x/\y/\z in {C/H/B}\pic[draw,angle radius=1.5mm]{right angle=\x--\y--\z}; % Đánh dấu góc vuông
	%	\draw (2.6,-0.5) node[below] {\textit{Hình 2.01}};
	\end{tikzpicture}
	}\begin{itemize}
	\item Xét $\triangle ABD$, ta có $S_{\triangle ABD}=\dfrac{1}{2}BA\cdot BD\cdot\sin \widehat{ABD}=\dfrac{1}{2}50\cdot 80\cdot\sin \widehat{60^{\circ}}=1000\sqrt{3}$(m$^2$).
	\tagEX{1}
	Mặc khác $S_{\triangle BCD}=\dfrac{1}{2}BC\cdot BD\cdot\sin \widehat{DBC}=\dfrac{1}{2}60\cdot 80\cdot\sin \widehat{45^{\circ}}=1200\sqrt{2}$(m$^2$).
	\tagEX{2}
	Từ $(1)$ và $(2)$ suy ra $S_{ACBD}=S_{\triangle ABD}+S_{\triangle BCD}=1000\sqrt{3}+1200\sqrt{2}\approx 3429{,}1 $(m$^2$).	
\end{itemize}}
\end{bt}
%Câu 3...........................
\begin{bt}%[0H2K2-5]%[Dự án đề kiểm tra Toán 11 GHKI NH23-24- NguyenHuuDuc]%[THPT - Tp HCM]
	($1$ điểm) Cho hình thoi $ABCD$ có cạnh bằng $2$, $\widehat{ABC}=60^{\circ}$. Tính $\left|\vec{AB}+\vec{AD}+\vec{BC}\right|$, $\left|\vec{AB}+\vec{AD}\right|$. 
	\loigiai{
	\immini
	{Do $ABCD$ là hình thoi mà $\widehat{ABC}=60^{\circ}$ nên $ \triangle ABC$ và $ \triangle ACD$ là các tam giác đều cạnh bằng $2$. Gọi $ M $ trung điểm $ DC $. \\
	$ \Rightarrow AM $ là đường trung tuyến của $ \triangle ACD$, $ AM=\sqrt{AD^2-DM^2}=\sqrt{2^2-1^2}=\sqrt{3} $.
		\begin{itemize}
			\item $\left|\vec{AB}+\vec{AD}+\vec{BC}\right|=\left|\vec{AC}+\vec{BC}\right|=\left|\vec{AC}+\vec{AD}\right|=\left|2\vec{AM}\right|=2\sqrt{3}$.
		\end{itemize}
	}
	{
		\begin{tikzpicture}[declare function={gA=30;a=3;}]
			\path
			(0,0) coordinate (D)
			(gA:a) coordinate (A)
			(-gA:a) coordinate (C)
			($(C)+(A)-(D)$) coordinate (B)
			($(D)!.5!(C)$) coordinate (M)
			\foreach \x/\y/\z in {A/B/E,B/C/F,C/D/G,D/A/H}{($(\x)!.5!(\y)$) coordinate (\z)}
			;
			%\foreach \x/\y in {A/E,E/B,B/F,F/C,C/G,G/D,D/H,H/A}{\khmotgach(\x,\y)}
			\draw (A)--(B)--(C)--(D)--cycle
			%(A)--(C) (B)--(D)
			(A)--(C) (B)--(D) (A)--(M);
			\foreach \d/\g in {A/90, B/0, C/-90, D/180, M/-140}
			\path[draw=black,fill=white] (\d) circle(1pt) node[shift={(\g:7pt)}] {$\d$};
		\end{tikzpicture}
	}\begin{itemize}
		\item Xét $\triangle ACD$ là tam giác đều nên $ AC=2 $, ta có $\left|\vec{AB}+\vec{AD}\right|=\left|\vec{AC}\right|=2$.	
\end{itemize}}
\end{bt}

