\de{ĐỀ THI GIỮA HỌC KỲ I NĂM HỌC 2023-2024}{THPT Bà Điểm - Tp HCM}
%Câu 1...........................
\begin{bt}%[1D1V4-2]%[Dự án đề kiểm tra Toán 11 GHKI NH23-24- Tổng Nguyễn]%[THPT Bà Điểm - Tp HCM]
	Tìm tập xác định của hàm số $y=\dfrac{\cot 2x}{2\sin x-1}$.
	\loigiai{Hàm số $y=\dfrac{\cot 2x}{2\sin x-1}$
		xác định khi 
		\[\heva{&2\sin x -1 \neq 0\\&\sin 2x \neq 0}\Leftrightarrow \heva{& x \neq \dfrac{\pi}{6}+k2\pi\\& x\neq \dfrac{5\pi}{6}+k2\pi\\& 2x \neq k\pi}\Leftrightarrow \heva{& x \neq \dfrac{\pi}{6}+k2\pi\\& x\neq \dfrac{5\pi}{6}+k2\pi\\& x \neq \dfrac{k\pi}{2}}, \  \ k \in \mathbb{Z}.\]
		Vậy tập xác định của hàm số đã cho là $\mathscr{D}=\mathbb{R} \setminus \left\{\dfrac{\pi}{6}+k2\pi, \ \dfrac{5\pi}{6}+k2\pi, \  \dfrac{k\pi}{2}\  \text{với } \ k \in \mathbb{Z}\right\}$.
		
	}
\end{bt}
%Câu 2...........................
\begin{bt}%[1D1V5-5]%[Dự án đề kiểm tra Toán 11 GHKI NH23-24- Tổng Nguyễn]%[THPT Bà Điểm - Tp HCM]
	Giải các phương trình sau
	\begin{listEX}[3]
		\item $\cos \left(x+\dfrac{5\pi}{6}\right)=-1$; 
		\item $\sin \left(2x-\dfrac{\pi}{6}\right)-\cos x =0$; 
		\item $\cos^2 3x+\cos^2 5x =1$.
	\end{listEX}
	\loigiai{ 
		\begin{listEX}[1]
			\item $\cos \left(x+\dfrac{5\pi}{6}\right)=-1$.\\
			Ta có 
			\[\cos \left(x+\dfrac{5\pi}{6}\right)=-1\Leftrightarrow x+\dfrac{5\pi}{6} =\pi +k2\pi\Leftrightarrow x =\dfrac{\pi}{6}+k2\pi, \ k \in \mathbb{Z}.\]
			Vậy phương trình đã cho  có nghiệm  $x =\dfrac{\pi}{6}+k2\pi, \ k \in \mathbb{Z}$.
			\item $\sin \left(2x-\dfrac{\pi}{6}\right)-\cos x =0$.\\
			Ta có 
			\begin{eqnarray*}
				\sin \left(2x-\dfrac{\pi}{6}\right)-\cos x =0
				&\Leftrightarrow&\sin \left(2x-\dfrac{\pi}{6}\right)=\cos x\\
				&\Leftrightarrow& \sin \left(2x-\dfrac{\pi}{6}\right)=\sin\left(\dfrac{\pi}{2}-x\right)\\
				&\Leftrightarrow& \hoac{&2x-\dfrac{\pi}{6}=\dfrac{\pi}{2}-x+k2\pi\\&2x-\dfrac{\pi}{6}=\pi-\dfrac{\pi}{2}+x+k2\pi}\\
				&\Leftrightarrow &\hoac{&x=\dfrac{2\pi}{9}+\dfrac{k2\pi}{3}\\&x=\dfrac{2\pi}{3}+k2\pi}, \ k \in \mathbb{Z}.
			\end{eqnarray*}
			Vậy phương trình đã cho có nghiệm $x=\dfrac{2\pi}{9}+\dfrac{k2\pi}{3}, \ k \in \mathbb{Z} $ và $x=\dfrac{2\pi}{3}+k2\pi, \ k \in \mathbb{Z}$.
			\item $\cos^2 3x+\cos^2 5x =1$.\\
			Ta có 
			\begin{eqnarray*}
				\cos^2 3x+\cos^2 5x =1 &\Leftrightarrow& \dfrac{1+\cos 6x}{2}+\dfrac{1+\cos 10x}{2}=1\\
				&\Leftrightarrow& \cos 6x +\cos 10=0\\
				&\Leftrightarrow& \cos 6x = \cos \left(\pi -10x\right)\\
				&\Leftrightarrow& \hoac{&6x=\pi-10x+k2\pi\\&6x=-\pi+10x+k2\pi}\\
				&\Leftrightarrow&\hoac{&x=\dfrac{\pi}{16}+\dfrac{k\pi}{8}\\&x=\dfrac{\pi}{4}-\dfrac{k\pi}{2}}, \ k \in \mathbb{Z}.
			\end{eqnarray*}
			Vậy phương trình đã cho có nghiệm $x=\dfrac{\pi}{16}+\dfrac{k\pi}{8},\ k \in \mathbb{Z}$ và $ x=\dfrac{\pi}{4}-\dfrac{k\pi}{2}, \ k \in \mathbb{Z}$.
		\end{listEX}
	}
\end{bt}
%Câu 3...........................
\begin{bt}%[1D2H2-4]%[Dự án đề kiểm tra Toán 11 GHKI NH23-24- Tổng Nguyễn]%[THPT Bà Điểm - Tp HCM]
	Tìm số hạng đầu tiên và công sai của cấp số cộng $(u_n)$ biết $\heva{&u_2-u_3+u_5=7\\&u_1+u_6=12.}$
	\loigiai{ Cấp số cộng $(u_n)$ có số hạng đầu tiên $u_1$ và công sai $d$. 
		\begin{eqnarray*}
			\heva{&u_2-u_3+u_5=7\\&u_1+u_6=12} &\Leftrightarrow& \heva{&(u_1+d)-(u_1+2d)+(u_1+4d)=7\\&u_1+(u_1+5d)=12}\\
			&\Leftrightarrow& \heva{&u_1+3d=7\\&2u_1+5d=12}\\
			&\Leftrightarrow& \heva{&u_1=1\\&d=2.}
		\end{eqnarray*}
		Vậy cấp số cộng $(u_n)$ có số hạng đầu tiên $u_1=1$ và công sai $d=2$.
		
	}
\end{bt}
%Câu 4...........................
\begin{bt}%[1D2V2-7]%[Dự án đề kiểm tra Toán 11 GHKI NH23-24- Phạm Phương]%[THPT Bà Điểm - Tp HCM]
	%(1,0 điểm)
	Anh Nam được nhận vào làm việc ở một công ty về công nghệ với mức lương khởi điểm là $100$ triệu đồng một năm. Công ty sẽ tăng thêm lương cho anh Nam mỗi năm là $20$ triệu đồng. Tính tổng số tiền lương mà anh Nam nhận được sau $10$ năm làm việc cho công ty đó.
	\loigiai{
	Số tiền anh Nam nhận được qua các năm lập thành một cấp số cộng với $u_1=100$ và công sai $d=20$.
	\\
	Vậy tổng số tiền lương mà anh Nam nhận được sau $10$ năm làm việc cho công ty đó là
	\allowdisplaybreaks{\begin{eqnarray*}
			S_{10}&= &nu_1+\dfrac{10(10-1)d}{2} \\
			&= & 10\cdot 100+\dfrac{10(10-1)\cdot 20}{2}= 1\,900 \text{ (triệu đồng).}
	\end{eqnarray*} }	
}
\end{bt}
%Câu 5...........................
\begin{bt}%[1H4C3-6]%[Dự án đề kiểm tra Toán 11 GHKI NH23-24- Phạm Phương]%[THPT Bà Điểm-Tp HCM]
	%(3,0 điểm)
	Cho hình chóp $S.ABCD$ có đáy $ABCD$ là hình bình hành. Gọi $M$, $N$, $K$ lần lượt là trung điểm của $AB$, $AD$, $SC$.
	\begin{enumerate}
		\item Tìm giao tuyến của $(SMN)$ và $(SBD)$.
		\item Chứng minh $SA$ song song với $(KBD)$.
		\item Gọi $G$ là trọng tâm của tam giác $SBD$. Mặt phẳng $(MNG)$ cắt $SC$ tại điểm $H$. Tính tỉ số $\dfrac{SH}{SC}$.
	\end{enumerate}
	\loigiai{
	\begin{center}
		\begin{tikzpicture}[scale=1,>=stealth, font=\footnotesize, line join=round, line cap=round]
			\coordinate (A) at (0,0);
			\coordinate (B) at (-1.5,-1.8);
			\coordinate (D) at (4,0);
			\coordinate (C) at ($(B)+(D)-(A)$);
			\coordinate (S) at ($(A)+(-0.3,2.6)$);
			\path ($(A)!0.5!(B)$) coordinate (M)
			($(A)!0.5!(D)$) coordinate (N)
			($(S)!0.5!(C)$) coordinate (K)
			($(S)+(M)-(N)$) coordinate (d)
			(intersection of A--C and B--D) coordinate (O)
			($(S)!2/3!(O)$) coordinate (G)
			(intersection of A--C and M--N) coordinate (E)
			(intersection of S--C and E--G) coordinate (H)
			;
			\draw[red] ($(S)!-1.5cm!(d)$)--(d)node[above]{$d$};
			\draw (S)--(B)--(C)--(D)--cycle
			(S)--(B) (S)--(C)
			(B)--(K)--(D);
			\draw[dashed] (A)--(B) (A)--(D) (S)--(A)
			(M)--(S)--(N)--cycle
			(B)--(D)(A)--(C)
			(K)--(O)--(S)
			(M)--(G)--(N)
			(E)--(H);
			\foreach \p/\r in {S/90,A/180,B/-90,C/-45,D/0,M/180,N/-90,K/75,O/-90,G/170,H/80,E/-100}
			\fill (\p) circle (1pt) node[shift={(\r:3mm)}]{$\p$};
		\end{tikzpicture}
		\hspace*{1cm}
		\begin{tikzpicture}[scale=1,>=stealth, font=\footnotesize,line join=round,line cap=round]
			\coordinate (A) at (0,0);
			\coordinate (C) at ($(A)+(3.5,0)$);
			\coordinate (S) at (1,4);
			\path ($(A)!0.5!(C)$) coordinate (O)
			($(A)!0.5!(O)$) coordinate (E)
			($(S)!2/3!(O)$) coordinate (G)
			(intersection of S--C and E--G) coordinate (H)
			($(S)!0.5!(G)$) coordinate (I)
			($(S)!0.5!(H)$) coordinate (J)
			;
			\draw (S)--(A)--(C)--cycle
			(S)--(O)
			(A)--(J)(E)--(H)			
			;
			\foreach \p/\r in {A/-145,S/90,C/45,E/-90,O/-90,H/45,J/45,I/-160,G/-20}
			\fill (\p) circle (1pt) node[shift={(\r:3mm)}]{$\p$};
		\end{tikzpicture}
	\end{center}
	\begin{enumerate}
		\item Ta có $\heva{&S\in(SMN)\\&S\in(SBD)}
			\Rightarrow \{S\}\in (SMN) \cap (SBD)$.\hfill (1)
		\\
		Lại có $MN\parallel BD$ (vì $MN$ là đường trung bình của $\triangle ABD$). \hfill (2)
		\\
		Mà $MN \subset (SMN)$, $BD\subset (SBD)$.\hfill (3)
		\\
		Từ (1), (2), (3) ta suy ra giao tuyến của $(SMN)$ và $(SBD)$ là đường thẳng $d$ qua $S$ và song song với $MN$ và $BD$.
		\item Trong hình bình hành $ABCD$ gọi $O=AC \cap BD$ suy ra $O$ là trung điểm của $AC$.
		\\
		$\Rightarrow KO \parallel SA$ (vì $KO$ là đường trung bình của $\triangle SAC$). \hfill (4)
		\\
		Mà $O\in BD \subset (KBD)$ nên $KO\subset (SBD)$ và $SA\not\subset (KBD)$.  \hfill (5)
		\\
		Từ (4), (5) suy ra $SA \parallel (KBD)$.
		\item Trong $(ABCD)$ gọi $E=MN\cap AC$. Khi đó $(MNG)$ cắt $SC$ tại điểm $H=EG\cap SC$.
		\\
		Gọi $I$, $J$ lần lượt là trung điểm của $SG$, $SH$.
		\\
		Ta có $\heva{&IJ\parallel HG\\& IA\parallel GE}\Rightarrow A,I,J$ thẳng hàng.
		\\
		Xét $\triangle ACJ$ có $EH\parallel AJ \Rightarrow \dfrac{CH}{HJ}=\dfrac{CE}{EA}=3 \Rightarrow CH=3HJ$.
		\\
		Lại có $SH=2HJ$ nên $SC=5HJ$. Vậy $\dfrac{SH}{SC}=\dfrac{2}{5}$.
	\end{enumerate}
	}
\end{bt}
