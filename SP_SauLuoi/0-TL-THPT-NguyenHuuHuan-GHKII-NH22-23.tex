
\de{ĐỀ THI GIỮA HỌC KỲ II NĂM HỌC 2022-2023}{THPT Nguyễn Hữu Huân}

\begin{bt}%[0T7B1-1]%[GHKII-2022-2023-TRƯỜNG THPT NGUYỄN HỮU HUÂN-Thehung Nguyen]
	 Cho đồ thị của hàm số bậc hai $f(x)$ bên dưới. Tìm tập nghiệm của bất phương trình tương ứng.
	 \begin{center}
	 	\begin{tikzpicture}[>=stealth,line join=round,line cap=round,font=\footnotesize,scale=1]
	 		\draw[->] (-2.5,0)--(1.7,0)node[above right]{$x$};
	 		\draw[->] (0,-2.5)--(0,1)node[left]{$y$};
	 		\fill (0,0)node[below left]{ $O$};
	 		\draw[black,samples=150,smooth,domain=-2.2:1.2] plot(\x,{(\x)^2+(\x)-2});
	 		\draw (-2,0)node[below left]{$-2$} (1,0)node[below right]{$1$}
	 		(-0.5,-3)node {$\mathrm{a})\,f(x) \geq 0.$}
	 		;
	 	\end{tikzpicture}
 		\begin{tikzpicture}[>=stealth,line join=round,line cap=round,font=\footnotesize,scale=1]
 		\draw[->] (-.5,0)--(4,0)node[above right]{$x$};
 		\draw[->] (0,-2.5)--(0,1)node[left]{$y$};
 		\fill (0,0)node[below left]{ $O$};
 		\draw[black,samples=150,smooth,domain=.45:3.5] plot(\x,{-1*(\x)^2+4*(\x)-4});
 		\draw (2,0)node[above]{$2$}
 		(2,-3)node {$\mathrm{b})\,f(x)<0.$}
 		;
 	\end{tikzpicture}
	 \end{center}
	 
\loigiai{
	\begin{enumerate}
		\item Dựa vào đồ thị $f(x) \geq 0\Leftrightarrow\hoac{&x\le -2\\&x\ge 1.}$\\
		Vậy $S=(-\infty;-2]\cup[1;+\infty)$ là tập nghiệm của bất phương trình.
		\item Dựa vào đồ thị $f(x)<0\Leftrightarrow x\neq 2$.\\
		Vậy $S=\mathbb{R}\setminus\{2\}$ là tập nghiệm của bất phương trình.
	\end{enumerate}
}
\end{bt} 

\begin{bt}%[0T7B2-1]%[GHKII-2022-2023-TRƯỜNG THPT NGUYỄN HỮU HUÂN-Thehung Nguyen]
  Định $m$ để bất phương trình $-x^2+2(m+1) x-2 m-2 \leq 0$ có tập nghiệm là $\mathbb{R}$.
\loigiai{
	Bất phương trình bậc hai có $a=-1<0$, để thỏa yêu cầu bài toán thì	
	\begin{eqnarray*}
	\Delta'\le 0	& \Leftrightarrow& (m+1)^2-(-1)(-2m-2)\le 0\Leftrightarrow m^2-1\le 0\\
		&\Leftrightarrow & |m|\le 1\Leftrightarrow -1\le m\le 1.
	\end{eqnarray*}
	}
\end{bt}
\begin{bt}%[0T7B3-2]%[GHKII-2022-2023-TRƯỜNG THPT NGUYỄN HỮU HUÂN-Thehung Nguyen]
Giải phương trình $\sqrt{15-2 x-x^2}-x=-3$.
	\loigiai{Ta có 
		\allowdisplaybreaks
		\begin{eqnarray*}
			&&\sqrt{15-2 x-x^2}-x=-3\\
			&\Leftrightarrow&\sqrt{15-2 x-x^2}=x-3\\
			&\Leftrightarrow&\heva{&x-3\ge 0\\&15-2x-x^2=(x-3)^2}\\
			&\Leftrightarrow&\heva{&x\ge 3\\&2x^2-4x-6=0}\\
			&\Leftrightarrow&\heva{&x\ge 3\\&\hoac{&x=-1\\&x=3}}\\
			&\Leftrightarrow&x=3.
		\end{eqnarray*}
	Vậy $x=3$ là nghiệm của phương trình.
	}
\end{bt} 

\begin{bt}%[0T8B2-1]%[GHKII-2022-2023-TRƯỜNG THPT NGUYỄN HỮU HUÂN-Thehung Nguyen]
Từ các chữ số $0;\,1;\,2;\,3;\,4;\,5;\,6$ có thể lập được bao nhiêu số tự nhiên lẻ có $3$ chữ số khác nhau đôi một và một trong hai chữ số đầu tiên là số $3$?
	\loigiai{
		Gọi số có ba chữ số là $\overline{abc}$. Theo yêu cầu bài toán ta có hai trường hợp sau
		\begin{itemize}
			\item Trường hợp $\overline{3bc}$\\
			 Chọn $c\in\{1;5\}$ có hai cách chọn.\\
			 Chọn $b\in\{0;1;2;4;5;6\}\setminus\{c\}$ có năm cách chọn.\\
			 Do đó có $2\cdot 5=10$ số cho trường hợp này.
			 \item  Trường hợp $\overline{a3c}$\\
			 Chọn $c\in\{1;5\}$ có hai cách chọn.\\
			 Chọn $a\in\{1;2;4;5;6\}\setminus\{c\}$ có bốn cách chọn.\\
			 Do đó có $2\cdot 4=8$ số cho trường hợp này.
		\end{itemize}
		Vậy có $10+8=18$ số thỏa yêu cầu bài toán.
	}
\end{bt}
\begin{bt}%[0T9B2-5]%[GHKII-2022-2023-TRƯỜNG THPT NGUYỄN HỮU HUÂN-Thehung Nguyen]
	Trong mặt phẳng $O x y$, cho tam giác $A B C$ với tọa độ các đỉnh $A(1; 2)$, $B(3 ;-4)$, $C(0 ;-1)$.
	\begin{enumerate}
		\item Tìm côsin của góc giữa hai véc-tơ $\overrightarrow{AB}$, $\overrightarrow{A C}$.
		\item Viết phương trình tổng quát của cạnh $B C$.
		\item Cho đường thẳng $d\colon\heva{&x=-4-2 t\\&y=-3+t}\,\,(t\in \mathbb{R})$. Tìm tọa độ điểm $M$ trên đường thẳng $d$ biết điểm $M$ cách đều $A$ và $B$.
	\end{enumerate}
	\loigiai{
		\begin{enumerate}
			\item Ta có $\overrightarrow{AB}=(2;-6)$, $\overrightarrow{AC}=(-1;-3)$.\\
			Suy ra $\cos\left(\overrightarrow{AB},\overrightarrow{AC}\right)=\dfrac{\overrightarrow{AB}\cdot\overrightarrow{AC}}{|\overrightarrow{AB}|\cdot|\overrightarrow{AC}|}=\dfrac{2\cdot(-1)+(-6)\cdot(-3)}{\sqrt{40}\cdot\sqrt{10}}=\dfrac{16}{20}=\dfrac{4}{5}$.
			\item Lại có $\overrightarrow{BC}=(-3;3)$. Suy ra véc-tơ pháp tuyến của đường thẳng $BC$ là $\vec{n}=(1,1)$.\\
			Do đó phương trình tổng quát đường thẳng $BC$ là
			$$x-3+y+4=0\Leftrightarrow x+y+1=0.$$
			\item Gọi $M(-4-2t;-3+t)$ thuộc $d$.\\
			Theo đề bài 
				\allowdisplaybreaks
				\begin{eqnarray*}
					&&MA=MB\\
					&\Leftrightarrow&(1+4+2t)^2+(2+3-t)^2=(3+4+2t)^2+(-4+3-t)^2\\
					&\Leftrightarrow&4t^2+20t+25+t^2-10t+25=4t^2+28t+49+t^2+2t+1\\
					&\Leftrightarrow&20t=0\\
					&\Leftrightarrow&t=0.
				\end{eqnarray*}
			Vậy điểm $M(-4;-3)$ thỏa mãn yêu cầu bài toán.
		\end{enumerate}
		
	}
\end{bt} 

\begin{bt}%[0T9B3-3]%[GHKII-2022-2023-TRƯỜNG THPT NGUYỄN HỮU HUÂN-Thehung Nguyen] 
	\begin{enumerate}
		\item Lập phương trình của đường tròn $(\mathrm{C})$ có tâm $I(3 ;-2)$ và đi qua điểm $M(-1 ; 1)$.
		\item Cho đường tròn $(C_1)\colon(x+2)^2+(y+2)^2=25$. Viết phương trình tiếp tuyến của $(C_1)$ biết tiếp tuyến song song với đường thẳng $d\colon 4x+3y-11=0$.
	\end{enumerate}
	\loigiai{
		\begin{enumerate}
			\item 
			Ta có $R=MI=\sqrt{(-4)^2+3^2}=5$ với $\overrightarrow{MI}=(-4;3)$.\\
			 Phương trình của đường tròn $(C)$ có tâm $I(3 ;-2)$ và $R=5$ là
			$$(x-3)^2+(y+2)^2=25.$$
			\item Đường tròn $(C_1)$ có tâm $I_1(-2;-2)$ và bán kính $R_1=5$.\\
			Phương trình tiếp tuyến của $(C_1)$ song song với $d$ có dạng $$d_1\colon 4x+3y+C=0\,\,(C\neq -11).$$
			Điều kiện tiếp xúc
			$$\mathrm{d}(I_1,d_1)=R_1\Leftrightarrow\dfrac{|-8-6+C|}{\sqrt{25}}=5\Leftrightarrow\hoac{&C-14=25\\&C-14=-25}\Leftrightarrow\hoac{&C=39\,\text{(nhận)}\\&C=-11\,\text{(loại)}.}$$
			Vậy tiếp tuyến cần tìm là $3x+4y+39=0$.
		\end{enumerate}
	}
\end{bt}
\begin{bt}%[0T7K2-2]%[GHKII-2022-2023-TRƯỜNG THPT NGUYỄN HỮU HUÂN-Thehung Nguyen] 
	Một đoạn kẽm dài $6$ mét được cắt ra làm hai phần: một phần uốn thành hình vuông có cạnh bằng $x$ mét, phần còn lại uốn thành hình chữ nhật có chiều dài gấp $3$ lần chiều rộng. Tìm điều kiện của $x$ để diện tích hình chữ nhật lớn hơn diện tích hình vuông (Làm tròn kết quả đến hàng phần chục).
	\loigiai{
		Đoạn kẽm uốn thành hình vuông có cạnh là $x$ (m) nên đoạn kẽm đó là $4x$ (m).\\
		 Điều kiện $0<x<\dfrac{3}{2}$.\\
		 Diện tích hình vuông là $x^2$.\\
		 Gọi $y$ (m) là chiều rộng hình chữ nhật, suy ra $2(y+3y)=6-4x$ hay $y=\dfrac{3}{4}-\dfrac{1}{2}x$.\\
		 Diện tích hình chữ nhật là $3y\cdot y=3\cdot\left(\dfrac{3}{4}-\dfrac{1}{2}x\right)^2$.\\
		 Theo đề bài ta có bất phương trình
		 \allowdisplaybreaks
		 \begin{eqnarray*}
		 	&&3\cdot\left(\dfrac{3}{4}-\dfrac{1}{2}x\right)^2>x^2\\
		 	&\Leftrightarrow&\dfrac{27}{16}-\dfrac{9}{4}x+\dfrac{3}{4}x^2>x^2\\
		 	&\Leftrightarrow&4x^2+36x-27<0\\
		 	&\Leftrightarrow&-9{,}696\approx\dfrac{-9-6\sqrt{3}}{2}<x<\dfrac{-9+6\sqrt{3}}{2}\approx 0{,}696.
		 \end{eqnarray*}
		Kết hợp với điều kiện $0<x<0{,}7$ (đơn vị của $x$ là mét) thỏa yêu cầu bài toán.
	}
\end{bt} 

