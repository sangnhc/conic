
\de{ĐỀ THI ÔN TẬP  HỌC KỲ II NĂM HỌC 2022-2023}{THPT Nguyễn Thượng Hiền 10TH2}


\begin{bt}%[0T7Y3-2]%Nguyễn Sĩ Đạt%THPT Nguyễn Thượng Hiền 10TH2
	Giải phương trình $\sqrt{3x^2+7x-1}=\sqrt{6x^2+6x-11}$.
	\loigiai
	{Bình phương hai vế của phương trình đã cho, ta có
		\begin{eqnarray*}
			&&\sqrt{3x^2+7x-1}=\sqrt{6x^2+6x-11}\\
			&\Rightarrow &3x^2+7x-1=6x^2+6x-11\\
			&\Rightarrow &3x^2-x-10=0\\
			&\Rightarrow &\hoac{&x=2\\&x=-\dfrac{5}{3}.}
		\end{eqnarray*}
		Thử lại, ta có $x=2$ là nghiệm của phương trình.
	}
\end{bt}

\begin{bt}%[0T7B1-1]%Nguyễn Sĩ Đạt%THPT Nguyễn Thượng Hiền 10TH2
	Tìm $m$ để $f(x)=mx^2-2x+m$ luôn âm với mọi $x$ thuộc $\mathbb{R}$.
	\loigiai{
		TH1: $a=0\Leftrightarrow m=0$, thay $m=0$ vào $f(x)$, khi  đó
		$f(x)<0 \Leftrightarrow -2x<0 \Leftrightarrow x>0 $.\\
		Suy ra loại $m=0$.\\
		TH2: $a \ne 0 \Leftrightarrow m \ne 0$.\\
		Khi đó
		\begin{eqnarray*}
			&& f(x) < 0 ,\,\, \forall x \in \mathbb{R}\\& \Leftrightarrow & \heva{& m<0\\& \Delta'<0}\\& \Leftrightarrow &  \heva{& m<0\\& 1-m^2<0}\\& \Leftrightarrow & \heva{& m \in (-\infty,0)\\& m \in (-\infty;-1) \cup (1;+\infty)}\\& \Leftrightarrow & m \in (-\infty,-1).   
		\end{eqnarray*}
		Vậy với $m \in (-\infty,-1)$ thì $f(x)=mx^2-2x+m$ luôn âm với mọi $x$ thuộc $\mathbb{R}$.
	}
\end{bt}

\begin{bt}%[0T8Y3-1]%Nguyễn Sĩ Đạt%THPT Nguyễn Thượng Hiền 10TH2
	Khai triển biểu thức sau $(x+3y)^4$.
	\loigiai{
		Ta có $\begin{aligned}[t]
			(x+3y)^4 &=x^4+4\cdot x^3\cdot (3y)+6\cdot x^2 \cdot (3y)^2+4\cdot x\cdot (3y)^3 + (3y)^4 \\
			&=x^4+12x^3y+54x^2y^2+108xy^3+81y^4.
		\end{aligned}$
	}
\end{bt}

\begin{bt}%[0T8B2-1] %Nguyễn Sĩ Đạt%THPT Nguyễn Thượng Hiền 10TH2
	Tổ $1$ có $10$ học sinh trong đó có $3$ học sinh nam và $7$ học sinh nữ. Có bao nhiêu cách chọn $3$ học sinh làm lớp trưởng, lớp phó và thủ quỹ? (Mỗi bạn chỉ làm $1$ nhiệm vụ).
	\loigiai{
		Mỗi cách chọn $3$ trong $10$ học sinh làm lớp trưởng, lớp phó và thủ quỹ là một chỉnh hợp chập $3$ của $10$ học sinh.\\
		Do đó có $\mathrm{A}_{10}^4=720$ cách chọn thỏa đề.}
\end{bt}

\begin{bt}%[0T0B2-2]%Nguyễn Sĩ Đạt%THPT Nguyễn Thượng Hiền 10TH2
	Trong hộp có $3$ viên bi xanh và $5$ viên bi đỏ. Lấy ngẫu viên trong hộp $3$ viên bi. Tính xác suất của biến cố $A\colon$\lq\lq Lấy ra được $3$ viên bi màu đỏ\rq\rq.
	\loigiai{
		Số phần tử của không gian mẫu là $n(\Omega) = \mathrm{C}^3_8$.\\
		Số cách lấy được $3$ viên bi đỏ là $n(A)=\mathrm{C}^3_5$.\\
		Xác suất để xảy ra biến cố A là: $P(A)=\dfrac{n(A)}{n(\Omega)}=\dfrac{5}{28}$.
	}
\end{bt}

	\begin{bt}%[0T9B1-3]%Võ Thị Thùy Trang%THPT Nguyễn Thượng Hiền 10TH2
	Trong mặt phẳng $Oxy$ cho $A(3;3), B(1;7), C(-4;-8)$. Tìm toạ độ điểm $D$ sao cho $A BCD$ là hình bình hành.
	\loigiai{
		Ta có $\overrightarrow{AB}=\left(-2;4\right)$, $\overrightarrow{DC}=\left(-4-x_D;-8-y_D\right)$.\\
		$ABCD$ là hình bình hành khi và chỉ khi $\overrightarrow{AB}=\overrightarrow{DC} \Leftrightarrow \heva{&-2=-4-x_D\\& 4=-8-y_D} \Leftrightarrow \heva{&x_D=-2\\&y_D=-12.}$\\
		Vậy $D=\left(-2;-12\right)$.
	}
\end{bt}
\begin{bt}%[0T9B2-2]%Võ Thị Thùy Trang%THPT Nguyễn Thượng Hiền 10TH2
	Cho ba điểm $A(-3 ; 2)$, $B(3 ; 4)$, $C(9 ;-2)$. Viết phương trình tổng quát và phương trình tham số của đường trung tuyến $AM$.
	\loigiai{
		Do $M$ là trung điểm của $BC$ nên\\ $x_M=\dfrac{x_B+x_C}{2}=\dfrac{3+9}{2}=6$; $y_M=\dfrac{y_B+y_C}{2}=\dfrac{4-2}{2}=1$.\\ Suy ra $M=(6;1)$.\\
		Ta có $\overrightarrow{AM}=(9;-1)$ là véc-tơ chỉ phương của đường trung tuyến $AM$.\\
		Do đó đường trung tuyến $AM$ có một véc-tơ pháp tuyến là $\overrightarrow{n}=(1;9)$.\\
		Phương trình tham số của đường trung tuyến $AM$ là $\heva{&x=-3+9t\\&y=2-t.}$\\
		Phương trình tổng quát của đường trung tuyến $AM$ là $x+9y-15=0$.
	}
\end{bt}

%%=====Bài 8
\begin{bt}%[0T9K3-6]%Thầy Hoá%THPT Nguyễn Thượng Hiền 10TH2
Một cái cổng hình bán nguyệt rộng $8{,}4$ m, cao $4{,}2$ m như hình vẽ. Mặt đường dưới cổng được chia làm hai làn cho xe ra vào
\begin{center}
	\begin{tikzpicture}[scale=.8, font=\footnotesize, line join=round, line cap=round,>=stealth]
		\definecolor{capri}{rgb}{0.0, 0.75, 1.0}
		\definecolor{orange(colorwheel)}{rgb}{1.0, 0.5, 0.0}
		\definecolor{kellygreen}{rgb}{0.3, 0.73, 0.09}
		\fill[color=capri] (-5,2)--(-5,5)--(5,5)--(5,2);
		\fill[color=gray!40!white] (-3.5,0)--(-3.5,3.05)--(3.5,3.05)--(3.5,0);
		\def\Q{  (3.5,3)..controls +(120:1) and +(60:1) .. (-3.5,3);}
		\draw[color=white, line width=5] \Q;
		\fill[color=kellygreen] (-5,-1.5)--(-5,2)--(5,2)--(5,-1.5);
		\fill[color=orange(colorwheel)] (-5,0)--(-5,3)--(5,3)--(5,0);
		\pattern[pattern color=white, pattern=bricks] (-5,0)--(-5,3)--(5,3)--(5,0);
		\fill[gray!50!white] (3.5,0)--(3.5,3)..controls +(120:1) and +(60:1) .. (-3.5,3)--(-3.5,0)--cycle;
		\fill[white] (1,4.5)..controls +(120:.8) and +(30:1) .. (2.5,4.4)
		..controls +(80:.5) and +(120:1) .. (4.8,4.3)
		..controls +(-60:1) and +(-120:.6) .. (3.5,4)
		..controls +(-60:.7) and +(-120:.6) .. (2,4.1)
		..controls +(-60:.6) and +(-120:.6) .. (-1,4.3)
		..controls +(60:.4) and +(30:.2)  .. (1,4.5);
		\draw[color=white, line width=2.5] (-5,3)--(-3.5,3)--(-3.5,0) (3.5,0)--(3.5,3)--(5,3);
		\fill[color=gray!80!white, line width=2] (3,0) arc (0:180:3 cm and 3cm)--cycle;
		\draw[color=cyan, line width=2] (3,0) arc (0:180:3 cm and 3cm)--cycle;
		\fill[color=kellygreen, line width=2,yshift=.4cm] (2.5,0) arc (0:180:2.5 cm and 2.5cm)--cycle;
		\draw[color=gray, line width=2,yshift=.4cm] (2.5,0) arc (0:180:2.5 cm and 2.5cm)--cycle;
		\fill[color=capri, line width=2,yshift=1.93cm] (1.914,0) arc (40:140:2.5 cm and 2.6cm)--cycle;
		\fill[color=white] (-.6,2)--(-5.4,-2)--(5.4,-2)--(.6,2)--cycle;
		\fill[color=gray!80!white] (-.4,2)--(-4.2,-1.5)--(4.2,-1.5)--(.4,2)--cycle;
		\fill[color=white](-.2,-1.1)--(.2,-1.1)--(.3,-1.5)--(-.3,-1.5)--cycle (-.15,.7)--(.15,.7)--(.25,0)--(-.25,0)--cycle (-.1,2)--(.1,2)--(.2,1.5)--(-.2,1.5)--cycle;
		\draw[color=red, line width=1.5,->] (0,0)--(3,0);
		\draw[color=red, line width=1.5,->] (0,0)--(-3,0);
		\node[color=red] at (0,-.5){$8{,}4$ m};
		\node[color=red] at (0,1.2){$4{,}2$ m};
		\draw[color=cyan, line width=1.5,->] (0,0.9)--(0,0);
		\draw[color=cyan, line width=1.5,->] (0,1.5)--(0,3);
	\end{tikzpicture}
\end{center}
\begin{enumerate}
	\item Viết phương trình mô phỏng cái cổng.
	\item Một chiếc xe tải rộng $2{,}2$ m, cao $2{,}6$ m đi đúng làn đường quy định có thể đi qua cổng mà không làm hư hỏng cổng hay không?
\end{enumerate}
\loigiai{
\immini{
\begin{enumerate}
	\item Chọn hệ trục tọa độ có gốc tọa độ $O(0,0)$ là tâm của hình bán nguyệt.\\
	Hình bán nguyệt có dạng nửa đường tròn tâm $O$, bán kính $R=4{,}2$ m.\\
	Phương trình mô phỏng cái cổng:
	\[\heva{&x^2+y^2=\left(4{,}2\right)^2\\&y\geq 0}\Leftrightarrow \heva{&x^2+y^2=17{,}64\\&y\geq 0.} \]
\end{enumerate}
}{
\begin{tikzpicture}[>=stealth,line join=round,line cap=round,font=\footnotesize,scale=0.7]
	\def\r{4.2};
	\path
	(0,0) coordinate (O)
	(0:\r) coordinate (C)
	(180:\r) coordinate (B)
	(90:\r) coordinate (A)
	(0:2.2) coordinate (M)
	(90:2.6) coordinate (N)
	($(M)+(N)-(O)$) coordinate (D)
	({atan(13/11)}:\r) coordinate (E)
	;
	\draw[->] (-\r-1,0)--(\r+1,0)node[below right]{\scriptsize $x$};
	\draw[->] (0,-1)--(0,\r+1)node[above right]{\scriptsize $y$};
	\draw (C) arc(0:180:\r)
	(B)--(C) (M)--(D)--(N) (O)--(E)
	;
	\foreach \x/\g in {B/-135,O/-135,C/-45,A/45,M/-90,N/180,E/45,D/-45}\fill[black](\x) circle (1pt) +(\g:3mm) node {$\x$};
\end{tikzpicture}
}
\begin{enumerate}
	\item[b)] Giả sử xe tải đi làn đường bên phải, vị trí xe thể hiện ở hình chữ nhật $OMDN$ (như hình vẽ). Khi đó ta có
	\[OD=\sqrt{OM^2+ON^2}=\sqrt{2{,}2^2+2{,}6^2}\approx 3{,}4\ (\mathrm{m}). \]
	Vì $OD<OE=R$ nên xe tải có thể đi qua mà không làm hư hỏng cổng.
\end{enumerate}
}
\end{bt}

%%=====Bài 9
\begin{bt}%[0T9T4-1]%Thầy Hoá%THPT Nguyễn Thượng Hiền 10TH2
Mặt Trăng chuyển động quanh Trái Đất theo một quỹ đạo là một elip mà Trái Đất là một tiêu điểm. Elip có chiều dài trục lớn và trục nhỏ lần lượt là $769\,266 \mathrm{~km}$ và $768\,106 \mathrm{~km}$. Tính khoảng cách ngắn nhất và khoảng cách dài nhất từ tâm Trái Đất đến tâm Mặt Trăng, biết rằng các khoảng cách đó đạt được khi Trái Đất và Mặt Trăng nằm trên trục lớn của elip.
\loigiai{
\immini{
Giả sử Trái Đất ở tiêu điểm $F_1$.\\
Ta có $\heva{&2a=769\,266\\&2b=768\,106}\Leftrightarrow\heva{&a=384\,633\\&b=384\,503.}$\\
Khi đó
\[c=\sqrt{a^2-b^2}=\sqrt{384\,633^2-384\,503^2}\approx 9\,999{,}38. \]
Khoảng cách ngắn nhất từ tâm Trái Đất đến Mặt Trăng là
\[A_1F_1=a-c\approx 384\,633-9\,999{,}38=374\,633{,}62\ (\mathrm{km}). \]
Khoảng cách dài nhất từ tâm Trái Đất đến Mặt Trăng là
\[A_2F_1=a+c\approx 384\,633+9\,999{,}38=394\,632{,}38\ (\mathrm{km}). \]
}{
\begin{tikzpicture}[>=stealth,line join=round,line cap=round,font=\footnotesize,scale=0.7]
	\def\a{3.84633};
	\def\b{3.84503};
	\def\c{2.1115};
	\path
	(0,0) coordinate (O)
	(0:\a) coordinate (A_2)
	(180:\a) coordinate (A_1)
	(0:\c) coordinate (F_2)
	(180:\c) coordinate (F_1)
	(130:\a cm and \b cm) coordinate (MT)
	;
	\draw[->] (-\a-1,0)--(\a+1,0)node[below right]{\scriptsize $x$};
	\draw[->] (0,-\b-1)--(0,\b+1)node[above right]{\scriptsize $y$};
	\draw (A_2) arc(0:360:\a cm and \b cm)
	;
	\foreach \x/\g in {A_1/-135,O/-135,A_2/-45,F_2/-90,F_1/-90,MT/130}\fill[black](\x) circle (1pt) +(\g:3mm) node {$\x$};
	\fill[shape=ball,red] (MT) circle (3pt);
	\fill[shape=ball,red] (F_1) circle (3pt);
\end{tikzpicture}
}
}
\end{bt}

