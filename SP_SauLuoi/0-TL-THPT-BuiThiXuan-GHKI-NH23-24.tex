\de{ĐỀ THI GIỮA HỌC KỲ I NĂM HỌC 2023-2024}{TRƯỜNG THPT BÙI THỊ XUÂN }
\begin{bt}%[0D1H3-1][0D1H3-2]%[Dự án đề kiểm tra Toán 10 GHKI NH23-24- Viết Tường]%[THPT Bùi Thị Xuân]
	Cho hai tập hợp $A=\{1;2;3;4;5\}$ và $B=\left \{x\in\mathbb{Z}\,|\,-3\le x<4 \right \}$. Hãy xác định $A\cup B$, $A\cap B$, $A\setminus B$, $B\setminus A$.
	\loigiai
	{
		Ta có $B=\{-3;-2;-1;0;1;2;3\}$. Khi đó
		\begin{itemize}
			\item $A\cup B=\{-3;-2;-1;0;1;2;3;4;5\}$.
			\item $A\cap B=\{1;2;3\}$.
			\item $A\setminus B=\{4;5\}$.
			\item $B\setminus A=\{-3;-2;-1;0\}$.
		\end{itemize}
	}
\end{bt}

\begin{bt}%[0D1V3-5]%[Dự án đề kiểm tra Toán 10 GHKI NH23-24- Viết Tường]%[THPT Bùi Thị Xuân]
	Lớp $10$A có $20$ bạn thích nhảy dance sport và $15$ bạn thích yoga. Trong đó có $5$ bạn thích cả hai môn dance sport và yoga. Hỏi có bao nhiêu bạn của lớp $10$A thích ít nhất một trong hai môn dance sport, yoga?
	\loigiai
	{Gọi $A$, $B$ lần lượt là tập hợp các bạn thích nhảy dance sport và bạn thích yoga.\\
	Theo giả thiết ta có $n(A)=20$, $n(B)=15$ và $n(A\cap B)=5$.\\
		Số bạn của lớp $10$A thích ít nhất một trong hai môn dance sport, yoga là
		 \[ n(A\cup B)=n(A)+n(B)- n(A\cap B)=20+15-5=30\text{ (bạn)}.\]
	}
\end{bt}

\begin{bt}%[0D2V2-3]%[Dự án đề kiểm tra Toán 10 GHKI NH23-24- Viết Tường]%[THPT Bùi Thị Xuân]
	Tổ $2$ của lớp $10$B dự định làm hộp quà bằng giấy thủ công để bán trong hội xuân của trường. Hộp quà loại nhỏ cần $2$ giờ để làm và hộp quà loại lớn cần $3$ giờ để làm. Biết rằng tổ $2$ cần làm ít nhất $8$ hộp quà và thời gian tổ $2$ thu xếp được để làm hộp quà là không quá $18$ giờ. Gọi $x$, $y$ lần lượt là số hộp quà loại nhỏ và số hộp quà loại lớn dự định làm.
	\begin{enumerate}
		\item Hãy xác định hệ bất phương trình biểu diễn các điều kiện ràng buộc của $x$, $y$.
		\item Hãy biểu diễn miền nghiệm của hệ bất phương trình ở câu a.
	\end{enumerate}
	\loigiai
	{
		\begin{enumerate}
			\item Vì $x$, $y$ lần lượt là số hộp quà loại nhỏ và số hộp quà loại lớn dự định làm nên $x$, $y\in\mathbb{N}$.\\
			Vì tổ $2$ cần làm ít nhất $8$ hộp quà nên ta có $x+y\ge 8$.\\
			Thời gian tổ $2$ làm $x$ hộp quà loại nhỏ là $2x$ (giờ).\\
			Thời gian tổ $2$ làm $y$ hộp quà loại lớn là $3y$ (giờ).\\
			Vì thời gian tổ $2$ thu xếp được để làm hộp quà là không quá $18$ giờ nên $2x+3y\le 18$.\\
			Khi đó ta có hệ phương trình $$\heva{& x+y\ge 8\\&2x+3y\le 18\\&x,y\in\mathbb{N} }.$$
			\item Vẽ đường thẳng $d_1\colon x+y=8$. Vì $0+0=0<8$ nên tọa độ $O(0;0)$ không thỏa mãn bất phương trình $x+y\ge 8$. Do đó, miền nghiệm $D_1$ của bất phương trình $x+y\ge 8$ là nửa mặt phẳng bờ $d_1$ không chứa gốc tọa độ $O$.\\
			Vẽ đường thẳng $d_2\colon 2x+3y=18$. Vì $2\cdot 0+3\cdot 0=0<18$ nên tọa độ $O(0;0)$ thỏa mãn bất phương trình $2x+3y\le 18$. Do đó, miền nghiệm $D_2$ của bất phương trình $2x+3y\le 18$ là nửa mặt phẳng bờ $d_2$ chứa gốc tọa độ $O$.\\
			Miền nghiệm $D_3$ của bất phương trình $x\ge 0$ là nửa mặt phẳng bờ $Oy$ chứa điểm $(1;0)$.\\
			Miền nghiệm $D_4$ của bất phương trình $y\ge 0$ là nửa mặt phẳng bờ $Ox$ chứa điểm $(0;1)$.\\
			Vậy miền nghiệm của hệ phương trình ở câu a là miền không bị gạch trong hình sau.
			\begin{center}
				\begin{tikzpicture}[font=\footnotesize ,line cap=round,line join=round,scale=.7,>=stealth]
					\begin{scope}
						\clip (-1.5,-1.5) rectangle (9.5,9.5);
						\fill[pattern=north east lines,pattern color=blue!50](-1.5,-1.5)--(10,-2)--(-1,9)--(-1.5,9)--cycle;
						\fill[pattern=north east lines,pattern color=red!50](10,9)--(10,-0.667)--(-1.5,7)--(-1.5,9)--cycle;
						\fill[pattern=north west lines,pattern color=yellow!50](-1.5,-1.5)--(10,-1.5)--(10,0)--(-1.5,0)--cycle;
						\fill[pattern=north west lines,pattern color=purple!50](-1.5,-1.5)--(0,-1.5)--(0,9)--(-1.5,9)--cycle;
						\draw[blue,smooth,domain=-1:10]plot(\x,-\x+8);
						\draw[red,smooth,domain=-1.5:10]plot(\x,-2/3*\x+6);
					\end{scope}
					\draw (4,4) node [above,rotate=-45] {$d_1\colon x + y =8$};
					\draw (1.8,4.7) node [above,rotate=-35] {$d_2\colon 2x+3y=18$};
					\draw[->] (-1.5,0)--(9.5,0) node[above right] {$x$};
					\draw[->] (0,-1.5)--(0,9) node[above left] {$y$};
					\draw (0,0) node [below left] {$O$};
				\end{tikzpicture}
			\end{center}
		\end{enumerate}
	}
\end{bt}

\begin{bt}%[0D3H1-2]%[Dự án đề kiểm tra Toán 10 GHKI NH23-24- Đào Hoàng Vũ]%[THPT Bùi Thị Xuân ]
	Tìm tập xác định của hàm số: $y=\sqrt{x+3}+\dfrac{5}{x^2-4}$
	\loigiai{Hàm số xác định khi và chỉ khi
\begin{align*}
&\heva{&x+3\geq 0\\&x^2-4\neq 0}
\Leftrightarrow\heva{&x\geq -3\\&x\neq \pm2.}
\end{align*}
\noindent Vậy tập xác định $\mathscr{D}=[-3;+\infty)\setminus \{-2; 2\}$.}
\end{bt}
%Câu 5...........................
\begin{bt}%[0D3H1-5]%[Dự án đề kiểm tra Toán 10 GHKI NH23-24- Đào Hoàng Vũ]%[THPT Bùi Thị Xuân ]
	Xét tính đồng biến, nghịch biến của hàm số $f(x)=x^2+2x+3$ trên khoảng $\left(-\infty;-1\right)$.
	\loigiai{
		Lấy $x_1$,  $x_2$ là 2 số tùy ý trên $\left(-\infty; -1\right)$ sao cho $x_1\neq x_2$.\\
		$\begin{aligned}
			\text{Xét} \ \dfrac{f(x_1)-f(x_2)}{x_1-x_2}&=\dfrac{x_1^2+2x_1+3-x_2^2-2x_2-3}{x_1-x_2}\\
			&=\dfrac{(x_1-x_2)(x_1+x_2)+2(x_1-x_2)}{x_1-x_2}\\
			&=\dfrac{(x_1-x_2)(x_1+x_2+2)}{x_1-x_2}\\
			&=x_1+x_2+2
		\end{aligned}$\\
		Do $x_1$, $x_2 \in\left(-\infty; -1\right)$ nên $x_1+x_2<-2$ hay $x_1+x_2+2<0$.\\
		Vậy hàm số nghịch biến trên khoảng $\left(-\infty; -1\right)$.
	}
\end{bt}

\begin{bt}%[0H4H3-1]%[Dự án đề kiểm tra Toán  10 GHKI NH23-24- Bùi Thanh Cương]%[Bùi Thị Xuân - Tp HCM]
Cho tam giác $ABC$ có $AB=5$, $BC=7$ và $AC=8$.
\begin{listEX}[1]
	\item Tính diện tích tam giác $ABC$ và bán kính đường tròn ngoại tiếp tam giác $ABC$.
	\item Tính độ dài đường trung tuyến $A M$ và độ dài đường cao $AH$ của tam giác $ABC$.
\end{listEX}
\loigiai{
\immini
{\begin{enumerate}[a)]
	\item Nửa chu vi tam giác $ABC$ là
	\[ p=\dfrac{AB+BC+CA}{2}=10.\]
	Diện tích tam giác $ABC$ là \\
	$S=\sqrt{p(p-5)(p-7)(p-8)}= \sqrt{10\cdot5\cdot3\cdot2}=10\sqrt{3}$.\\
	Ta có $S=\dfrac{abc}{4R}\Rightarrow R=\dfrac{4S}{abc}=\dfrac{4\cdot 10\sqrt{3}}{5\cdot7\cdot8}=\dfrac{\sqrt{3}}{7}$.
	\item 
	Áp dụng công thức 
	\[AM^{2}=\dfrac{2AB^2+2AC^2-BC^2}{4}=\dfrac{2\cdot 25+2\cdot 64-49}{4}=\dfrac{129}{4}\Rightarrow AM=\dfrac{\sqrt{129}}{2}. \]
		Từ công thức $S=\dfrac{1}{2}\cdot BC\cdot AH\Rightarrow AH=\dfrac{2\cdot S}{BC}=\dfrac{20\sqrt{3}}{7}.$
	\end{enumerate}
}
{
\begin{tikzpicture}[line join = round, line cap = round,>=stealth,scale=0.6,thick]
	\def\a{7}
	\def\b{8}
	\def\c{5}
	\path
	(0,0) coordinate (B) ($(0:\a)+(B)$) coordinate (C);
	\path ($(81:\c)+(B)$) coordinate (A);
	\draw (A)--(B)--(C)--cycle;
	\foreach \x/\g in {A/90,B/210,C/-30}
	\fill (\x) circle (1pt)($(\x)+(\g:3mm)$) node{$\x$};
	\coordinate (H) at ($(B)!(A)!(C)$);
	\coordinate (M) at ($(B)!0.5!(C)$);
	\draw (A)--(H) (A)--(M);
	%\draw pic [draw, angle eccentricity=1.5,angle radius =7 mm] {angle = C--B--A};
	%\draw pic [draw, double,angle eccentricity=1.5,angle radius =7 mm] {angle = A--C--B};
	%\draw pic [draw, angle eccentricity=1.5,angle radius =7 mm] {angle =B--A--C};
	%\draw pic [draw,"$ $", angle eccentricity=1.5,angle radius =6 mm] {angle = B--A--C};
	\path (A)--(B) node[above,midway,sloped]{$5$};
	\path (A)--(C) node[above,midway,sloped]{$8$};
	%\path (B)--(C) node[below,midway,sloped]{$7$};
	\foreach \x/\g in {H/-90,M/-90}
	\fill (\x) circle (1pt)($(\x)+(\g:5mm)$) node{$\x$};
\end{tikzpicture}	
}
}
\end{bt}

\begin{bt}%[0H4V3-2]%[Dự án đề kiểm tra Toán  10 GHKI NH23-24- Bùi Thanh Cương]%[Bùi Thị Xuân - Tp HCM]
	\immini[thm]{Hai trạm quan sát ở hai thành phố Đà Nẵng (giả sử là điểm $B$) và Nha Trang (giả sử là điểm $C$) đồng thời nhìn thấy vệ tinh (giả sử là điểm $A$) với góc nâng lần lượt là $75^\circ$ và $60^\circ$. Tính khoảng cách từ vệ tinh đến trạm quan sát tại thành phố Đà Nẵng và khoảng cách từ vệ tinh đến trạm quan sát tại Nha Trang, biết rằng khoảng cách giữa hai trạm quan sát là $520 \mathrm{~km}$ \textit{(Kết quả làm tròn một chữ số thập phân sau dấu phẩy)}.
	}
	{	
	\begin{tikzpicture}[line join=round, line cap=round,scale=1,transform shape]
			\clip (-1.5,-1.14) rectangle (4.5,3.8);
			\begin{pgfinterruptboundingbox}
				\tikzset{vetinh/.pic={
						\def\D{ 
							(.94,.77)--(1.2,.94)--(.95,.91)--(.85,.85)--cycle
							;}
						\draw \D;
						\fill[gray] \D;
						\def\V{ 
							(.6,.82)--(1.3,.92)--(1.4,1.05)--(.7,.95)--
							cycle;
						}
						\draw \V;
						\fill[pattern=north east lines] \V;
						\def\T{ 
							(.8,.72)--(.84,.9)
							..controls +(40:0.05) and +(120:0.1) .. (1,.75)--cycle;
						}
						\draw \T;
						\fill[black!50!white] \T;
				}}
				%Vẽ nhà 1
				\tikzset{nha/.pic={
						\draw (-1.05,-.25)coordinate [label=above:] (A)--(.12,3.52)coordinate (C)--(3,-.25)coordinate [label=above:] (B)--cycle;
						%\node at (-.68,0) [above]{\tiny $360$ km};
						\node at (.77,-0.7) [above]{\tiny $520$ km};
						%\node at (C) [left]{\tiny $13{,}2^\circ$};
						%\node at ([xshift=0.7cm,yshift=0.3cm]A){\tiny$75^\circ$};
						%\node at ([xshift=-0.7cm,yshift=0.3cm]B){\tiny$60^\circ$};
						%\tkzMarkAngles[size=.5](B,A,C);
						%\tkzMarkAngles[size=.5,double](C,B,A);
						\draw [yellow!40!gray,line width=0.7cm,decorate,decoration={random steps, segment length=1cm}]([yshift=-0.7cm,xshift=-.5]A) -- ([yshift=-0.7cm, xshift=0.3cm]B);
						\node at ([xshift=0.1cm,yshift=-0.75cm]A){\tiny \text{Đà Nẵng}};
						\node at ([yshift=-0.75cm]B){\tiny \text{Nha Trang}};
						\node at ([xshift=0.6cm]C){\tiny \text{Vệ tinh}};
						\def\N{ 
							(-1.22,-.25)--(-.82,-.25)--(-.82,-.85)--(-.82,-.85)--(-1.22,-.85)--cycle;}
						\draw \N;
						\fill[brown] \N;
				}}
				%ve nha 2
				\tikzset{nhahai/.pic={
						\def\NH{ 
							(-1.22,-.25)--(-.82,-.25)--(-.82,-.85)--(-.82,-.85)--(-1.22,-.85)--cycle;}
						\draw \NH;
						\fill[cyan] \NH;
				}}
				\tikzset{cuaso/.pic={
						\def\C{ 
							(-1.22,-.25)--(-.82,-.25)--(-.82,-.85)--(-.82,-.85)--(-1.22,-.85)--cycle;}
						\draw \C;
						\fill[white] \C;
				}}
				\tikzset{cay/.pic={
						\def\C{ 
							(-1.25,-.85)--(-1.24,-.75)--(-1.22,-.75)--(-1.2,-.85)--cycle
							;}
						\def\L{ 
							(-1.23,-.78)
							..controls +(-40:.13) and +(-160:0.16) .. (-1.27,-.7)
							..controls +(-40:.13) and +(-160:0.14) .. (-1.23,-.6)
							..controls +(120:.0) and +(50:0.11) .. (-1.2,-.7)
							..controls +(60:.1) and +(0:0.2) .. (-1.23,-.78)
							--cycle
							;}
						
						\draw \C;
						\fill[black!50!white] \C;
						\draw \L;
						\fill[green] \L;
				}}
				\path(0,0)pic[scale=1]{nha}
				(0,0)pic[fill=white,scale=.2,,xshift=-4.7cm,,yshift=-1.5cm]{cuaso}
				(0,0)pic[fill=white,scale=.2,,xshift=-4.1cm,,yshift=-1.5cm]{cuaso}
				(0,0)pic[fill=white,scale=.2,,xshift=-3.5cm,,yshift=-1.5cm]{cuaso}
				(0,0)pic[fill=white,scale=.4,,xshift=-1.55cm,,yshift=-1.25cm]{cuaso}
				(0,0)pic[scale=1]{cay}
				(4,0)pic[scale=1]{nhahai}
				(3.25,0)pic[fill=white,scale=.4,,xshift=0.35cm,,yshift=-1.25cm]{cuaso}
				(3.25,0)pic[fill=white,scale=.2,,xshift=-1cm,,yshift=-1.5cm]{cuaso}
				(3.25,0)pic[fill=white,scale=.2,,xshift=-.4cm,,yshift=-1.5cm]{cuaso}
				(3.25,0)pic[fill=white,scale=.2,,xshift=.2cm,,yshift=-1.5cm]{cuaso}
				(-0.5,3)pic[scale=.7]{vetinh}
				(3,0.1)pic[scale=1.3]{cay}
				(2,0.2)pic[scale=1.5]{cay}
			;
			\end{pgfinterruptboundingbox}
			\draw pic [draw,"$60^\circ$", angle eccentricity=1.5,angle radius =6 mm] {angle = C--B--A};
			\draw pic [draw,"$75^\circ$",double, angle eccentricity=1.5,angle radius =6 mm] {angle = B--A--C};
	\end{tikzpicture}
	}
	\loigiai{
		\immini{
			Bài toán đưa về tính cạnh $AB, AC$ của tam giác $ABC$ biết $\widehat{B}=75^\circ$, $\widehat{C}=60^\circ$ và $CB=520\text{ km}.$\\
			Ta có $ \widehat{A}=180^\circ-(\widehat{B}+\widehat{C})=45^\circ$\\
			Áp dụng định lí Sin trong tam giác $ABC,$ ta có:
			\begin{itemize}
				\item $\dfrac{AB}{\sin C}=\dfrac{BC}{\sin A}\Rightarrow AB=\dfrac{BC\cdot \sin C}{\sin A}=\dfrac{520\cdot\sin 60^\circ }{\sin 45^\circ}\approx 636{,}9 (\text{km}).$
				\item $\dfrac{AC}{\sin B}=\dfrac{BC}{\sin A}\Rightarrow AC=\dfrac{BC\cdot \sin B}{\sin A}=\dfrac{520\cdot\sin 75^\circ }{\sin 45^\circ}\approx 710{,}3 (\text{km}).$
			\end{itemize}
				}
		{\begin{tikzpicture}[scale=0.5,font=\scriptsize]
				\path
				(0,0) coordinate [label=left:$B$](B)
				(5,0) coordinate [label=right:$C$](C)
				($(B)!2cm!75:(C)$)coordinate[label=left:](m)
				($(C)!2cm!-60:(B)$)coordinate[label=right:](n)
				(intersection of B--m and C--n) coordinate [label=above:$A$](A)
				;
				\foreach \x in{B,C,A}
				{\fill (\x)circle (1.5pt);}
				\draw (A)--(B)--(C)--cycle;
				\path (B)--node [below]{$520\text{ km}$}(C);
				\draw pic ["$75^\circ$",draw, angle eccentricity=1.6,angle radius =0.5cm] 
				{angle = C--B--A};
				\draw pic ["$60^\circ$",double, draw, angle eccentricity=1.6,angle radius =0.5cm] 
				{angle = A--C--B};
		\end{tikzpicture}}
	\noindent Vậy khoảng cách từ vệ tinh đến trạm quan sát tại thành phố Đà Nẵng và trạm quan sát tại Nha Trang lần lượt là $636{,}9 (\text{km})$ và $710{,}3 (\text{km})$.
	
	}
\end{bt}
