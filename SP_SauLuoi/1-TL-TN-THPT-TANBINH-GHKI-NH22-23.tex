
\de{ĐỀ THI GIỮA HỌC KỲ I NĂM HỌC 2023-2024}{THPT TÂN BÌNH}
\begin{center}
	\textbf{PHẦN 1 - TRẮC NGHIỆM}
\end{center}
\Opensolutionfile{ans}[ans/ans]

%Câu 1
\begin{ex}%[1D1N1-2]%[Dự án đề kiểm tra Toán 11 GHKI NH23-24- Dương Công Tạo]%[THPT - Tp HCM]
	Khi đổi $\dfrac{5\pi }{3}\,\mathrm{rad}$ sang đơn vị độ ta được kết quả là
	\choice
	{$360^\circ$}
	{\True $300^\circ$}
	{$250^\circ$}
	{$270^\circ$}
	\loigiai{
		Ta có $\dfrac{5\pi }{3}\,=\dfrac{5\pi }{3}\,\cdot \dfrac{180^\circ}{\pi }=300^\circ$.
	}
\end{ex}
%Câu 2
\begin{ex}%[1D1N5-3]%[Dự án đề kiểm tra Toán 11 GHKI NH23-24- Dương Công Tạo]%[THPT - Tp HCM]
	Nghiệm của phương trình $\sin x=\sin \dfrac{\pi }{5}$ là
	\choice
	{\True $\hoac{& x=\dfrac{\pi }{5}+k2\pi \\& x=\dfrac{4\pi }{5}+k2\pi}\,\left( k\in \mathbb{Z} \right)$}
	{$\hoac{& x=\dfrac{\pi }{5}+k\pi \\& x=-\dfrac{\pi }{5}+k\pi}\,\left( k\in \mathbb{Z} \right)$}
	{$\hoac{& x=\dfrac{\pi }{5}+k2\pi \\& x=-\dfrac{\pi }{5}+k2\pi}\,\left( k\in \mathbb{Z} \right)$}
	{$\hoac{& x=\dfrac{\pi }{5}+k\pi \\& x=\dfrac{4\pi }{5}+k\pi}\,\left( k\in \mathbb{Z} \right)$}
	\loigiai{
		Ta có $\sin x=\sin \dfrac{\pi }{5}\Leftrightarrow \hoac{& x=\dfrac{\pi }{5}+k2\pi \\& x=\dfrac{4\pi }{5}+k2\pi}\,\left( k\in \mathbb{Z} \right)$.
	}
\end{ex}
%Câu 3
\begin{ex}%[1D1N4-2]%[Dự án đề kiểm tra Toán 11 GHKI NH23-24- Dương Công Tạo]%[THPT - Tp HCM]
	Tập xác định của hàm số $y=\sin x$ là
	\choice
	{$\left( 0;+\infty \right)$}
	{$\left( -1;1 \right)$}
	{\True $\mathbb{R}$}
	{$\left[ -1;1 \right]$}
	\loigiai{
		Hàm số $y=\sin x$ có tập xác định của hàm số là $\mathscr{D}=\mathbb{R}$.
	}
\end{ex}
%Câu 4
\begin{ex}%[1D2N3-3]%[Dự án đề kiểm tra Toán 11 GHKI NH23-24- Dương Công Tạo]%[THPT - Tp HCM]
	Cho cấp số nhân $(u_n)$ với $u_1=3$ và công bội $q=2$. Công thức số hạng tổng quát của $(u_n)$ là
	\choice
	{\True $u_n=3\cdot 2^{n-1}$}
	{$u_n=3\cdot 2^{n+1}$}
	{$u_n=3\cdot 2^{n+2}$}
	{$u_n=3\cdot 2^n$}
	\loigiai{
		Công thức số hạng tổng quát của $(u_n)$ là $u_n=u_1\cdot q^{n-1}=3\cdot 2^{n-1}$.
	}
\end{ex}
%Câu 5
\begin{ex}%[1D2N1-3]%[Dự án đề kiểm tra Toán 11 GHKI NH23-24- Dương Công Tạo]%[THPT - Tp HCM]
	Cho dãy số $(u_n)$, biết $u_n=\dfrac{(-1)^n}{2^n+1}\,, n\in \mathbb{N}^*$. Số hạng thứ $6$ của dãy số bằng
	\choice
	{\True $\dfrac{1}{65}$}
	{$\dfrac{6}{13}$}
	{$\dfrac{6}{65}$}
	{$\dfrac{1}{13}$}
	\loigiai{
		Số hạng thứ $6$ của dãy số là $u_6=\dfrac{(-1)^6}{2^6+1}=\dfrac{1}{65}$.
	}
\end{ex}
%Câu 6
\begin{ex}%[1D1H1-5]%[Dự án đề kiểm tra Toán 11 GHKI NH23-24- Dương Công Tạo]%[THPT - Tp HCM]
	Cho góc $\alpha \in \left( \dfrac{\pi }{2};\pi \right)$. Mệnh đề nào sao đây đúng?
	\choice
	{\True $\sin \alpha >0$}
	{$\cos \alpha >0$}
	{$\tan \alpha >0$}
	{$\cot \alpha >0$}
	\loigiai{
	}
\end{ex}
%Câu 7
\begin{ex}%[1D2N2-3]%[Dự án đề kiểm tra Toán 11 GHKI NH23-24- Dương Công Tạo]%[THPT - Tp HCM]
	Cho cấp số cộng $(u_n)$ với số hạng đầu $u_1=-3$ và công sai $d=\dfrac{1}{2}$. Công thức số hạng tổng quát của $(u_n)$ là
	\choice
	{$u_n=-3+\dfrac{1}{2}n-1$}
	{\True $u_n=-3+\dfrac{1}{2}(n-1)$}
	{$u_n=-3+\dfrac{1}{2}(n+1)$}
	{$u_n=-3+\dfrac{1}{4}(n+1)$}
	\loigiai{
		Ta có: $u_n=u_1+(n-1)d=-3+\dfrac{1}{2}(n-1)$.
	}
\end{ex}
%Câu 8
\begin{ex}%[1D2N3-2]%[Dự án đề kiểm tra Toán 11 GHKI NH23-24- Dương Công Tạo]%[THPT - Tp HCM]
	Cho cấp số nhân $(u_n)$ với $u_1=3$,$u_2=1$. Công bội $q$ của $(u_n)$ là
	\choice
	{\True $\dfrac{1}{3}$}
	{$-2$}
	{$2$}
	{$3$}
	\loigiai{
		Ta có;
		$q=\dfrac{{{u}_2}}{u_1}=\dfrac{1}{3}$.
	}
\end{ex}
%Câu 9
\begin{ex}%[1D1N3-2]%[Dự án đề kiểm tra Toán 11 GHKI NH23-24- Dương Công Tạo]%[THPT - Tp HCM]
	Xét $a,b$ là các góc tùy ý, biểu thức $\cos a\cos b+\sin a\sin b$ bằng
	\choice
	{$\cos (a+b)$}
	{\True $\cos (a-b)$}
	{$\sin (a+b)$}
	{$\sin (a-b)$}
	\loigiai{
		Ta có: $\cos a\cos b+\sin a\sin b=\cos (a-b)$.
	}
\end{ex}
%Câu 10
\begin{ex}%[1D2H3-4]%[Dự án đề kiểm tra Toán 11 GHKI NH23-24- Dương Công Tạo]%[THPT - Tp HCM]
	Cho cấp số nhân $(u_n)$ với $u_1=-2$ và$u_2=-5$. Tính ${{u}_3}$.
	\choice
	{${{u}_3}=23$}
	{\True ${{u}_3}=-50$}
	{${{u}_3}=-12$}
	{${{u}_3}=50$}
	\loigiai{
		Ta có;
		$u_3=u_1\cdot q^2=-2\cdot (-5)^2=-50$.
	}
\end{ex}
%Câu 11
\begin{ex}%[1D1N3-3]%[Dự án đề kiểm tra Toán 11 GHKI NH23-24- Dương Công Tạo]%[THPT - Tp HCM]
	Cho $\alpha $ là góc tùy ý. Mệnh đề nào sau đây là đúng?
	\choice
	{$\sin 2\alpha =2\sin \alpha $ }
	{$\sin 2\alpha =1-2\sin^2 \alpha $}
	{\True $\sin 2\alpha =2\sin \alpha \cos \alpha $ }
	{$\sin 2\alpha =2\sin^2 \alpha -1$}
	\loigiai{
		Mệnh đề đúng là $\sin 2\alpha =2\sin \alpha \cos \alpha $.
	}
\end{ex}
%Câu 12
\begin{ex}%[1D2H2-2]%[Dự án đề kiểm tra Toán 11 GHKI NH23-24- Dương Công Tạo]%[THPT - Tp HCM]
	Trong các dãy số sau, dãy nào là một cấp số cộng hữu hạn?
	\choice
	{\True $2;5;8$}
	{$1;3;4$}
	{$1;3;7$}
	{$2;4;8$}
	\loigiai{
		Ta có $5-2=3;\,8-5=3$ nên $2;5;8$ là một cấp số cộng hữu hạn.
	}
\end{ex}

%%=====Câu 13
\begin{ex}%[1D2H2-3]%[Dự án đề kiểm tra Toán 11 GHKI NH23-24- Huỳnh Quy]%[THPT Tân Bình - Tp HCM]
	Cấp số cộng $(u_n)$ có số hạng đầu tiên $u_1=2$ và công sai $d=3$. Số hạng $u_3$ bằng
	\choice
	{$6$}
	{\True $8$}
	{$10$}
	{$9$}
	\loigiai{
		Ta có $u_3=u_1+2d=2+2\cdot 3=8$.
	}
\end{ex}
%%=====Câu 14
\begin{ex}%[1D2H1-2]%[Dự án đề kiểm tra Toán 11 GHKI NH23-24- Huỳnh Quy]%[THPT Tân Bình - Tp HCM]
	Cho dãy số $(u_n)$, biết $\heva{&u_1=-1\\&u_{n+1}=2u_n+1}$ ($n\in\mathbb{N}^{*}$). Số hạng thứ hai của dãy số bằng
	\choice
	{$2$}
	{$3$}
	{$1$}
	{\True $-1$}
	\loigiai{
		Ta có $u_2=2\cdot u_1+1=2\cdot (-1)+1=-1$.	
	}
\end{ex}

%%=====Câu 15
\begin{ex}%[1D1H1-1]%[Dự án đề kiểm tra Toán 11 GHKI NH23-24- Huỳnh Quy]%[THPT Tân Bình - Tp HCM]
	Khi đổi $70^{\circ}$ sang đơn vị radian ta được kết quả là
	\choice
	{$\dfrac{7\pi}{10}$}
	{$\dfrac{7\pi}{15}$}
	{\True $\dfrac{7\pi}{18}$}
	{$\dfrac{7\pi}{12}$}
	\loigiai{
		Ta có $180^{\circ}=\pi$ rad $\Rightarrow 70^{\circ}=\dfrac{\pi\cdot 70}{180}=\dfrac{7\pi}{18}$.	
	}
\end{ex}
%%=====Câu 16
\begin{ex}%[1D1H2-3]%[Dự án đề kiểm tra Toán 11 GHKI NH23-24- Huỳnh Quy]%[THPT Tân Bình - Tp HCM]
	Cho $\alpha\in\mathbb{R}$ tùy ý, mệnh đề nào dưới đây đúng?
	\choice
	{$\cos\left(\dfrac{\pi}{2}-\alpha\right)=-\cos\alpha$}
	{$\cos\left(\dfrac{\pi}{2}-\alpha\right)=-\sin\alpha$}
	{$\cos\left(\dfrac{\pi}{2}-\alpha\right)=\cos\alpha$}
	{\True $\cos\left(\dfrac{\pi}{2}-\alpha\right)=\sin\alpha$}
	\loigiai{
		Ta có $\dfrac{\pi}{2}-\alpha$ và $\alpha$ là hai góc phụ nhau nên $\cos\left(\dfrac{\pi}{2}-\alpha\right)=\sin\alpha$.
	}
\end{ex}
%%=====Câu 17
\begin{ex}%[1D2H3-1]%[Dự án đề kiểm tra Toán 11 GHKI NH23-24- Huỳnh Quy]%[THPT Tân Bình - Tp HCM]
	Cho cấp số nhân $(u_n)$ có $u_2=6$, $u_5=162$. Công bội của cấp số nhân đó bằng
	\choice
	{$27$}
	{\True $3$}
	{$2$}
	{$9$}
	\loigiai{
		Ta có $\heva{&u_2=6\\&u_5=162}\Leftrightarrow\heva{&u_1\cdot q=6&(1)\\&u_1\cdot q^4=162.&(2)}$\\
		Lấy $(2)$ chia $(1)$ vế theo vế, ta được $q^3=27\Rightarrow q=3$.	
	}
\end{ex}
%%=====Câu 18
\begin{ex}%[1D1H5-3]%[Dự án đề kiểm tra Toán 11 GHKI NH23-24- Huỳnh Quy]%[THPT Tân Bình - Tp HCM]
	Nghiệm của phương trình $\sin 3x=\sin x$ là
	\choice
	{$x=k\dfrac{\pi}{2}$ ($k\in\mathbb{Z}$)}
	{$x=k\dfrac{\pi}{4}$ ($k\in\mathbb{Z}$)}
	{$\hoac{&x=k2\pi\\&x=\dfrac{\pi}{4}+k2\pi}$ ($k\in\mathbb{Z}$)}
	{\True $\hoac{&x=k\pi\\&x=\dfrac{\pi}{4}+k\dfrac{\pi}{2}}$ ($k\in\mathbb{Z}$)}
	\loigiai{
		Ta có 
		\begin{eqnarray*}
			&&\sin 3x=\sin x\\
			&\Leftrightarrow& \hoac{&3x=x+k2\pi\\&3x=\pi-x+k2\pi} (k\in\mathbb{Z})\\
			&\Leftrightarrow&\hoac{&2x=k2\pi\\&4x=\pi+k2\pi}\\
			&\Leftrightarrow&\hoac{&x=k\pi\\&x=\dfrac{\pi}{4}+k\dfrac{\pi}{2}.}
		\end{eqnarray*}	
	}
\end{ex}
%%=====Câu 19
\begin{ex}%[1D1H4-5]%[Dự án đề kiểm tra Toán 11 GHKI NH23-24- Huỳnh Quy]%[THPT Tân Bình - Tp HCM]
	Tìm tập giá trị $T$ của hàm số $y=5-3\sin\left(x-\dfrac{\pi}{3}\right)$.
	\choice
	{\True $T=[2;8]$}
	{$T=[-3;3]$}
	{$T=[-1;1]$}
	{$T=[8;10]$}
	\loigiai{
		Ta có
		\begin{eqnarray*}
			&&-1\leq \sin\left(x-\dfrac{\pi}{3}\right)\leq 1	\\
			&\Leftrightarrow& 3\geq -3\sin\left(x-\dfrac{\pi}{3}\right)\geq -3\\
			&\Leftrightarrow&8\geq 5-3\sin\left(x-\dfrac{\pi}{3}\right)\geq 2.
		\end{eqnarray*}	
		Vậy $T=[2;8]$.
	}
\end{ex}
%%=====Câu 20
\begin{ex}%[1D1H2-2]%[Dự án đề kiểm tra Toán 11 GHKI NH23-24- Huỳnh Quy]%[THPT Tân Bình - Tp HCM]
	Cho $\tan x=\dfrac{3}{4}$ và góc $x$ thỏa mãn $\pi<x<\dfrac{3\pi}{2}$. Giá trị $\cos x$ bằng
	\choice
	{$\dfrac{3}{5}$}
	{$\dfrac{4}{5}$}
	{$-\dfrac{3}{5}$}
	{\True $-\dfrac{4}{5}$}
	\loigiai{
		Ta có $\pi<x<\dfrac{3\pi}{2}\Rightarrow \cos x<0$.\\
		Lại có 
		\begin{eqnarray*}
			&&1+\tan^2x=\dfrac{1}{\cos^{2}x}\\
			&\Rightarrow& \cos^{2}x=\dfrac{1}{1+\tan^2x}\\
			&\Rightarrow& \cos x=-\sqrt{\dfrac{1}{1+\tan^2x}}=-\sqrt{\dfrac{1}{1+\left(\dfrac{3}{4}\right)^{2}}}=-\dfrac{4}{5}.
		\end{eqnarray*}
	}
\end{ex}
%%=====Câu 21
\begin{ex}%[1D2H2-2]%[Dự án đề kiểm tra Toán 11 GHKI NH23-24- Huỳnh Quy]%[THPT Tân Bình - Tp HCM]
	Cho một cấp số cộng $(u_n)$ có $u_1=5$ và tổng của $50$ số hạng đầu bằng $5150$. Tìm công thức của số hạng tổng quát $u_n$.
	\choice
	{$u_n=3+2n$}
	{$u_n=2+3n$}
	{\True $u_n=1+4n$}
	{$u_n=5n$}
	\loigiai{
		Ta có 
		\begin{eqnarray*}
			&&S_{50}=50u_1+\dfrac{50\cdot 49\cdot d}{2}\\
			&\Leftrightarrow&\dfrac{50\cdot 49\cdot d }{2}=5150-50\cdot 5\\
			&\Leftrightarrow&d=\dfrac{2(5150-50\cdot 5)}{50\cdot 49}=4.
		\end{eqnarray*}	
		Số hạng tổng quát của khai triển là $u_n=u_1+(n-1)d=5+(n-1)4=4n+1$.
	}
\end{ex}
%%=====Câu 22
\begin{ex}%[1D2V2-6]%[Dự án đề kiểm tra Toán 11 GHKI NH23-24- Huỳnh Quy]%[THPT Tân Bình - Tp HCM]
	Trên một bàn cờ có nhiều ô vuông. Người ta đặt $7$ hạt dẻ vào ô vuông đầu tiên, sau đó đặt tiếp vào ô thứ hai số hạt dẻ nhiều hơn ô đầu tiên là $5$, tiếp tục đặt vào ô thứ ba số hạt dẻ nhiều hơn ô thứ hai là $5$, $\ldots$ và cứ thế tiếp tục đến ô cuối cùng. Biết rằng đặt hết số ô trên bàn cờ người ta đã phải sử dụng hết $25450$ hạt dẻ. Hỏi bàn cờ đó có bao nhiêu ô?
	\choice
	{$98$ ô}
	{$104$ ô}
	{$102$ ô}
	{\True $100$ ô}
	\loigiai{
		Số hạt dẻ trên các ô theo thứ tự từ nhỏ đến lớn lập thành một cấp số cộng với số hạng đầu $u_1=7$, công sai $d=5$.\\
		Theo đề ta có
		\begin{eqnarray*}
			&&S_n=15450\\
			&\Leftrightarrow&nu_1+\dfrac{n(n-1)d}{2}=25450\\
			&\Leftrightarrow&7n+\dfrac{n(n-1)5}{2}=25450\\
			&\Leftrightarrow&\dfrac{5}{2}n^2+\dfrac{9}{2}n-25450=0\\
			&\Leftrightarrow&\hoac{&n=100&(\text{nhận})\\&n=-\dfrac{509}{5}&(\text{loại}).}
		\end{eqnarray*}
	}
\end{ex}
%%=====Câu 23
\begin{ex}%[1D1K4-1]%[cau 23]%[Dự án đề kiểm tra Toán 11 GHKI NH23-24- Huỳnh Quy]%[THPT Tân Bình - Tp HCM]
	Hàm số $y=\dfrac{1}{2\cos x-1}$ xác định trên khoảng nào sau đây?
	\choice
	{$(0;\pi)$}
	{$\left(\dfrac{\pi}{2};\dfrac{3\pi}{2}\right)$}
	{$\left(-\dfrac{\pi}{2};\dfrac{\pi}{2}\right)$}
	{\True $(\pi;2\pi)$}
	\loigiai{
		Hàm số đã cho xác định khi và chỉ khi
		\begin{eqnarray*}
			&&2\cos x-1\ne 0\\
			&\Leftrightarrow&\cos x\ne\dfrac{1}{2}\\
			&\Leftrightarrow&\cos x\ne\cos\dfrac{\pi}{3}\\
			&\Leftrightarrow&\heva{&x\ne\dfrac{\pi}{3}+k2\pi\\&x\ne-\dfrac{\pi}{3}+k2\pi} (k\in\mathbb{Z}).
		\end{eqnarray*}
		Các khoảng $(0;\pi)$, $\left(\dfrac{\pi}{2};\dfrac{3\pi}{2}\right)$, $\left(-\dfrac{\pi}{2};\dfrac{\pi}{2}\right)$ đều chứa $\dfrac{\pi}{3}$ nên hàm số không xác định trên các khoảng này.
	}
\end{ex}
%%=====Câu 24
\begin{ex}%[1D1K3-1]%[cau 24]%[Dự án đề kiểm tra Toán 11 GHKI NH23-24- Huỳnh Quy]%[THPT Tân Bình - Tp HCM]
	Cho $x\in\left(-\dfrac{\pi}{2};0\right)$ thỏa mãn $\sin x\cdot\sin 2x+\cos x\cdot\cos2x=\dfrac{1}{3}$. Giá trị của biểu thức $M=\sin x\cdot \cos3x+\cos x\cdot \sin 3x$ bằng
	\choice
	{$\dfrac{4\sqrt{2}}{9}$}
	{$-\dfrac{56\sqrt{2}}{81}$}
	{\True $\dfrac{56\sqrt{2}}{81}$}
	{$-\dfrac{4\sqrt{2}}{9}$}
	\loigiai{
		Ta có $x\in\left(-\dfrac{\pi}{2};0\right)\Rightarrow \sin x<0,\, \cos x>0$.\\
		Theo đề, ta có
		\begin{eqnarray*}
			&&\sin x\cdot\sin 2x+\cos x\cdot\cos2x=\cos(2x-x)=\cos x=\dfrac{1}{3}\\
			&\Rightarrow&\sin x=-\sqrt{1-\cos^{2}x}=-\sqrt{1-\left(\dfrac{1}{3}\right)^2}=-\dfrac{2\sqrt{2}}{3}.
		\end{eqnarray*}
		Do đó, $M=\sin x\cdot \cos3x+\cos x\cdot \sin 3x=\sin(3x+x)=\sin4x=2\sin2x\cdot\cos2x$.\\
		Mà
		\begin{itemize}
			\item $\sin2x=2\sin x\cos x=2\cdot \left(-\dfrac{2\sqrt{2}}{3}\right)\cdot\dfrac{1}{3}=-\dfrac{4\sqrt{2}}{9}$.
			\item $\cos 2x=2\cos^{2}x-1=2\cdot\left(\dfrac{1}{3}\right)^{2}-1=-\dfrac{7}{9}$.
		\end{itemize}
		Từ đó, ta có $M=2\cdot\left(-\dfrac{4\sqrt{2}}{9}\right)\cdot\left(-\dfrac{7}{9}\right)=\dfrac{56\sqrt{2}}{81}$.
	}
\end{ex}


\Closesolutionfile{ans}
%\begin{center}
%	\textbf{ĐÁP ÁN}
%	\inputansbox{10}{ans/ans}	
%\end{center}
\begin{center}
	\textbf{PHẦN 2 - TỰ LUẬN}
\end{center}

%Câu 1...........................
\begin{bt}%[1D1H3-1]%[Dự án đề kiểm tra Toán 11 GHKI NH23-24- Quang Vinh NT]%[THPT Tân Bình - Tp HCM]
	Cho $\sin x=\dfrac{3}{4}$ với $\dfrac{\pi}{2}<x<\pi$. Tính giá trị của $\cos x$ và $\cos \left(x+\dfrac{\pi}{6}\right)$
	\loigiai{
		Do $\dfrac{\pi}{2}<x<\pi$ nên $\cos x<0$.\\
		Suy ra $\cos x =-\sqrt{1-\sin^2 x}=-\sqrt{1-\left(\dfrac{3}{4}\right)^2}=-\dfrac{\sqrt{7}}{4}$.\\
		Ta có 
		\begin{eqnarray*}
		\cos \left(x+\dfrac{\pi}{6}\right)	
		&= & \cos x \cos \dfrac{\pi}{6}-\sin x \sin \dfrac{\pi}{6}\\
			&= & -\dfrac{\sqrt{7}}{4}\cdot \dfrac{\sqrt{3}}{2} - \dfrac{3}{4}\cdot \dfrac{1}{2}\\
				&= & -\dfrac{\sqrt{21}+3}{8}.
		\end{eqnarray*}
	}
\end{bt}


%Câu ...........................
\begin{bt}%[1D2H2-3]%[Dự án đề kiểm tra Toán 11 GHKI NH23-24- Quang Vinh NT]%[THPT Tân Bình - Tp HCM]
	Cho cấp số cộng $\left(u_n\right)$ có $u_2+u_3=20, u_5+u_7=-29$. Hãy tìm số hạng đầu $u_1$ và công sai $d$.
	\loigiai{
	Ta có
	\begin{eqnarray*}
		& & \heva{&u_2+u_3=20\\&u_5+u_7=-29}\\
		&\Leftrightarrow & \heva{&u_1+d+u_1+2d=20\\&u_1+4d+u_1+6d=-29}\\
		&\Leftrightarrow & \heva{&2u_1+3d=20\\&2u_1+10d=-29}\\
		&\Leftrightarrow & \heva{&u_1=\dfrac{41}{2}\\&d=-7.}\\
	\end{eqnarray*}
	}
\end{bt}


%Câu ...........................
\begin{bt}%[1D1H5-5]%[Dự án đề kiểm tra Toán 11 GHKI NH23-24- Quang Vinh NT]%[THPT Tân Bình - Tp HCM]
	Giải phương trình $\cos 2x +\cos \left(x-\dfrac{2\pi}{3}\right)=0$ và tìm các nghiệm thuộc $\left(-\pi;2\pi\right)$.
	\loigiai{
	Ta có
	\begin{eqnarray*}
		& & \cos 2x +\cos \left(x-\dfrac{2\pi}{3}\right)=0\\
		&\Leftrightarrow & \cos 2x =-\cos \left(x-\dfrac{2\pi}{3}\right)\\
		&\Leftrightarrow & \cos 2x =\cos \left(\pi+x-\dfrac{2\pi}{3}\right)\\		
		&\Leftrightarrow & \cos 2x =\cos \left(x+\dfrac{\pi}{3}\right)\\
		&\Leftrightarrow & \hoac{&2x = x+\dfrac{ \pi}{3}+k2\pi\\&2x = -x-\dfrac{ \pi}{3}+k2\pi}, ~k\in \mathbb{Z}\\
		&\Leftrightarrow & \hoac{&x = \dfrac{ \pi}{3}+k2\pi\\&3x = -\dfrac{ \pi}{3}+k2\pi}, ~k\in \mathbb{Z}\\
		&\Leftrightarrow & \hoac{&x = \dfrac{ \pi}{3}+k2\pi\\&x = -\dfrac{ \pi}{9}+\dfrac{k2\pi}{3}}, ~k\in \mathbb{Z}.
	\end{eqnarray*}
	Suy ra tập hợp các nghiệm thuộc khoảng $\left(-\pi;2\pi\right)$ là  $S=\left\{ \dfrac{\pi}{3};-\dfrac{\pi}{9};\dfrac{5\pi}{9};\dfrac{11\pi}{9}\right\}$.
	}
\end{bt}


%Câu ...........................
\begin{bt}%[1D2T3-7]%[Dự án đề kiểm tra Toán 11 GHKI NH23-24- Quang Vinh NT]%[THPT Tân Bình - Tp HCM]
	Quốc gia $X$ có trữ lượng dầu mỏ là $120$ triệu thùng vào đầu năm $2022$. Theo dự báo với mức khai thác dầu không đổi như năm $2022$ thì trữ lượng dầu của nước này sẽ hết sau đúng $50$ năm nữa. Do nhu cầu phát triển kinh tế, kể từ năm $2023$ trở đi mức khai thác dầu mỏ mỗi năm của quốc gia $X$ tăng lên $5 \%$ so với năm trước. Hỏi đến hết năm $2035$ thì trữ lượng dầu của nước này còn lại bao nhiêu thùng (làm tròn đến hàng đơn vị)?
	\loigiai{
	Trữ lượng dầu mỏ khai thác trong năm $2022$ là $\dfrac{120}{50}=\dfrac{12}{5}$ triệu thùng.\\
	Trữ lượng khai thác dầu mỏ hàng năm tính từ năm $2022$ đến năm $2035$ là
	$$\dfrac{12}{5}; \dfrac{12}{5}\cdot 1{,}05;\dfrac{12}{5}\cdot \left(1{,}05\right)^2;\dfrac{12}{5}\cdot \left(1{,}05\right)^3;\cdots;\dfrac{12}{5}\cdot \left(1{,}05\right)^{13}.$$
	Dãy trên là một cấp số nhân có $u_1=\dfrac{12}{5}$, công bội $q=1{,}05$.\\
	Tổng trữ lượng dầu khai thác từ năm $2022$ đến năm $2035$ là
	$$S_{14}=u_1\cdot \dfrac{1-q^{14}}{1-q}=\dfrac{12}{5}\cdot \dfrac{1-\left(1{,}05\right)^{14}}{1-1{,}05}\approx 47 \text{ (triệu thùng).}$$
	Vậy tính đến hết năm $2035$ thì trữ lượng dầu của nước này còn lại $120-47=73$ triệu thùng.
	}
\end{bt}