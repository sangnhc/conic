
\de{ĐỀ THI GIỮA HỌC KỲ I NĂM HỌC 2023-2024}{THPT Nguyễn Thị Minh Khai}


%Câu 1...........................
\begin{bt}%[1D1V5-5]%[Dự án đề kiểm tra Toán 11 GHKI NH23-24- AHL]%[THPT - NguyenThiMinhKhai-HCM]
Giải các phương trình sau
\begin{multicols}{2}
\begin{enumerate}
\item $\sin 2 x=\sin \dfrac{\pi}{5}$.
\item $\cot \left(x+50^{\circ}\right)=\sqrt{3}$.
\item $\cos \left(3 x-\dfrac{\pi}{4}\right)=\cos 2 x$.
\item $\sin (x+6)=2 \cos ^2 5 x-1$.
\end{enumerate}
\end{multicols}
\loigiai{
\begin{enumerate}
\item  $\sin 2 x=\sin \dfrac{\pi}{5}$\\
$
\Leftrightarrow \hoac{ 
& 2 x = \dfrac { \pi } { 5 } + k 2 \pi  \\
& 2 x = \pi - \dfrac { \pi } { 5 } + k 2 \pi 
} \Leftrightarrow 
\hoac{
& x=\dfrac{\pi}{10}+k \pi \\
& x=\dfrac{2 \pi}{5}+k \pi
} (k \in \mathbb{Z})
$.
\item 
$  \cot \left(x+50^{\circ}\right)=\sqrt{3} \\
 \Leftrightarrow \cot \left(x+50^{\circ}\right)=\cot 30^{\circ} \\
 \Leftrightarrow x+50^{\circ}=30^{\circ}+k 180^{\circ} \\
  \Leftrightarrow x=-20^{\circ}+k 180^{\circ} \quad(k \in \mathbb{Z})
$.
\item $
 \cos \left(3 x-\dfrac{\pi}{4}\right)=\cos 2 x \\
 \Leftrightarrow \hoac{
& 3 x - \dfrac { \pi } { 4 } = 2 x + k 2 \pi  \\
& 3 x - \dfrac { \pi } { 4 } = - 2 x + k 2 \pi 
} \Leftrightarrow \hoac{
& x=\dfrac{\pi}{4}+k 2 \pi \\
& x=\dfrac{\pi}{20}+k \dfrac{2 \pi}{5}
} (k \in \mathbb{Z})
$.
\item  $\sin (x+6)=2 \cos ^2 5 x-1 \\
 \Leftrightarrow \sin (x+6)=\cos 10 x \\
 \Leftrightarrow \sin (x+6)=\sin \left(\dfrac{\pi}{2}-10 x\right) \\
 \Leftrightarrow \hoac{
& x + 6 = \dfrac { \pi } { 2 } - 1 0 x + k 2 \pi  \\
& x + 6 = \pi - \dfrac { \pi } { 2 } + 1 0 x + k 2 \pi 
} \Leftrightarrow \hoac{
& x=\dfrac{\pi}{22}-\dfrac{6}{11}+k \dfrac{2 \pi}{11} \\
& x=-\dfrac{\pi}{18}+\dfrac{2}{3}-k \dfrac{2 \pi}{9}
}(k \in \mathbb{Z})
$.
\end{enumerate}
}
\end{bt}

%Câu 2...........................
\begin{bt}%[1D2H2-2]%[Dự án đề kiểm tra Toán 11 GHKI NH23-24- AHL]%[THPT - NguyenThiMinhKhai-HCM]
Chứng minh dãy số $\left(u_n\right)$ với $u_n=1912+n\left(n \in \mathbb{N}^*\right)$ là cấp số cộng. Xác định công sai, số hạng đầu của $\left(u_n\right)$.
\loigiai{
$\left(u_n\right) \colon u_n=1912+n \quad\left(n \in \mathbb{N}^*\right)$.\\
$u_{n+1}=1912+(n+1)=1913+n$.\\
$u_{n+1}-u_n=(1913+n)-(1912+n)=1, \forall n \in \mathbb{N}^*$.\\
Vậy $\left(u_n\right)$ là một cấp số cộng với công sai $d=1$, $u_1=1912+1=1913$.
}

\end{bt}


	\begin{bt}%[1D2H2-4]%[Dự án đề kiểm tra Toán 11 GHKI NH23-24- Dương Phước Sang]%[THPT Nguyễn Thị Minh Khai - Tp HCM]
		Tìm số hạng đầu và công sai của cấp số cộng $(u_n)$ biết 
		$\heva{&u_7-u_3=-8\\&u_6 \cdot u_9=72.}$
		\loigiai{
			Với $d$ là công sai của cấp số cộng $(u_n)$, ta có 
			$$\begin{aligned}
				\heva{&u_7-u_3=-8\\&u_6 \cdot u_9=72}
				&\Leftrightarrow \heva{&u_1+6d-(u_1+2d)=-8\\&\left(u_1+5d\right)\left(u_1+8d\right)=72}\\
				&\Leftrightarrow \heva{&4d=-8\\&u_1^2+13d\,u_1+40d^2=72}\\
				&\Leftrightarrow \heva{&d=-2\\&u_1^2-26u_1+88=0}\\
				&\Leftrightarrow \heva{&d=-2\\&u_1=22} 
				\text{ hoặc } \heva{&d=-2\\&u_1=4.}
			\end{aligned}$$
			Vậy có $2$ cấp số cộng thỏa mãn yêu cầu bài toán là $(u_n)\colon \heva{&u_1=22\\&d=-2} 
			\text{ và } (u_n)\colon \heva{&u_1=4\\&d=-2.}$
		}
	\end{bt}
	
	\begin{bt}%[1D1V5-6]%[Dự án đề kiểm tra Toán 11 GHKI NH23-24- Dương Phước Sang]%[THPT Nguyễn Thị Minh Khai - Tp HCM]
		Cho hai vật dao động điều hòa theo phương trình lần lượt là $x_1(t)=8\cos\left(4\pi t+\dfrac{\pi}{2}\right)$ $(\mathrm{cm})$ và $x_2(t)=-8\cos (4\pi t)$ $(\mathrm{cm})$, trong đó $x_1(t)$, $x_2(t)$ là li độ của hai vật tại thời điểm $t$ (giây). Khi hai vật dao động trong thời gian từ $0$ đến $10$ giây, hỏi chúng có cùng li độ mấy lần?
		\loigiai{
			Thời điểm hai vật có cùng li độ ta có $8\cos\left(4\pi t+\dfrac{\pi}{2}\right)=-8\cos (4\pi t)$
			$$\begin{aligned}
				&\Leftrightarrow 
				 \cos\left(4\pi t+\dfrac{\pi}{2}\right)=\cos (\pi-4\pi t)\\
				 & \Leftrightarrow \hoac{&4\pi t+\dfrac{\pi}{2}=\pi-4\pi t+k2\pi\\&4\pi t+\dfrac{\pi}{2}=-\pi+4\pi t+k2\pi}\\
				 & \Leftrightarrow \hoac{&8\pi t=\dfrac{\pi}{2}+k2\pi\\&0t=-\dfrac{3\pi}{2}+k2\pi\text{: vô nghiệm}}\\
				 & \Leftrightarrow 
				 t=\dfrac{1}{16}+\dfrac{k}{4} \ (k \in \mathbb{Z}).
			\end{aligned}$$
			Xét $t \in [0;10]$, ta có $0 \leq \dfrac{1}{16}+\dfrac{k}{4} \leq 10 \Leftrightarrow -\dfrac{1}{4} \leq t \leq \dfrac{159}{4} \Leftrightarrow k \in \{0;1;2;\ldots;39\}$ (do $k \in \mathbb{Z}$).\\
			Vậy trong $10$ giây đầu tiên, hai vật có cùng li độ $40$ lần.
		}
	\end{bt}
