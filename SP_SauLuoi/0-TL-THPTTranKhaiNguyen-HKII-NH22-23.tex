\de{ĐỀ THI ÔN TẬP  HỌC KỲ II NĂM HỌC 2022-2023}{THPT Trần Khai Nguyên}


\begin{bt}%[0T7B3-2]%[Đề Cuối Kì II-THPT Trần Khai Nguyên]%[Trương Đăng Khoa]
	Giải phương trình $\sqrt{2x^2-4x-2}=\sqrt{x^2-x-2}$.
	\loigiai{
		Bình phương hai vế của phương trình, ta được
		\begin{eqnarray*}
			&&2x^2-4x-2=x^2-x-2\\
			&\Rightarrow&  x^2-3x=0\\
			&\Rightarrow& x=0\,\ \text{hoặc}\,\ x=3.
		\end{eqnarray*}
		Thay lần lượt các giá trị trên vào phương trình đã cho, ta thấy chỉ có $x=3$ thỏa mãn.\\
		Vậy nghiệm của phương trình đã cho là $x=3$.
	}
\end{bt}

\begin{bt}%[0T7T2-3]%[Đề Cuối Kì II-THPT Trần Khai Nguyên]%[Trương Đăng Khoa]
	Một công ty đồ gia dụng sản xuất bình nước thấy rằng khi đơn giá của bình đựng nước là $x$ nghìn đồng thì doanh thu $R$ (tính theo đơn vị nghìn đồng) sẽ là $R(x)=-560x^2+50000x$. Với khoảng đơn giá nào của bình đựng nước thì doanh thu từ việc bán bình đựng nước vượt mức $1$ tỉ đồng? Đơn giá được làm tròn đến số nguyên.
	\loigiai{
		Doanh thu từ việc bán bình đựng nước vượt mức $1$ tỉ đồng khi và chỉ khi 
		\begin{eqnarray*}
			&&R(x)>1000000\\
			&\Leftrightarrow& -560x^2+50000x>1000000\\
			&\Leftrightarrow& -560x^2+50000x-1000000>0\\
			&\Leftrightarrow& \dfrac{625-25\sqrt{65}}{14}< x< \dfrac{625+25\sqrt{65}}{14}.
		\end{eqnarray*}
		Mà $x\in\mathbb{Z}$ nên $x\in\{31;32;\ldots;58;59\}$.\\
		Vậy đơn giá của bình nước trong từ khoảng $31$ nghìn đồng đến $59$ nghìn đồng thì doanh thu việc bán bình đựng nước vượt mức $1$ tỉ đồng.
	}
\end{bt}

\begin{bt}%[0T8K1-3]%[Đề Cuối Kì II-THPT Trần Khai Nguyên]%[Trương Đăng Khoa]
	Bạn Hoa có $10$ quyển sách Toán khác nhau và $8$ quyển sách Lý khác nhau. Bạn Hoa cần chọn $5$ quyển sách đề đọc. Hỏi bạn Hoa có bao nhiêu sự lựa chọn biết phải có cả môn Toán và Lý đồng thời số sách Toán nhiều hơn sách Lý?
	\loigiai{
		Bạn Hoa cần chọn $5$ quyển sách về đọc trong đó có cả môn Toán và Lý đồng thời số sách Toán nhiều hơn sách Lý, khi đó có $2$ trường hợp xảy ra.
		\begin{enumerate}
			\item \textbf{Trường hợp 1.} Bạn Hoa chọn $4$ quyển sách Toán và $1$ quyển sách Lý.
			\begin{itemize}
				\item Chọn $4$ quyển sách Toán trong $10$ quyển sách Toán có $\mathrm{C}_{10}^4$ cách chọn.
				\item Chọn $1$ quyển sách Lý trong $8$ quyển sách Lý có $\mathrm{C}_8^1$ cách chọn.
			\end{itemize}
			Do đó số cách bạn Hoa chọn $4$ quyển sách Toán và $1$ quyển sách Lý là $\mathrm{C}_{10}^4\cdot \mathrm{C}_8^1=1680$ cách.
			\item \textbf{Trường hợp 2.} Bạn Hoa chọn $3$ quyển sách Toán và $2$ quyển sách Lý.
			\begin{itemize}
				\item Chọn $3$ quyển sách Toán trong $10$ quyển sách Toán có $\mathrm{C}_{10}^3$ cách chọn.
				\item Chọn $2$ quyển sách Lý trong $8$ quyển sách Lý có $\mathrm{C}_8^2$ cách chọn.
			\end{itemize}
			Do đó số cách bạn Hoa chọn $3$ quyển sách Toán và $2$ quyển sách Lý là $\mathrm{C}_{10}^3\cdot \mathrm{C}_8^2=3360$ cách.
		\end{enumerate}
		Vậy theo quy tắc cộng, bạn Hoa có $3360$ cách chọn $5$ quyển sách về đọc trong đó có cả môn Toán và Lý đồng thời số sách Toán nhiều hơn sách Lý.
	}
\end{bt}

\begin{bt}%[0T8B3-1]%[Đề Cuối Kì II-THPT Trần Khai Nguyên]%[Trương Đăng Khoa]
	Sử dụng công thức nhị thức Newton, hãy khai triển biểu thức $(3x+2y)^4$.
	\loigiai{
		Ta có $\begin{aligned}[t]
			(3x+2y)^4
			=&\ \mathrm{C}_4^0\cdot (3x)^4 +\mathrm{C}_4^1\cdot (3x)^3\cdot 2y +\mathrm{C}_4^2\cdot (3x)^2\cdot(2y)^2+\mathrm{C}_4^3\cdot (3x)\cdot (2y)^3+\mathrm{C}_4^4\cdot (2y)^4\\
			=&\ 81x^4+216x^3y+216x^2y^2+96xy^3+16y^4.
		\end{aligned}$
	}
\end{bt}

\begin{bt}%%[0D0B2-2]%[THPT Trần Khai Nguyên-Trần Lê Vĩnh Phúc]
Một hộp có $ 3 $ bi trắng, $ 4 $ bi xanh, $ 5 $ bi vàng. Lấy ngẫu nhiên từ hộp ra $ 3 $ viên bi. Tính xác suất của mỗi biến cố sau
\begin{enumerate}
\item $ A $: “Ba viên bi lấy ra có cùng màu”.
\item $ B $: “Ba viên bi lấy ra không có viên bi màu trắng”.
\end{enumerate}
\loigiai{
\begin{enumerate}
\item $\mathrm{P}(A)=\dfrac{\mathrm{C}_3^3+\mathrm{C}_4^3+\mathrm{C}_5^3}{\mathrm{C}_{12}^3}=\dfrac{3}{44} $.
\item $ \mathrm{P}(B)=\dfrac{\mathrm{C}_9^3}{\mathrm{C}_{12}^3}=\dfrac{7}{22} $.
\end{enumerate}
}
\end{bt}


\begin{bt}%%[0D0K2-2]%[THPT Trần Khai Nguyên-Trần Lê Vĩnh Phúc]
Chọn ngẫu nhiên $ 2 $ số nguyên dương không vượt quá $ 20 $. Tính xác suất để chọn được $ 2 $ số có tích là một số chẵn.
\loigiai{
Số các kết quả có thể xảy ra là $ n(\Omega)= \mathrm{C}_{20}^2=190$.\\
Gọi $ C $ là biến cố “2 số có tích là một số chẵn” nên 
$ n(C)=\mathrm{C}_{10}^2+\mathrm{C}_{10}^1\cdot \mathrm{C}_{10}^1=145 $ cách.\\
Do đó $ \mathrm{P}(C)=\dfrac{\mathrm{C}_{10}^2+\mathrm{C}_{10}^1\cdot \mathrm{C}_{10}^1}{\mathrm{C}_{20}^2}=\dfrac{29}{38} $.
}
\end{bt}


\begin{bt}%%[0H9K3-3]%[THPT Trần Khai Nguyên-Trần Lê Vĩnh Phúc]
Trong mặt phẳng $Oxy$, cho hai điểm $A(-1 ; 2), B(3 ; 5)$, đường thẳng $d \colon2x-y+2=0$ và đường tròn $(C) \colon (x-1)^2+(y+1)^2=5$.
\begin{enumerate}
\item  Viết phương trình tổng quát của đường thẳng $AB$;
\item  Tìm tọa độ điểm $M$ nằm trên đoạn thẳng $A B$ thỏa mãn $AM=3\cdot BM$;
\item  Viết phương trình tiếp tuyến $\Delta$ với đường tròn $(C)$ biết tiếp tuyến $\Delta$ song song với đường thẳng $d$.
\end{enumerate}
\loigiai{
\begin{enumerate}
\item  Phương trình tổng quát của đường thẳng $AB$ đi qua $ A $ và nhận vectơ $ \vec{n}=(3;-4) $ làm vectơ pháp tuyến.\\
Do đó $ d\colon 3x-4y+11=0 $.
\item  $M$ nằm trên đoạn thẳng $A B$ nên $ M(3+4t;5+3t) $.\\
Ta có: $ AM=3MB \Leftrightarrow \vec{AM}=3\vec{MB} \Leftrightarrow \heva{&4+4t=-12t\\&3+3t=-9t} \Leftrightarrow t=-\dfrac{1}{4}  $.\\
Vậy $ M\left(2; \dfrac{17}{4}\right) $.
\item 
Đường tròn có tâm $ I(1;-1) $ và bán kính $ R=\sqrt{5} $.\\
Vì $ d $ song song $ \Delta $ nên $ \Delta \colon 2x-y+m=0$ với $ m \neq 2 $.\\
Vì $ \Delta $ tiếp xúc với $ (C) $ nên $ \mathrm{d}\left(I; \Delta\right)=\dfrac{\left|2.1-(-1)+m\right|}{\sqrt{2^2+1^2}}=\sqrt{5} \Leftrightarrow \left|3+m\right|=5 \Leftrightarrow \hoac{&m=2\\&m=-8}$.\\
So điều kiện ta nhận $ m=-8 $.\\
Vậy $ \Delta \colon 2x-y-8=0 $.
\end{enumerate}
}
\end{bt}


\begin{bt}%%[0H9K4-10]%[THPT Trần Khai Nguyên-Trần Lê Vĩnh Phúc]
Một đường hầm cho xe đi hai chiều có mặt cắt là một nửa hình elip, chiều rộng của hầm là $12 \mathrm{~m}$, khoảng cách từ điểm cao nhất của elip so với mặt đường là $3 \mathrm{~m}$.
\begin{center}
\begin{tikzpicture}[scale=.8, font=\footnotesize, line join=round, line cap=round,>=stealth]
\fill[color=orange] (3.5,0) arc (0:180:3.5 cm and 1.3cm)--(-4.5,0)--(-4.5,3.4)--(4.5,3.4)--(4.5,0)--cycle;
\draw[pattern color=white, pattern=bricks] (3.5,0) arc (0:180:3.5 cm and 1.3cm)--(-4.5,0)--(-4.5,3.4)--(4.5,3.4)--(4.5,0)--cycle;
\draw[blue] (3.5,0) arc (0:180:3.5 cm and 1.3cm);
\draw[->] (-5,0)--(5,0) node [below]{$x$};
\draw[->] (0,-1)--(0,4) node [left]{$y$};
\node at (0,0) [below left]{$O$};
\node at (3.5,0) [below]{$A$};
\node at (0,1.2) [below right]{$B$};
\end{tikzpicture}
\end{center}
\begin{enumerate}
\item  Viết phuơng trình chính tắc của elip đó.
\item  Một chiếc xe tải có chiều rộng $2{,}2 \mathrm{~m}$ và chiều cao $2{,}5 \mathrm{~m}$. Hỏi nếu xe đi đúng làn đường quy định thì có thể qua đường hầm không? Giả sử đường hầm luôn thẳng và bề rộng vạch kẻ ngăn cách hai làn đường không đáng kể.
\end{enumerate}
\loigiai{
\begin{enumerate}
\item $ \dfrac{x^2}{36}+\dfrac{y^2}{9}=1 $;
\item Gọi $ N $ là điểm nằm trên elip có hoành độ là $ 2{,}2 $, khi đó tung độ của $ N $ là $ y=\sqrt{9.\left(1-\dfrac{2{,}2^2}{36}\right)}\approx 2{,}79 \mathrm{~m}$. \\
Khi xe tải qua cổng, chiều rộng xe tải là $2{,}2 \mathrm{~m}$ và cao $2{,}5 \mathrm{~m}$ thấp hơn $ 2{,}79 \mathrm{~m} $ nên có thể đi qua.
\end{enumerate}
}
\end{bt}
