\de{ĐỀ THI HỌC KỲ II NĂM HỌC 2022-2023}{THPT Gia Định}

%Câu 1
\begin{bt}%[0T9K4-1]%[Dự án đề kiểm tra HKII NH22-23-Phan Trung Hiếu]%[THPT Gia Định]
Trong mặt phẳng tọa độ $Oxy$, cho $(E)\colon5x^2+9y^2=45$.
\begin{enumerate}
	\item Tìm tọa độ các tiêu điểm, tọa độ các đỉnh, tiêu cự và độ dài các trục của $(E)$.
	\item Gọi $M$, $N$ là các điểm trên $(E)$ sao cho $NF_1+MF_2=7$. Tính giá trị $MF_1+NF_2$ (với $F_1$, $F_2$ là hai tiêu điểm của $(E)$).
\end{enumerate}
\loigiai{
	\begin{equation*}
		(E)\colon\dfrac{x^2}{9}+\dfrac{y^2}{5}=1.
	\end{equation*}
Ta có $a=3$, $b=\sqrt{5}$, $c=\sqrt{3^2-5}=2$.
\begin{enumerate}
	\item Tọa độ các tiêu điểm là $F_1(-2;0)$ và $F_2(2;0)$.\\
	Tọa độ các đỉnh là $A_1(-3;0)$, $A_2(3;0)$, $B_1(0;-\sqrt{5})$, $B_2(0;\sqrt{5})$.\\
	Tiêu cự $F_1F_2=2c=4$.\\
	Độ dài trục lớn $2a=6$, độ dài trục bé $2b=2\sqrt{5}$.
	\item Vì $N\in (E)$ và $M\in(E)$ nên ta có hệ phương trình
	\begin{equation*}
		\heva{&NF_1+NF_2=2a=6\\&MF_1+MF_2=2a=6}
		\Rightarrow NF_1+NF_2+MF_1+MF_2=12.
	\end{equation*}
	Mà $NF_1+MF_2=7$ nên $MF_1+NF_2=12-7=5$.
\end{enumerate}
}
\end{bt}


%Câu 2
\begin{bt}%[0T9K3-5]%[Dự án đề kiểm tra HKII NH22-23-Phan Trung Hiếu]%[THPT Gia Định]
Trong mặt phẳng với hệ trục tọa độ $Oxy$, cho đường tròn
\begin{equation*}
	(C)\colon x^2+y^2-6x+8y-11=0.
\end{equation*}
\begin{enumerate}
	\item Tìm tọa độ tâm $I$ và tính bán kính của đường tròn $(C)$.
	\item Viết phương trình tiếp tuyến $(d)$ với đường tròn $(C)$, biết tiếp tuyến $(d)$ song song với đường thẳng $(\Delta)\colon3x+4y-23=0$.
\end{enumerate}	
\loigiai{
	Ta có $a=3$, $b=-4$, $c=-11$
\begin{enumerate}
	\item Tọa độ tâm $I$ của đường tròn $(C)$ là $I(3;-4)$.\\
	Bán kính $R$ của đường tròn là $R=\sqrt{a^2+b^2-c}=\sqrt{3^2+(-4)^2-(-11)}=6$.
	\item Gọi phương trình tiếp tuyến $(d)$ với đường tròn $(C)$ là $ax+by+c=0$.\\
	Vì $(d)$ song song với đường thẳng $(\Delta)\colon3x+4y-23=0$ nên phương trình tiếp tuyến $(d)$ có dạng
	\begin{equation*}
		3x+4y+c=0,\quad (c\ne -23).
	\end{equation*}
	Mặt khác,
	\allowdisplaybreaks
	\begin{eqnarray*}
		&&\mathrm{d}(I,(d))=R\\
		&\Rightarrow&\dfrac{|3x_I+4y_I+c|}{\sqrt{3^2+4^2}}=6\\
		&\Leftrightarrow&\dfrac{|3\cdot3+4\cdot(-4)+c|}{5}=6\\
		&\Leftrightarrow&|c-7|=30\\
		&\Leftrightarrow&\hoac{&c=37\quad\text{(nhận)}\\&c=-23\quad(\text{loại}).}
	\end{eqnarray*}
Với $c=37$, phương trình tiếp tuyến với đường tròn $(C)$ là $(d)\colon3x+4y+37=0$.
\end{enumerate}
}
\end{bt}


%Câu 3
\begin{bt}%[0T7B3-2]%[Dự án đề kiểm tra HKII NH22-23-Phan Trung Hiếu]%[THPT Gia Định]
	Giải phương trình $\sqrt{2x^2-59x+299}=x-23$.
\loigiai{
Ta có
\allowdisplaybreaks
\begin{eqnarray*}
	&&\sqrt{2x^2-59x+299}=x-23\\
	&\Rightarrow&\heva{&x-23\geq 0\\&2x^2-59x+299=(x-23)^2}\\
	&\Leftrightarrow&\heva{&x\geq 23\\&2x^2-59x+299=x^2-46x+529}\\
	&\Leftrightarrow&\heva{&x\geq 23\\&x^2-13x-230=0}\\
	&\Leftrightarrow&\heva{&x\geq 23\\&\hoac{&x=-10\quad(\text{loại})\\&x=23\quad(\text{nhận}).}}
\end{eqnarray*}
Vậy nghiệm của phương trình là $S=\{23\}$.
}
\end{bt}

\begin{bt}%[0T7K2-1]%[Dự án đề kiểm tra HKII NH22-23- Nguyễn Cường]%[THPT Gia Định]
Tìm tất cả các giá trị nguyên của $m$ để $x^2+2x-m^2+3m+5\ge 0$, $\forall x\in\mathbb{R}$.
\loigiai{
	\allowdisplaybreaks
	\begin{eqnarray*}
		x^2+2x-m^2+3m+5\ge 0,\,\forall x\in\mathbb{R}&\Leftrightarrow&\heva{&a>0\\&\Delta'\le 0}\\
		&\Leftrightarrow&\heva{&1>0\\&1+m^2-3m-5\le 0}\\
		&\Leftrightarrow&m^2-3m-4\le 0\\
		&\Leftrightarrow&-1\le m\le 4.
	\end{eqnarray*}
Do $m\in\mathbb{Z}$ nên $m\in\{-1;0;1;2;3;4\}$.
}
\end{bt}

\begin{bt}%[0T0B2-2]%[Dự án đề kiểm tra HKII NH22-23- Nguyễn Cường]%[THPT Gia Định]
	Một hộp có $18$ quả cầu, trong đó có $8$ quả cầu trắng, $6$ quả cầu vàng và $4$ quả cầu đen (các quả cầu đôi một khác nhau). Lấy ngẫu nhiên $4$ quả cầu từ hộp. Tính xác suất của các biến cố sau
	\begin{enumerate}
		\item $A$ \lq\lq Trong $4$ quả cầu chọn ra nhiều nhất một quả cầu màu vàng\rq\rq.
		\item $B$ \lq\lq Trong $4$ quả cầu chọn ra ít nhất hai quả cầu màu trắng\rq\rq.
		\item $C$ \lq\lq $4$ quả cầu chọn ra có đủ ba màu\rq\rq.
	\end{enumerate}
	\loigiai{
		Ta có $n(\Omega)=\mathrm{C}_{18}^4=3060$.
		\begin{enumerate}
			\item Biến cố $A$ \lq\lq Trong $4$ quả cầu chọn ra nhiều nhất một quả cầu màu vàng\rq\rq.
			\begin{itemize}
				\item Chọn $4$ quả cầu từ $12$ quả cầu trắng, đen có $\mathrm{C}_{12}^4=495$.
				\item Chọn $1$ quả cầu vàng và $3$ quả cầu từ $12$ quả cầu trắng, đen có $\mathrm{C}_6^1\cdot \mathrm{C}_{12}^3=1320$.
			\end{itemize}
			Suy ra $n(A)=495+1320=1815$.\\
			Vậy xác suất biến cố $A$ là $\mathrm{P}(A)=\dfrac{n(A)}{n(\Omega)}=\dfrac{1815}{3060}=\dfrac{121}{204}$.
			\item Biến cố $B$ \lq\lq Trong $4$ quả cầu chọn ra ít nhất hai quả cầu màu trắng\rq\rq.
			\begin{itemize}
				\item Chọn $2$ quả cầu trắng và $2$ quả cầu từ $10$ quả cầu vàng, đen có $\mathrm{C}_8^2\cdot \mathrm{C}_{10}^2=1260$.
				\item Chọn $3$ quả cầu trắng và $1$ quả cầu từ $10$ quả cầu vàng, đen có $\mathrm{C}_8^3\cdot \mathrm{C}_{10}^1=560$.
				\item Chọn $4$ quả cầu trắng có $\mathrm{C}_8^4=70$.
			\end{itemize}
			Suy ra $n(B)=1260+560+70=1890$.\\
			Vậy xác suất biến cố $B$ là $\mathrm{P}(B)=\dfrac{n(B)}{n(\Omega)}=\dfrac{1890}{3060}=\dfrac{21}{34}$.
			\item Biến cố $C$ \lq\lq $4$ quả cầu chọn ra có đủ ba màu\rq\rq.
			\begin{itemize}
				\item Chọn $1$ quả cầu trắng, $1$ quả cầu vàng, $2$ quả cầu đen có $\mathrm{C}_8^1\cdot \mathrm{C}_6^1\cdot\mathrm{C}_4^2=288$.
				\item Chọn $1$ quả cầu trắng, $2$ quả cầu vàng, $1$ quả cầu đen có $\mathrm{C}_8^1\cdot \mathrm{C}_6^2\cdot\mathrm{C}_4^1=480$.
				\item Chọn $2$ quả cầu trắng, $1$ quả cầu vàng, $1$ quả cầu đen có $\mathrm{C}_8^2\cdot \mathrm{C}_6^1\cdot\mathrm{C}_4^1=672$.
			\end{itemize}
			Suy ra $n(C)=288+480+672=1440$.\\
			Vậy xác suất biến cố $C$ là $\mathrm{P}(C)=\dfrac{n(C)}{n(\Omega)}=\dfrac{1440}{3060}=\dfrac{8}{17}$.
		\end{enumerate}
	}
\end{bt}


\begin{bt}%[0T7K3-2]%[Dự án đề kiểm tra HKII NH22-23- Nguyễn Cường]%[THPT Gia Định]
	Cho tam giác $ABC$ vuông tại $A$, độ dài cạnh huyền $BC$ lớn hơn độ dài cạnh $AC$ là $2$ (cm). Tính độ dài các cạnh của $\triangle ABC$, biết chu vi $\triangle ABC$ là $24$ (cm).
	\loigiai{
\immini{
Đặt $AC=x$ (cm) với $x>0$, ta có
\begin{itemize}
	\item $BC=x+2$.
	\item Áp dụng định lý Pi-ta-go, ta có
	\[AB=\sqrt{BC^2-AC^2}=\sqrt{(x+2)^2-x^2}=\sqrt{4x+4}. \]
\end{itemize}
}{
\begin{tikzpicture}[>=stealth,line join=round,line cap=round,font=\footnotesize,scale=1]
	\def\r{2};
	\def\g{110};
	\path
	(0,0) coordinate (O)
	(0:\r) coordinate (C)
	(180:\r) coordinate (B)
	(\g:\r) coordinate (A)
	;
	\draw
	(A)--(B)--(C)--cycle
	;
	\foreach \x/\g in {B/180,C/0,A/130}\fill[black](\x) circle (1pt) +(\g:3mm) node {$\x$};
	\foreach \x/\y/\z in {B/A/C}{\draw pic[draw,angle radius=2mm]{right angle=\x--\y--\z};}
\end{tikzpicture}
}	
\noindent Do chu vi $\triangle ABC$ là $24$ (cm) nên 
\allowdisplaybreaks
\begin{eqnarray*}
	x+x+2+\sqrt{4x+4}=24&\Leftrightarrow& 2x+2+2\sqrt{x+1}=24\\
	&\Leftrightarrow&x+1+\sqrt{x+1}=12\\
	&\Leftrightarrow& \sqrt{x+1}=11-x \quad \quad (1)\\
	&\Rightarrow&x+1=(11-x)^2\\
	&\Leftrightarrow&x^2-23x+120=0\\
	&\Leftrightarrow&\hoac{&x=8\\&x=15.}
\end{eqnarray*}
Với $x=15$ thử lại phương trình (1), ta có $\sqrt{16}\ne -4$, nên $x=15$ không là nghiệm.\\
Với $x=8$ thử lại phương trình (1), ta có $\sqrt{16}=4$, nên $x=8$ là nghiệm của phương trình.\\
Vậy độ dài ba cạnh của $\triangle ABC$ là $AC=8$ cm, $BC=10$ cm và $AB=6$ cm.
	}
\end{bt}


