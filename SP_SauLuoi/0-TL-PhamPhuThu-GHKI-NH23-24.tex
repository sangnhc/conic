\de{ĐỀ THI GIỮA HỌC KỲ I NĂM HỌC 2023-2024}{TRƯỜNG THPT NGUYỄN PHÚ THỨ}
\setcounter{bt}{0}

\begin{bt}%[Dự án Tex đề GHK-HK 2023-2024]%[Nhật Thiện]%[0D1B2-1]
	Viết các tập hợp sau dưới dạng liệt kê phần tử.
	\begin{enumEX}{2}
		\item $A=\left\{x\in \mathbb{N}\mid 3<x^2\le 16\right\}$;
		\item $B=\left\{x\in \mathbb{R}\mid(x^2-4)(|x|-1)=0\right\}$.
	\end{enumEX}
\loigiai{
\begin{enumerate}
	\item $A=\{2;3;4\}$.
	\item Ta có $(x^2-4)(|x|-1)=0\Leftrightarrow \hoac{&x^2-4=0\\&|x|-1=0}\Leftrightarrow \hoac{&x=2\\&x=-2\\&x=1\\&x=-1.}$\\
	Suy ra $B=\{2;-2;1;-1\}$.
\end{enumerate}
}
\end{bt}
\begin{bt}%[Dự án Tex đề GHK-HK 2023-2024]%[Nhật Thiện]%[0D1B2-2]
	Cho tập hợp $X=\{1;2;3;a\}$. Tìm tất cả các tập hợp con có nhiều hơn $2$ phần tử của tập hợp $X$.
	\loigiai{
	Tập con của tập $X$ gồm: $\{1;2\}$, $\{1;3\}$, $\{1;a\}$, $\{2;3\}$, $\{2;a\}$, $\{3;a\}$, $\{1;2;3\}$, $\{1;2;a\}$, $\{1;3;a\}$, $\{2;3;a\}$, $\{1;2;3;a\}$.
}
\end{bt}
\begin{bt}%[Dự án Tex đề GHK-HK 2023-2024]%[Nhật Thiện]%[0D1B3-5]
	\begin{enumerate}
		\item Cho hai tập hợp $B=\{1;3;5;7\}$, $C=\{x\in \mathbb{R}\big| x^2-7x+10=0\}$. Xác định các tập hợp $B\cap C$, $B\cup C$, $C\setminus B$.
		\item Cho hai tập hợp $A=[-2;+\infty)$ và $B(-\sqrt{5};3)$. Xác định các tập hợp $A\cap B$, $A\cup B$, $A\setminus B$, $C_{\mathbb{R}}B$.
	\end{enumerate}
\loigiai{
\begin{enumerate}
	\item Ta có $C=\{2;5\}$.\\
	Suy ra $B\cap C=\{5\}$, $B\cup C=\{1;3;5;7;2\}$, $C\setminus B=\{2\}$.
	\item $A\cap B=[-2;3)$, $A\cup B=(-\sqrt{5};+\infty)$, $A\setminus B=[3;+\infty)$, $C_{\mathbb{R}}B=(-\infty;-\sqrt{5}]\cup [3;+\infty)$.
\end{enumerate}
}
\end{bt}

\begin{bt}%[Dự án đề kiểm tra Toán 10 GHKI NH23-24- Hieu Phan]%[THPT PHAM PHÚ THỨ - Tp HCM]%[0D1H2-3]
    Viết các tập hợp sau dưới dạng khoảng, đoạn, nửa khoảng.
    \begin{enumerate}
        \item $A=\{x \in \mathbb{R} \mid x<3\}$.
        \item $B=\{x \in \mathbb{R} \mid 2 \leq x \leq 5\}$.
        \item $C=\left\{x \in \mathbb{R} \mid 3 \leq x<\sqrt{17}\right\}$.
        \item $D=\{x \in \mathbb{R} \mid 3-2 x \leq 5+3 x\}$.
    \end{enumerate}
    \loigiai
    {
          \begin{enumerate}
            \item $A=(-\infty;3)$.
            \item $B=[2;5]$.
            \item $C=\left[3;\sqrt{7}\right)$.
            \item $D=\{x \in \mathbb{R} \mid 3-2 x \leq 5+3 x\}=\left\{x \in \mathbb{R} \mid x \geq - \dfrac{2}{5}\right\}$.\\
            Suy ra $D=\left[-\dfrac{2}{5};+\infty\right)$.
        \end{enumerate}
    }
\end{bt}
\begin{bt}%[Dự án đề kiểm tra Toán 10 GHKI NH23-24- Hieu Phan]%[THPT PHAM PHÚ THỨ - Tp HCM]%[0D1V3-5]
     Lớp $10 \mathrm{A}$ có $26$ học sinh tham gia câu lạc bộ thể thao, $19$ học sinh tham gia câu lạc bộ âm nhạc và $9$ học sinh tham gia cả hai câu lạc bộ trên.
     \begin{enumerate}
         \item Hỏi có bao nhiêu học sinh lớp 10A tham gia câu lạc bộ thể thao và không tham gia câu lạc bộ âm nhạc?
         \item Biết rằng lớp $10 \mathrm{A}$ có $40$ học sinh. Hỏi có bao nhiêu học sinh lớp $10 \mathrm{A}$ không tham gia cả hai câu lạc bộ trên?
     \end{enumerate}

    \loigiai
    {
         \begin{enumerate}
            \item Gọi $B$, $C$ lần lượt là tập hợp các học sinh lớp 10A tham gia câu lạc bộ thể thao và câu lạc bộ âm nhạc.\\
            Ta có $n(B)=26$, $n(C)=19$, $n(B\cap C)=9$.\\
            Số học sinh tham gia câu lạc bộ thể thao và không tham gia câu lạc bộ âm nhạc là $$n(A)-n(A\cap B)=26-9=17.$$
            \item Số học sinh không tham gia cả hai câu lạc bộ là $$40-(n(A)+n(B)-n(A\cap B))=40-(26+19-9)=4.$$
        \end{enumerate}
    }
\end{bt}

\begin{bt}%[Dự án đề kiểm tra Toán 10 GHKI NH23-24- Hieu Phan]%[THPT PHAM PHÚ THỨ - Tp HCM]%[0H4V2-2]
    Cho tam giác $MNP$ có $MN=3, NP=5$ và $MP=7$.
    \begin{enumerate}
        \item Tính số đo các góc của tam giác $MNP$ và diện tích tam giác $M N P$.
        \item Tính bán kính đường tròn ngoại tiếp tam giác $MNP$.
        \item Tính độ dài đường trung tuyến $MI$ của tam giác $MNP$.
    \end{enumerate}
    \loigiai
    {        \begin{center}           
\begin{tikzpicture}
    	\def\a{7}
	\def\b{3}
	\def\c{5}
	\path 	(0:0) coordinate (P)
			++(0:\a) coordinate (M);
	\path[name path=c1]  (P) circle (\c);
	\path[name path=c2]  (M) circle (\b);
	\path[name intersections={of=c1 and c2,by={N,D}}];
	\pgfresetboundingbox %Co khung hình 
    \coordinate (I) at ($(N)!0.5!(P)$);
	\draw (M)--(P)--(N)--cycle (M)--(I);
	\foreach \x/ \goc in {N/135,P/-135,M/-45, I/135} 
			\fill (\x) circle (1pt)
			($(\x)+(\goc:3mm)$) node {$\x$};	
\end{tikzpicture}
        \end{center}
          \begin{enumerate}
            \item Ta có
            \begin{itemize}
                \item $\cos N=\dfrac{NM^2+NP^2-MP^2}{2\cdot NM\cdot NP}=\dfrac{9+25-49}{30}=-\dfrac{1}{2}\Rightarrow \widehat{N}=120^\circ$.
                \item $\cos M=\dfrac{NM^2+MP^2-NP^2}{2\cdot NM\cdot MP}=\dfrac{9+49-25}{42}=\dfrac{33}{42}\Rightarrow \widehat{M}\approx 38^\circ13'$.
                \item $\widehat{P}\approx180^\circ - 120^\circ-38^\circ13'=21^\circ47'$.
                \item $S_{\triangle MNP}=\dfrac{1}{2}\cdot NP\cdot NM\cdot \sin N =\dfrac{1}{2}\cdot 5\cdot 3\cdot \sin 120^\circ=\dfrac{15\sqrt{3}}{4}$.
            \end{itemize}
            
            \item Ta có $2R=\dfrac{MP}{\sin N}\Rightarrow R=\dfrac{7}{2\cdot \dfrac{\sqrt{3}}{2}}=\dfrac{7\sqrt{3}}{3}$.
            \item $MI^2=\dfrac{2\left(MN^2+MP^2\right)-NP^2}{4}=\dfrac{2(9+49)-25}{4}=\dfrac{\sqrt{91}}{2}$.
        \end{enumerate}
    }
\end{bt}

\begin{bt}%[Dự án đề kiểm tra Toán 10 GHK1 NH 23-24]%[Lương Như Quỳnh]%[0H4H1-2]
	Cho $ \sin x=\dfrac{\sqrt{3}}{2} $ và $ 90^\circ <x<180^\circ $. Tính giá trị biểu thức $ B=\cos^2 x-2\tan x $.
	\loigiai{
Vì $ 90^\circ <x<180^\circ $ nên $ \cos x<0 $.\\
Ta có $ \sin ^2x+ \cos ^2x=1 \Leftrightarrow \cos ^2x=1-\sin ^2x=1-\left(\dfrac{\sqrt{3}}{2}\right)^2=\dfrac{1}{4}$.\\
Suy ra $ \cos x=-\dfrac{1}{2} $.\\
Ta có $ \tan x=\dfrac{\sin x}{\cos x}=\dfrac{\sqrt{3}}{2}:\left(-\dfrac{1}{2}\right)=-\sqrt{3} $.\\
Do đó $ B=\cos^2 x-2\tan x =\dfrac{1}{4}-2(-\sqrt{3})=\dfrac{1}{4}+2\sqrt{3} $.
	}
\end{bt}
\begin{bt}%[Dự án đề kiểm tra Toán 10 GHK1 NH 23-24]%[Lương Như Quỳnh]%[0H4V3-2]
	Đứng ở vị trí $ A $ trên bờ biển, bạn An đo được góc nghiêng so với bờ biển tới một vị trí $C$ trên đảo là $ 32^\circ $. Sau đó di chuyển dọc bờ biển đến vị trí $B$ cách vị trí $A$ một khoảng $ 110 $ m và đo được góc nghiêng so với bờ biển tới vị trí C đã chọn là $ 42^\circ $. Tính khoảng cách từ vị trí $C$ trên đảo tới bờ biển theo đơn vị mét.
	\begin{center}
\definecolor{lightcornflowerblue}{rgb}{0.6, 0.81, 0.93}
\definecolor{cadmiumgreen}{rgb}{0.0, 0.42, 0.24}
\definecolor{trueblue}{rgb}{0.0, 0.45, 0.81}
\definecolor{tumbleweed}{rgb}{0.87, 0.67, 0.53}%màu cát
	
\begin{tikzpicture}[line join=round, line cap=round,scale=1.5,transform shape]
\clip (-4,-2.5) rectangle (4,2.5);
%\path (0,0) node[opacity=.5,scale=.3] {\includegraphics{h1}};
%\draw[gray!50] (-4,-2) grid (4,2);

\tikzset{dao/.pic={
\def\H{ 
(-1.34,1.1)
..controls +(60:.1) and +(-160:0) ..  (-1,1.2)
..controls +(60:.1) and +(140:0) ..  (-.18,1.29)
..controls +(10:.06) and +(140:0.27) .. (1.1,1.02)
..controls +(-40:.1) and +(150:0.2) .. (.55,1)
..controls +(-140:.3) and +(-45:0.2) .. (-.8,1)
..controls +(170:.2) and +(-60:0.1) .. cycle
;}
\draw \H;
\fill[tumbleweed] \H;
}}
\tikzset{nui/.pic={
\def\T{ %Trời
(-3.65,1.75)
..controls +(60:.05) and +(-160:0) ..  (-3.4,1.8)
..controls +(45:.1) and +(-150:0) ..  (-3.1,1.78)
..controls +(45:.1) and +(-150:0) ..  (-2.7,1.75)
..controls +(60:.1) and +(-160:0) ..  (-2.4,1.8)
..controls +(60:.1) and +(-160:0) ..  (-1.45,1.74)
..controls +(60:.1) and +(-160:0) ..  (-.4,1.64)
..controls +(30:.1) and +(160:0.1) ..  (.98,1.74)
..controls +(40:.1) and +(130:0.1) ..  (2.2,1.76)
..controls +(30:.1) and +(150:0.1) ..  (3,1.8)
..controls +(10:.1) and +(170:0.1) ..  (3.66,1.78)--(3.66,1.95)--(-3.65,1.95)--cycle
;}
\draw \T;
\fill[lightcornflowerblue] \T;
\def\N{ 
(-3.65,1.75)
..controls +(60:.05) and +(-160:0) ..  (-3.4,1.8)
..controls +(45:.1) and +(-150:0) ..  (-3.1,1.78)
..controls +(45:.1) and +(-150:0) ..  (-2.7,1.75)
..controls +(60:.1) and +(-160:0) ..  (-2.4,1.8)
..controls +(60:.1) and +(-160:0) ..  (-1.45,1.74)
..controls +(60:.1) and +(-160:0) ..  (-.4,1.64)
..controls +(30:.1) and +(160:0.1) ..  (.98,1.74)
..controls +(40:.1) and +(130:0.1) ..  (2.2,1.76)
..controls +(30:.1) and +(150:0.1) ..  (3,1.8)
..controls +(10:.1) and +(170:0.1) ..  (3.66,1.78)--(3.66,1.16)
..controls +(175:.2) and +(20:0) ..  (0,1.5)
..controls +(20:0) and +(20:.7) ..  (-3.65,1.28)--cycle
;}
\draw \N;
\fill[cadmiumgreen] \N;
}}

\tikzset{bien/.pic={
\def\B{ 
(3.66,1.16)
..controls +(175:.2) and +(20:0) ..  (0,1.5)
..controls +(20:0) and +(20:.7) ..  (-3.65,1.28)--(-3.65,-.65)
..controls +(15:.2) and +(120:0) .. (0,-.62)
..controls +(-60:0) and +(-160:0.1) .. (3.66,-.65)--(3.66,1.16)
;}
\draw \B;
\fill[trueblue!70!] \B;
\def\C{ %Cát
(-3.65,-.65)
..controls +(15:.2) and +(120:0) .. (0,-.62)
..controls +(-60:0) and +(-160:0.1) .. (3.66,-.65)--(3.66,-1.9)--(-3.65,-1.9)--cycle
;}
\draw \C;
\fill[tumbleweed] \C;
}}

\tikzset{nguoi/.pic={
\def\N{ 
(-3.04,-.64)--(-3.025,-.6)
..controls +(90:.1) and +(85:0.06) .. (-3.122,-.59)--(-3.11,-.63)
..controls +(-160:.07) and +(110:0.05) .. (-3.14,-.81)
..controls +(-40:.01) and +(60:0) .. (-3.14,-.85)
..controls +(-120:.07) and +(80:0.05) .. (-3.16,-1.1)
..controls +(-120:.09) and +(80:0.05) .. (-3.18,-1.28)
..controls +(-120:.02) and +(170:0.01) .. (-3.165,-1.318)--(-3.1,-1.31)--(-3.1,-1.29)
..controls +(120:.01) and +(-30:0) .. (-3.135,-1.278)
..controls +(120:.02) and +(-120:0.04) .. (-3.1,-1.1)
..controls +(-70:.04) and +(120:0.01) ..(-3.096,-1.14)
..controls +(-120:.04) and +(120:0.01) ..(-3.09,-1.3)
..controls +(-120:.02) and +(170:0.02) .. (-3.08,-1.34)--(-2.98,-1.34)
..controls +(70:.03) and +(-70:0.01) .. (-3.045,-1.296)
..controls +(110:.04) and +(-70:0.01) .. (-3.03,-1.1)
..controls +(70:.03) and +(-100:0.01) .. (-3.014,-1.04)--(-3,-1.02)
..controls +(70:.03) and +(-100:0.01) .. (-3.014,-.85)--(-2.97,-.84)
..controls +(40:.04) and +(-10:0) .. (-2.98,-.7)
..controls +(110:.04) and +(-20:0.02) .. (-3.043,-.66)--cycle
;}
\draw \N;
\fill[black!70!] \N;
\def\Q{ %quần
(-3.045,-.9)
..controls +(120:.01) and +(-30:0) .. (-3.04,-1.04)
-- (-3.1,-1.038)
..controls +(80:.07) and +(-10:0.01) .. (-3.13,-.94)--(-3.124,-.9)--cycle
(-3.14,-1.04)--(-3.12,-1.04)
..controls +(80:.07) and +(-10:0.01) .. (-3.134,-.96)--cycle
;}
%\draw \Q;
\fill[green] \Q;

}}

\path
(0,0)pic[scale=1]{bien} (0,0)pic[scale=1]{dao}
(0,0)pic[scale=1]{nui}
(0,0)pic[scale=1]{nguoi} (-1.1,0)pic[scale=1,rotate=180,yscale=-1]{nguoi}
;

\draw[dashed] (1.9,-.6)node[right]{\tiny $B$}--(-.1,1.1)node[above]{\tiny $C$}--(-3,-.54)node[left]{\tiny $A$} (-.1,1.1)--(-.1,-.6);
\draw (-2.6,-.46)--(-2.45,-.4);%gạch góc
\draw[<->,red,line width=1] (-3,-.6)--(1.9,-.6);
\node at (-.5,-1) [above]{\tiny $110$ m};
\node at (0,0) [above]{\tiny d};
\node at (-2.2,-.6) [above]{\tiny $32^\circ$};
\node at (1.3,-.6) [above]{\tiny $42^\circ$};
%góc
\draw (-2.6,-.3)
..controls +(-50:.09) and +(60:0.01) .. (-2.5,-.6);
\draw[scale=1,rotate=180,xscale=.6,yscale=-1] (-2.6,-.3)
..controls +(-50:.09) and +(60:0.01) .. (-2.5,-.6);
\end{tikzpicture}

	\end{center}
	\loigiai{
\begin{center}
\begin{tikzpicture}[line join=round, line cap=round,scale=1.5,transform shape]
	\draw (1.9,-.6) node[right]{\tiny $B$}--(-.1,1.1)node[above]{\tiny $C$}--(-3,-.6)node[left]{\tiny $A$}--cycle (-.1,1.1)--(-.1,-.6)node[below]{\tiny $H$};
\draw (-2.6,-.46)--(-2.45,-.4);%gạch góc
\node at (-.8,-1) [above]{\tiny $110$ m};
\node at (0,0) [above]{\tiny d};
\node at (-2.2,-.6) [above]{\tiny $32^\circ$};
\node at (1.3,-.6) [above]{\tiny $42^\circ$};
%góc
\draw (-2.6,-.37)
..controls +(-50:.09) and +(60:0.01) .. (-2.5,-.6);
\draw[scale=1,rotate=180,xscale=.6,yscale=-1] (-2.6,-.3)
..controls +(-50:.09) and +(60:0.01) .. (-2.5,-.6);
\end{tikzpicture}
\end{center}
Trong tam giác $ ABC $, ta có $ \widehat{ACB}=180^\circ -( \widehat{CAB}+ \widehat{CBA})=180^\circ -(32^\circ+42^\circ)=106^\circ$.\\
Theo định lý $ \sin $ ta có 
\[\dfrac{AC}{\sin B}=\dfrac{AB}{\sin C}\Leftrightarrow \dfrac{AC}{\sin 42^\circ}=\dfrac{110}{\sin 106^\circ}\]
suy ra $AC\approx 76{,}57$.\\
Diện tích tam giác $ ABC $ là 
\[S=\dfrac{1}{2}AB\cdot AC\cdot \sin \widehat{CAB} = \dfrac{1}{2}CH\cdot AB\]
suy ra $ CH=AC\cdot \sin \widehat{CAB} \approx 76{,}57\cdot \sin 32^\circ \approx 40{,}58$.\\
Vậy khoảng cách từ vị trí $C$ trên đảo tới bờ biển xấp xỉ $ 40{,}58 $ m.
	}
\end{bt}
\begin{bt}%[Dự án đề kiểm tra Toán 10 GHK1 NH 23-24]%[Lương Như Quỳnh]%[0H4V2-4]
	Cho tam giác $ ABC $ có $ S=2R^2\sin A\sin B $, với $ S $ là diện tích tam giác $ ABC $ và $ R $ là bán kính đường tròn ngoại tiếp tam giác $ ABC $. Chứng minh tam giác $ ABC $ vuông tại $ C $.
	\loigiai{
Ta có diện tích tam giác $ ABC $ là $ S=\dfrac{abc}{4R} $.\\
Theo định lý $ \sin $ ta có
\[\dfrac{a}{\sin A}=\dfrac{b}{\sin B}=\dfrac{c}{\sin C}=2R\]
suy ra $ a=2R\sin A $, $ b=2R\sin B $, $ c=2R\sin C $.\\
Do đó $ S=\dfrac{(2R\sin A)(2R\sin B)(2R\sin C)}{4R}=2R^2\sin A\sin B\sin C $.\\
Mặt khác, theo giả thiết $ S=2R^2\sin A\sin B $.\\
Suy ra $ \sin C=1 \Rightarrow C=90^\circ$.\\
Vậy tam giác $ ABC $ vuông tại $ C $.
	}
\end{bt}