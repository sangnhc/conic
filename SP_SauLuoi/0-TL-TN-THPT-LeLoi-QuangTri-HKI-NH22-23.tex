
\de{ĐỀ THI HỌC KỲ I NĂM HỌC 2022-2023}{Trường THPT Lê Lợi - Quảng Trị}
\begin{center}
	\textbf{PHẦN 1 - TRẮC NGHIỆM}
\end{center}
\Opensolutionfile{ans}[ans/ans]
%Câu 1...........................
\begin{ex}%[0D1B2-1]%[Dự án đề kiểm tra HKI NH22-23- Thy  Nguyen Vo Diem]%[Lê Lợi - Quảng Trị]
Cho tập hợp $A=\left\lbrace x \in \mathbb{N} \left| \dfrac{1}{2}<x<4 \right.\right\rbrace$. Mệnh đề nào sau đây đúng?
	\choice
	{\True $3 \in A$}
	{$\dfrac{1}{2} \in A$}
	{$4 \in A$}
	{$\dfrac{3}{2} \in A$}
	\loigiai{
	Ta có $A=\{1;2;3\}$. Do đó $3 \in A$.	
	}
\end{ex}

\begin{ex}%[0X1Y3-5]%[Dự án đề kiểm tra HKI NH22-23- Thy  Nguyen Vo Diem]%[Lê Lợi - Quảng Trị]
	Các giá trị xuất hiện nhiều nhất trong mẫu số liệu được gọi là
	\choice
	{Số trung bình}
	{Độ lệch chuẩn}
	{\True Mốt}
	{Số trung vị}
	\loigiai{
	Các giá trị xuất hiện nhiều nhất trong mẫu số liệu được gọi là mốt.
	}
\end{ex}

\begin{ex}%[0H2B1-5]%[Dự án đề kiểm tra HKI NH22-23- Thy  Nguyen Vo Diem]%[Lê Lợi - Quảng Trị]
	Cho tam giác đều $ABC$. Đẳng thức nào sau đây đúng?
	\choice
	{\True $\left| \overrightarrow{AC}\right|=\left| \overrightarrow{BC}\right|$}
	{$\overrightarrow{AB}=\overrightarrow{BC}$}
	{$\left| \overrightarrow{BC}\right|=\left| \overrightarrow{AB}+\overrightarrow{AC}\right|$}
	{$\overrightarrow{AB}=-\overrightarrow{AC}$}
	\loigiai{
	Do tam giác $ABC$ đều nên $AC=BC$. Vậy $\left| \overrightarrow{AC}\right|=\left| \overrightarrow{BC}\right|$.
	}
\end{ex}

\begin{ex}%[0H1Y1-1]%[Dự án đề kiểm tra HKI NH22-23- Thy  Nguyen Vo Diem]%[Lê Lợi - Quảng Trị]
	Với mỗi góc $\alpha$ ($0^{\circ} \le \alpha \le 180^{\circ}$). Gọi $M\left(x_0;y_0\right)$ là điểm trên nửa đường tròn đơn vị sao cho $\widehat{xOM}=\alpha$. Mệnh đề nào sau đây \textbf{sai}?
	\choice
	{\True tang của góc $\alpha$ là $\dfrac{x_0}{y_0}$ ($y_0 \ne 0$)}
	{cô-tang của góc $\alpha$ là $\dfrac{x_0}{y_0}$ ($y_0 \ne 0$)}
	{$\sin$ của góc $\alpha$ là tung độ $y_0$ của điểm $M$}
	{côsin của góc $\alpha$ là hoành độ $x_0$ của điểm $M$}
	\loigiai{
	tang của góc $\alpha$ là $\dfrac{y_0}{x_0}$ ($x_0 \ne 0$).
	}
\end{ex}

\begin{ex}%[0D2Y1-1]%[Dự án đề kiểm tra HKI NH22-23- Thy  Nguyen Vo Diem]%[Lê Lợi - Quảng Trị]
	Bất phương trình nào sau đây là bất phương trình bậc nhất hai ẩn?
	\choice
	{$2021x-2022y+2023z \ge 0$}
	{$2022x^2-2023y<0$}
	{$2022x-2023\sqrt{y} \le 1$}
	{\True $2022x-0y+2023 \ge 0$}
	\loigiai{
	$2022x-0y+2023 \ge 0$ có dạng $ax+by \ge c$ (với $a$, $b$ không đồng thời bằng $0$) nên là bất phương trình bậc nhất hai ẩn.	
	}
\end{ex}

\begin{ex}%[0H3Y1-3]%[Dự án đề kiểm tra HKI NH22-23- Thy  Nguyen Vo Diem]%[Lê Lợi - Quảng Trị]
	Trong mặt phẳng tọa độ $Oxy$, cho $M(x_1;y_1)$ và $N(x_2;y_2)$. Tọa độ trung điểm $I$ của đoạn thẳng $MN$ là 
	\choice
	{$I\left(\dfrac{x_1+y_1}{2};\dfrac{x_2+y_2}{2}\right)$}
	{\True $I\left(\dfrac{x_1+x_2}{2};\dfrac{y_1+y_2}{2}\right)$}
	{$I\left(\dfrac{x_1+x_2}{3};\dfrac{y_1+y_2}{3}\right)$}
	{$I\left(\dfrac{x_1-x_2}{2};\dfrac{y_1-y_2}{2}\right)$}
	\loigiai{
	Tọa độ trung điểm $I$ của đoạn thẳng $MN$ là $I\left(\dfrac{x_1+x_2}{2};\dfrac{y_1+y_2}{2}\right)$.	
	}
\end{ex}

\begin{ex}%[0D2B1-2]%[Dự án đề kiểm tra HKI NH22-23- Thy  Nguyen Vo Diem]%[Lê Lợi - Quảng Trị]
	Cặp số $(x;y)$ nào sau đây là nghiệm của hệ bất phương trình $\heva{&3x+2y \ge 1\\&4x-y-3<0}$?
	\choice
	{$(2;0)$}
	{$(1;1)$}
	{\True $(-1;3)$}
	{$(-3;0)$}
	\loigiai{
	Với $(-1;3)$, ta có $3 \cdot (-1)+2\cdot 3 \ge 1$ và $4\cdot (-1) -3-3 <0$.\\
	Vậy $(-1;3)$ là nghiệm của hệ bất phương trình đã cho.
	}
\end{ex}

\begin{ex}%[0H3B2-1]%[Dự án đề kiểm tra HKI NH22-23- Thy  Nguyen Vo Diem]%[Lê Lợi - Quảng Trị]
	Trong mặt phẳng tọa độ $Oxy$, cặp véc-tơ nào sau đây vuông góc với nhau?
	\choice
	{$\vec{a}=(2;0)$ và $\vec{b}=(-1;0)$}
	{$\vec{i}=(1;0)$ và $\vec{e}=(2;1)$}
	{\True $\vec{c}=(2;-5)$ và $\vec{d}=(10;4)$}
	{$\vec{u}=(3;2)$ và $\vec{v}=(2;3)$}
	\loigiai{
	Ta có $\vec{c} \cdot \vec{d}=2 \cdot 10+(-5)\cdot 4=0$ nên $\vec{c}$ và $\vec{d}$ vuông góc với nhau.
	}
\end{ex}

\begin{ex}%[0X1B4-1]%[Dự án đề kiểm tra HKI NH22-23- Thy  Nguyen Vo Diem]%[Lê Lợi - Quảng Trị]
	Điều tra số km chạy bộ của 10 học sinh trong một tháng ta có các số liệu bên dưới. Hãy tìm khoảng biến thiên của mẫu số liệu.
	\[22 \quad 24 \quad 33 \quad 17 \quad 11 \quad 4 \quad 18 \quad 87 \quad 72 \quad 30\]
	\choice
	{$89$}
	{\True $83$}
	{$33$}
	{$82$}
	\loigiai{
	Giá  trị lớn nhất và giá trị nhỏ nhất của mẫu số liệu lần lượt là $87$ và $4$.\\
	Khoảng biến thiên của mẫu số liệu là $R=87-4=83$.
	}
\end{ex}

\begin{ex}%[0X1Y1-1]%[Dự án đề kiểm tra HKI NH22-23- Thy  Nguyen Vo Diem]%[Lê Lợi - Quảng Trị]
	Đại lượng nào sau đây phản ánh mức độ sai lệch giữa số đúng và số gần đúng?
	\choice
	{Số đúng}
	{Số gần đúng}
	{\True Sai số tuyệt đối}
	{Sai số tương đối}
	\loigiai{
	Sai số tuyệt đối phản ánh mức độ sai lệch giữa số đúng và số gần đúng.
	}
\end{ex}

\begin{ex}%[0H1Y2-2]%[Dự án đề kiểm tra HKI NH22-23- Thy  Nguyen Vo Diem]%[Lê Lợi - Quảng Trị]
	Cho tam giác $ABC$ với $BC=a$, $AC=b$, $AB=c$ và $p=\dfrac{a+b+c}{2}$. Diện tích $S$ của $\triangle ABC$ được tính bằng công thức nào?
	\choice
	{\True $S=\sqrt{p(p-a)(p-b)(p-c)}$}
	{$S=p(p-a)(p-b)(p-c)$}
	{$S=\sqrt{(p-a)(p-b)(p-c)}$}
	{$S=\dfrac{1}{2}\sqrt{p(p-a)(p-b)(p-c)}$}
	\loigiai{
	Diện tích $S$ của $\triangle ABC$ là  $S=\sqrt{p(p-a)(p-b)(p-c)}$, trong đó $p=\dfrac{a+b+c}{2}$.	
	}
\end{ex}

\begin{ex}%[0H3B1-3]%[Dự án đề kiểm tra HKI NH22-23- Thy  Nguyen Vo Diem]%[Lê Lợi - Quảng Trị]
	Cho $\vec{a}=\left(\dfrac{1}{3};-\dfrac{2}{3}\right)$, $\vec{b}=\left(-\dfrac{5}{2};\dfrac{7}{2}\right)$. Tọa độ của véc-tơ $3\vec{a}+2\vec{b}$ là
	\choice
	{$(4;-5)$}
	{$(-6;9)$}
	{$(6;-9)$}
	{\True $(-4;5)$}
	\loigiai{
	$3\vec{a}+2\vec{b}=3\left(\dfrac{1}{3};-\dfrac{2}{3}\right)+2\left(-\dfrac{5}{2};\dfrac{7}{2}\right)=(-4;5)$.	
	}
\end{ex}

\begin{ex}%[0H2B2-3]%[Dự án đề kiểm tra HKI NH22-23- Thy  Nguyen Vo Diem]%[Lê Lợi - Quảng Trị]
	Cho $I$ là trung điểm của đoạn thẳng $AB$, $M$ là một điểm tùy ý. Khẳng định nào sau đây là khẳng định \textbf{sai}?
	\choice
	{\True $\overrightarrow{MA}+\overrightarrow{MB}=\overrightarrow{AB}$}
	{$\overrightarrow{AI}=\overrightarrow{IB}$}
	{ $\overrightarrow{IA}+\overrightarrow{IB}=\vec{0}$}
	{$\overrightarrow{MA}+\overrightarrow{MB}=2\overrightarrow{MI}$}
	\loigiai{
	$I$ là trung điểm của đoạn thẳng $AB$ thì 	$\overrightarrow{MA}+\overrightarrow{MB}=2\overrightarrow{MI}$.\\
	Vậy $\overrightarrow{MA}+\overrightarrow{MB}=\overrightarrow{AB}$ là sai.
	}
\end{ex}

\begin{ex}%[0X1B1-3]%[Dự án đề kiểm tra HKI NH22-23- Thy  Nguyen Vo Diem]%[Lê Lợi - Quảng Trị]
	Viết số quy tròn của số $345678910$ đến hàng nghìn.
	\choice
	{$345678000$}
	{\True $345679000$}
	{$345678$}
	{$345679$}
	\loigiai{
		Số quy tròn của số $345678910$ đến hàng nghìn là $345679000$.
	}
\end{ex}

\begin{ex}%[0H2B2-2]%[Dự án đề kiểm tra HKI NH22-23- Thy  Nguyen Vo Diem]%[Lê Lợi - Quảng Trị]
	Xác định véc-tơ $\vec{u}=\overrightarrow{AB}+\overrightarrow{DE}-\overrightarrow{AC}+\overrightarrow{BD}$.
	\choice
	{\True $\vec{u}=\overrightarrow{CE}$}
	{$\vec{u}=\overrightarrow{EC}$}
	{$\vec{u}=\overrightarrow{AD}$}
	{$\vec{u}=\overrightarrow{AE}$}
	\loigiai{
		\allowdisplaybreaks{\begin{eqnarray*}
			\vec{u}	&=& \overrightarrow{AB}+\overrightarrow{DE}-\overrightarrow{AC}+\overrightarrow{BD} \\
				&=& \overrightarrow{AB}+\overrightarrow{DE}+\overrightarrow{CA}+\overrightarrow{BD} \\
				&=&  \left(\overrightarrow{CA}+\overrightarrow{AB}\right)+\left(\overrightarrow{BD}+\overrightarrow{DE}\right)\\
				&=& \overrightarrow{CB}+\overrightarrow{BE} \\
				&=& \overrightarrow{CE}.
		\end{eqnarray*}}
	}
\end{ex}

\begin{ex}%[0H2B3-1]%[Dự án đề kiểm tra HKI NH22-23- Thy  Nguyen Vo Diem]%[Lê Lợi - Quảng Trị]
	Cho $\vec{a}=-5\vec{b}$. Khẳng định nào sau đây là \textbf{sai}?
	\choice
	{Hai vec-tơ $\vec{a}$, $\vec{b}$ cùng phương}
	{$\left|\vec{a} \right|=-5\left| \vec{b} \right|$}
	{Hai vec-tơ $\vec{a}$, $\vec{b}$ ngược hướng}
	{\True $\left|\vec{a} \right|=5\left| \vec{b} \right|$}
	\loigiai{
	Ta có $\left|\vec{a} \right|=5\left| \vec{b} \right|$ nên $\left|\vec{a} \right|=-5\left| \vec{b} \right|$ là sai.	
	}
\end{ex}

\begin{ex}%[0H2B3-1]%[Dự án đề kiểm tra HKI NH22-23- Thy  Nguyen Vo Diem]%[Lê Lợi - Quảng Trị]
	Trên đường thẳng $AB$ lấy điểm $M$ sao cho $\overrightarrow{MA}=-\dfrac{1}{3}\overrightarrow{MB}$. Hình vẽ nào sau đây xác định đúng vị trí điểm $M$?
	\choice
	{\begin{tikzpicture}[>=stealth,scale=1, line join = round, line cap = round]
	\path (0,0) coordinate (M)
	(4,0) coordinate (A)
	($(M)!1/3!(A)$) coordinate (B)
	;
	\draw (M)--(A);
\foreach \x/\goc in {A/90,B/90,M/90}\fill[black]
(\x) circle (1pt)
($(\x)+(\goc:3mm)$)node{$\x$};
	\draw[fill] ($(M)!2/3!(A)$) circle(1pt);
	\end{tikzpicture}}
	{\begin{tikzpicture}[>=stealth,scale=1, line join = round, line cap = round]
			\path (0,0) coordinate (A)
			(4,0) coordinate (B)
			($(A)!3/4!(B)$) coordinate (M)
			;
			\draw (B)--(A);
			\foreach \x/\goc in {A/90,B/90,M/90}\fill[black]
			(\x) circle (1pt)
			($(\x)+(\goc:3mm)$)node{$\x$};
			\draw[fill] ($(A)!2/4!(B)$) circle(1pt) ($(A)!1/4!(B)$) circle(1pt);
	\end{tikzpicture}}
	{\True \begin{tikzpicture}[>=stealth,scale=1, line join = round, line cap = round]
			\path (0,0) coordinate (A)
			(4,0) coordinate (B)
			($(A)!1/4!(B)$) coordinate (M)
			;
			\draw (B)--(A);
			\foreach \x/\goc in {A/90,B/90,M/90}\fill[black]
			(\x) circle (1pt)
			($(\x)+(\goc:3mm)$)node{$\x$};
			\draw[fill] ($(A)!2/4!(B)$) circle(1pt) ($(A)!3/4!(B)$) circle(1pt);
	\end{tikzpicture}}
	{\begin{tikzpicture}[>=stealth,scale=1, line join = round, line cap = round]
			\path (0,0) coordinate (M)
			(4,0) coordinate (B)
			($(M)!1/3!(B)$) coordinate (A)
			;
			\draw (M)--(B);
			\foreach \x/\goc in {A/90,B/90,M/90}\fill[black]
			(\x) circle (1pt)
			($(\x)+(\goc:3mm)$)node{$\x$};
			\draw[fill] ($(M)!2/3!(B)$) circle(1pt);
	\end{tikzpicture}}
	\loigiai{$\overrightarrow{MA}=-\dfrac{1}{3}\overrightarrow{MB}$ nên $MA=\dfrac{1}{3}MB$ và hai véc-tơ $\overrightarrow{MA}$ và $\overrightarrow{MB}$ ngược hướng.
	Vậy hình vẽ xác định đúng vị trí điểm $M$ là
		\begin{center}
			\begin{tikzpicture}[>=stealth,scale=1, line join = round, line cap = round]
				\path (0,0) coordinate (A)
				(4,0) coordinate (B)
				($(A)!1/4!(B)$) coordinate (M)
				;
				\draw (B)--(A);
				\foreach \x/\goc in {A/90,B/90,M/90}\fill[black]
				(\x) circle (1pt)
				($(\x)+(\goc:3mm)$)node{$\x$};
				\draw[fill] ($(A)!2/4!(B)$) circle(1pt) ($(A)!3/4!(B)$) circle(1pt);
			\end{tikzpicture}
		\end{center}
	}
\end{ex}

\begin{ex}%[0X1K3-2]%[Dự án đề kiểm tra HKI NH22-23- Thy  Nguyen Vo Diem]%[Lê Lợi - Quảng Trị]
	Cho bảng phân bố tần số về điểm kiểm tra giữa kì môn Toán của $20$ học sinh.
\begin{center}
		\begin{tabular}{|c|c|c|c|c|c|c|c|}
		\hline
		Điểm & 3 & 4 & 5 & 6 & 7 & 8 & 9 \\
		\hline
		Tần số & 2 & 3 & 5 & 3 & 4 & 2 & 1 \\
		\hline
	\end{tabular}
\end{center}
Số trung vị của mẫu số liệu trên là
	\choice
	{$6$}
	{\True $5{,}5$}
	{$5$}
	{$5{,}7$}
	\loigiai{
	Trung vị của mẫu số liệu $M_e=\dfrac{x_{10}+x_{11}}{2}=\dfrac{5+6}{2}=5{,}5$.
	}
\end{ex}

\begin{ex}%[0H1B2-1]%[Dự án đề kiểm tra HKI NH22-23- Thy  Nguyen Vo Diem]%[Lê Lợi - Quảng Trị]
	Cho tam giác $MNP$ có $MN=6$, $MP=10$, $\widehat{M}=120^{\circ}$. Tính $NP$.
	\choice
	{$196$}
	{\True $14$}
	{$8$}
	{$16$}
	\loigiai{
	Áp dụng định lý hàm số cô-sin, ta có \allowdisplaybreaks{\begin{eqnarray*}
		NP^2	&=& MN^2+MP^2-2MN \cdot MP \cdot \cos \widehat{M} \\
			&=&  6^2+10^2-2\cdot 6 \cdot 10 \cdot \cos 120^{\circ}\\
			&=& 196 \\
			\Rightarrow NP &=& 14.
	\end{eqnarray*}}	
	}
\end{ex}

\begin{ex}%[0X1K4-1]%[Dự án đề kiểm tra HKI NH22-23- Thy  Nguyen Vo Diem]%[Lê Lợi - Quảng Trị]
	Số lượng ly trà sữa một quán nước bán được trong $20$ ngày qua là 
	\[ 4 \quad 16 \quad 5 \quad 6 \quad 8 \quad 33 \quad 9 \quad 11 \quad 25 \quad 13 \quad 16 \quad 40 \quad 18 \quad 20 \quad 21 \quad 30 \quad 31 \quad 36 \quad 37 \quad 41\]
	Khoảng tứ phân vị của mẫu số liệu trên là
	\choice
	{$26$}
	{$20$}
	{$24$}
	{\True $22$}
	\loigiai{
	Sắp xếp mẫu số liệu theo thứ tự tăng dần 
	\[ 4  \quad 5 \quad 6 \quad 8  \quad 9 \quad 11  \quad 13 \quad 16 \quad 16 \quad 18 \quad 20 \quad 21 \quad 25 \quad 30 \quad 31 \quad  33  \quad 36 \quad 37 \quad 40 \quad 41\]
	Tứ phân vị thứ hai  $Q_2=\dfrac{x_{10}+x_{11}}{2}=\dfrac{18+20}{2}=19$.\\
	Tứ phân vị thứ nhất $Q_1=\dfrac{x_5+x_6}{2}=\dfrac{9+11}{2}=10$.\\
	Tứ phân vị thứ ba  $Q_3=\dfrac{x_{15}+x_{16}}{2}=\dfrac{31+33}{2}=32$.\\
	Khoảng tứ phân vị $Q_3-Q_1=32-10=22$.
	}
\end{ex}



\Closesolutionfile{ans}
%\begin{center}
%	\textbf{ĐÁP ÁN}
%	\inputansbox{10}{ans/ans}	
%\end{center}
\begin{center}
	\textbf{PHẦN 2 - TỰ LUẬN}
\end{center}


\begin{bt}%[0D1B3-4]%[Dự án đề kiểm tra HKII NH22-23 - Phạm Phương]%[THPT Lê Lợi - Quảng Trị]
	\begin{enumerate}
		\item Cho tập hợp $A=\left\{x \in \mathbb{Q} \mid\left(x^2-5\right)\left(4x^2+5x-6\right)=0\right\}$. Liệt kê các phần tử của tập hợp $A$.
		\item Cho hai tập $B=(-\infty;-2)$, $C=\{x \in \mathbb{R} \mid-2 \leq x<5\}$. Xác định các tập hợp $B \cup C$, $B \cap C$.	
	\end{enumerate}
	\loigiai{
		\begin{enumerate}
			\item Ta có
			$$\left(x^2-5\right)\left(4x^2+5x-6\right)=0 \Leftrightarrow \hoac{&x^2-5=0 \\& 4x^2+5x-6=0} \Leftrightarrow \hoac{& x=\pm \sqrt{5} \\& x=\dfrac{3}{4}\\& x=-2.}$$
			Vì $x \in \mathbb{Q}$ nên $A=\left\{-2;\dfrac{3}{4}\right\}$.
			\item Ta có $C=\left[-2;5\right)$. Khi đó
			\begin{itemize}
				\item $B \cup C=(-\infty;-2) \cup \left[-2;5\right)=(-\infty;5)$.
				\item $B \cap C=(-\infty;-2) \cap \left[-2;5\right)=\varnothing$.
			\end{itemize}			
		\end{enumerate}
}
\end{bt}
\begin{bt}%[0H3B1-4]%[Dự án đề kiểm tra HKII NH22-23- Phạm Phương]%[THPT Lê Lợi - Quảng Trị]
	Trong mặt phẳng tọa độ $Oxy$, cho các điểm $A(6;-3)$ và $B(3;2)$.
	\begin{enumerate}
		\item Tìm tọa độ điểm $D$ sao cho $B$ là trung điểm của $AD$.
		\item Tìm tọa độ điểm $C$ để tứ giác $OABC$ là hình bình hành (với $O$ là gốc tọa độ).
	\end{enumerate}
	\loigiai{
	\begin{enumerate}
		\item $B$ là trung điểm của $AD$ nên ta có
		$$\heva{& x_B=\dfrac{x_A+x_D}{2} \\& y_B=\dfrac{y_A+y_D}{2}} \Leftrightarrow \heva{& 3=\dfrac{6+x_D}{2} \\& 2=\dfrac{-3+y_D}{2}} \Leftrightarrow \heva{& x_D=0 \\& y_D=7.}$$
		Vậy $D(0;7)$.
		\immini{
		\item Ta có $\vec{OC}=\left(x_C;y_C\right)$, $\vec{AB}=\left(-3;5\right)$. Để tứ giác $OABC$ là hình bình hành thì $\vec{OC}=\vec{AB}$. Khi đó
		$$\heva{& x_C=-3 \\& y_C=5} \Rightarrow C\left(-3;5\right).$$
	}{
	\begin{tikzpicture}[scale=1,>=stealth, font=\footnotesize,line join=round,line cap=round]
		\foreach \i/\j/\k in{0/0/O,4/0/C,1/2/A}
		\coordinate (\k) at(\i,\j);
		\coordinate (B) at($(A)+(C)-(O)$);
		\draw (O)--(A)--(B)--(C)--cycle;
		\draw [->,red,thick] (O)--(C)(A)--(B);
		\foreach \p/\r in {O/-135,A/135,B/45,C/-45}
		\fill (\p) circle (1pt) node[shift={(\r:3mm)}]{$\p$};
	\end{tikzpicture}
	}
	\end{enumerate}
}
\end{bt}
\begin{bt}%[0H2K3-5]%[Dự án đề kiểm tra HKII NH22-23- Phạm Phương]%[THPT Lê Lợi - Quảng Trị]
	Cho tam giác $MNP$, gọi $A$ là điểm thuộc cạnh $NP$ sao cho $2AN=3AP$. Chứng minh rằng: $\overrightarrow{MA}=\dfrac{2}{5} \overrightarrow{MN}+\dfrac{3}{5} \overrightarrow{MP}$.
	\loigiai{
	\immini{
	Ta có 
	\allowdisplaybreaks{
		\begin{eqnarray*}
			\vec{MA} &= & \vec{MN}+\vec{NA}\\
			&= & \vec{MN}+\dfrac{3}{5}\vec{NP} \\
			&= & \vec{MN}+\dfrac{3}{5}\left(\vec{MP}-\vec{MN}\right) \\
			&= & \dfrac{2}{5}\vec{MN}+\dfrac{3}{5}\vec{MP}.
	\end{eqnarray*} }
	}{
	\begin{tikzpicture}[scale=1,>=stealth, font=\footnotesize,line join=round,line cap=round]
		\foreach \i/\j/\k in{0/0/N,4/0/P,1/2.7/M}
		\coordinate (\k) at (\i,\j);
		\coordinate (A) at ($(N)!{3/5}!(P)$);
		\draw (M)--(N)--(P)--cycle
		(M)--(A);
		\foreach \p/\r in {N/-135,P/45,M/90,A/-90}
		\fill (\p) circle (1pt) node[shift={(\r:3mm)}]{$\p$};
		\foreach \x in {1,2,4}
		\draw ($(N)!{\x/5}!(P)+(0,-0.07)$)--($(N)!{\x/5}!(P)+(0,0.07)$);
	\end{tikzpicture}
	}
}
\end{bt}
\begin{bt}%[0X1B3-4]%[Dự án đề kiểm tra HKII NH22-23- Phạm Phương]%[THPT Lê Lợi - Quảng Trị]
	Thu nhập theo tháng (đơn vị: triệu đồng) của các công nhân trong một công ty nhỏ được cho như bảng sau:
	\begin{center}
		\begin{tabular}{|c|c|c|c|c|c|c|c|}
		\hline $5{,}5$ & $6$ & $8$ & $7$ & $7$ & $8{,}5$ & $7$ & $9{,}5$ \\
		\hline $12$ & $10$ & $4{,}5$ & $11$ & $13$ & $9{,}5$ & $8{,}5$ & $4$ \\
		\hline
	\end{tabular}
	\end{center}
	\begin{enumerate}
		\item Tính số trung bình, số trung vị và mốt của mẫu số liệu trên.
		\item Trong đại dịch Covid-19 công ty có chính sách hỗ trợ cho $25\%$ công nhân có thu nhập thấp nhất. Số nào trong các tứ phân vị giúp xác định các công nhân trong diện hỗ trợ? Tính giá trị tứ phân vị đó. (Các kết quả làm tròn đến hàng phần trăm)
	\end{enumerate}
	\loigiai{
	\begin{enumerate}
		\item Sắp xếp mẫu số liệu theo thứ tự không giảm ta được
		$$4;\ 4{,}5;\ 5{,}5;\ 6;\ 7;\ 7;\ 7;\ 8;\ 8{,}5;\ 8{,}5;\ 9{,}5;\ 9{,}5;\ 10;\ 11;\ 12;\ 13.$$
		Ta có
		\begin{itemize}
			\item $\bar{x}=\dfrac{4+4{,}5+5{,}5+6+7\cdot 3+8+8{,}5\cdot 2+9{,}5\cdot 2+10+11+12+13}{16}\approx 8{,}19$.
			\item $M_e=\dfrac{8+8{,}5}{2}=8{,}25$.
			\item Mốt $M_o=7$.
		\end{itemize}
		\item Vì hỗ trợ cho $25\%$ công nhân có thu nhập thấp nhất nên cần dựa vào tứ phân vị thứ nhất $Q_1$.\\
		Ta có $Q_1=\dfrac{6+7}{2}=6{,}5$.
	\end{enumerate}
}
\end{bt}
\begin{bt}%[0H3G2-5]%[Dự án đề kiểm tra HKII NH22-23- Phạm Phương]%[THPT Lê Lợi - Quảng Trị]
	Trong mặt phẳng tọa độ $Oxy$, cho các điểm $A(1;1)$, $B(3;5)$. Tìm tọa độ các điểm $M$, $N$ sao cho $AMBN$ là hình vuông.
	\loigiai{
	
	Gọi $M(x;y)$ là đỉnh của hình vuông $AMBN$ có đường chéo $AB$ ta có
	\immini{
	\allowdisplaybreaks{
		\begin{eqnarray*}
			& &\heva{& AM=BM \\& \vec{AM} \perp \vec{BM} } \Leftrightarrow \heva{& AM^2=BM^2 \\& \vec{AM}\cdot \vec{BM}=0} \\
			&\Leftrightarrow & \heva{& (x-1)^2+(y-1)^2=(x-3)^2+(y-5)^2 \\& (x-1)(x-3)+(y-1)(y-5)=0 } \\
			&\Leftrightarrow & \heva{& x+2y=8 \\& x^2+y^2-4x-6y+8=0} \Leftrightarrow \hoac{& \heva{& x=0 \\& y=4}\\& \heva{& x=4 \\& y=2.}}
	\end{eqnarray*} }
}{
	\begin{tikzpicture}[scale=1,>=stealth, font=\footnotesize,line join=round,line cap=round]
		\foreach \i/\j/\k in{0/0/A,3/0/M,3/3/B}
		\coordinate (\k) at(\i,\j);
		\coordinate (N) at($(A)+(B)-(M)$);
		\draw (A)--(M)--(B)--(N)--cycle;
		\foreach \p/\r in {A/-135,M/-45,B/45,N/135}
		\fill (\p) circle (1pt) node[shift={(\r:3mm)}]{$\p$};
	\end{tikzpicture}
}
	Vậy $M(0;4)$, $N(4;2)$ hoặc $M(4;2)$, $N(0;4)$.
}
\end{bt}