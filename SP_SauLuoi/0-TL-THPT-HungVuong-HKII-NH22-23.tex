\de{ĐỀ THI HỌC KỲ II NĂM HỌC 2022-2023}{THPT Hùng Vương}



\begin{bt}%[0D7Y3-2]%[Dự án đề kiểm tra HKII NH22-23- Nguyễn Ngọc Dũng]%[THPT Hùng Vương]
Giải phưong trình $\sqrt{-x^2+4x-2}=x-2$.
\loigiai{
Ta có 
\allowdisplaybreaks 
\begin{eqnarray*}
\sqrt{-x^2+4x-2}=x-2 &\Rightarrow& -x^2+4x-2=x^2-4x+4 \\
&\Rightarrow& -2x^2+8x-6=0 \\
&\Rightarrow& x=1 \text{ hoặc } x=3
\end{eqnarray*}
Thay $x=1$ và $x=3$ vào phương trình đã cho, ta thấy chỉ có $x=3$ thỏa mãn phương trình.\\
Vậy $S=\{3\}$.
}
\end{bt}

\begin{bt}%[0D7B2-1]%[Dự án đề kiểm tra HKII NH22-23- Nguyễn Ngọc Dũng]%[THPT Hùng Vương]
Định giá trị của tham số $m$ để bất phương trình $(m-2)x^2+2mx-m<0$ nghiệm đúng với mọi $x$ thuộc $\mathrm{R}$.
\loigiai{
\begin{enumerate}[\bf TH1.]
\item $m-2=0\Leftrightarrow m=2$. Khi đó phương trình đã cho trở thành
$$4x-2<0 \Leftrightarrow x< \dfrac{1}{2}.$$
Suy ra $m=2$ không thỏa yêu cầu bài toán.
\item $m-2\neq 0 \Leftrightarrow m\neq 2$.\\
Yêu cầu bài toán $\Leftrightarrow \heva{&m-2<0 \\ &\Delta ' <0} \Leftrightarrow \heva{&m<2\\ &2m^2 -2m <0} \Leftrightarrow \heva{&m<2\\ & 0<m<1} \Leftrightarrow 0<m<1$.
\end{enumerate}
Vậy $m\in (0;1)$.
}
\end{bt}

\begin{bt}%[0D8B1-2]%[Dự án đề kiểm tra HKII NH22-23- Nguyễn Ngọc Dũng]%[THPT Hùng Vương]
Từ các chữ số $0$; $1$; $2$; $3$; $4$; $5$ có thể lập được bao nhiêu số tự nhiên lẻ có bốn chữ số khác nhau?
\loigiai{
Gọi số cần tìm là $\overline{abcd}$.
\begin{itemize}
\item Chọn $d\in \{1;3;5\}$: có $3$ cách.
\item Chọn $a\neq 0, a\neq d$: có $4$ cách.
\item Chọn $b\neq a, b\neq d$: có $4$ cách.
\item Chọn $c\neq a, c\neq b, c\neq d$: có $3$ cách.
\end{itemize}
Vậy có tất cả $3\cdot 4\cdot 4\cdot 3 = 144$ (số).
}
\end{bt}

\begin{bt}%[0D8Y3-1]%[Dự án đề kiểm tra HKII NH22-23- Nguyễn Ngọc Dũng]%[THPT Hùng Vương]
Sử dụng công thức nhị thức Newton, hãy khai triển biểu thức $(3x+1)^5$.
\loigiai{
Ta có
\allowdisplaybreaks 
\begin{eqnarray*}
&& (3x+1)^5 \\
&=& 1\cdot (3x)^5 1^0 + 5\cdot (3x)^4 1^1 + 10\cdot (3x)^3 1^2 + 10\cdot (3x)^2 1^3 + 5\cdot (3x)^1 1^4 + 1\cdot (3x)^0 1^5\\
&=& 243x^5 + 405x^4 + 270x^3 + 90x^2 + 15x + 1.
\end{eqnarray*}
}
\end{bt}

\begin{bt}%[0D0Y2-2]%[Dự án đề kiểm tra HKII NH22-23- Nguyễn Ngọc Dũng]%[THPT Hùng Vương]
Một tổ học sinh có $4$ nam, $5$ nữ. Chọn ngẫu nhiên $3$ học sinh. Tính xác suất để chọn có $2$ nam, $1$ nữ.
\loigiai{
Ta có $n(\Omega) = \mathrm{C}^3_9$.\\
Gọi $A$ là biến cố: "có $2$ nam và $1$ nữ". Khi đó $n(A)= \mathrm{C}^2_4\cdot \mathrm{C}^1_5$.\\
Vậy $\mathrm{P}(A) = \dfrac{n(A)}{n(\Omega)} = \dfrac{\mathrm{C}^2_4\cdot \mathrm{C}^1_5}{\mathrm{C}^3_9} = \dfrac{5}{14}$.
}
\end{bt}

%%==========Bài 5
\begin{bt}%[0T0B2-2]%[Dự án đề kiểm tra HKII NH22-23- Nguyễn Sĩ Đạt]%[THPT Hùng Vương]
	Một tổ học sinh có $4$ nam, $5$ nữ. Chọn ngẫu nhiên $3$ học sinh. Tính xác suất để trong $3$ học sinh được chọn có $2$ nam, $1$ nữ.
	\loigiai
	{
		Ta có $n(\Omega)=\mathrm{C}^3_9=84$.\\
		Gọi $A$ là biến cố \lq\lq Trong $3$ học sinh được chọn có $2$ nam và $1$ nữ\rq\rq. \\
		Ta có $n(A)=\mathrm{C}^2_4\cdot \mathrm{C}^1_5=30$.\\
		Vậy $\mathrm{P}(A)=\dfrac{n(A)}{n(\Omega)}=\dfrac{5}{14}$.
	}
\end{bt}
%%==========Bài 6
\begin{bt}%[0T9B2-2]%[Dự án đề kiểm tra HKII NH22-23- Nguyễn Sĩ Đạt]%[THPT Hùng Vương]
	Viết phương trình tổng quát của đường thẳng $d$ đi qua điểm $M(-1;4)$ và vuông góc với đường thẳng $D\colon 2x-y+5=0$.
	\loigiai{
		Ta có $d$ vuông góc $D\colon2x-y+5=0$ $\Rightarrow d\colon x+2y+c=0$.\\
		Lại có $d$ đi qua $M(-1;4) \Leftrightarrow -1+2\cdot4+c=0 \Leftrightarrow c=-7$.\\
		Vậy $d\colon x+2y-7=0$.
	}
\end{bt}
%%==========Bài 7
\begin{bt}%[0T9B3-2]%[Dự án đề kiểm tra HKII NH22-23- Nguyễn Sĩ Đạt]%[THPT Hùng Vương]
	Viết phương trình đường tròn $(C)$ có tâm $I(2;3)$ và tiếp xúc với đường thẳng $\Delta \colon 3x-4y+9=0$. \\
	\loigiai{
		Bán kính $R = \mathrm{d}(I, \Delta) = \dfrac{|3\cdot 2 - 4\cdot 3 + 9|}{\sqrt{3^2+(-4)^2}}=\dfrac{3}{5}$. \\
		Phương trình đường tròn $(C)$ có dạng $(x-2)^2+(y-3)^2=\dfrac{9}{25}$.
	}
\end{bt}

%%==========Bài 8
\begin{bt}%[0T9B4-2]%[Dự án đề kiểm tra HKII NH22-23- Nguyễn Sĩ Đạt]%[THPT Hùng Vương]
	Viết phương trình chính tắc của $(E)$ đi qua điểm $A(5;0)$ và có một tiêu điểm $F_2(3;0)$.
	\loigiai{
		Gọi phương trình chính tắc của $(E) \colon \dfrac{x^2}{a^2}+\dfrac{y^2}{b^2}=1$ $(a>b>0)$.\\
		Vì $(E)$ đi qua điểm $A(5;0)$ nên  $\dfrac{5^2}{a^2}=1 \Leftrightarrow a^2=25 \Leftrightarrow a=5$.\\
		Vì  $(E)$ có một tiêu điểm $F_2(3;0)$ nên $a^2-b^2=c^2 \Leftrightarrow 5^2-b^2=3^2 \Leftrightarrow b^2=16 \Leftrightarrow b=4$.\\
		Vậy phương trình chính tắc của $(E) \colon \dfrac{x^2}{25}+\dfrac{y^2}{16}=1$.
	}
\end{bt}

%%==========Bài 9
\begin{bt}%[0T7T2-1]%[Dự án đề kiểm tra HKII NH22-23- Nguyễn Sĩ Đạt]%[THPT Hùng Vương]
	Để xây dựng phương án kinh doanh cho một loại sản phẩm, doanh nghiệp tính toán lợi nhuận $I$ (đồng) theo công thức sau: $I(x)=-200x^2+90000x-8204200$, trong đó $x$ là số sản phẩm được bán ra. Hỏi với số lượng sản phẩm bán ra như thế nào thì doanh nghiệp không bị lỗ?
	\loigiai{
		Doanh nghiệp không bị lỗ thì $I(x) \geq 0 \Leftrightarrow -200x^2+90000x-8204200\ge 0 \Leftrightarrow 127 \leq x \leq 323$.\\
		Vậy để không bị lỗ thì doanh nghiệp này phải bán trong khoảng từ 127 đến 323 sản phẩm.
	}
\end{bt}
%%==========Bài 10
\begin{bt}%[0T3T1-1] %[Dự án đề kiểm tra HKII NH22-23- Nguyễn Sĩ Đạt]%[THPT Hùng Vương]
	Theo Google Maps, sân bay Tân Sơn Nhất tại thành phố Hồ Chí Minh có vĩ độ $10{,}8^\circ$ Bắc, kinh độ $106{,}7^\circ$ Đông, sân bay Incheon tại Hàn Quốc có vĩ độ $37{,}5^\circ$ Bắc, kinh độ $126{,}4^\circ$ Đông. Một máy bay, bay thẳng từ sân bay Tây Sơn Nhất đến sân bay Incheon. Tại thời điểm $t$ (giờ), tính từ lúc xuất phát, máy bay ở vị trí có vĩ độ $x^\circ$ Bắc, kinh độ $y^\circ$ Đông được tính theo công thức
	\begin{center}
		$\heva{&x=10{,}8+5{,}34t \\ &y=106{,}7+3{,}94t.}$
	\end{center}
	Hỏi chuyến bay từ sân bay Tân Sơn Nhất đến sân bay Incheon mất mấy giờ?
	\loigiai{
		Vì sân bay Tân Sơn Nhất tại thành phố Hồ Chí Minh có vĩ độ $10{,}8^\circ$ Bắc, kinh độ $106{,}7^\circ$ Đông, sân bay Incheon tại Hàn Quốc có vĩ độ $37{,}5^\circ$ Bắc, kinh độ $126{,}4^\circ$ Đông nên ta có:
		\begin{center}
			$\heva{&x=10{,}8+5{,}34t \\ &y=106{,}7+3{,}94t}
			\Rightarrow
			\heva{&37{,}5=10{,}8+5{,}34t \\& 126{,}4=106{,}7+3{,}94t} \Rightarrow t=5$.
		\end{center}
		Vậy chuyến bay từ sân bay Tân Sơn Nhất đến sân bay Incheon mất $5$ giờ.
	}
\end{bt}

