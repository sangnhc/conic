
\de{ĐỀ THI GIỮA HỌC KỲ I NĂM HỌC 2022-2023}{THPT Trưng Vương}

%%%===============Câu 1
\begin{bt}%[0T1B1-3]]%[Dự án đề kiểm tra HKI NH22-23- Nguyễn Ngọc Nguyên]%[THPT Trưng Vương]
	Cho mệnh đề $P:\forall x\in\mathbb{R},x^2+x+1>0$. Viết mệnh đề phủ định của mệnh đề $P$.
	\loigiai{
		Mệnh đề phủ định của mệnh đề $P$ là
		$$\overline{P}:\exists x\in\mathbb{R}, x^2+x+1\le 0.$$
}
\end{bt}
%%%===============Câu 2
\begin{bt}%[0T1B2-1]%[Dự án đề kiểm tra HKI NH22-23- Nguyễn Ngọc Nguyên]%[THPT Trưng Vương]
	Cho tập hợp $A=\left\{x\in\mathbb{R}\,\middle|\ (x^2-3x)(4x^2-3x-1)=0\right\}$. Xác định tập $A$ bằng cách liệt kê phần tử của nó.
	\loigiai{
		Ta có $(x^2-3x)(4x^2-3x-1)=0\Leftrightarrow\hoac{&x=0 \in\mathbb{R}\\&x=3 \in\mathbb{R}\\&x=1\in\mathbb{R} \\&x=-\dfrac{1}{4}\in\mathbb{R} .}$\\
		Do đó $A=\left\{-\dfrac{1}{4};0;1;3\right\}$.
}
\end{bt}
%%%===============Câu 3
\begin{bt}%[0T1B3-5]%[Dự án đề kiểm tra HKI NH22-23- Nguyễn Ngọc Nguyên]%[THPT Trưng Vương]
	Cho tập $A=(-\infty;7]$, $B=(-3;10]$. Tìm $A\cap B$, $A\cup B$, $A\setminus B$ và ghi kết quả dưới dạng khoảng, đoạn, nửa khoảng.	
	\loigiai{
		$\bullet\, A\cap B=(-3;7]$.\\
		$\bullet\, A\cup B=(-\infty;10]$.\\
		$\bullet\, A\setminus B=(-\infty;-3]$.
			
	}
\end{bt}
%%%===============Câu 4
\begin{bt}%[0T1K3-3]%[Dự án đề kiểm tra HKI NH22-23- Nguyễn Ngọc Nguyên]%[THPT Trưng Vương]
	\textbf{Bài toán về \lq\lq Thống kê người dùng Mạng xã hội\rq\rq:}\\
	Tại một quốc gia, số người có dùng ít nhất một trong hai ứng dụng Zalo và Facebook Messenger là $77$ triệu người. Trong đó có $74{,}7$ triệu người dùng Zalo và $67{,}8$ triệu người dùng Facebook Messenger. Hỏi có bao nhiêu người dùng cả hai ứng dụng trên?\\
	\textbf{Chú thích:} thống kê vào đầu năm 2022, tại Việt Nam có tất cả $76{,}95$ triệu người dùng mạng xã hội, số người dùng thường xuyên hàng tháng của Zalo đạt $74{,}7$ triệu, cao hơn ứng dụng nhắn tin Messenger của Meta ($67{,}8$ triệu). (Nguồn: Internet)
	\loigiai{
		Gọi $A$ là tập hợp số người dùng Zalo $\Rightarrow n(A)=74{,}7$ triệu người.\\
		Gọi $B$ là tập hợp số người dùng Facebook Messenger $\Rightarrow n(B)=67{,}8$ triệu người.\\
		Do đó số người dùng ít nhất một trong hai ứng dụng Zalo và Facebook Messenger là $A\cup B\Rightarrow n(A\cup B)=77$ triệu người.\\
		Yêu cầu bài toán tìm số người dùng cả hai ứng dụng trên là tìm $n(A\cap B)$.\\
		Mà $n(A\cap B)=n(A)+n(B)-n(A\cup B)=74{,}7+67{,}8-77=65{,}5$ triệu người.\\
		Vậy có $65{,}5$ triệu người dùng cả hai ứng dụng Zalo và Facebook Messenger.
	}
\end{bt}
%%%===============Câu 5
\begin{bt}%[0T4K2-1]%[Dự án đề kiểm tra HKI NH22-23- Nguyễn Ngọc Nguyên]%[THPT Trưng Vương]
	Cho tam giác $ABC$ có diện tích là $10$, $AB=5$ và $AC=8$. Tính số đo góc $A$ và độ dài cạnh $BC$ của tam giác $ABC$ biết góc $A$ là góc tù.
	\loigiai{
		Ta có $S_{ABC}=10\Leftrightarrow \dfrac{1}{2}\cdot AB\cdot AC\cdot\sin A=10\Leftrightarrow \sin A=\dfrac{1}{2}\Leftrightarrow\cos A=\pm\dfrac{\sqrt{3}}{2}$.\\
		Do góc $A$ tù suy ra $\cos A=-\dfrac{\sqrt{3}}{2}\Rightarrow\widehat{A}=150^\circ$.\\ 
		Áp dụng định lý cô-sin ta có\\
		$BC^2=AB^2+AC^2-2AB\cdot AC\cdot\cos A\Leftarrow BC^2=5^2+8^2-2\cdot5\cdot8\cdot \left(-\dfrac{\sqrt{3}}{2}\right) \Leftrightarrow BC\approx 12{,}6$.\\
		Vậy $BC\approx 12{,}6$ và $\widehat{A}=150^\circ$.
	}
\end{bt}
%%%===============Câu 6
\begin{bt}%[0T4K3-1]%[Dự án đề kiểm tra HKI NH22-23- Nguyễn Ngọc Nguyên]%[THPT Trưng Vương]
	\textbf{Bài toán về \lq\lq Đặt ghế VIP trong rạp chiếu phim\rq\rq:}\\
	\immini{Tại một phòng chiếu phim, màn hình có chiều ngang $18$ mét. Người ta thiết kế ghế VIP tại vị trí mà khán giả có góc quan sát lý tưởng nhất. Giả sử rằng góc quan sát lý tưởng nhất của khán giả đến màn hình từ $50^\circ$ đến $58^\circ$. Tính khoảng cách từ màn hình đến vị trí có thể đặt ghế VIP trong phòng chiếu phim, biết rằng ghế VIP đặt chính giữa màn hình (kết quả làm tròn đến hàng phần trăm).}
	{	\begin{tikzpicture}[scale=.9, font=\footnotesize, line join=round, line cap=round, >=stealth]
			\path
			(0,0) coordinate (A)node[left]{Ghế VIP}
			(6,3) coordinate (B)
			(6,-3)coordinate (C)
			(6,0)node[right]{Màn hình}
			(1,0)node[right] {Góc quan sát}
			;
			\draw[line width=2mm](B)--(C);
			\draw 
			(B)--(A)--(C)
			pic[angle radius=9mm,draw=blue,fill=green!50,opacity=.3] {angle = C--A--B}
			;
			%\foreach \p/\g in {A/180,B/90,H/0,C/-90} \draw[fill=black](\p)circle(.04)node[shift={(\g:.25)}]{$\p$};
			
	\end{tikzpicture}}
	\loigiai{
		\immini{Gọi $BC$ là chiều ngang của màn hình, $A$ là vị trí đặt ghế VIP và $H$ là trung điểm $BC$ như hình bên.\\
		 Do đó $AH$ là khoảng cách từ màn hình đến ghế VIP.\\
		 Suy ra $AH\perp BC$ tại $H$ và $\widehat{BAC}$ là góc quan sát lý tưởng.\\
	 	 Khi đó $BC=18$ mét, $BH=9$ mét và $\widehat{BAH}=\dfrac{1}{2}\widehat{BAC}$.
		\begin{itemize}
			\item Trường hợp  $\widehat{BAC}=58^\circ$\\
				 Xét $\triangle AHB$ vuông tại H, có $\tan\widehat{BAH}=\dfrac{BH}{AH}$\\
				 $\Rightarrow AH=\dfrac{BH}{\tan 29^\circ}=\dfrac{9}{\tan 29^\circ}\approx 16{,}24$ mét.
			\item Trường hợp  $\widehat{BAC}=50^\circ$\\
				Xét $\triangle AHB$ vuông tại H, có
				$\tan\widehat{BAH}=\dfrac{BH}{AH}$\\
				$\Rightarrow AH=\dfrac{BH}{\tan 25^\circ}=\dfrac{9}{\tan 25^\circ}\approx 19{,}30$ mét.
		\end{itemize}}
			{\begin{tikzpicture}[scale=1, font=\footnotesize, line join=round, line cap=round, >=stealth]
				\path
				(0,0) coordinate (A)
				(6,3) coordinate (B)
				(6,-3)coordinate (C)
				(6,0) coordinate (H)
				;
				\draw (A)--(B)--(C)--cycle (A)--(H);
				
				\foreach \p/\g in {A/180,B/90,H/0,C/-90} \draw[fill=black](\p)circle(.04)node[shift={(\g:.25)}]{$\p$};
				
		\end{tikzpicture}}
		\noindent Vậy khoảng cách từ màn hình đến vị trí có thể đặt ghế VIP trong phòng chiếu phim trong khoảng $16{,}24$ mét đến $19{,}30$ mét.
		
	}
\end{bt}
%%%===============Câu 7
\begin{bt}%[0T2K2-2]%[Dự án đề kiểm tra HKI NH22-23- Nguyễn Ngọc Nguyên]%[THPT Trưng Vương]
	\textbf{Bài toán về \lq\lq Dán tem thư trên phong bì\rq\rq:}\\
	Để tham gia hội chợ tem sưu tập, chú Nam dán một số con tem nhỏ và lớn vào phong bì để bán cho khách. Diện tích của tem nhỏ, tem lớn và phong bì thư (không kể viền) lần lượt là $7$ cm$^2$, $14$ cm$^2$, $196$ cm$^2$ (các con tem và bì thư đều có dạng hình chữ nhật và diện tích con tem được tính bằng chiều dài $a$ nhân chiều rộng $b$ như hình minh họa). Theo quy định ban tổ chức hội chợ, số lượng tem dán vào phong bì không vượt quá $19$ con. Giá tiền phong bì có dán tem tính trên số con tem được dán trên bì thư: tem nhỏ có giá bán $5\,000$ đồng/1 con, tem lớn là $9\,000$ đồng/1 con. Hỏi chú Nam phải dán bao nhiêu tem nhỏ, tem lớn vào phong bì để bán được giá cao nhất?
	\begin{center}
		\includegraphics[scale=0.7]{images/Tem}
		\includegraphics[scale=0.5]{images/PhongThu}
	\end{center}
	\loigiai{
		Gọi $x$, $y$ lần lượt là số con tem nhỏ và con tem lớn cần dán vào phong bì để bán được giá cao nhất. Ta có hệ bất phương trình
		$$\heva{&x+y\le 19\\&7x+14y\le 196\\&x\ge 0\\&y\ge 0}\Leftrightarrow\heva{&x+y\le 19\\&x+2y\le 28\\&x\ge 0\\&y\ge 0.}$$
		Miền nghiệm của hệ bất phương trình trên là miền không bị gạch trong hình sau
		\begin{center}
			\begin{tikzpicture}[>=stealth,x=1cm,y=1cm,scale=1,font=\footnotesize]
				\path
				(0,14/5)coordinate (A)node[above right]{$14$}
				(2,9/5)coordinate (B)
				(19/5,0)coordinate (C)
				(0,0) coordinate (O)
				(19/5,0)node[below left]{$19$}
				(0,19/5)node[left]{$19$}
				(28/5,0)node[below]{$28$}
				(2,0) coordinate (10)
				(0,9/5)coordinate (9)%node[left,scale=.8]{$9$}
				;	
				\draw[->] (-2,0) -- (8,0) node[below] {\scriptsize $x$};
				\draw[->] (0,-2) -- (0,6) node[left] {\scriptsize $y$};
				\draw (0,0)node[below left]{\scriptsize $O$};
				\clip (-2,-2)rectangle(8,6);
				\fill[pattern=north east lines](29/5,-2)--(-2,29/5)--(7.85,6)|-cycle;
				\fill[pattern=north west lines](48/5,-2)--(-2,19/5)--(-2,29/5)--(7.85,6)--(7.85,-9/8);
				\fill[pattern=dots](7.8,0)--(7.8,-2)--(-2,-2)--(-2,0)--cycle;
				\fill[pattern=crosshatch dots](0,-2)--(0,5.85)--(-2,5.85)--(-2,-2)--cycle;
				\draw[thick] 
				(29/5,-2)--(-2,29/5) node [pos=0.1,below,sloped, scale=.9] {$x+y-19=0$}
				(7.85,-9/8)--(-2,19/5)node [pos=0.4,above,sloped, scale=.9] {$x+2y-28=0$}
				;
				\draw[dashed](2,0)--(B)--(0,9/5);
				\foreach \p/\g in {A/-160,B/-135,C/60,O/45,10/-90,9/180} \draw[fill=black](\p)circle(.05)node[shift={(\g:.3)}]{$\p$};
			\end{tikzpicture}
		\end{center}
	Miền nghiệm là miền tứ giác $OABC$ với các đỉnh: $O(0;0)$; $A(0;14)$; $B(10;9)$; $C(19;0)$.\\
	Gọi $F$ là số tiền (ngàn đồng) bán được tem giá cao nhất, ta có $F=5x+9y$.\\
	Tính giá trị của $F$ tại các đỉnh của tứ giác\\
	$\bullet\,O(0;0)\Rightarrow F=0$;\\
	$\bullet\,A(0;14)\Rightarrow F=126$;\\
	$\bullet\,B(10;9)\Rightarrow F=131$;\\
	$\bullet\,C(19;0)\Rightarrow F=95$.\\
	$F$ đạt giá trị lớn nhất bằng $131$ tại $C(10;9)$.\\
	Vậy chú Nam dán $10$ tem nhỏ $9$ tem lớn thì phong bì bán được giá cao nhất là $131$ ngàn đồng.
	}
\end{bt}

