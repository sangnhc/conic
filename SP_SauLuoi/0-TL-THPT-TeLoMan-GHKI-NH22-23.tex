
\de{ĐỀ THI GIỮA HỌC KỲ I NĂM HỌC 2022-2023}{THPT Ten Lơ Man}

\begin{bt}%[0T1B3-1-1]%[Dự án đề kiểm tra HKI NH22-23- Phan Trung Hiếu]%[THPT Ten-Lo-Man]
 Cho $A=\left\{x \in \mathbb{N}\,|\left(x^2-1\right)(2-x)=0\right\}, B=\{x \in \mathbb{Z}|-2 \leq x<3\}$. Tìm $ A \cap B$.
  \loigiai{
 	Xét phương trình $(x^2-1)(2-x)=0$
 $	\Leftrightarrow$
 $\hoac{& x^2 -1=0 \\& 2-x=0 } \Leftrightarrow \hoac{&x=1\\&x=-1\\&x=2}$ $\Rightarrow A=\{1;2\}$.\\
 Và $B=\{x \in \mathbb{Z}|-2 \leq x<3\}$ $\Rightarrow B=\{-2;-1;0;1;2\}$.\\
 Vậy $ A \cap B = \{1;2\}$.
  }
\end{bt}

\begin{bt}%[0T1B2-2]%[Dự án đề kiểm tra HKI NH22-23- Phan Trung Hiếu]%[THPT Ten-Lo-Man]
	Cho tập hợp $A=\left\{x \in \mathbb{N}^* | 6\right.$ chia hết cho $\left.x\right\}$. Tìm hai tập con của $A$ có chứa hai phần tử.
	 \loigiai{
	Tập hợp $A=\left\{x \in \mathbb{N}^* | 6\right.$ chia hết cho $\left.x\right\}$ $\Rightarrow A=\{1;2;3;6\}$.\\
	Hai tập con của $A$ có chứa 2 phần tử là $\{1;2\}$; $\{1;3\}$.
		}
\end{bt}

\begin{bt}%[0T1B3-1]%[0T1B3-2]%[Dự án đề kiểm tra HKI NH22-23- Phan Trung Hiếu]%[THPT Ten-Lo-Man]
Cho $A=[-3 ; 5], B=(1 ;+\infty)$. Tìm $A \cup B ; B  \setminus A$.
 \loigiai{
 Ta có	$A \cup B = [-3; + \infty)$.\\
$B\setminus A=(5;+\infty)$.	
}
\end{bt}

\begin{bt}%[0T1K3-3]%[Dự án đề kiểm tra HKI NH22-23- Phan Trung Hiếu]%[THPT Ten-Lo-Man]
 Lớp $10A$ có $22$ bạn thích môn Toán, $25$ bạn thích môn Văn và $15$ bạn thích cả hai môn Toán và Văn. Hỏi lớp $10A$ có bao nhiêu học sinh thích ít nhất một trong hai môn Văn và Toán.
  \loigiai{
\immini{ 	Gọi $A$ là tập hợp các bạn thích môn Toán, $B$ là tập hợp các bạn thích môn Văn.\\
 	Khi đó, $A \cup B$ là tập hợp các bạn thích ít nhất một trong 2 môn Toán và Văn, $A \cap B$ là tập hợp các bạn thích cả hai môn Toán và Văn.\\
 	Ta có $n(A)=22$, $n(B)=25$, $n(A \cap B)=15$.\\
Vậy số học sinh thích ít nhất một trong 2 môn Toán và Văn là $n(A \cup B)=n(A)+n(B)-n(A \cap B)=22+25-15=32$ (học sinh). }{ \begin{tikzpicture}[scale=0.54]
	\def\firstven{(0,0) ellipse (3cm and 2cm)}
	\def\secondven{(2.5,1) ellipse (2.8cm and 2cm)}
	\begin{scope}
		\clip \firstven;
		\fill[pattern=north east lines,opacity=0.95] \secondven;
	\end{scope}
	\draw \firstven \secondven;
	\node at (-2.2,2) {$A$};
	\node at (5.6,2.2){$B$};
	\node at (1.3,0.5){$A \cap B$};
	\node at (3.5,-1.7){$A \cup B$};
\end{tikzpicture}}
 }
\end{bt}

\begin{bt}%[0T4B3-1]%[Dự án đề kiểm tra HKI NH22-23- Phan Trung Hiếu]%[THPT Ten-Lo-Man]
 Cho tam giác $ABC$ có $a=13 ; b=14 ; c=15$. Tính diện tích tam giác $ABC$.
 \loigiai{
 	Ta có $p=\dfrac{13+14+15}{2}=21$.\\
 	Áp dụng công thức Hê-rông ta có diện tích tam giác $ABC$ là $$S_{\triangle ABC} = \sqrt{p(p-a)(p-b)(p-c) }= \sqrt{21\cdot (21-13)\cdot(421-14)\cdot(21-15)} = 84.$$
}
\end{bt}
\newpage
\begin{bt}%[0T4B3-1]%[Dự án đề kiểm tra HKI NH22-23- Phan Trung Hiếu]%[THPT Ten-Lo-Man]
 Người ta muốn nối một sợi dây trực tiếp từ $A$ đến $C$ nhưng không thể vì phải qua một hố sâu nên người ta làm như sau: Nối sợi dây từ $A$ đến $B$ rồi từ $B$ đến $C$ biết khoảng cách $AB=120$ m, $BC=50 $ m và đo được góc $\widehat{ACB}=37^{\circ}$. Hỏi nếu nối dây từ $A$ đến $C$ thì tiết kiệm bao nhiêu dây so với đi đường vòng (đi từ $A$ đến $B$ rồi từ $B$ đến $C$),
(làm tròn kết quả đến chữ số thâp phân thứ nhất).
\loigiai{
	\begin{center}
	\begin{tikzpicture}[scale=1, font=\footnotesize, line join=round, line 
		cap=round, >=stealth]
		\path 
		(1,3) coordinate (B)
		(0,0) coordinate (C)
		(6,0) coordinate (A)
		;
		
		\draw (A)--(B)--(C)--cycle;
		\draw[line width=25pt,color=blue] ($(C)!0.3!(A)$)--($(A)!0.3!(C)$);
		\foreach \p/\r in {B/90,C/-120,A/-60}
		\fill (\p) circle (1.5pt) node[shift={(\r:3mm)}]{$\p$};
	\end{tikzpicture}
\end{center}
Theo định lý cô-sin ta có 
$$\begin{aligned}
&	\cos C = \dfrac{BC^2 +AC^2-AB^2}{2BC\cdot AC}\\
	\Leftrightarrow &	\cos 37^\circ = \dfrac{50^2 +AC^2-120^2}{2\cdot50\cdot AC}\\
	\Leftrightarrow & AC \approx 156{,1} \,\text{m}.
\end{aligned}$$
Vậy khoảng cách $AC$ là $156{,}1$ m.
}
\end{bt}

\begin{bt}%[0T2K1-2]%[Dự án đề kiểm tra HKI NH22-23- Phan Trung Hiếu]%[THPT Ten-Lo-Man]
 Phần không gạch chéo (không kể bờ $d$) trong hình vẽ sau biễu diễn miền nghiệm một bất phương trình. Hãy tìm bất phương trình đó.
\loigiai{
	\begin{center}
	\begin{tikzpicture}[>=stealth,line join=round, line cap=round, scale=0.7]
		\draw[->] (-1.5,0)--(6.5,0) node[above left]{$x$};
		\draw[->] (0,-1.5)--(0,3.5) node[below right]{$y$};
		\fill[pattern=north east 
		lines]plot[domain=-1.5:6.5](\x,{(2*\x)/5})--plot[domain=6.5:-1.5](\x,{-1.5})--cycle;
		\draw[samples=200,smooth,red,line width=1] plot[domain=-1.5:6.5] 
		(\x,{(2*\x)/5}) node[above]{$d$};
		\foreach \x in {-1,1,2,3,4,5,6} \draw[fill] (\x,0) circle (1.5pt) 
		node[below]{$\x$};
		\foreach \y in {-1,1,2,3} \draw[fill] (0,\y) circle (1.5pt) 
		node[left]{$\y$};
		\draw[dashed] (5,0)--(5,2)--(0,2);
		\fill (0,0) circle (1.5pt) node[above left]{$O$};
		\fill (5,2) circle (1.5pt) node[above right]{$M$};
	\end{tikzpicture}
\end{center}
Đường thẳng $d$ có phương trình $y=ax+b$ đi qua 2 điểm $O(0;0); M(5;2)$ nên ta có hệ phương trình 
$\begin{aligned}
\heva{&b=0\\&5a+b=2} \Leftrightarrow  \heva{&b=0\\&a=\dfrac{5}{2} . } 
\end{aligned}$\\
$\Rightarrow y=\dfrac{5}{2} x \Rightarrow 5x -2y=0$.\\
Vậy bất phương trình là $5x -2y  \leq 0$.
}
\end{bt}
\begin{bt}%[0T1K2-2]%[Dự án đề kiểm tra HKI NH22-23- Phan Trung Hiếu]%[THPT Ten-Lo-Man]
 Cho ba tập hợp $A=\{2 ; 5\}, B=\{5 ; x+1\}, C=\{2 ; y-3\}$. Cho $A=B=C$, hãy tìm tổng các bình phương của $x$ và $y$.
  \loigiai{
 $$\begin{aligned}
 	A=B=C  \Leftrightarrow  \heva{&x+1=2\\&y-3=5} \Leftrightarrow \heva{&x=1\\&y=8} \Rightarrow x^2 + y^2 = 65.
 \end{aligned} $$ 
 	
 }
\end{bt}