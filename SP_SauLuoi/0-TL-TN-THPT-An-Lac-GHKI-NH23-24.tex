
\de{ĐỀ THI GIỮA HỌC KỲ I NĂM HỌC 2023-2024}{THPT AN LẠC}
\begin{bt}%[0D1N1-5]%[Dự án đề kiểm tra Toán 10 GHKI NH23-24- Phan Trung Hiếu]%[THPT An Lạc- Tp HCM]
	\begin{enumerate}
		\item Lập mệnh phủ định của các mệnh đề sau\\
		$A$: \lq\lq$\forall x\in\mathbb{R},x^2\leq 2x-1$\rq\rq.\\
		$B$: \lq\lq$\exists x\in\mathbb{R},x^2-x+1<0$\rq\rq.
		\item Xét tính đúng sai của các mệnh đề sau\\
		$C$: \lq\lq Nếu $a+b+c=0$ thì phương trình $ax^2+bx+c=0$ có một nghiệm bằng 1\rq\rq.\\
		$D$: \lq\lq Vì $120$ chia hết cho $6$ nên $120$ chia hết cho $9$\rq\rq.
	\end{enumerate}
	\loigiai{
		\begin{enumerate}
			\item $\overline{A}$: \lq\lq$\exists \in\mathbb{R}, x^2>2x-1$\rq\rq. \\
			$\overline{B}$: \lq\lq$\forall x \in\mathbb{R}, x^2-x+1\geq0$\rq\rq.
			\item $C$ là mệnh đề đúng do thế $x=1$ vào phương trình ta có $a+b+c=0$.\\
			$D$ là mệnh đề sai vì $120$ không chia hết cho $9$.
		\end{enumerate}
	}
\end{bt}
%Câu 2...........................
\begin{bt}%[0D1H2-1]%[Dự án đề kiểm tra Toán 10 GHKI NH23-24- Phan Trung Hiếu]%[THPT An Lạc- Tp HCM]
	Xác định các tập hợp sau bằng cách liệt kê
%	\begin{multicols}{2}
		\begin{enumerate}
			\item $A=\{x\in\mathbb{R}\big|x^2-10x+21=0\}$.
			\item $B=\{x\in\mathbb{Z}\big|-3<x<4\}$.
			\item $C=\{x\in\mathbb{Q}\big|(4x^2-3x-7)(2x^2-5)=0\}$.
			\item $D=\{x\in\mathbb{N}\big||2x-3|\leq7\}$.
		\end{enumerate}
%	\end{multicols}
	\loigiai{
		\begin{enumerate}
			\item Ta có $x^2-10x+21=0\Leftrightarrow\hoac{&x=3~(\text{nhận})\\&x=7~(\text{nhận}).}$\\
			Vậy $A=\{3;7\}$.
			\item $B=\{-2;-1;0;1;2;3\}$.
			\item Ta có
			$(4x^2-3x-7)(2x^2-5)=0\Leftrightarrow\hoac{&4x^2-3x-7=0\\&2x^2-5=0}\Leftrightarrow\hoac{&x=-1~(\text{nhận})\\&x=\dfrac{7}{4}~(\text{nhận})\\&x=\dfrac{\sqrt{5}}{2}~(\text{loại})\\&x=-\dfrac{\sqrt{5}}{2}~(\text{loại}).}$\\
			Vậy $C=\left\{-1;\dfrac{7}{4}\right\}$.
			\item Ta có $|2x-3|\leq7\Leftrightarrow-7\leq2x-3\leq7\Leftrightarrow-2\leq x\leq5$.\\
			Vậy $D=\{0;1;2;3;4;5\}$.
		\end{enumerate}
	}
\end{bt}
%Câu 3...........................
\begin{bt}%[0D1H3-4]%[Dự án đề kiểm tra Toán 10 GHKI NH23-24- Phan Trung Hiếu]%[THPT An Lạc- Tp HCM]
	\begin{enumerate}
		\item Cho hai tập hợp $A=(2;7)$, $B=(-\infty;5]$. Xác định $A\cap B$, $A\cup B$, $A\setminus B$, $C_\mathbb{R} A$.
		\item Viết các tập hợp sau dưới dạng khoảng, đoạn, nửa khoảng
		\begin{equation*}
			C=\{x\in\mathbb{R}|-7<x\leq13\}, \quad D=\left\{x\in\mathbb{R}\Big|x>\dfrac{2}{5}\right\}.
		\end{equation*}
	\end{enumerate}
	\loigiai{
		\begin{enumerate}
			\item $A\cap B=(2;5]$, $A\cup B =(-\infty;7)$, $A\setminus B=(5;7)$, $C_\mathbb{R}A=(-\infty;2]\cup[7;+\infty)$.
			\item $C=(-7;13]$, $D=\left(\dfrac{2}{5};+\infty\right)$.
		\end{enumerate}
	}
\end{bt}
\begin{bt}%[0H4H1-2]%[Dự án đề kiểm tra Toán 10 GHKI NH23-24- Lâm Chính]%[THPT An Lạc- Tp HCM]
	Cho $\cos x=\dfrac{-4}{5}$ với $90^\circ < x < 180^\circ$. Tính $\sin x, \tan x, \cot x, P=2\sin^2 x+5\cot^2 x$.
	\loigiai{
		Ta có $\sin^2 x+\cos ^2x=1$
		$\Rightarrow \sin ^2x=1-\cos^2x $
		$\Rightarrow \sin x=\pm\sqrt{1-\cos ^2x}$\\
		$\Rightarrow \sin x=\pm\sqrt{1-\left(-\dfrac{4}{5} \right)^2}$
		$\Rightarrow \hoac{	& \sin x=\dfrac{3}{5} \\
			& \sin x=-\dfrac{3}{5}.}$\\
		Vì  $90^\circ < x < 180^\circ$ nên $\sin x >0$. Nhận $\sin x=\dfrac{3}{5}$.\\
		$\Rightarrow \tan x=\dfrac{\sin x}{\cos x}=\dfrac{3}{5}:\left(-\dfrac{4}{5} \right)=-\dfrac{3}{4}$ và 
		$\cot x=\dfrac{\cos x}{\sin x}=-\dfrac{4}{3}$.\\
		$P=2\sin^2x+5\cot^2 x=2\cdot\left(\dfrac{3}{5} \right)^2+5\cdot \left(-\dfrac{4}{3} \right)^2=\dfrac{2162}{225}$.\\
	}
\end{bt}
\begin{bt}%[0H4H2-2]%[Dự án đề kiểm tra Toán 10 GHKI NH23-24- Lâm Chính]%[THPT An Lạc- Tp HCM]
	Cho tam giác $ABC$ biết $AB=7$ cm, $AC=10$ cm, góc $\widehat{BAC}=60^\circ$.
	\begin{enumerate}
		\item Tính độ dài cạnh $BC$ và diện tích tam giác $ABC$.
		\item Gọi $BH$ là đường cao của tam giác $ABC$. Tính độ dài đoạn $AH$, $BH$.
	\end{enumerate}
	\loigiai{
		\begin{center}
			\begin{tikzpicture}[scale=0.8, font=\footnotesize,line join=round, line cap=round, >=stealth]
				\coordinate (A) at (0,0);
				\coordinate (C) at (5,0);
				\coordinate (B) at ($(A) + (60:3.5)$);
				\coordinate (H) at ($(A)!(B)!(C)$);
				\draw(A)--(B)--(C)--cycle (B)--(H);
%				\pic[draw,thin,angle radius=6mm] {angle = C--A--B};
				\pic[draw,thin,angle radius=2mm] {right angle = B--H--C};
				\draw  pic["\tiny $60^\circ$", draw=black, angle eccentricity=1.6,angle radius=.4cm]
				{angle=C--A--B};
				%		\draw (0.5,0.5) node[right]{$60^\circ$};
				\foreach \i/\g in {A/-90,B/90,C/-90,H/-90}{\draw[fill=black](\i) circle (1.5pt) ($(\i)+(\g:3mm)$) node[scale=1]{$\i$};}
			\end{tikzpicture}
		\end{center}
		\begin{enumerate}
			\item Áp dụng định lý hàm số cô-sin ta có
			\allowdisplaybreaks
			\begin{eqnarray*}
				BC^2&=&AB^2+AC^2-2AB\cdot AC\cdot\cos A\\
				&=&7^2+10^2-2\cdot7\cdot10\cdot\cos 60^\circ\\
				&=&79.\\
				\Rightarrow BC &=& \sqrt{79}.
			\end{eqnarray*}
			Diện tích tam giác $ABC$ bằng $S_{\triangle ABC} =\dfrac{1}{2}AB\cdot AC\cdot\sin A =\dfrac{1}{2}\cdot 7\cdot 10\cdot\sin60 ^\circ  = \dfrac{35\sqrt3}{2}$.
			\item Tam giác $ABH$ vuông tại $H$, ta có \\
			$\cos \widehat{BAH} = \dfrac {AH}{AB} \Rightarrow AH = AB\cdot \cos \widehat{BAH} = 7 \cdot \cos 60 ^\circ =\dfrac{7}{2} $.\\
			$\sin \widehat{BAH} = \dfrac {BH}{AB} \Rightarrow BH = AB\cdot \sin \widehat{BAH} = 7 \cdot \sin 60 ^\circ = \dfrac{7\sqrt3}{2}$.
		\end{enumerate}
	}
\end{bt}
%Câu 1...........................
\begin{bt}%[0H4H1-3]%[Dự án đề kiểm tra Toán 10 GHKI NH23-24- Nguyễn Cường]%[THPT An Lạc ]
	Cho tam giác $ABC$. Chứng minh các đẳng thức sau
	\begin{multicols}{2}
		\begin{enumerate}
			\item $\sin\left(\dfrac{A+B-2C}{2}\right)=\cos\dfrac{3C}{2}$; 
			\item $\cos(A+B-C)=-\cos 2C$.
		\end{enumerate}
	\end{multicols}
	\loigiai{
		\begin{enumerate}
			\item Ta có
			\allowdisplaybreaks
			\begin{eqnarray*}
				\sin\left(\dfrac{A+B-2C}{2}\right)&=&\sin\left(\dfrac{A+B+C-3C}{2}\right)\\
				&=&\sin\left(\dfrac{180^\circ-3C}{2}\right)\\
				&=&\sin\left(90^\circ-\dfrac{3C}{2}\right)=\cos\dfrac{3C}{2}.
			\end{eqnarray*}
			\item $\cos(A+B-C)=\cos\left(A+B+C-2C\right)=\cos\left(180^\circ-2C\right)=-\cos 2C$.
		\end{enumerate}
	}
\end{bt}
%Câu 1...........................
\begin{bt}%[0H4V3-2]%[Dự án đề kiểm tra Toán 10 GHKI NH23-24- Nguyễn Cường]%[THPT An Lạc]
	\immini
	{
		Hai người đứng trên bờ biển ở hai vị trí $A$, $B$ cách nhau $500$ m cùng nhìn thấy mép của một hòn đảo ở vị trí $C$ trên đảo với các góc so với bờ biển lần lượt là $60^\circ$ và $70^\circ$. Tính khoảng cách $d$ từ mép hòn đảo đến bờ biển (làm tròn đơn vị m).
	}
	{
		\begin{tikzpicture}[line join=round, line cap=round,scale=1.5,transform shape]
			\definecolor{darkbrown}{rgb}{0.4, 0.26, 0.13}	
			\definecolor{lightcornflowerblue}{rgb}{0.6, 0.81, 0.93}
			\definecolor{forestgreen(web)}{rgb}{0.13, 0.55, 0.13}
			\definecolor{darkpastelgreen}{rgb}{0.01, 0.75, 0.24}
			\definecolor{bronze}{rgb}{0.8, 0.5, 0.2}
			\definecolor{deepskyblue}{rgb}{0.0, 0.75, 1.0}
			
			\clip (-1.8,-2.5) rectangle (2.3,1.5);
			\tikzset{gon_song/.pic={
					\def\S{ %sóng
						(-1.35,-2.15)
						..controls +(160:.5) and +(-40:.5) ..(-2.5,-2.15)
						(3.5,-2.2)
						..controls +(160:.2) and +(-40:.5) ..(2,-2.2)
						%----
						(3,-2.5)
						..controls +(160:.5) and +(-40:.5) ..(1.4,-2.5)
						..controls +(160:.5) and +(-40:.5) ..(.3,-2.5)
						..controls +(160:.5) and +(-40:.5) ..(-.75,-2.55)
						;}
					\draw[color=deepskyblue] \S;
			}}
			\tikzset{nuoc/.pic={
					\def\N{ %nước sông
						(-4,.3)
						%..controls +(160:.2) and +(-80:.3)..(1.5,.1)
						..controls +(120:.5) and +(-100:0.2)..(4,.8)--(4,-1.1)
						..controls +(-150:.3) and +(60:.3) ..(2,-1.2)
						..controls +(-100:.5) and +(120:1) ..(-4,-1.9)
						--cycle
						;}
					%\draw \N;
					\fill[lightcornflowerblue] \N;
			}}
			
			\tikzset{dat/.pic={
					\def\T{ %Đất
						(0,0)%trái
						..controls +(170:.2) and +(40:.3) ..  (-.7,-.1)
						..controls +(-120:.15) and +(40:.15) ..  (-1.1,-.25)
						..controls +(-150:.4) and +(170:.4) ..  (-.7,-.55)
						..controls +(-35:.4) and +(-150:.2) ..  (.2,-.7)
						..controls +(50:.4) and +(-120:.35) ..  (.8,-.5)
						..controls +(60:.3) and +(-120:.2) ..  (1.3,-.3)
						..controls +(60:.2) and +(-50:.1) ..  (.7,-.1)
						..controls +(60:.2) and +(-40:.1) ..  (0,0)
						
						;}
					\draw \T;
					\fill[darkbrown] \T;
			}}
			\tikzset{cay/.pic={
					\def\T{ %Thân
						(-.33,0)%trái
						..controls +(-50:.25) and +(40:.45) ..  (-.57,-1.45)
						..controls +(20:.1) and +(-160:.15) ..  (-.1,-1.3)
						..controls +(-120:.1) and +(60:.15) ..  (-.2,-1.6)
						..controls +(-30:.1) and +(-140:.15) ..  (.15,-1.3)
						..controls +(-20:.1) and +(-160:.15) ..  (.57,-1.4)
						..controls +(170:.4) and +(-160:.1) ..  (.35,0)
						..controls +(110:.5) and +(80:.5) ..  (-.33,0)
						;}
					%\draw \T;
					\fill[bronze] \T;
					\def\C{ 
						(0,.3)
						..controls +(-100:.25) and +(-60:.2) ..  (-.3,.1)
						..controls +(-100:.25) and +(-60:.2) ..  (-.6,0)
						..controls +(-120:.45) and +(-110:.35) ..  (-1,.2)
						..controls +(-150:.5) and +(-140:.35) ..  (-1.15,.7)%nút giao
						..controls +(-170:.4) and +(-170:.35) ..  (-1,1.15)
						..controls +(140:.35) and +(110:.4) ..  (-.37,1.35)
						..controls +(110:.25) and +(80:.3) ..  (-.15,1.35)
						..controls +(80:.3) and +(95:.8) ..  (.55,1.1)
						..controls +(80:.2) and +(95:.2) ..  (.8,1.1)
						..controls +(20:.1) and +(95:.1) ..  (.95,1)
						..controls +(-20:.4) and +(35:.25) ..  (1,.47)
						..controls +(-30:.3) and +(-20:.3) ..  (.75,0.05)%nút giao
						..controls +(-120:.3) and +(-60:.2) ..  (.35,0)
						..controls +(175:.2) and +(-160:.1) ..  (.2,0.2)
						..controls +(-160:.1) and +(-70:.1) ..  (0,.3)
						;}
					\draw \C;
					\fill[forestgreen(web)] \C;
					\def\C1{ 
						(-1.15,.7)%nút giao
						..controls +(-170:.4) and +(-170:.35) ..  (-1,1.15)
						..controls +(140:.35) and +(110:.4) ..  (-.37,1.35)
						..controls +(110:.25) and +(80:.3) ..  (-.15,1.35)
						..controls +(80:.3) and +(95:.8) ..  (.55,1.1)
						..controls +(80:.2) and +(95:.2) ..  (.8,1.1)
						..controls +(20:.1) and +(95:.1) ..  (.95,1)
						..controls +(-20:.4) and +(35:.25) ..  (1,.47)
						..controls +(-50:.5) and +(-85:.6) ..  (.65,.55)% gần nút giao
						..controls +(-160:.4) and +(-120:.4) ..  (0,.7)
						..controls +(-150:.2) and +(-80:.4) ..  (-.63,.6)
						..controls +(-140:.5) and +(-130:.4) ..  (-1.15,.7)
						;}
					%\draw \C1;
					\fill[darkpastelgreen] \C1;
					\def\G{ %Gân
						(-.8,.2)
						..controls +(-35:.1) and +(130:.35) ..  
						(-.32,0)%nút giao
						..controls +(-40:.1) and +(45:.35) ..  (-.42,-1.25)
						(-.32,0)%nút giao
						..controls +(120:.1) and +(-45:.35) ..  (-.58,.45)
						(-.32,0)%nút giao
						..controls +(80:.1) and +(-170:.35) ..  (-.05,.45)
						(-.28,.3)
						..controls +(80:.1) and +(-60:.05) ..  (-.35,.55)
						%Gân phải
						(.45,-1.3)
						..controls +(130:.6) and +(-160:.3) ..  (.6,0.35)
						(.37,0)
						..controls +(35:.2) and +(-150:.1) ..  (.66,0.15)
						(.37,0)
						..controls +(80:.2) and +(-40:.1) ..  (.18,0.4)
						(.31,0.25)
						..controls +(80:.1) and +(-150:.1) ..  (.38,0.5)
						(.25,-0.15)
						..controls +(-110:.05) and +(110:.05) ..  (.2,-0.5)%gân dọc
						(.2,-0.25)
						..controls +(-110:.05) and +(110:.05) ..  (.17,-0.45)
						(-.2,-0.15)
						..controls +(-70:.1) and +(110:.05) ..  (-.18,-0.5)
						(-.15,-0.15)
						..controls +(-70:.1) and +(110:.05) ..  (-.15,-0.7)
						(-.05,-0.8)
						..controls +(-80:.15) and +(-10:.15) ..  (-.3,-1.28)
						(-.1,-0.9)
						..controls +(-110:.1) and +(20:.05) ..  (-.2,-1.1)
						(.1,-1)
						..controls +(-50:.05) and +(120:.05) ..  (.15,-1.2)
						(.1,-1.15)
						..controls +(-50:.05) and +(120:.05) ..  (.12,-1.2)
						(.15,-.75)
						..controls +(-120:.1) and +(-140:.1) ..  (.22,-.9)
						..controls +(70:.12) and +(120:.05) ..  (.18,-.9)
						;}
					
					\draw \G;
					
			}}
			\path
			(0,0)pic[scale=1]{nuoc}
			(0,.45)pic[scale=1]{dat}
			%	(0,.5)pic[scale=.6]{cay}
			(0,.5)pic[scale=.6]{gon_song}
			(2,1.5)pic[scale=.6]{gon_song}
			(-2.5,1.3)pic[scale=.5]{gon_song}
			;
			\path 	(2,-1.5) coordinate (A)
			(-1,-1.5) coordinate (B)
			(0,-.3) coordinate (C)
			;
			\node at (B) [left]{\tiny $B$};
			\node at (A) [right]{\tiny $A$};
			\node[white] at (C) [above]{\tiny $C$};
			\node at (0.5,-1.4) [below]{\tiny $500$ m};
			\draw (B)--(C)--(A)--cycle;
			\draw    pic["\tiny $60^\circ$", draw=black, angle eccentricity=1.6,angle radius=.4cm, color=blue]
			{angle=C--A--B};
			\draw    pic["\tiny $70^\circ$", draw=black, angle eccentricity=1.6, angle radius=.4cm, color=blue]
			{angle=A--B--C};
			\draw pic[draw, double, cap = butt, angle radius = 12pt]{angle = A--B--C};
		\end{tikzpicture}
	}
	\loigiai{
		\immini
		{
			Gọi $CH$ là đường cao của $\triangle ABC$.\\
			Ta có $\widehat{C}=180^\circ-\widehat{A}-\widehat{B}=180^\circ-60^\circ-70^\circ=50^\circ$.\\
			Áp dụng định lý sin vào $\triangle ABC$ có
			$$\dfrac{AC}{\sin B}=\dfrac{AB}{\sin C}\Rightarrow AC=\dfrac{AB\cdot \sin B}{\sin C}=\dfrac{500\cdot \sin 70^\circ}{\sin 50^\circ}\approx 613.$$
		}
		{
			\begin{tikzpicture}[line join=round, line cap=round,scale=1.5,transform shape]
				\path 	(2,-1.5) coordinate (A)
				(-1,-1.5) coordinate (B)
				(0,-.3) coordinate (C)
				($(A)!(C)!(B)$)coordinate(H)
				;
				\node at (B) [left]{\tiny $B$};
				\node at (A) [right]{\tiny $A$};
				\node at (C) [above]{\tiny $C$};
				%	\node at (0.5,-1.4) [below]{\tiny $500$ m};
				\node at (H) [below]{\tiny $H$};
				\draw (B)--(C)--(A)--cycle (C)--(H);
				\draw  pic["\tiny $60^\circ$", draw=black, angle eccentricity=1.6,angle radius=.4cm, color=blue]
				{angle=C--A--B};
				\draw pic["\tiny $70^\circ$", draw=black, angle eccentricity=1.6, angle radius=.4cm, color=blue]
				{angle=A--B--C};
				\draw pic[draw, double, cap = butt, angle radius = 12pt]{angle = A--B--C};
					\pic[draw,thin,angle radius=2mm] {right angle = B--H--C};
			\end{tikzpicture}
		}
		\noindent
		Ta lại có $\sin A=\dfrac{CH}{AC}\Rightarrow CH=AC\cdot\sin A=613\cdot\sin 60^\circ\approx 531$ (m).\\
		Vậy khoảng cách $d$ gần bằng $531$ m.
	}
\end{bt}