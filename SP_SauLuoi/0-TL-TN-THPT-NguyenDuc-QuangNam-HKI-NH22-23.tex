
\de{ĐỀ THI HỌC KỲ I NĂM HỌC 2022-2023}{Trường THPT Nguyễn Dục - Quảng Nam}
\begin{center}
	\textbf{PHẦN 1 - TRẮC NGHIỆM}
\end{center}
\Opensolutionfile{ans}[ans/ans]
\begin{ex}%Câu 1%[0H3B2-1]%[Dự án đề kiểm tra HKII NH22-23- Nguyễn Tài Tuệ]%[THPT Nguyễn Đức-QuangNam]
	Cho hình vuông $ABCD$ có độ dài cạnh bằng 2 . Tính $\overrightarrow{AB}\cdot \overrightarrow{AC}$.
	\choice
	{$ 0  $}
	{$8 \sqrt{2}$}
	{\True $ 4  $}
	{$2 \sqrt{2}$}	
	\loigiai{
		Ta có $ \overrightarrow{AB}\cdot \vec{AC}=AB\cdot AC\cdot \cos \widehat{BAC}=2\cdot 2\sqrt{2} \cdot \dfrac{\sqrt{2}}{2} =4$.
	}
\end{ex}

\begin{ex}%Câu 2%[0D1Y2-3]%[Dự án đề kiểm tra HKII NH22-23- Nguyễn Tài Tuệ]%[THPT Nguyễn Đức-QuangNam]
	Sử dụng các kí hiệu khoảng, đoạn để viết tập hợp $A=\{x \in \mathbb{R}\mid x \leq 3\}$.
	\choice
	{\True $A=(-\infty ; 3]$}
	{ $A=[3 ;+\infty)$}
	{$A=(-\infty ; 3)$}
	{$A=(3 ;+\infty)$}
	\loigiai{
	Ta có $A=(-\infty ; 3]$.
	}
\end{ex}

\begin{ex}%Câu 3%[0H1B1-2]%[Dự án đề kiểm tra HKII NH22-23- Nguyễn Tài Tuệ]%[THPT Nguyễn Đức-QuangNam]
	Trong các đẳng thức sau, đẳng thức nào \textbf{sai}?
	\choice
	{$\sin \left(180^{\circ}-\alpha\right)=\sin \alpha$}
	{\True $\cos \left(180^{\circ}-\alpha\right)=\cos \alpha$}
	{$\tan \left(180^{\circ}-\alpha\right)=-\tan \alpha$}
	{$\cot \left(180^{\circ}-\alpha\right)=-\cot \alpha$}
	\loigiai{
	Theo định giá trị lượng giác của hai góc bù nhau ta có $\cos \left(180^{\circ}-\alpha\right)=-\cos \alpha$.
	}
\end{ex}

\begin{ex}%Câu 4%[0D1B3-1]%[Dự án đề kiểm tra HKII NH22-23- Nguyễn Tài Tuệ]%[THPT Nguyễn Đức-QuangNam]
	Cho $S=(-\infty ; m)$, $T=[0 ; 2]$. Tìm $m$ để $S \cap T=\varnothing$
	\choice
	{$m<2$}
	{\True $m \leq 0$}
	{$m<0$}
	{$m \leq 2$}
	\loigiai{
		Ta có $S \cap T=\varnothing \Leftrightarrow m\le 0$.
	}
\end{ex}

\begin{ex}%Câu 5%[0D1B1-3]%[Dự án đề kiểm tra HKII NH22-23- Nguyễn Tài Tuệ]%[THPT Nguyễn Đức-QuangNam]
	Theo thống kê, dân số tỉnh A được ghi lại như sau $ 1495812\pm 200 $ (người). Số quy tròn của số gần đúng $ 1495812 $ là
	\choice
	{$ 1495900  $}
	{\True $ 1496000 $ }
	{$ 1495800 $ }
	{$ 1495000 $ }
	\loigiai{
		Số quy tròn của số gần đúng $ 1495812 $ là $ 1496000 $.
	}
\end{ex}

\begin{ex}%Câu 6%[0H2B1-3]%[Dự án đề kiểm tra HKII NH22-23- Nguyễn Tài Tuệ]%[THPT Nguyễn Đức-QuangNam]
\immini{	Cho hình thoi $ ABCD $ như hình vẽ. Khẳng định nào sau đây đúng?
	\choice
	{\True $\overrightarrow{B C}=\overrightarrow{A D}$}
	{$\overrightarrow{B C}=\overrightarrow{D A}$}
	{$\overrightarrow{B C}=\overrightarrow{C D}$}
	{$\overrightarrow{B C}=\overrightarrow{A B}$}}{
		\begin{tikzpicture}
			\path 
			(0,0) coordinate (A)
			($ (A)+(-30:2) $) coordinate (D)
			($ (A)+(30:2) $) coordinate (B)
			($ (B)+(D)-(A)$) coordinate (C);
			\draw (A)--(B)--(C)--(D)--cycle  (A)--(C) (B)--(D);
			\foreach \p/\r in {A/180, B/ 90, C/0,D/-90}
			\fill (\p) circle (1.25pt) node[shift={(\r:3mm)}]{$\p$};
			
		\end{tikzpicture}
	}
\loigiai{Ta có $\overrightarrow{B C}=\overrightarrow{A D}$.}
\end{ex}

\begin{ex}%Câu 7%[0H1B3-2]%[Dự án đề kiểm tra HKII NH22-23- Nguyễn Tài Tuệ]%[THPT Nguyễn Đức-QuangNam]
	\immini{Hình bên dưới biểu diễn hai lực $\vec{F}_1, \vec{F}_2$ cùng tác động lên một vật. Biết độ lớn của các lực $\vec{F}_1, \vec{F}_2$ lần lượt là $4 \mathrm{~N}$ và $3 \mathrm{~N}$. Độ lớn của hợp lực $\vec{F}_1+\vec{F}_2$ tác động lên vật là.
	\choice
	{$\sqrt{37}$}
	{$ 7  $}
	{$ 5  $}
	{\True $\sqrt{13}$}}{
	\begin{tikzpicture}
		\clip(-3,-0.5) rectangle (4,3);
		\path
		(0,0)coordinate (A)
		(4,0) coordinate (B)
		(135:3) coordinate (D)
		($ (B)+(C)-(A) $) coordinate (C);
		
		 \draw[->] (A)--(B);
		 \draw[->] (A)--(D);
		 \draw
		 pic [draw,angle radius=4mm] {angle = B--A--D};
		 
		 \draw (A) node[shift={(60:0.8)}]{$120^\circ$};
	\end{tikzpicture}}
\loigiai{
	\begin{center}
		\begin{tikzpicture}
			\path
			(0,0)coordinate (A)
			(4,0) coordinate (B)
			(120:3) coordinate (D)
			($ (B)+(D)-(A) $) coordinate (C);
			
			\draw[->] (A)--(B);
			\draw[->] (A)--(D);
			\draw (A)--(B)--(C)--(D)--cycle;
			\foreach \p/\r in {A/-90, B/ -90, C/90,D/90}
			\fill (\p) circle (1.25pt) node[shift={(\r:3mm)}]{$\p$};
				\draw
			pic [draw,angle radius=3mm] {angle = B--A--D} ;
			\draw (A) node[shift={(60:0.8)}]{$120^\circ$};
			
		\end{tikzpicture}
	\end{center}
	Theo quy tắc hình bình hành ta có $ \vec{F}_1 +\vec{F_2}=\vec{AC}$.\\
	Ta có $ \widehat{BAC}=120^\circ \Rightarrow \widehat{ABC} =60^\circ$.\\
	Áp dụng định lý cosin vào tam giác $ \triangle ACB $ có\\ $ AC^2=AB^2+BC^2-2AB\cdot BC\cdot \cos\widehat{ABC} =4^2+3^2-2\cdot 4\cdot 3\cdot \dfrac{1}{2} =13$.\\
	Do đó độ lớn của lực cần tìm là $ \sqrt{13} $.
}
\end{ex}

\begin{ex}%Câu 8%[0D1Y1-1]%[Dự án đề kiểm tra HKII NH22-23- Nguyễn Tài Tuệ]%[THPT Nguyễn Đức-QuangNam]
	Trong các phát biểu sau, phát biểu nào \textbf{không} phải là mệnh đề?
	\choice
	{$ 13 $ là số nguyên tố}
	{Số $ 15 $ chia hết cho $ 2  $}
	{Tam Kỳ là một thành phố của tỉnh Quảng Nam}
	{\True Bạn có thích học môn Toán không?}
	\loigiai{
		
	Phát biểu~\lq\lq Bạn có thích học môn Toán không?\rq\rq~không phải là mệnh đề.
	}
\end{ex}

\begin{ex}%Câu 9%[0D2Y1-1]%[Dự án đề kiểm tra HKII NH22-23- Nguyễn Tài Tuệ]%[THPT Nguyễn Đức-QuangNam]
	Trong các cặp số sau, cặp nào là nghiệm của bất phương trình $x+y-2>0$
	\choice
	{\True $(1 ; 2)$}
	{$(1 ; 1)$}
	{$(-1 ; 1)$}
	{$(0 ; 0)$}
	\loigiai{
	Thay cặp số $(1 ; 2)$ vào bất phương trình ta thấy thỏa mãn.
}
\end{ex}

\begin{ex}%Câu 10%[0H2Y4-1]%[Dự án đề kiểm tra HKII NH22-23- Nguyễn Tài Tuệ]%[THPT Nguyễn Đức-QuangNam]
	Cho tam giác $A B C$ đều. Khẳng định nào sau đây đúng?
	\choice
	{$(\overrightarrow{A B}, \overrightarrow{A C})=120^{\circ}$}
	{$(\overrightarrow{A B}, \overrightarrow{A C})=90^{\circ}$}
	{\True $(\overrightarrow{A B}, \overrightarrow{A C})=60^{\circ}$}
	{$(\overrightarrow{A B}, \overrightarrow{A C})=45^{\circ}$}
	\loigiai{
	Theo định nghĩa góc giữa hai véc-tơ ta thấy khẳng định ~\lq\lq $(\overrightarrow{A B}, \overrightarrow{A C})=60^{\circ}$\rq\rq~  là đúng.
	}
\end{ex}

\begin{ex}%Câu 11%[0H1B3-2]%[Dự án đề kiểm tra HKII NH22-23- Nguyễn Tài Tuệ]%[THPT Nguyễn Đức-QuangNam]
\immini{	Để rút ngắn khoảng cách và thuận lợi cho việc đi lại, người ta dự kiến xây dựng một đường hầm xuyên núi (từ $ A $ đến $ B $). Để ước tính được chiều dài của hầm, một kĩ sư thực hiện các phép đo đạc (từ vị trí  $ C $ ) và cho ra kết quả như hình vẽ bên dưới. Từ các số liệu đã khảo sát được, chiều dài của đường hầm gần nhất với kết quả nào sau đây?
	\choice
	{$429 $m}
	{$181 $m}
	{$466 $m}
	{\True $417 $m}}{
	\begin{tikzpicture}
		\path 
		(0,0) coordinate (A)
		(4,0) coordinate (B)
		($ (A)+(-40:3) $) coordinate (C)
		(2,3) coordinate (D)
		;
		\fill[gray] (A)--(B) --(D)--cycle;
		\draw (A)--(C) node[pos=0.5,sloped,below]{$ 388 $m}--(B) node[pos=0.5,sloped,below]{$ 212 $m};
		\foreach \p/\r in {A/-90, B/ 90, C/-90 }
		\fill (\p) circle (1.25pt) node[shift={(\r:3mm)}]{$\p$};
		\draw pic[draw,angle radius=3mm,angle eccentricity=1.5] {angle = B--C--A};
		\draw (C) node[shift={(90:0.6)}] {$ 82,4^\circ $};
	\end{tikzpicture}}
\loigiai{
	Chiều dài của hầm bằng độ dài đoạn $ AB $.\\
	Theo định lí cô-sin ta có $ AB^2=AC^2+BC^2-2AC\cdot CB\cos \widehat{ACB} \approx 416$ m.
}
\end{ex}

\begin{ex}%Câu 12%[0H3Y1-2]%[Dự án đề kiểm tra HKII NH22-23- Nguyễn Tài Tuệ]%[THPT Nguyễn Đức-QuangNam]
	Trong mặt phẳng $O x y$, tọa độ của vec tơ $\vec{u}=5 \vec{i}-2 \vec{j}$ là 
	\choice
	{$(2 ; 5)$}
	{\True $(5 ;-2)$}
	{$(-2 ; 5)$}
	{$(5 ; 2)$}
	\loigiai{
		Ta có $ \vec{u}=(5;-2) $.
	}
\end{ex}

\begin{ex}%Câu 13%[0D1B1-3]%[Dự án đề kiểm tra HKII NH22-23- Nguyễn Tài Tuệ]%[THPT Nguyễn Đức-QuangNam]
	Bạn An vừa đậu vào lớp $ 10 $ năm học $ 2022 - 2023 $, ba mẹ bạn thưởng cho bạn một chiếc laptop. Khi mang về, bạn phát hiện trên bao bì có ghi trọng lượng $1,5 \mathrm{~kg}\pm 0,02 \mathrm{~kg}$. Gọi $\bar{a}$ là khối lượng thực của máy tính. Hỏi $\bar{a}$ nằm trong đoạn nào sau đây?
	\choice
	{\True $[1,48 ; 1,52]$}
	{$[1,3 ; 1,7]$}
	{$[1,52 ; 1,54]$}
	{$[1,5 ; 1,52]$}
	\loigiai{
		Khối lượng thực của máy tính $ \overline{a}\in  [1,48 ; 1,52]  $.
	}
\end{ex}

\begin{ex}%Câu 14%[0H1B3-1]%[Dự án đề kiểm tra HKII NH22-23- Nguyễn Tài Tuệ]%[THPT Nguyễn Đức-QuangNam]
	Cho tam giác $A B C$ có $A B=10, A C=20$ và $\widehat{A}=60^{\circ}$. Tính diện tích của tam giác $ ABC $.
	\choice
	{$ 50  $}
	{$100 \sqrt{3}$}
	{\True $50 \sqrt{3}$}
	{$ 100  $}
 
		\loigiai{
			Diện tích của tam giác $ ABC $ là $ S_{ABC}=\dfrac{1}{2}AB\cdot AC\cdot \sin A =50\sqrt{3}$.
		}
 
\end{ex}

\begin{ex}%Câu 15%[0H2B3-1]%[Dự án đề kiểm tra HKII NH22-23- Nguyễn Tài Tuệ]%[THPT Nguyễn Đức-QuangNam]
	Cho 3 điểm $A$, $B$, $C$ như hình vẽ. 
	\begin{center}
		\begin{tikzpicture}
			\path 
			(0,0) coordinate (A)
			(2,0) coordinate (B)
			(4,0) coordinate (D)
			(6,0) coordinate (C)
			;
			\draw (A)--(C);
			\foreach \p/\r in {A/-90, B/-90, C/-90}
			\fill (\p) circle (1.25pt) node[shift={(\r:3mm)}]{$\p$};
			\fill (D) circle (1.25pt);
		\end{tikzpicture}
	\end{center}
\noindent Khẳng định nào sau đây là đúng?
	\choice
	{$\overrightarrow{B C}=-2 \overrightarrow{A B}$}
	{$\overrightarrow{A B}=\dfrac{1}{2}\overrightarrow{A C}$}
	{$\overrightarrow{A B}=-\dfrac{1}{2}\overrightarrow{A C}$}
	{\True $\overrightarrow{B C}=-2 \overrightarrow{B A}$ }
	\loigiai{
	Ta thấy $ \vec{BC} $ và $ \vec{BA} $ ngược  hướng và $ BC=2BA $. Do đó $\overrightarrow{B C}=-2 \overrightarrow{B A}$ .
	}
\end{ex}



\Closesolutionfile{ans}
%\begin{center}
%	\textbf{ĐÁP ÁN}
%	\inputansbox{10}{ans/ans}	
%\end{center}
\begin{center}
	\textbf{PHẦN 2 - TỰ LUẬN}
\end{center}

\begin{bt}%[0D1Y3-2]%[0H1B1-2]%[Dự án đề kiểm tra HKI NH22-23- NguyenHuynh]%[THPT Nguyễn Dực Quảng Nam]
	\begin{enumerate}[a)]
		\item Cho hai tập hợp $A=\{2;3;4;5\} ; B=\{4;5;6\}$. Hãy xác định các tập hợp sau $A \cap B ; A \setminus B$.
		\item Cho $\cos \alpha=\dfrac{-4}{5}\left(90^{\circ}<\alpha<180^{\circ}\right)$. Tính các giá trị lượng giác $\sin \alpha, \tan \alpha$.
	\end{enumerate}
	\loigiai{
		\begin{enumerate}[a)]
			\item Ta có $A \cap B =\{4;5\}$ và $A \setminus B=\{2;3\}$.
			\item Ta có $\sin^2\alpha +\cos^2\alpha =1$\\
			$\begin{array}{l}
				\Rightarrow \sin^2\alpha +\left(\dfrac{-4}{5}\right)^2=1\\
				\Rightarrow \sin^2\alpha=\dfrac{9}{25}\\
				\Rightarrow \hoac{&\sin\alpha=\dfrac{3}{5} \\& \sin\alpha=\dfrac{-3}{5}.}
			\end{array}$\\
			Vì $90^{\circ}<\alpha<180^{\circ}$ nên $\sin\alpha=\dfrac{3}{5}$. Khi đó $\tan\alpha=\dfrac{\sin\alpha}{\cos\alpha}=\dfrac{-3}{4}$.
		\end{enumerate}
	}
\end{bt}
\begin{bt}%[0H2B2-1]%[0H3B1-3]%[Dự án đề kiểm tra HKI NH22-23- NguyenHuynh]%[THPT Nguyễn Dực Quảng Nam]
	\begin{enumerate}[a)]
		\item Cho hình bình hành ${ABCD}$. Chứng minh rằng với mọi điểm ${E}$ ta luôn có $\overrightarrow{E A}+\overrightarrow{E C}=\overrightarrow{E B}+\overrightarrow{E D}$.
		\item Trong mặt phẳng ${Oxy}$, cho ba điểm ${A}(-1 ; 3), {B}(4 ; 2), {C}(3 ; 5)$. Tìm tọa độ điểm ${D}$ sao cho $\overrightarrow{A D}=-3 \overrightarrow{B C}$.
	\end{enumerate}
	\loigiai{
		\begin{enumerate}[a)]
			\item Ta có
			\[\begin{array}{ll}
				\overrightarrow{E A}+\overrightarrow{E C}&=\overrightarrow{EB}+\overrightarrow{BA}+\overrightarrow{ED}+\overrightarrow{DC}\\
				&=\overrightarrow{EB}+\overrightarrow{AB}+\overrightarrow{ED}+\overrightarrow{DC}\\
				&=\overrightarrow{EB}+\overrightarrow{ED}+\overrightarrow{BA}+\overrightarrow{DC}\\
				&=\overrightarrow{EB}+\overrightarrow{ED}.
			\end{array}\]
			Vậy $\overrightarrow{E A}+\overrightarrow{E C}=\overrightarrow{E B}+\overrightarrow{E D}$.
			\item Gọi $D\left(x_D;y_D\right)$ thỏa đề bài.\\
			Khi đó $\overrightarrow{AD}=\left(x_D+1;y_D-3\right)$ và $\overrightarrow{BC}=(-1;3)$.
			\[\begin{array}{ll}
				&\overrightarrow{A D}=-3 \overrightarrow{B C}\\
				\Leftrightarrow &\heva{&x_D+1=-3\cdot (-1)\\&y_D-3=-3\cdot 3}\\
				\Leftrightarrow &\heva{&x_D=2\\&y_D=-6.}
			\end{array}\]
			Vậy $D(2;-6)$.
			
		\end{enumerate}
	}
\end{bt}
\begin{bt}%[0H2K2-6]%[Dự án đề kiểm tra HKI NH22-23- NguyenHuynh]%[THPT Nguyễn Dực Quảng Nam]
	Cho hai lực $\vec{F}_1=\overrightarrow{M A}, \vec{F}_2=\overrightarrow{M B}$ cùng tác động vào một vật tại điểm ${M}$, cho biết cường độ của lực $\overrightarrow{F}_1, \overrightarrow{F}_2$ lần lượt bằng $100 {~N}$ và $100 \sqrt{2} {~N}$ góc $\widehat{A M B}=45^{\circ}$. Tính công sinh ra khi tổng hợp lực $\vec{F}$ tác dụng lên và kéo vật di chuyển theo phương ngang góc $60^{\circ}$ một đoạn ${d}=10 \sqrt{5}$m.
	\loigiai{Dựng hình bình hành $MACB$, khi đó tổng hợp lực  $\overrightarrow{F}=\overrightarrow{F_1}+\overrightarrow{F_2}=\overrightarrow{MA}+\overrightarrow{MB}=\overrightarrow{MC}$.\\
		Áp dụng định lí côsin, ta có
		\[\begin{array}{ll}
			MC^2&=MA^2+MB^2-2\cdot MA\cdot MB\cos \widehat{MAC}\\
			&=100^2+\left(100\sqrt{2}\right)^2-2\cdot 100\cdot 100\sqrt{2}\cos \widehat{135^{\circ}}\\
			&=MA^2+MB^2-2\cdot MA\cdot MB\cos \widehat{MAC}\\
			&=500\\
		\end{array}\]
		Suy ra $\left|\overrightarrow{F}\right|=MC=100\sqrt{5}$.\\
		Công sinh ra bởi $\overrightarrow{F}$
		\[A=\left|\overrightarrow{F}\right|\cdot\left|\overrightarrow{d}\right|\cdot\cos 60^{\circ}=2500 \;(J)\]
	}
\end{bt}
\begin{bt}%[0H2K3-4]%[Dự án đề kiểm tra HKI NH22-23- NguyenHuynh]%[THPT Nguyễn Dực Quảng Nam]
	Cho tam giác ${ABC}$. Gọi ${M}$, ${N}$ là hai điểm xác định bởi các hệ thức $3 \overrightarrow{MA}+4 \overrightarrow{M B}=\vec{0}$; $\overrightarrow{B C}=2\overrightarrow{CN}$, điểm ${G}$ là trọng tâm tam giác ${ABC}$. Chứng minh ${M}$, ${N}$, ${G}$ thẳng hàng.
	\loigiai{
		Ta có $3 \overrightarrow{MA}+4 \overrightarrow{M B}=\vec{0}\Leftrightarrow-3\overrightarrow{AM}+4\left( \overrightarrow{AB}-\overrightarrow{AM}\right) =\vec{0} \Leftrightarrow\overrightarrow{AM}=\dfrac{4}{7}\overrightarrow{AB}$.\\
		Ta có $\overrightarrow{B C}=2\overrightarrow{CN}\Leftrightarrow\overrightarrow{AC}-\overrightarrow{AB}=2\left(\overrightarrow{AN}-\overrightarrow{AC} \right) \Leftrightarrow\overrightarrow{AN}=\dfrac{3}{2}\overrightarrow{AC}-\dfrac{1}{2}\overrightarrow{AB}$.\\
		Khi đó
		\[\begin{array}{ll}
			\overrightarrow{MG}&=\overrightarrow{AG}-\overrightarrow{AM}\\
			&=\dfrac{1}{3}\left(\overrightarrow{AB}+\overrightarrow{AC}\right)-\dfrac{4}{7}\overrightarrow{AB}\\
			&=\dfrac{-5}{21}\overrightarrow{AB}+\dfrac{1}{3}\overrightarrow{AC}\\
		\end{array}\]
		và
		\[\begin{array}{ll}
			\overrightarrow{NG}&=\overrightarrow{AG}-\overrightarrow{AN}\\
			&=\dfrac{1}{3}\left(\overrightarrow{AB}+\overrightarrow{AC}\right)-\left(\dfrac{3}{2}\overrightarrow{AC}-\dfrac{1}{2}\overrightarrow{AB}\right)\\
			&=\dfrac{5}{6}\overrightarrow{AB}-\dfrac{7}{6}\overrightarrow{AC}\\
			&=\dfrac{-7}{2}\left(\dfrac{-5}{21}\overrightarrow{AB}+\dfrac{1}{3}\overrightarrow{AC}\right)\\
			&=\dfrac{-7}{2}\overrightarrow{MG}.
		\end{array}\]
		Vậy 3 điểm $M$, $N$, $G$ thẳng hàng.
	}
\end{bt}