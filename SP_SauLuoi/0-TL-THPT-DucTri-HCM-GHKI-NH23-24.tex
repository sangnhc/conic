

\de{ĐỀ THI GIỮA HỌC KỲ I NĂM HỌC 2023-2024}{THPT Đức Trí Tp Hồ Chí Minh\\Toán 10}
\begin{center}
	\textbf{ĐỀ A}
\end{center}

\begin{bt}%[Câu 1]%[0H1Y1-2]
	(1 điểm) Tính giá trị biểu thức $$3\sin\left(90^\circ-\alpha\right)+2\cos\left(180^\circ-\alpha\right)-\cos\alpha\, , \left(0^\circ<\alpha<90^\circ\right).$$
	\loigiai{
	Ta có $\begin{aligned}[t]
		\sin\left(90^\circ-\alpha\right) &= \cos\alpha\\
		\cos\left(180^\circ-\alpha\right) &= -\cos\alpha.\\
	\end{aligned}$\\
	Nên $3\sin\left(90^\circ-\alpha\right)+2\cos\left(180^\circ-\alpha\right)-\cos\alpha = 3\cos\alpha-2\cos\alpha-\cos\alpha=0.$
	}
\end{bt}

\begin{bt}%[Câu 2]%[0D1K3-3]
	(1 điểm) Lớp 10A có 42 học sinh, trong đó 16 học sinh tham gia cuộc thi vẽ đồ họa trên máy tính, 22 học sinh tham gia tin học văn phòng cấp trường và 7 học sinh không tham gia cả hai cuộc thi này. Hỏi có bao nhiêu học sinh lớp 10A tham gia đồng thời cả hai cuộc thi?
	\loigiai{
		Công thức $\left| A\cup B \right|= \left| A \right|  + \left| B \right| - \left| A\cap B \right|$.\\
		Gọi $A$ là tập hợp số học sinh tham gia thi vẽ đồ họa, có $|A|=16$.\\
		Gọi $B$ là tập hợp số học sinh tham gia thi tin học văn phòng, có $|B|=22$.\\
		Số học sinh tham gia một trong hai cuộc thi $|A|+|B|=22+16=38$.\\
		Số học sinh tham gia cuộc thi $|A\cup B|=42-7=35$.\\
		Số học sinh tham gia đồng thời hai cuộc thi $|A\cap B|=38-35=3$.
	}
\end{bt}

\begin{bt}%[Câu 3]%[0D1B3-4]
	(2 điểm) Cho các tập hợp $A=\left(-3;8\right)$, $B=\left(-\infty;4\right]$, $E=\left\{x\in\mathbb{R} \,\middle|\, x+7>0{,}2 \right\}$.\\
	Xác định các tập hợp: $A\cap B$; $A\setminus B$; $\left(A\cup B\right)\setminus\left(B\cap E\right)$; $\left(A\cup E\right)\setminus\left(C_{\mathbb{R}}B\right)$.
	\loigiai{
		Ta có 
		$\begin{aligned}[t]
			& A\cap B=\left( -3;4 \right] \\
			& A\setminus B=\left(4;8\right)\\
			& A\cup B=\left(-\infty;8\right)\\
			& B\cap E=\left(-6{,}8;4\right]\\
			& A\cup E=\left(6{,}8;+\infty\right)\\
			& C_{\mathbb{R}}B=\left(4;+\infty\right)\\
			& \left(A\cup B\right)\setminus\left(B\cap E\right)= \left(-\infty;8\right)\setminus\left(-6{,}8;4\right]= \left(-\infty;-6{,}8\right]\cup\left(4;8\right)\\
			& \left(A\cup E\right)\setminus C_{\mathbb{R}}B= \left(-6{,}8;+\infty\right)\setminus\left(4;+\infty\right)= \left(-6{,}8;4\right].
		\end{aligned}$\\
		}
\end{bt}

\begin{bt}%[Câu 4]%[0D1B2-1]
	(1 điểm) Viết các tập hợp sau dưới dạng liệt kê các phần tử:
	$$A=\left\{x\in\mathbb{R} \,\big|\, x^2-6x+5=0\right\}\ , \quad B=\left\{x\in\mathbb{Z} \,\big|\, -2\le x<5\right\}.$$
	\loigiai{
		Tập hợp $A=\left\{x\in\mathbb{R} \,\big|\, x^2-6x+5=0\right\}= \{1;5\}$.\\
		Tập hợp $B=\left\{x\in\mathbb{Z} \,\big|\, -2\le x<5\right\}=\{-2;-1;0;1;2;3;4\}$.\\
	}
\end{bt}

\begin{bt}%[Câu 5]%[0D1B1-5]
	(1 điểm) Cho mệnh đề: $P(x)\colon \text{\lq\lq}\exists x\in\mathbb{N} \,\big|\, 2x-4>0\text{\rq\rq}$.
	\begin{enumerate}
		\item Xét tính đúng sai của mệnh đề trên và giải thích.
		\item Lập mệnh đề phủ định của $P(x)$.
	\end{enumerate}
	\loigiai{
		\begin{enumerate}
			\item Mệnh đề $P(x)\colon \text{\lq\lq}\exists x\in\mathbb{N} \,\big|\, 2x-4>0\text{\rq\rq}$, tồn tại $n=3$ kiểm tra mệnh đề đúng.
			\item Mệnh đề phủ định $\overline{P(x)}\colon \text{\lq\lq}\forall x\in\mathbb{N} \,\big|\, 2x-4\le0\text{\rq\rq}$.
		\end{enumerate}
	}
\end{bt}

\begin{bt}%[Câu 6]%[0H2Y1-3]
	(1 điểm) Cho tam giác $ABC$ đều cạnh bằng $3$. Gọi $H$ là trung điểm của $BC$. Tìm vectơ bằng vectơ $\overrightarrow{BH}$ và tính độ dài của vectơ $\overrightarrow{AH}$.
	\loigiai{
		\immini{
			\begin{itemize}
				\item Vectơ bằng $\overrightarrow{BH}$ là vectơ $\overrightarrow{HC}$.
				\item Xét $\triangle HAC$ vuông tại $H$, có góc $\widehat{C}=60^{\circ}$.\\
				Ta có $\sin C=\dfrac{AH}{AC}  \Rightarrow $
				$\begin{aligned}[t]
					AH &= AC\cdot\sin C\\
					& = 3\cdot\sin60^{\circ}\\
					& = \dfrac{3\sqrt{3}}{2}.
				\end{aligned}$\\
			Vậy $\left|\overrightarrow{AH}\right|=AH=\dfrac{3\sqrt{3}}{2}$.
			\end{itemize}
		}{
		\begin{tikzpicture}[blue,thick,font=\footnotesize, line cap=round, line join=round]
			\def\a{3}
			\path 	
			(0:0) coordinate (A)
			(0:\a) coordinate (B)
			(60:\a) coordinate (C)
			($(B)!1/2!(C)$) coordinate (H)
			;
			\draw (B)--(A)--(C) (A)--(H);
			\pic[draw,angle radius=2mm,angle eccentricity=1.5] {right angle=A--H--C};
			\pic[draw,angle radius=4mm,angle eccentricity=1.5,"$60^\circ$"] { angle=A--C--H};
			\path 
			(A)--(B) node[pos=.5,below]{$a=3$}
			(A)--(C) node[pos=.5,above,sloped]{$a=3$}
			(A)--(H) node[pos=.6,above,sloped]{$h=?$}
			(B)--(H) node[pos=.6,sloped]{$||$}
			(C)--(H) node[pos=.6,sloped]{$||$}
			;
			\draw[-latex,red] (B)--(H); \draw[-latex,orange] (H)--(C);
			\foreach \x/\g in {A/210,B/-30,C/90,H/60}
				\fill[black] (\x) circle (1pt) ($(\g:3mm)+(\x)$) node {$\x$};
		\end{tikzpicture}
		}
		}
\end{bt}

\begin{bt}%[Câu 7]%[0H1K3-2]
	\immini[thm]{
	(1 điểm) Từ hai vị trí $A$ và $B$ của một tòa nhà, người ta quan sát đỉnh $C$ của ngọn núi. Biết rằng độ cao $AB=70\rm\,m$, phương nhìn $AC$ tạo với phương nằm ngang một góc $30^\circ$. Phương nhìn $BC$ tạo với phương nằm ngang một góc $15^\circ30'$. Khi đó chiều cao của ngọn núi so với mặt đất là bao nhiêu? (\textit{làm tròn đến hàng đơn vị})
	}{
	\begin{tikzpicture}[x=2.5mm,y=2.5mm,blue,thick, font=\footnotesize, line cap=round, line join=round]
		\def\a{1}
		\path 	
		(0:0) coordinate (A)
		($(A)+(0:1)$) coordinate (A1)
		($(A)+(30:1)$) coordinate (A2)
		(90:7*\a) coordinate (B)
		($(B)+(0:1)$) coordinate (B1)
		($(B)+(31/2:1)$) coordinate (B2)
		(180:3*\a) coordinate (A')
		($(B)+(A')-(A)$) coordinate (B')
		(intersection of A--A2 and B--B2) coordinate (C)
		(intersection of A--A2 and B--B1) coordinate (I)
		($(A)!(C)!(A1)$) coordinate (H)
		($(B)!(C)!(B1)$) coordinate (K)
		($(H)!1/3!(A)$) coordinate (Ht)
		($(H)!-1/3!(A)$) coordinate (Hp)
		($(C)!2/3!(Ht)$) coordinate (m)
		($(C)!2/3!(Hp)$) coordinate (n)
		($(C)!1/4!(Ht)$) coordinate (p)
		($(C)!1/4!(Hp)$) coordinate (q)
		($(C)+(Hp)-(H)$) coordinate (Cp)
		;
		\draw (A)--(B) node[pos=.575,right]{70 m}--(B')--(A')--cycle (C)--(H)--(A)--(C)--(B)--(K) ;
		\draw[teal,thin] 
		($(A)!1/4!(B)$)--($(A')!1/4!(B')$)
		($(A)!2/4!(B)$)--($(A')!2/4!(B')$)
		($(A)!3/4!(B)$)--($(A')!3/4!(B')$)
		($(A)!1/2!(A')$)--($(B)!1/2!(B')$)
		;
		\pic[draw,angle radius=6mm,angle eccentricity=1.75,"$30^\circ$"] {angle=H--A--C};
		\pic[draw,angle radius=12mm,angle eccentricity=1.65,"$15^\circ30'$"] {angle=K--B--C};
		\foreach \x/\g in {A/-90,B/90,C/90,I/-90}
		\fill[black] (\x) circle (1pt) ($(\g:3mm)+(\x)$) node {$\x$};
		
		\draw[clip]
		decorate [decoration={random steps,segment length=9pt,amplitude=3pt}]%
		{(Ht) -- ($(m)+(180:2mm)$) -- (C) -- ($(n)+(0:3mm)$) -- (Hp)}
		-- ($(Hp)+(-90:3mm)$) -- ($(Ht)+(-90:3mm)$) -- (Ht) ;
		\fill[brown,opacity=.85]($(Ht)+(-90:3mm)$) rectangle (Cp);
		\fill[green,opacity=.85]($(Ht)+(-90:3mm)$) rectangle (Hp);
		\fill[white,opacity=.85]
		decorate [decoration={random steps,segment length=3pt,amplitude=1pt}]%
		{(C) -- ($(p)+(0:.09mm)$) -- ($(p)+(-80:13mm)$) -- ($(p)+(-10:3mm)$) -- ($(p)+(-50:10mm)$) -- ($(p)+(15:6mm)$) -- ($(q)+(180:.25pt)$)} -- cycle ;
		\path (K) node[above right]{$K$} (H) node[above right]{$H$};
		\fill[black]  (K) circle(1pt) (H) circle(1pt) (C) circle(1pt) ;
		\draw (C)--(H);
	\end{tikzpicture}
	}
	\loigiai{
		Ta có 
		$\begin{aligned}[t]
			& \widehat{ABC}=\widehat{ABK}+\widehat{KBC}=90^{\circ}+15^{\circ}30'=105^{\circ}30'.\\
			& \widehat{BAC}=\widehat{BAH}-\widehat{CAH}=90^{\circ}-30^{\circ}=60^{\circ}.\\
			 \Rightarrow & \widehat{ACB}=180^{\circ}-\widehat{BAC}-\widehat{ABC}=14^{\circ}30'.
		\end{aligned}$\\
	Áp dụng định lí $\sin$ cho $\triangle ABC$, ta có \\
	$\dfrac{AB}{\sin\widehat{ACB}}=\dfrac{AC}{\sin\widehat{ABC}}$  
	$\Rightarrow AC=\dfrac{AB\cdot\sin\widehat{ABC}}{\sin\widehat{ACB}}= \dfrac{70\cdot\sin105^{\circ}30'}{\sin14^{\circ}30'}\approx269$ (m).\\
	Xét $\triangle ACH$ vuông tại $H$, ta có $CH=AC\cdot\sin\widehat{CAH}=\dfrac{70\cdot\sin105^{\circ}30'}{\sin14^{\circ}30'}\cdot\sin30^{\circ}\approx 135$ (m).\\
	}
\end{bt}

\begin{bt}%[Câu 8]%[0H1B3-1]
	(2 điểm) Cho $\triangle ABC$ có $AB=16$, $AC=18$ và $BC=20$.
	\begin{enumerate}
		\item Tính số đo góc $A$.
		\item Tính diện tích $\triangle ABC$ và bán kính $R$ của đường tròn ngoại tiếp $\triangle ABC$.
	\end{enumerate}
	\loigiai{
	\immini{
		\begin{enumerate}
			\item Áp dụng định lí cosin cho $\triangle ABC$, ta có 
			\begin{eqnarray*}
				\cos{A} &=& \dfrac{AB^2+AC^2-BC^2}{2\cdot AB\cdot AC}\\
				&=& \dfrac{16^2+18^2-20^2}{2\cdot16\cdot18}=\dfrac{5}{16}.\\
				 \Rightarrow \widehat{A} &=& 71^{\circ}47'24".
			\end{eqnarray*}
			\item Áp dụng công thức Heron cho $\triangle ABC$, ta có nửa chu vi $p=\dfrac{1}{2}\left(16+18+20\right)=27$
			\begin{eqnarray*}
				S=\sqrt{27(27-16)(27-18)(27-20)}=9\sqrt{231} \text{ (đvdt)}.
			\end{eqnarray*}
			Bán kính đường tròn ngoại tiếp $\triangle ABC$, ta có
			\begin{eqnarray*}
				S=\dfrac{abc}{4R}  \Rightarrow R=\dfrac{abc}{4S}=\dfrac{16\cdot18\cdot20}{4\cdot9\sqrt{231}}=\dfrac{160}{\sqrt{231}}\approx 10{,}52 \text{ (đvđd)}.
			\end{eqnarray*}
		\end{enumerate}
	}{
	\begin{tikzpicture}[blue,thick,font=\footnotesize, line cap=round, line join=round]
		\def\a{2}
		\path 	
		(0:0) coordinate (A)
		(0:1.6*\a) coordinate (B)
		(71.79:1.8*\a) coordinate (C)
		;
		\draw (B)--(A)--(C)--cycle ;
		\pic[draw,angle radius=4mm,angle eccentricity=1.5,"$?$"] { angle=B--A--C};
		\path 
		(A)--(B) node[pos=.5,below]{$16$}
		(A)--(C) node[pos=.5,above,sloped]{$18$}
		(B)--(C) node[pos=.5,above,sloped]{$20$}
		;
		\foreach \x/\g in {A/210,B/-30,C/90}
		\fill[black] (\x) circle (1pt) ($(\g:3mm)+(\x)$) node {$\x$};
	\end{tikzpicture}
	}	
	}
\end{bt}