
\de{ĐỀ THI GIỮA HỌC KỲ I NĂM HỌC 2023-2024}{THTH ĐHSP TP HCM }
\begin{center}
	\textbf{PHẦN 1 - TRẮC NGHIỆM}
\end{center}
\Opensolutionfile{ans}[ans/ans]
\begin{ex}%[0D1V2-2]%[Dự án đề kiểm tra Toán 10 GHKI NH23-24- Nguyen Tan Linh ]%[THSP ĐHSP - Tp HCM]
	Cho tập hợp $A=\left\{x;y;z\right\}$ và $B=\left\{x;y;z;t;u\right\}$. Có bao nhiêu tập $X$ thỏa mãn $A\subset X\subset B$?
	\choice
	{$8$}
	{$2$}
	{\True $4$}
	{$16$}
	\loigiai
	{
		Tập hợp $X$ thỏa mãn $A\subset X\subset B$ là $\{x;y;z\}$, $\{x;y;z;t\}$, $\{x;y;z;u\}$, $\{x;y;z;t;u\}$.
	}
\end{ex}
\begin{ex}%[0D1N3-2]%[Dự án đề kiểm tra Toán 10 GHKI NH23-24- Nguyen Tan Linh ]%[THSP ĐHSP - Tp HCM]
	Cho tập $A=\left\{0;2;4;6;8\right\}$; $B=\left\{3;4;5;6;7\right\}$. Tập $A\setminus B$ là
	\choice
	{\True $\left\{0;6;8\right\}$}
	{$\left\{3;6;7\right\}$}
	{$\left\{0;2\right\}$}
	{$\left\{0;2;8\right\}$}
	\loigiai
	{
		Ta có $A\setminus B=\left\{0;6;8\right\}$.
	}
\end{ex}
\begin{ex}%[0D2N2-1]%[Dự án đề kiểm tra Toán 10 GHKI NH23-24- Nguyen Tan Linh ]%[THSP ĐHSP - Tp HCM]
	Cho hệ bất phương trình $\heva{&x+y-2\le 0 \\&2x-3y+2>0}$. Trong các điểm sau, điểm nào không thuộc miền nghiệm của hệ bất phương trình đã cho?
	\choice
	{$O\left(0;0\right)$}
	{$M\left(1;1\right)$}
	{$P\left(-1;-1\right)$}
	{\True $N\left(-1;1\right)$}
	\loigiai
	{
		Xét điểm $N(-1;1)$, ta có $2x-3y+2=2\cdot(-1)-3\cdot 1+2=-3<0$. Do đó điểm $N$ không thuộc miền nghiệm của hệ bất phương trình đã cho.
	}
\end{ex}
\begin{ex}%[0H4N2-2]%[Dự án đề kiểm tra Toán 10 GHKI NH23-24- Nguyen Tan Linh ]%[THSP ĐHSP - Tp HCM]
	Cho $\Delta ABC$ với các cạnh $AB=c,AC=b,BC=a$. Gọi $R,r,S$ lần lượt là bán kính đường tròn ngoại tiếp, nội tiếp và diện tích của tam giác $ABC$. Trong các phát biểu sau, phát biểu nào \textbf{sai}?
	\choice
	{$a^2+b^2-c^2=2ab\cos C$}
	{$S=\dfrac{abc}{4R}$}
	{$S=\dfrac{1}{2}ab\sin C$}
	{\True $R=\dfrac{a}{\sin A}$}
	\loigiai
	{
		Ta có $\dfrac{a}{\sin A}=2R\Rightarrow R=\dfrac{a}{2\sin A}$.
	}
\end{ex}
\begin{ex}%[0D1N1-3]%[Dự án đề kiểm tra Toán 10 GHKI NH23-24- Nguyen Tan Linh ]%[THSP ĐHSP - Tp HCM]
	Cho mệnh đề \lq\lq $\forall x\in \mathbb{R},x^2+3x+5>0$\rq\rq. Mệnh đề phủ định của mệnh đề trên là
	\choice
	{$\forall x\in \mathbb{R},x^2+3x+5<0$}
	{\True $\exists x\in \mathbb{R},x^2+3x+5\le 0$}
	{$\forall x\in \mathbb{R},x^2+3x+5\le 0$}
	{$\exists x\in \mathbb{R},x^2+3x+5>0$}
	\loigiai
	{
		Mệnh đề phủ định của mệnh đề trên là $\exists x\in \mathbb{R},x^2+3x+5\le 0$.
	}
\end{ex}
\begin{ex}%[0H4N2-1]%[Dự án đề kiểm tra Toán 10 GHKI NH23-24- Nguyen Tan Linh ]%[THSP ĐHSP - Tp HCM]
	Cho tam giác $ABC$, mệnh đề nào sau đây đúng?
	\choice
	{\True $a^2=b^2+c^2-2bc\cos A$}
	{$a^2=b^2+c^2-2bc\cos B$}
	{$a^2=b^2+c^2-2bc\cos C$}
	{$a^2=b^2+c^2+2bc\cos A$}
	\loigiai
	{
		
	}
\end{ex}
\begin{ex}%[0D2H1-2]%[Dự án đề kiểm tra Toán 10 GHKI NH23-24- Nguyen Tan Linh ]%[THSP ĐHSP - Tp HCM]
	\immini
	{
		Biết rằng hình vẽ dưới đây biểu diễn miền nghiệm của một trong các bất phương trình bên dưới - đó là nửa mặt phẳng kể cả bờ là đường thẳng và không bị gạch chéo trong hình. Đó là bất phương trình nào?
		\choice
		{\True $3x+2y+6\ge 0$}
		{$3x+2y+6\le 0$}
		{$3x-2y+6\le 0$}
		{$2x+y+6\ge 0$}
	}
	{
		\begin{tikzpicture}[line join = round, line cap = round,>=stealth,font=\footnotesize,scale=.9]
			\begin{scope}
				\clip (-3,-4) rectangle (2,1.5);
				\fill[pattern=north east lines] (-4,3)--(-4,-7.5)--(3,-7.5)--cycle;
				\draw (-3,1.5)--(0.67,-4);
			\end{scope}
			\draw[->] (-3,0)--(2,0) node[below]{$x$};
			\draw[->] (0,-4)--(0,1.5) node[left]{$y$};
			\fill (-2,0) circle (1.5pt) node[above]{$-2$};
			\fill (0,-3) circle (1.5pt) node[right]{$-3$};
			\fill (0,0) circle (1.5pt) node[below left]{$O$};
		\end{tikzpicture}
	}
	\loigiai
	{
		Gọi $d$ là đường thẳng đi qua hai điểm $(-2;0)$ và $(0;-3)$. Suy ra phương trình đường thẳng $d\colon 3x+2y+6=0$.\\
		Ta thấy điểm $O(0;0)$ thuộc miền nghiệm. Thay $x=0$, $y=0$, ta có $3\cdot 0+2\cdot 0+6>0$.\\
		Suy ra bất phương trình cần tìm là $3x+2y+6\ge 0$.
	}
\end{ex}
\begin{ex}%[0D1N2-2]%[Dự án đề kiểm tra Toán 10 GHKI NH23-24- Nguyen Tan Linh ]%[THSP ĐHSP - Tp HCM]
	Tập hợp nào sau đây có đúng hai tập hợp con?
	\choice
	{$\left\{x;y\right\}$}
	{$\left\{x;0\right\}$}
	{\True $\left\{x\right\}$}
	{$\left\{x;y;0\right\}$}
	\loigiai
	{
		Tập hợp $\left\{x\right\}$ có hai tập hợp con là $\varnothing$ và $\left\{x\right\}$.
	}
\end{ex}
\begin{ex}%[0D1H3-5]%[Dự án đề kiểm tra Toán 10 GHKI NH23-24- Nguyen Tan Linh ]%[THSP ĐHSP - Tp HCM]
	Lớp 10A có $45$ học sinh, trong đó có $15$ bạn biết bơi lội, $20$ bạn biết chơi bóng rỗ, $10$ bạn vừa biết bơi lội vừa biết chơi bóng rỗ. Hỏi có bao nhiêu học sinh của lớp 10A biết ít nhất một môn thể thao là bơi lội hoặc chơi bóng rỗ?
	\choice
	{\True $25$}
	{$45$}
	{$10$}
	{$35$}
	\loigiai
	{
		Gọi $A$ là tập hợp học sinh biết bơi lội, $B$ là tập hợp học sinh biết chơi bóng rổ.\\
		Số học sinh biết ít nhất một môn thể thao là bơi lội hoặc chơi bóng rỗ là
		\[n(A\cup B)=n(A)+n(B)-n(A\cap B)=15+20-10-25\text{ (học sinh)}. \]
	}
\end{ex}
\begin{ex}%[0D1N1-1]%[Dự án đề kiểm tra Toán 10 GHKI NH23-24- Nguyen Tan Linh ]%[THSP ĐHSP - Tp HCM]
	Phát biểu nào sau đây là một mệnh đề?
	\choice
	{Đề thi môn Toán dễ quá!}
	{Mùa thu Hà Nội đẹp quá!}
	{\True Hà Nội là thủ đô của Việt Nam}
	{Bạn có đi học không?}
	\loigiai{}
\end{ex}

\Closesolutionfile{ans}
%\begin{center}
%	\textbf{ĐÁP ÁN}
%	\inputansbox{10}{ans/ans}	
%\end{center}
\begin{center}
	\textbf{PHẦN 2 - TỰ LUẬN}
\end{center}
\begin{bt}%[0D1N1-3]%[Dự án đề kiểm tra Toán 10 GHKI NH23-24- Van Nguyen ]%[THSP ĐHSP - Tp HCM]
	Cho mệnh đề sau:  $A$: \lq\lq$ \exists x \in \mathbb{Q}: 2 x^2-5 x+2=0 $\rq\rq. Xét tính đúng sai (có giải thích) của mệnh đề $A$ và tìm mệnh đề phủ định của $A$.
	\loigiai{
		* Ta có: $2x^2-5x+2=0\Leftrightarrow \hoac{&x=2\\&x=\dfrac{1}{2}.}$\\
		Vậy mệnh đề $A$ đúng.\\
		* Mệnh đề phủ định:  $\bar {A}$: \lq\lq $\forall x \in \mathbb{Q}: 2 x^2-5 x+2\neq 0 $\rq\rq}
\end{bt}
\begin{bt}%[0D1N3-3]%[Dự án đề kiểm tra Toán 10 GHKI NH23-24- Van Nguyen ]%[THSP ĐHSP - Tp HCM]
	Xáċ định các tập hợp sau đây:
	\begin{enumerate}
		\item $(1 ; 3) \cup[-2 ; 2]$.
		\item $\left (-\infty ; \sqrt{2}\right ] \cap\left [1 ;+\infty\right )$.
	\end{enumerate}
	\loigiai{
		\begin{enumerate}
			\item $(1 ; 3) \cup[-2 ; 2]=[-2;3)$.
			\item $\left (-\infty ; \sqrt{2}\right ] \cap\left [1 ;+\infty \right )=\left[1;\sqrt 2\right]$.
		\end{enumerate}
	}
\end{bt}

\begin{bt}%[0D2H2-2]%[Dự án đề kiểm tra Toán 10 GHKI NH23-24- Van Nguyen ]%[THSP ĐHSP - Tp HCM]
	Biểu diễn miền nghiệm của hệ bất phương trình sau: $\left\{\begin{array}{l}2 x+3 y \leq 18 \\ x \geq 0 \\ y \geq 0\end{array}\right.$.
	\loigiai{
		* Vẽ đường thẳng $d_1\colon  2x+3y=18$ đi qua điểm $A(0;6),B(-3;8)$\\
		* Biểu diễn miền nghiệm:\\
		\begin{center}
			\begin{tikzpicture}[scale=0.5,line join=round, line cap=round,>=stealth,thick]
				\tikzset{every node/.style={scale=0.7}}
				\begin{scope}
					\clip (-4,-4) rectangle (9.5,8.5);
					\fill[pattern=horizontal lines gray] (-5,9.33)--(15.5,9.33)--(15.5,-4.33)--cycle;
					\fill[pattern=horizontal lines gray] (0,-4)--(-4,-4)--(-4,8.5)--(0,8.5)--cycle;
					\fill[pattern=horizontal lines gray] (-4,0)--(-4,-4)--(6.5,-4)--(9.5,0)--cycle;
					\draw (-3.75,8.5)--(15,-4) node [pos=0.45, above, sloped] {$2x+3y-18=0$};
				\end{scope}
				\draw[->] (-4,0)--(9.5,0) node[below]{$x$};
				\draw[->] (0,-4)--(0,8.5) node[left]{$y$};
				\draw (0,0) node[below left]{$O$};
				\foreach \x in {-3}
				\draw[thin] (\x,1pt)--(\x,-1pt) node [below] {$\x$};
				\foreach \y in {6,8}
				\draw[thin] (1pt,\y)--(-1pt,\y) node [left] {$\y$};
				\draw[dashed,thin] (-3,0)--(-3,8)--(0,8);
			\end{tikzpicture}
		\end{center}
		Miền nghiệm là phần không bị gạch bỏ trên hình.}
\end{bt}
%Câu 4...........................
\begin{bt}%[0H4V2-1]%[Dự án đề kiểm tra Toán 10 GHKI NH23-24- VU Ngoc Hao]%[THSP ĐHSP - Tp HCM]
Cho tam giác $ABC$ có $\widehat{BCA}=38^\circ$, $\widehat{ABC}=64^\circ$, $BC=22$.
\begin{enumerate}[a)]
\item Tính cạnh $AB$ và $AC$.
\item Tính diện tích tam giác $ABC$ và bán kính đường tròn ngoại tiếp tam giác $ABC$.
$$\text{\emph{(Kết quả hàm tròn đến hàng phần trăm)}.} $$
\end{enumerate}
\loigiai{
	
\begin{center}
	\begin{tikzpicture}[scale=1,>=stealth, font=\footnotesize, line join=round, line cap=round]
	%	\draw[color=gray!50,dashed] (0,0) grid (4,3);
	\path
	(0,0) coordinate (B)
	(1,3) coordinate (A)
	(5,0) coordinate (C);
	\draw(A)--(B)--(C) node [pos=.5,sloped,below]{$22$}--cycle;
	\foreach \i/\j in {A/140,B/180,C/0} 
	\draw[fill] (\i) circle(1pt) ($(\i) + (\j:2mm)$)node{$\i$};
	\draw
	pic[draw,"$64^\circ$",angle radius=8mm]{angle=C--B--A}
	pic[draw,"$38^\circ$",angle radius=11mm]{angle=A--C--B};
\end{tikzpicture}
\end{center}

\begin{enumerate}[a)]
	\item 
	Ta có $\widehat{A}=180^\circ - \widehat{B}-\widehat{C}=180^\circ - 64^\circ - 38^\circ = 78^\circ$.\\
	Áp dụng định lí $\sin$ cho tam giác $ABC$, ta được:\\
	$\dfrac{BC}{\sin{\widehat{A}}}=\dfrac{AB}{\sin{\widehat{C}}}=\dfrac{AC}{\sin{\widehat{B}}} 	\Leftrightarrow
	\heva{
		& AB =\dfrac{BC\cdot \sin{\widehat{C}}}{\sin{\widehat{A}}} \\ 
		& AC =\dfrac{BC\cdot \sin{\widehat{B}}}{\sin{\widehat{A}}}} 
	\Leftrightarrow
	\heva{
		& AB =\dfrac{22 \cdot \sin{38^\circ}}{\sin{78^\circ}} \approx 13,8\\ 
		& AC =\dfrac{22 \cdot \sin{64^\circ}}{\sin{78^\circ}} \approx 20,2.}$.
	\item Diện tích tam giác $ABC$ là $S=\dfrac{1}{2} \cdot BA \cdot BC \cdot \sin{B}=\dfrac{1}{2} \cdot 13{,}8 \cdot 22 \cdot \sin{64^\circ} \approx 136{,}4 $.\\
	Ta có $S = \dfrac{AB \cdot AC \cdot BC}{4R} \Leftrightarrow R= \dfrac{AB \cdot AC \cdot BC}{4S} = \dfrac{13{,}8 \cdot 20{,}2 \cdot 22}{4\cdot 136{,}4} \approx 11{,}2$.

\end{enumerate}
}
\end{bt}

%Câu 5...........................
\begin{bt}%[0H4V2-2]%[Dự án đề kiểm tra Toán 10 GHKI NH23-24- VU Ngoc Hao]%[THSP ĐHSP - Tp HCM]
	Cho tam giác $ABC$ với $BC=15$, $AC=20$, $AB=25$. Tính bán kính đường tròn nội tiếp tam giác $ABC$.
	\loigiai{
	Ta có $a=15$, $b=20$, $c=25$. Gọi $r$ là bán kính đường tròn nội tiếp tam giác $ABC$.\\
	Ta có nửa chu vi tam giác $ABC$ là $p=\dfrac{15+20+25}{2}=30$.\\
	Diện tích tam giác $ABC$ là $$S=\sqrt{p(p-a)(p-b)(p-c)}=\sqrt{30(30-15)(30-20)(30-25)}=150.$$
	Mặt khác $S=pr \Rightarrow r=\dfrac{S}{p}=\dfrac{150}{30}=5$.
	
	
	}
\end{bt}

%Câu 6...........................
\begin{bt}%[0H4V3-2] %[Dự án đề kiểm tra Toán 10 GHKI NH23-24- VU Ngoc Hao]%[THPT - Tp HCM]
	Trong lần đến tham quan tháp Eiffel (ở công viên Champ-de-Mars, thủ đô Paris nước Pháp), bạn Phương muốn ước tính độ cao của tháp. Sau khi quan sát, bạn Phương đã minh họa lại kết quả đo đạc của mình ở \emph{Hình 1}. Em hãy giúp bạn Phương tính độ cao của tháp Eiffel theo đơn vị mét (kết quả làm tròn đến hàng đơn vị).
	\loigiai{
	\begin{center}
		\begin{tikzpicture}[scale=1, font=\footnotesize, line join=round, line cap=round,>=stealth]
			\path
			(1,0) coordinate (A)
			(2.5,0) coordinate (B)
			(4,3.5) coordinate (D)
			(4,0) coordinate (C)
			;
			\draw (A)--(D)--(C)node[pos=0.5,right]{$h$}
			(A)--(B)node[pos=0.5,below]{$154\,\text{m}$}--(D) ;
			\draw [dashed] (C)--(B);
			\foreach \x/\g in{A/180,B/-60, D/90, C/-90}
			\fill[black] (\x) circle (1pt)+(\g:3mm) node {$\x$};
			\draw[-] ($(A)!4mm!(C)$) to[bend right=60] node[pos=.7,right]{$50^\circ$} ($(A)!4mm!(D)$);
			\draw ($(B)!4mm!(C)$) to[bend right=60] node[pos=.7,right]{$70^\circ$} ($(B)!4mm!(D)$);	
			\draw (2.5,-1) node{Hình $1$};
		\end{tikzpicture}
	\end{center}
	Xét tam giác $ACD$ vuông tại $C$, có: $\tan 50^\circ =\dfrac{h}{AC} \Rightarrow AC = \dfrac{h}{\tan 50^\circ}$.\\
	Xét tam giác $BCD$ vuông tại $C$, có: $\tan 70^\circ =\dfrac{h}{BC} \Rightarrow BC = \dfrac{h}{\tan 70^\circ}$.\\
	$\Rightarrow AC - BC = \dfrac{h}{\tan 50^\circ} - \dfrac{h}{\tan 70^\circ} = h \left( \dfrac{1}{\tan 50^\circ} - \dfrac{1}{\tan 70^\circ} \right) = 154$\\
	$\Rightarrow h= \dfrac{154}{\dfrac{1}{\tan 50^\circ} - \dfrac{1}{\tan 70^\circ}}\approx 324$ (m).
		
		
	}
\end{bt}
