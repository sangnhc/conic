
\de{ĐỀ THI ÔN TẬP  HỌC KỲ II NĂM HỌC 2022-2023}{THPT Nguyễn Thị Minh Khai}

\begin{bt}%[0T7B3-2]%[Dự án đề kiểm tra HKII- Nguyen Hunh]%[THPT Nguyễn Thị Minh Khai]
	Giải phương trình $ \sqrt{3x^2-2x+1}=x+1 \quad (*)$
\end{bt}

\loigiai{Phương trình $(*) \Rightarrow 3x^2-2x+1=(x+1)^2 \Rightarrow 3x^2-2x+1=x^2+2x+1 \Rightarrow 2x^2-4x=0 \Rightarrow \hoac{&x=0\\&x=2.}$\\
	Thay $ x=2 $ và $ x=0 $ vào phương trình $(*)$, ta thấy cả hai giá trị đều thỏa mãn.\\
	Vây phương trình đã cho có hai nghiệm là $0$ và $2$.}


\begin{bt}%[0T7K1-2]%[Dự án đề kiểm tra HKII- Nguyen Hunh]%[THPT Nguyễn Thị Minh Khai]
	Tìm $ m $ sao cho $ f(x)=(m-2)x^2+(1-2m)x+m<0, \forall x \in \mathbb{R}. $
\end{bt}
\loigiai{\begin{itemize}
		\item TH1: $ m-2=0 \Leftrightarrow m=2$. Thay vào bất phương trình, ta được $ -3x+2<0, \forall x \in \mathbb{R}$ (mệnh đề sai).\\ Do đó, loại $ m=2 $.
		\item TH2: $ m-2 \neq 0 \Leftrightarrow m\neq 2 $. Khi đó $ f(x) $ là tam thức bậc hai.\\
		Ta có $ f(x)<0,\forall x \in \mathbb{R} \Leftrightarrow \heva{&m-2<0 \\& \Delta <0} \Leftrightarrow \heva{&m<2 \\ &(1-2m)^2-4m(m-2) <0} \Leftrightarrow \heva{&m<2 \\& 4m+1 <0} \Leftrightarrow \heva{&m<2 \\& m <\dfrac{-1}{4}} \Leftrightarrow m <\dfrac{-1}{4}$.
	\end{itemize}
	Vậy $ m <\dfrac{-1}{4} $ là các giá trị cần tìm.}
\begin{bt}%[0T8Y1-2] %[0T8Y3-1] %[0T0B2-3]%[Dự án đề kiểm tra HKII- Nguyen Hunh]%[THPT Nguyễn Thị Minh Khai]
	\begin{enumerate}
		\item Trong nhà sách hiện đang có 9 mẫu sổ lưu niệm, 7 mẫu thiệp chúc và 5 mẫu túi giấy. Hỏi bạn An có bao nhiêu cách chọn một phần quà gồm một sổ lưu niệm, một thiệp chúc và một túi giấy?
		\item 	Sử dụng nhị thức Newton, khai triển $\left(3x+2\right)^4$ và tìm số hạng chứa $x^3$ trong khai triển.
		\item 	Tổ toán trường Nguyễn Thị Minh Khai dự định tổ chức một buổi ngoại khóa và mỗi lớp sẽ chọn 5 học sinh tham gia. Lớp 10T có tất cả 30 học sinh, trong đó có 20 nữ. Tính xác suất để cả 5 học sinh được chọn đi ngoại khóa đều là nam.
	\end{enumerate}
	
	\loigiai{
		\begin{enumerate}
			\item 	Để thực hiện công việc trên, An có các công đoạn sau:
			\begin{itemize}
				\item Chọn 1 sổ lưu niệm có $9$ cách chọn.
				\item Chọn 1 thiệp chúc có $7$ cách chọn.
				\item Chọn 1 túi giấy có $5$ cách chọn. 
			\end{itemize}
			Theo quy tắc nhân, có $9\cdot7\cdot5=315$ cách thực hiện công việc.
			\item Ta có $$\begin{array}{ll}
				\left(3x+2\right)^4 &=(3x)^4+4\cdot(3x)^3\cdot2+6\cdot(3x)^2\cdot2^2+4\cdot(3x)\cdot2^3+2^4\\
				& =81x^4+108x^3+216x^2+96x+16.
			\end{array} $$\\
			Vậy số hạng chứa $x^3$ là $108x^3$.
			\item 		Không gian mẫu là tập hợp các cách chọn ra $5$ học sinh từ $30$ học sinh.\\
			Suy ra $n\left(\Omega\right)=\mathrm C^{5}_{30}$.\\
			Gọi $A$ là biến cố: \lq\lq5 học sinh được chọn đều là nam\rq\rq.\\
			Khi đó, $n\left(A\right)=\mathrm C^{5}_{10}$.\\
			Vậy xác suất của biến cố $A$ là $
			\mathrm P\left(A\right)=\dfrac{n\left(A\right)}{n\left(\Omega\right)}=\dfrac{2}{1131}$.
		\end{enumerate}
		
	} 
\end{bt}

\begin{bt}%[0T9K2-6] %ThyNguyenVoDiem%THPT Nguyễn Thị Minh Khai
	Trong mặt phẳng $Oxy$, cho đường thẳng $(d_1)\colon 4x+3y-6=0$ và điểm $A(-1;0)$.
	\begin{enumerate}
		\item Viết phương trình tổng quát của đường thẳng $(d_2)$ đi qua $A$ và vuông góc với $(d_1)$.
		\item Tìm tọa độ điểm $B$ sao cho $AB=\dfrac{5}{4}$ và khoảng cách từ $B$ đến $(d_1)$ là lớn nhất.
	\end{enumerate}
\loigiai
{
\begin{enumerate}
	\item $(d_2)$ vuông góc với $(d_1)$ nên $(d_2) \colon 3x-4y+m=0$.\\
	Mà $A(-1;0) \in (d_2) \Leftrightarrow 3\cdot (-1)-4\cdot 0+m=0 \Leftrightarrow m=3$.\\
	Vậy $(d_2) \colon 3x-4y+3=0$.
	\item $AB=\dfrac{5}{4}$ nên $B$ di động trên đường tròn $(C)$ tâm $A(-1;0)$, bán kính $R=\dfrac{5}{4}$.\\
	$(C) \colon \left(x+1\right)^2+y^2=\dfrac{25}{16}$.\\
	$(d_2)$ là đường thẳng đi qua $A$ và vuông góc $(d_1)$.\\ $(d_2) \colon 3x-4y+3=0 \Leftrightarrow y=\dfrac{3}{4}\left(x+1\right)$. Thay vào phương trình $(C)$, ta có
	\allowdisplaybreaks{\begin{eqnarray*}
			&& (x+1)^2+\dfrac{9}{16}(x+1)^2=\dfrac{25}{16}\\
			&\Leftrightarrow& \dfrac{25}{16}(x+1)^2=\dfrac{25}{16} \\
			&\Leftrightarrow& (x+1)^2=1 \\
			&\Leftrightarrow&  \hoac{&x+1=1\\&x+1=-1}\\
			&\Leftrightarrow&  \hoac{&x=0 \Rightarrow y=\dfrac{3}{4}\\&x=-2\Rightarrow y=-\dfrac{3}{4}.}
	\end{eqnarray*}}
	Thử lại:
	\begin{itemize}
		\item Với $B\left(0;\dfrac{3}{4}\right)$ thì $\mathrm{d}(B,d_1)=\dfrac{\left|4\cdot 0+3\cdot \dfrac{3}{4}-6\right|}{\sqrt{4^2+3^2}}=\dfrac{3}{4}$.
		\item Với $B\left(-2;-\dfrac{3}{4}\right)$ thì $\mathrm{d}(B,d_1)=\dfrac{\left|4\cdot (-2)+3\cdot \left(-\dfrac{3}{4}\right)-6\right|}{\sqrt{4^2+3^2}}=\dfrac{13}{4}$.
	\end{itemize}
Vậy $B\left(-2;-\dfrac{3}{4}\right)$ là điểm cần tìm.
\end{enumerate}	
}
\end{bt}
\begin{bt}%[0T9K3-5]%ThyNguyenVoDiem%THPT Nguyễn Thị Minh Khai
	Trong mặt phẳng $Oxy$,
	\begin{enumerate}
		\item Viết phương trình đường tròn đi qua điểm $E(-1;5)$ và tiếp xúc với đường thẳng $\Delta\colon 2x-y+2=0$ tại điểm $F(0;2)$.
		\item Viết phương trình tiếp tuyến của đường tròn $(C)\colon (x-1)^2+(y+5)^2=25$ tại điểm $M(4;-1)$.
	\end{enumerate}
	\loigiai
	{
		\begin{enumerate}
			\item Đường thẳng $\left(\Delta'\right)$ đi qua $F\left(0;2\right)$ và vuông góc với $\left(\Delta\right)$ có phương trình: $\left(\Delta'\right):x+2y-4=0$.\\
			Gọi $I$ là tâm của đường tròn $\Rightarrow I\in \left(\Delta'\right)$
			$\Rightarrow I\left(4-2a;a\right)$.\\
			Vì đường tròn qua $E\left(-1;5\right)$ và $F\left(0;2\right)$ nên
			\begin{eqnarray*}
				IE=IF&\Leftrightarrow& \sqrt{\left(2a-5\right)^2+\left(5-a\right)^2}=\sqrt{\left(2a-4\right)^2+\left(2-a\right)^2}\\
				&\Leftrightarrow& 5a^2-30a+50=5a^2-20a+20\\
				&\Leftrightarrow& -10a=-30\Leftrightarrow a=3\Rightarrow I\left(-2;3\right).
			\end{eqnarray*}
			Bán kính: $R=IE=\sqrt{5}$.\\
			Phương trình đường tròn: $\left(x+2\right)^2+\left(y-3\right)^2=5$.
			\item Đường tròn $\left(C\right):\left(x-1\right)^2+\left(y+5\right)^2=25$ có tâm $I\left(1;-5\right)$.\\
			Phương trình tiếp tuyến của đường tròn $\left(C\right)\colon\left(x-1\right)^2+\left(y+5\right)^2=25$ tại điểm $M\left(4;-1\right)$ là
			\begin{eqnarray*}
				\left(1-4\right)\left(x-4\right)+\left(-5+1\right)\left(y+1\right)=0&\Leftrightarrow& -3\left(x-4\right)-4\left(y+1\right)=0\\
				&\Leftrightarrow& 3x+4y-8=0.
			\end{eqnarray*}
		\end{enumerate}
	}
\end{bt}


\begin{bt}%[0T9Y4-4]%ThyNguyenVoDiem%THPT Nguyễn Thị Minh Khai
	Trong mặt phẳng $Oxy$, cho hyperbol $(H)\colon \dfrac{x^2}{225}-\dfrac{y^2}{64}=1$. Tính độ dài trục thực và tìm tọa độ hai tiêu điểm của $(H)$.
	\loigiai
	{
		Từ phương trình Hypebol $\left(H\right):\dfrac{x^2}{225}-\dfrac{y^2}{64}=1$ suy ra
		$a^2=225\Rightarrow a=15$ ($a>0$) và $b^2=64\Rightarrow b=8$ (vì $b>0$).\\
		Suy ra độ dài trục thực là $2a=2\cdot15=30$.\\
		Ta có $b^2=c^2-a^2\Rightarrow 64=c^2-225\Rightarrow c^2=289\Rightarrow c=17$.\\
		Vậy tọa độ hai tiêu điểm là $F_1\left(-17;0\right)$ và $F_2\left(17;0\right)$.
	}
\end{bt}

