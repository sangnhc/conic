
\de{ĐỀ THI GIỮA HỌC KỲ I NĂM HỌC 2023-2024}{THPT NGUYỄN THỊ MINH KHAI}

%Câu 1...........................
\begin{bt}%[0D1H3-1]%[Dự án đề kiểm tra Toán 10 GHKI NH23-24- Nguyễn Trần Phong]%[THPT NTMK - Tp HCM]
	\begin{listEX}[1]
		\item Cho hai tập hợp $A =\left\{1;2;3;4 \right\}$	và $B=\left\{3;4;5 \right\}$. Tìm $A \cap B$, $A \setminus B$.
		\item Cho hai tập hợp $A= (-\infty; 1)$ và $B=(0;5]$. Xác định $A \cup B$, $\mathrm C_{\mathbb{R}}A$.
	\end{listEX}
	
	\loigiai{\begin{listEX}[1]
			\item Ta có $A \cap B =\{3;4\}$ và $A \setminus B =\{1;2\}$.
			\item $A \cap B = (0;1)$ và $\mathrm C_{\mathbb{R}}A= [1; +\infty)$.	
		\end{listEX}
	}
\end{bt}

\begin{bt}%[0D3H1-2]%[Dự án đề kiểm tra Toán 10 GHKI NH23-24- Nguyễn Trần Phong]%[THPT NTMK - Tp HCM]
	Tìm tập xác định của các hàm số 
	\begin{listEX}[2]
		\item $y=f(x)= \sqrt{3-x} +1$.	
		\item $y=f(x) =\dfrac{1}{x(x+1)}$.
	\end{listEX}
	\loigiai{
		\begin{listEX}[1]
			\item Hàm số xác định khi và chỉ khi $3-x \ge 0 \Leftrightarrow x \le 3$.\\
			Vậy $\mathscr D= (-\infty; 3]$.
			\item Hàm số xác định khi và chỉ khi $x \cdot (x+1) \neq 0 \Leftrightarrow \heva{& x\neq 0 \\& x \neq -1.}$\\
			Vậy $\mathscr D= \mathbb{R} \setminus \{-1;0\}$.	
		\end{listEX}
	}
\end{bt}

\begin{bt}%[0D3H1-5]%[Dự án đề kiểm tra Toán 10 GHKI NH23-24- Nguyễn Trần Phong]%[THPT NTMK - Tp HCM]
	Xét tính đồng biến, nghịch biến của hàm số $y=f(x)=x^2 -2$ trên khoảng $(-\infty; 0)$ bằng định nghĩa.
	\loigiai{
		Với mọi $x_1$, $x_2 \in (-\infty; 0) \colon x_1 > x_2 \Leftrightarrow x_1 -x_2 >0$, ta có 
		$$f(x_1) - f(x_2) = x_1^2 -x_2^2 = (x_1-x_2) \cdot (x_1+x_2).$$
		Do $x_1$, $x_2 \in (-\infty; 0)$ và $x_1 -x_2>0$ nên $(x_1-x_2) \cdot (x_1+x_2) <0$.\\
		Suy ra $f(x_1) < f(x_2)$ với mọi $x_1$, $x_2 \in (-\infty; 0)$.\\
		Vậy hàm số đã cho nghịch biến trên khoảng $(-\infty; 0)$}
\end{bt}

\begin{bt}%[0H4H3-1]%[Dự án đề kiểm tra Toán 10 GHKI NH23-24- Nguyễn Trần Phong]%[THPT NTMK - Tp HCM]
	Cho tam giác $ABC$ với $BC=a$, $AC= b$, $AB=c$. Gọi $S$, $h_a$, $R$, $r$ lần lượt là diện tích, chiều cao kẻ từ $A$, bán kính đường tròn ngoại tiếp, nội tiếp tam giác $ABC$.
	\begin{listEX}[1]
		\item Chứng minh $h_a = 2 R \cdot \sin B\cdot \sin C$.
		\item Biết $b=8$, $c=6$, $\widehat {BAC}= 60^\circ$. Tính $a$, $S$, $r$.
	\end{listEX}
	\loigiai{
		\begin{listEX}[1]
			\item Ta có $$h_a = 2R \cdot \sin B \cdot \sin C \Leftrightarrow \dfrac{2S}{a}= 2R \cdot \dfrac{b}{2R}\cdot \dfrac{c}{2R} \Leftrightarrow S=\dfrac{abc}{4R} \text{ (đúng)}.$$
			Suy ra $h_a = 2 R \cdot \sin B\cdot \sin C$.
			\item Ta có $a^2 = b^2 + c^2 -2 \cdot b \cdot c \cdot \cos A = 8^2 + 6^2 -2 \cdot 8\cdot 6 \cdot \cos 60^\circ = 52.$\\
			Suy ra $a = 2\sqrt{13}$.\\
			$S=\dfrac{1}{2}\cdot b \cdot c \cdot \sin A = 12\sqrt{3}$.\\
			$p=\dfrac{a+b+c}{2}= 7 +\sqrt{13}$.\\
			Suy ra $S= p\cdot r \Rightarrow r=\dfrac{S}{p}= \dfrac{7\sqrt{3} -\sqrt{39}}{3}$.
	\end{listEX}}
\end{bt}


\begin{bt}%[0H4?3-2]%[Dự án đề kiểm tra Toán 10 GHKI NH23-24- Nguyễn Trần Phong]%[THPT NTMK - Tp HCM]
	Từ hai vị trí $A$, $B$ của một tòa nhà, người ta quan sát đỉnh $C$ của một ngọn núi (hình vẽ). Biết rằng độ cao $AB$ bằng $80$ mét, phương nhìn $AC$ tạo với phương nằm ngang một góc $30^\circ $, phương nhìn $BC$ tạo với phương nằm ngang một góc $14^\circ$. Tính chiều cao $CH$ của ngọn núi so với mặt đất (làm tròn kết quả đến hai chữ số thập phân).
	\begin{center}
		\begin{tikzpicture}[x=2.5mm,y=2.5mm,blue,thick, font=\footnotesize, line cap=round, line join=round]
			\def\a{1}
			\path 	
			(0:0) coordinate (A)
			($(A)+(0:1)$) coordinate (A1)
			($(A)+(30:1)$) coordinate (A2)
			(90:7*\a) coordinate (B)
			($(B)+(0:1)$) coordinate (B1)
			($(B)+(31/2:1)$) coordinate (B2)
			(180:3*\a) coordinate (A')
			($(B)+(A')-(A)$) coordinate (B')
			(intersection of A--A2 and B--B2) coordinate (C)
			(intersection of A--A2 and B--B1) coordinate (I)
			($(A)!(C)!(A1)$) coordinate (H)
			($(B)!(C)!(B1)$) coordinate (K)
			($(H)!1/3!(A)$) coordinate (Ht)
			($(H)!-1/3!(A)$) coordinate (Hp)
			($(C)!2/3!(Ht)$) coordinate (m)
			($(C)!2/3!(Hp)$) coordinate (n)
			($(C)!1/4!(Ht)$) coordinate (p)
			($(C)!1/4!(Hp)$) coordinate (q)
			($(C)+(Hp)-(H)$) coordinate (Cp)
			($(Ht)+(180:1.25cm)$) coordinate (Nt)
			;
			\draw (A)--(B) node[pos=.575,right]{70 m}--(B')--(A')--cycle (C)--(H)--(A)--(C)--(B)--(K) ;
			\draw[teal,thin] 
			($(A)!1/4!(B)$)--($(A')!1/4!(B')$)
			($(A)!2/4!(B)$)--($(A')!2/4!(B')$)
			($(A)!3/4!(B)$)--($(A')!3/4!(B')$)
			($(A)!1/2!(A')$)--($(B)!1/2!(B')$)
			;
			\pic[draw,angle radius=6mm,angle eccentricity=1.75,"$30^\circ$"] {angle=H--A--C};
			\pic[draw,angle radius=12mm,angle eccentricity=1.65,"$15^\circ30'$"] {angle=K--B--C};
			\foreach \x/\g in {A/-90,B/90,C/90,I/-90}
			\fill[black] (\x) circle (1pt) ($(\g:3mm)+(\x)$) node {$\x$};
			
			\draw[clip]
			decorate [decoration={random steps,segment length=9pt,amplitude=3pt}]%
			{($(Ht)+(180:1.25cm)$) -- ($(Ht)+(90:1.95cm)$)-- ($(m)+(210:4mm)$) -- (C) -- ($(n)+(0:3mm)$) -- (Hp)}
			-- ($(Hp)+(-90:3mm)$) -- ($(Nt)+(-90:3mm)$) -- (Nt) ;
			\fill[brown,opacity=.85](Nt) rectangle (Cp);
			\fill[green,opacity=.85]($(Nt)+(-90:3mm)$) rectangle (Hp);
			\fill[white,opacity=.85]
			decorate [decoration={random steps,segment length=3pt,amplitude=1pt}]%
			{(C) -- ($(p)+(0:.09mm)$) -- ($(p)+(-80:13mm)$) -- ($(p)+(-10:3mm)$) -- ($(p)+(-50:10mm)$) -- ($(p)+(15:6mm)$) -- ($(q)+(180:.25pt)$)} -- cycle ;
			\path (K) node[above right]{$K$} (H) node[above right]{$H$};
			\fill[black]  (K) circle(1pt) (H) circle(1pt) (C) circle(1pt) ;
			\draw (C)--(H);
		\end{tikzpicture}
	\end{center}
	\loigiai{
		Ta có $\widehat{BAC}= 90^\circ - 30^\circ = 60^\circ$, $\widehat{ABC}= 90^\circ + 14^\circ =104^\circ $ và $\widehat {BCA}=180^\circ - 104^\circ - 60^\circ = 16^\circ$.\\
		Áp dụng định lý hàm sin trong tam giác $ABC$ ta được $$\dfrac{AC}{\sin B}= \dfrac{AB}{\sin C} \Rightarrow AC = \dfrac{80}{\sin 104^\circ} \cdot \sin 16^\circ$$
		Xét tam giác $ACH$ vuông tại $H$ có $$CH = \sin 30^\circ \cdot AC = \sin 30^\circ \cdot  \dfrac{80}{\sin 104^\circ} \cdot \sin 16^\circ \approx 140{,} 81 \text{ mét}.$$}	
\end{bt}

