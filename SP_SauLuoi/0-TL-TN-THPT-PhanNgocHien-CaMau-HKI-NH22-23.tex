\de{ĐỀ KIỂM TRA HỌC KÌ I NĂM HỌC 2022-2023}{TRƯỜNG THPT CHUYÊN PHAN NGỌC HIỂN - Cà Mau}
\begin{center}
	\textbf{PHẦN 1 - TRẮC NGHIỆM}
\end{center}
\Opensolutionfile{ans}[ans/ans]
%%=====Câu 1
\begin{ex}%[0H1B2-1]%[Dự án đề kiểm tra HKII NH22-23- Thầy Hóa]%[THPT Phan Ngọc Hiển, Cà Mau]
Cho tam giác $ABC$ có $c=14$, $\widehat{C}=80^\circ$, $\widehat{B}=40^\circ$. Cạnh $b$ xấp xĩ bằng
\choice
{$b\approx 0{,}11$}
{\True $b\approx 9{,}14$}
{$b\approx 0{,}05$}
{$b\approx 21{,}45$}
\loigiai{
Áp dụng định lý $\sin$, ta có
\[\dfrac{b}{\sin B}=\dfrac{c}{\sin C}\Rightarrow \dfrac{b}{\sin 40^\circ}=\dfrac{14}{\sin 80^\circ}\Rightarrow b=\dfrac{14\cdot\sin 40^\circ}{\sin 80^\circ}\approx 9{,}14. \]
}
\end{ex}

%%=====Câu 2
\begin{ex}%[0D1Y3-4]%[Dự án đề kiểm tra HKII NH22-23- Thầy Hóa]%[THPT Phan Ngọc Hiển, Cà Mau]
Cho tập $A=(2;5)$, $B=[3;7]$. Tìm $A\cap B$.
\choice
{$(2;7)$}
{$(3;5)$}
{$(2;7]$}
{\True $[3;5)$}
\loigiai{
Ta có $A\cap B=[3;5)$.
}
\end{ex}

%%=====Câu 3
\begin{ex}%[0D1Y2-1]%[Dự án đề kiểm tra HKII NH22-23- Thầy Hóa]%[THPT Phan Ngọc Hiển, Cà Mau]
Viết tập $A=\left\{x \in \mathbb{R}\middle| x^2-10 x+16=0\right\}$ bằng cách liệt kê phần tử.
\choice
{$A=\{2\}$}
{$A=\{8\}$}
{\True $A=\{2;8\}$}
{$A=\{-2;-8\}$}
\loigiai{
Ta có $x^2-10x+16=0\Leftrightarrow\hoac{&x=2\\&x=8.}$\\
Do đó $A=\{2;8\}$.
}
\end{ex}

%%=====Câu 4
\begin{ex}%[0H3Y1-3]%[Dự án đề kiểm tra HKII NH22-23- Thầy Hóa]%[THPT Phan Ngọc Hiển, Cà Mau]
Trong mặt phẳng tọa độ $Oxy$, cho $A(6;3)$, $B(2;-1)$. Tọa độ trung điểm $M$ của đoạn thẳng $AB$ là
\choice
{\True $(4;1)$}
{$(8;2)$}
{$(-4;-4)$}
{$(4;4)$}
\loigiai{
Ta có $\heva{&x_M=\dfrac{6+2}{2}=4\\&y_M=\dfrac{3+(-1)}{2}=1.}$\\
Tọa độ trung điểm $M$ của đoạn thẳng $AB$ là $(4;1)$.
}
\end{ex}

%%=====Câu 5
\begin{ex}%[0H1Y2-1]%[Dự án đề kiểm tra HKII NH22-23- Thầy Hóa]%[THPT Phan Ngọc Hiển, Cà Mau]
Cho tam giác $ABC$, kí hiệu $A,B,C$ là các góc của tam giác tại các đỉnh tương ứng và $AB=c$, $AC=b$, $BC=a$. Khẳng định nào dưới đây đúng?
\choice
{$c^2=a^2+b^2-2ab\sin C$}
{$c^2=a^2+b^2+2ab\sin C$}
{\True $c^2=a^2+b^2-2ab\cos C$}
{$c^2=a^2+b^2+2ac\cos C$}
\loigiai{
Theo định lý cô-sin, ta có $c^2=a^2+b^2-2ab\cos C$.
}
\end{ex}

%%=====Câu 6
\begin{ex}%[0D1Y2-2]%[Dự án đề kiểm tra HKII NH22-23- Thầy Hóa]%[THPT Phan Ngọc Hiển, Cà Mau]
Tập nào sau đây là tập con của $A=\left\{x\in\mathbb{N}\middle| x<2\right\}$.
\choice
{$\{-2;1\}$}
{$\{-1;2\}$}
{$\{-1;0\}$}
{\True $\{1\}$}
\loigiai{
Ta có $A=\{0;1\}$. Do đó tập con của $A$ là $\{1\}$.
}
\end{ex}

%%=====Câu 7
\begin{ex}%[0D1Y1-1]%[Dự án đề kiểm tra HKII NH22-23- Thầy Hóa]%[THPT Phan Ngọc Hiển, Cà Mau]
Có bao nhiêu phát biểu dưới đây là mệnh đề?
\begin{listEX}[2]
	\item Mấy giờ rồi?
	\item Tôi thích học môn Toán!
	\item $17$ là số nguyên tố.
	\item Cả lớp nộp bài kiểm tra!
	\item $972$ chia hết cho $3$.
\end{listEX}
\choice
{4}
{3}
{1}
{\True 2}
\loigiai{
Có 2 phát biểu là mệnh đề là ``$17$ là số nguyên tố'' và ``$972$ chia hết cho $3$''.
}
\end{ex}

%%=====Câu 8
\begin{ex}%[0H3B1-3]%[Dự án đề kiểm tra HKII NH22-23- Thầy Hóa]%[THPT Phan Ngọc Hiển, Cà Mau]
Trong mặt phẳng $Oxy$, cho $A(1;3)$, $B(2;7)$. Tính độ dài đoạn thẳng $AB$.
\choice
{$AB=17$}
{$AB=5$}
{$AB=\sqrt{5}$}
{\True $AB=\sqrt{17}$}
\loigiai{
Ta có $AB=\sqrt{(2-1)^2+(7-3)^2}=\sqrt{17}$.
}
\end{ex}

%%=====Câu 9
\begin{ex}%[0H3B1-2]%[Dự án đề kiểm tra HKII NH22-23- Thầy Hóa]%[THPT Phan Ngọc Hiển, Cà Mau]
Trong mặt phẳng $Oxy$, cho $\vec{a}=(1;4)$, $\vec{b}=(3;-2)$. Khi đó $\vec{a}+\vec{b}$ có tọa độ là
\choice
{$(4;-2)$}
{$(-2;6)$}
{\True $(4;2)$}
{$(2;-6)$}
\loigiai{
Véc-tơ $\vec{a}+\vec{b}$ có tọa độ là $(4;2)$.
}
\end{ex}

%%=====Câu 10
\begin{ex}%[0D1B1-3]%[Dự án đề kiểm tra HKII NH22-23- Thầy Hóa]%[THPT Phan Ngọc Hiển, Cà Mau]
Cho mệnh đề $P:\forall x \in \mathbb{R}, x=\dfrac{1}{x}$. Tìm mệnh đề phủ định $\overline{P}$.
\choice
{$\overline{P}: \forall x \in \mathbb{R}, x>\dfrac{1}{x}$}
{$\overline{P}: \exists x \in \mathbb{R}, x=\dfrac{1}{x}$}
{$\overline{P}: \forall x \in \mathbb{R}, x\neq\dfrac{1}{x}$}
{\True $\overline{P}: \exists x \in \mathbb{R}, x\neq\dfrac{1}{x}$}
\loigiai{
Mệnh đề phủ định $\overline{P}: \exists x \in \mathbb{R}, x\neq\dfrac{1}{x}$.
}
\end{ex}

%%=====Câu 11
\begin{ex}%[0H1B2-1]%[Dự án đề kiểm tra HKII NH22-23- Thầy Hóa]%[THPT Phan Ngọc Hiển, Cà Mau]
Cho tam giác $ABC$ có $a=8$, $b=3$, $\widehat{C}=60^\circ$. Cạnh $c$ bằng
\choice
{$97$}
{$\sqrt{97}$}
{$49$}
{\True $7$}
\loigiai{
Áp dụng định lý cô-sin, ta có
\allowdisplaybreaks
\begin{eqnarray*}
c^2&=&a^2+b^2-2ab\cos C\\
&=&8^2+3^2-2\cdot 8\cdot 3\cdot\cos 60^\circ\\
&=&49.
\end{eqnarray*}
Suy ra $c=\sqrt{49}=7$.
}
\end{ex}

%%=====Câu 12
\begin{ex}%[0D1Y3-1]%[Dự án đề kiểm tra HKII NH22-23- Thầy Hóa]%[THPT Phan Ngọc Hiển, Cà Mau]
Cho tập hợp $A=\{0;1;2;3\}$, $B=\{-2;-1;2;5\}$. Tìm $A\cup B$.
\choice
{\True $\{-2;-1;0;1;2;3;5\}$}
{$\{-2;-1;5\}$}
{$\{0;1;3\}$}
{$\{-2;-1;0;1;2;3;4;5\}$}
\loigiai{
Ta có $A\cup B=\{-2;-1;0;1;2;3;5\}$.
}
\end{ex}

%%=====Câu 13
\begin{ex}%[0H1K2-2]%[Dự án đề kiểm tra HKII NH22-23- Thầy Hóa]%[THPT Phan Ngọc Hiển, Cà Mau]
Cho tam giác $ABC$ có $a=21$, $b=17$, $c=10$. Bán kính đường tròn nội tiếp $r$ bằng
\choice
{$r=\dfrac{85}{8}$}
{$r=\dfrac{7\sqrt{6}}{24}$}
{\True $r=\dfrac{7}{2}$}
{$r=\dfrac{2}{7}$}
\loigiai{
Ta có nửa chu vi $p=\dfrac{a+b+c}{2}=\dfrac{21+17+10}{2}=24$.\\
Diện tích tam giác $ABC$:
\[S_{\triangle ABC}=\sqrt{p(p-a)(p-b)(p-c)}=\sqrt{24\cdot (24-21)\cdot (24-17)\cdot (24-10)}=84. \]
Lại có $S_{\triangle ABC}=pr\Rightarrow r=\dfrac{S_{\triangle ABC}}{p}=\dfrac{84}{24}=\dfrac{7}{2}$.
}
\end{ex}

%%=====Câu 14
\begin{ex}%[0D1Y1-3]%[Dự án đề kiểm tra HKII NH22-23- Thầy Hóa]%[THPT Phan Ngọc Hiển, Cà Mau]
Sử dụng máy tính bỏ túi, hãy viết giá trị gần đúng của $\sqrt{19}$ chính xác đến hàng phần trăm.
\choice
{$4{,}35$}
{$4{,}359$}
{\True $4{,}36$}
{$4{,}4$}
\loigiai{
Ta có $\sqrt{19}\approx 4{,}36$.
}
\end{ex}

\begin{ex}%[0D2Y1-2]%[Dự án đề kiểm tra HKI NH22-23 -Ngô Quang Anh]%[Chuyên Phan ngọc Hiển]
 Miền nghiệm của bất phương trình $3x+y<5$ là nửa mặt phẳng chứa điểm nào trong các điểm sau?
	\choice
	{\True $(1;-1)$}
	{$(0; 6)$}
	{$(4; 2)$}
	{ $(2; 7)$}
	\loigiai{
	Vì $3\cdot 1-1<5$ là mệnh đề đúng nên miền nghiệm của bất phương trình $3x+y<5$ là nửa mặt phẳng chứa điểm $(1;-1)$.
	}
\end{ex}
\begin{ex}%[0H3Y2-1]%[Dự án đề kiểm tra HKI NH22-23 -Ngô Quang Anh]%[Chuyên Phan ngọc Hiển]
	Trong mặt phẳng $Oxy$, cho $\overrightarrow{a}=(2; 3); \overrightarrow{b}=(4;-2)$. Khi đó $\overrightarrow{a} \cdot \overrightarrow{b}$ bằng
	\choice
	{$14$}
	{$(8;-6)$}
	{\True $2$}
	{$12$}
	\loigiai{
	$\overrightarrow{a} \cdot \overrightarrow{b}=2\cdot4+3\cdot(-2)=2$. 
	}
\end{ex}
\begin{ex}%[0H1Y1-2]%[Dự án đề kiểm tra HKI NH22-23 -Ngô Quang Anh]%[Chuyên Phan ngọc Hiển]
 Giá trị của $\cos 60^\circ \cdot \cos 30^\circ+\sin 60^\circ \cdot \sin 30^\circ$ bằng
	\choice
	{$\dfrac{\sqrt{3}}{3}$}
	{$\dfrac{1}{2}$}
	{$1$}
	{\True $\dfrac{\sqrt{3}}{2}$}
	\loigiai{
	$\cos 60^\circ \cdot \cos 30^\circ+\sin 60^\circ \cdot \sin 30^\circ=\dfrac{1}{2} \cdot \dfrac{\sqrt{3}}{2}+\dfrac{\sqrt{3}}{2} \cdot \dfrac{1}{2}=\dfrac{\sqrt{3}}{2}$.
	}
\end{ex}
\begin{ex}%[0H3Y1-3]%[Dự án đề kiểm tra HKI NH22-23 -Ngô Quang Anh]%[Chuyên Phan ngọc Hiển]
Trong mặt phẳng $Oxy$, cho $\overrightarrow{m}=5 \overrightarrow{i}-6 \overrightarrow{j}$, khi đó tọa độ của véc-tơ $\overrightarrow{m}$ là
	\choice
	{$(-6; 5)$}
	{$(5; 6)$}
	{$(6; 5)$}
	{\True $(5;-6)$}
	\loigiai{
	$\overrightarrow{m}=5 \overrightarrow{i}-6 \overrightarrow{j}\Rightarrow \vec{m}=(5;-6)$.
	}
\end{ex}
\begin{ex}%[0H3Y1-3]%[Dự án đề kiểm tra HKI NH22-23 -Ngô Quang Anh]%[Chuyên Phan ngọc Hiển]
Trong mặt phẳng $Oxy$, cho $A(-2; 4), B(3; 1), C(5;-2)$. Trọng tâm của $\triangle ABC$ là
	\choice
	{$G_4(1; 2)$}
	{$G_1(6; 3)$}
	{$G_2(3; 1)$}
	{\True $G_3(2; 1)$}
	\loigiai{
	Trọng tâm của $\triangle ABC$ là $G_3(2; 1)$.		
	}
\end{ex}
\begin{ex}%[0D2Y1-1]%[Dự án đề kiểm tra HKI NH22-23 -Ngô Quang Anh]%[Chuyên Phan ngọc Hiển]
Cặp số $(x; y)=(3; 2)$ là một nghiệm của bất phương trình nào dưới đây?
	\choice
	{\True $2x+y>1$}
	{$3x-6y>5$}
	{$x+5y<-3$}
	{$x+2y<1$}
	\loigiai{
	Vì $2\cdot 3+2>1$ là mệnh đề đúng nên cặp số $(x; y)=(3; 2)$ là một nghiệm của bất phương trình $2x+y>1$.
	}
\end{ex}
\begin{ex}%[0H1T3-2]%[Dự án đề kiểm tra HKI NH22-23 -Ngô Quang Anh]%[Chuyên Phan ngọc Hiển]
Từ hai vị trí $A$ và $B$ của một tòa nhà, người ta quan sát đỉnh $C$ của ngọn núi. Biết rằng độ cao $AB=70$ m, phương nhìn $AC$ tạo với phương nằm ngang góc $30^{\circ}$, phương nhìn $BC$ tạo với phương nằm ngang góc $15^{\circ} 30^{\prime}$.
Ngọn núi đó có độ cao so với mặt đất gần nhất với giá trị nào sau đây?
\begin{center}
	\begin{tikzpicture}[xscale=.7,yscale=0.4, font=\footnotesize, line join=round, >=stealth]
		\coordinate (O) at (0,0);
		\coordinate[label=below:$A$] (A) at (2.2,0);
		\coordinate[label=above left:$B$] (B) at (2.2,7);
		\coordinate[label=above left:$C$] (C) at (11.27,13.13);
		\coordinate (D) at (11.27,7);
		\coordinate[label=below:$H$] (E) at (11.27,0);
		\coordinate[label=right:$70$m] (I) at ($(A)!0.5!(B)$);
		\coordinate[label=above left:$15^{\circ} 30^{\prime}$] () at (5,6.8);
		\coordinate[label=above right:$30^\circ$] () at (3,0);
		\draw[dashed](B)--(D);
		\draw (A)--(C) (B)--(C) (A)--(E)--(C) (E)--(C);
		\def\nhathap{6} %so tang nha thap
		\def\dai{1} %chieu dai 1 khoi
		\def\cao{1} %chieu cao 1 khoi
		\newcommand{\xaynha}[2]{
			\draw[very thick] (#1,#2)rectangle(#1+\dai,#2+\cao);
			\draw[very thick,fill=black] (#1,#2)rectangle(#1+\dai,#2+0.25*\cao);
			\draw[very thick] (#1+0.2*\dai,#2+0.25*\cao)rectangle(#1+0.8*\dai,#2+0.9*\cao); 
		}
		%Xay nha thap
		\foreach \j in {0,...,\nhathap}{
			\foreach \i in {0,1}{
				\xaynha{\i*1.1}{\j*\cao}	} 	}
		\fill [black!40]plot [smooth ] coordinates{(6.64,0)(7.36,3.91) (7.96,5.47) (8.28,4.92) (8.63,4.77)(8.84,3.99) (9.47,5.08) (9.87,7.42)(10.4,8.36)(10.37,9.77)(11.27,13.13)(11.93,12.11)(11.83,10.55)(12.44,9.14)(12.91,6.33)(12.75,4.77)(13.94,0.7)(13.78,0)};
		\draw pic[draw,blue,angle radius=5mm] {angle = D--B--C}; 
		\draw pic[draw,blue,angle radius=5mm] {angle = E--A--C}; 
	\end{tikzpicture}
\end{center}
	\choice
	{$140$m}
	{$233{,}3$m}
	{$9{,}1$m}
	{\True $134{,}7$m}
	\loigiai{
	\begin{itemize}
		\item Ta có $\widehat{ACB}=180^\circ-\left(105^\circ30^{\prime}+60^\circ\right)=14^\circ 30^{\prime}$.
		\item Áp dụng định lí sin vào tam giác $ABC$ ta có \[\dfrac{70}{\sin 14^\circ 30^{\prime}}=\dfrac{AC}{\sin 105^\circ30^{\prime}} \Rightarrow AC=70\cdot \dfrac{\sin 105^\circ30^{\prime}}{\sin 14^\circ 30^{\prime}}.\]
		Xét tam giác vuông $AHC$, ta có $CH=AC\cdot \sin 30^\circ\approx 134{,}7$ m.
	\end{itemize}
	}
\end{ex}
\begin{ex}%[0D1Y3-1]%[Dự án đề kiểm tra HKI NH22-23 -Ngô Quang Anh]%[Chuyên Phan ngọc Hiển]
	Cho tập $A=\{-1; 2; 4\}, B=\{0; 2; 4\}$. Tìm $A \cap B$.
	\choice
	{$\{-1 ; 0 ; 2 ; 4\}$}
	{$(2; 4)$}
	{\True $\{2; 4\}$}
	{$\{-1\}$}
	\loigiai{
	$A \cap B=\{2; 4\}$.
	}
\end{ex}
\begin{ex}%[0H3Y1-3]%[Dự án đề kiểm tra HKI NH22-23 -Ngô Quang Anh]%[Chuyên Phan ngọc Hiển]
Trong mặt phẳng $Oxy$, cho $M(4; 5), N(2; 8)$. Tọa độ $\vec{MN}$ bằng
	\choice
	{$(6; 13)$}
	{\True $(-2; 3)$}
	{$(8; 40)$}
	{$(2;-3)$}
	\loigiai{
	Ta có $\vec{MN}=(x_N-x_M; y_N-y_M)=(-2; 3)$.
	}
\end{ex}
\begin{ex}%[0D1Y3-4]%[Dự án đề kiểm tra HKI NH22-23 -Ngô Quang Anh]%[Chuyên Phan ngọc Hiển]
	Một tổ học sinh học sinh có điểm kiểm tra cuối học kì một môn Toán như sau: $$4 ; 5 ; 6 ; 6 ; 7 ; 8 ; 7 ; 5 ; 6 ; 8 ; 9 ; 10 ; 6.$$ Tìm mốt của dãy số liệu trên.
	\choice
	{$9$}
	{$5$}
	{\True $6$}
	{$7$}
	\loigiai{
	Vì điểm $6$ có số lần xuất hiện nhiều nhất nên $M_0=6$.
	}
\end{ex}
\begin{ex}%[0H3Y1-3]%[Dự án đề kiểm tra HKI NH22-23 -Ngô Quang Anh]%[Chuyên Phan ngọc Hiển]
Trong mặt phẳng $Oxy$, cho $\triangle A B C$ có $A(1; 2), B(3; 4), C(4;-1)$. Tìm tọa độ điểm $D$ sao cho tứ giác $ABCD$ là hình bình hành
	\choice
	{ $(-2; 3)$}
	{$(7; 10)$}
	{$(0 ;-3)$}
	{\True $(2 ;-3)$}
	\loigiai{
	Vì tứ giác $ABCD$ là hình bình hành nên $\heva{&x_D=x_A+x_C-x_B=2\\&y_D=y_A+y_C-y_B=-3.}$
	}
\end{ex}
\begin{ex}%[0D2Y2-2]%[Dự án đề kiểm tra HKI NH22-23 -Ngô Quang Anh]%[Chuyên Phan ngọc Hiển]
Một tổ học sinh gồm $10$ học sinh có điểm kiểm tra giữa học kì 1 môn Toán như sau: $$5 ; 4 ; 7 ; 8 ; 8 ; 9 ; 9 ; 7 ; 8 ; 10.$$ Điểm trung bình của cả tổ gần nhất với số nào dưới đây?
	\choice
	{$7{,}6$}
	{$7{,}8$}
	{\True$7{,}5$}
	{$7{,}4$}
	\loigiai{
	Điểm trung bình của cả tổ là $\overline{x}=\dfrac{5+4+7+8+8+9+9+7+8+10}{10}=7{,}5$.
	}
\end{ex}
\begin{ex}%[0H1Y2-2]%[Dự án đề kiểm tra HKI NH22-23 -Ngô Quang Anh]%[Chuyên Phan ngọc Hiển]
Cho tam giác $ABC$, kí hiệu $A, B, C$ là các góc của tam giác tại các đỉnh tương ứng và $AB=c, AC=b, BC=a$. Diện tích tam giác $ABC$ bằng
	\choice
	{$S_{\triangle ABC}=\dfrac{1}{2} b c \sin B$}
	{\True $S_{\triangle ABC}=\dfrac{1}{2} b c \sin A$}
	{$S_{\triangle ABC}=\dfrac{1}{2} b c \sin C$}
	{$S_{\triangle ABC}=\dfrac{1}{2} b a \sin B$}
	\loigiai{
	$S_{\triangle ABC}=\dfrac{1}{2} b c \sin A$.
	}
\end{ex}
\begin{ex}%[0D1Y3-2]%[Dự án đề kiểm tra HKI NH22-23 -Ngô Quang Anh]%[Chuyên Phan ngọc Hiển]
Điểm thi môn Toán cuối năm của một nhóm các học sinh lớp $10$ là $$1;2;4;4;5;6;6;7;10.$$ Tìm số trung vị của dãy số liệu trên.
	\choice
	{$8$}
	{$5{,}5$}
	{\True $5$}
	{$6$}
	\loigiai{
	Vì cở mẫu $n=9$ nên $M_e=5$.
	}
\end{ex}



\Closesolutionfile{ans}
%\begin{center}
%	\textbf{ĐÁP ÁN}
%	\inputansbox{10}{ans/ans}	
%\end{center}




\begin{bt}%[0D1Y3-2] 
	Cho hai tập hợp $E=\{4 ; 5 ; 6 ; 7 ; 8\}, D=\{6 ; 7 ; 8 ; 9 ; 10\}$. Xác định các tập hợp sau: $E \cap D, E \cup D, E \setminus  D$.
	\loigiai{Ta có $E\cap D=\{6;7;8\}$, $E\cup D=\{4;5;6;7;8;9;10\}$ và $E\setminus  D=\{4;5\}$.}
\end{bt}
\begin{bt}%[0H1Y2-1]
	Cho tam giác $\triangle A B C$ có $a=5, c=4, \widehat{B}=60^{\circ}$. Tính cạnh $b$.
	\loigiai{Áp dụng định lý cosin cho tam giác $ABC$, ta có 
		$$b^2=a^2+c^2-2ac\cos B=5^2+4^2-2\cdot 5\cdot 4\cdot \dfrac{1}{2}=21\Rightarrow b=\sqrt{21}.$$
	}
\end{bt}
\begin{bt}%[0H3Y1-3]
	Trong mặt phẳng $Oxy$, cho $\triangle ABC$ có $A(2;3), B(-2;4), C(-5;-1)$.
	\begin{enumerate}
		\item Tìm tọa độ điểm $M$ là trung điểm $BC$.
		\item Tìm tọa độ điểm $G$ là trọng tâm $\triangle ABC$.
		\item Tìm tọa độ điểm $D$ sao cho tứ giác $ABCD$ là hình bình hành.
	\end{enumerate}
	\loigiai{
		\begin{enumerate}
			\item Ta có $x_{M} = \dfrac{x_{B} + x_{C}}{2}$ và $y_{M} = \dfrac{y_{B} + y_{C}}{2}$. Vậy $M \left( \dfrac{-7}{2}; \dfrac{3}{2} \right)$.
			\item Ta có $x_{G} = \dfrac{x_{A}+x_{B}+x_{C}}{3}$ và $y_{G} = \dfrac{y_{A}+y
				_{B}+y_{C}}{3}$. Vậy $G \left( \dfrac{-5}{3};2 \right) $.
			\item Tứ giác $ABCD$ là hình bình hành khi và chỉ khi $$\overrightarrow{AB} = \overrightarrow{DC} \Leftrightarrow (-4;1)=(-5-x_{D};-1-y_{D}) \break \Leftrightarrow \heva{& x_{D}=-1 \\ & y_{D}=-2.}$$
		\end{enumerate}
	}
\end{bt}

\begin{bt}%[0H3K1-3]
	Trong mặt phẳng tọa độ $Oxy$, cho ba điểm $A(1;0), B(0;3)$ và $C(-3;-5)$. Tìm tọa độ điểm $M$ thuộc trục hoành sao cho biểu thức $P=\left| 2\overrightarrow{MA} - 3\overrightarrow{MB} + 2\overrightarrow{MC}\right|$ đạt giá trị nhỏ nhất.
	\loigiai{
		Gọi $M(x;0) \in Ox$ thỏa mãn điều kiện đề ra.\\
		Ta có $2\overrightarrow{MA} - 3\overrightarrow{MB} + 2\overrightarrow{MC} = (2(1-x)-3(0-x)+2(-3-x); 2\cdot0-3\cdot3+2\cdot(-5) ) = (-4-x;-19)$.\\
		Suy ra $P(x)=\sqrt{(-4-x)^{2}+(-19)^{2}} = \sqrt{(x+4)^{2}+19^{2}} \geq 19$.\\
		$P(x)=19 \Leftrightarrow (4+x)^2=0 \Leftrightarrow x = -4$.\\
		Vậy $\min P = 19$ khi $M(-4;0)$.
	}
\end{bt}
