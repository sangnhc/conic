
\de{ĐỀ THI ÔN TẬP  HỌC KỲ II NĂM HỌC 2022-2023}{THPT Võ Thị Sáu}



	\begin{bt}%[0T7B3-2]%Dương Phước Sang%THPT Võ Thị Sáu
		Giải phương trình $\sqrt{x^2+5x-20}=\sqrt{3x^2-10x+8}$.
		\loigiai{
			Ta có 
			\allowdisplaybreaks
			$\begin{aligned}[t]
				\sqrt{x^2+5x-20}=\sqrt{3x^2-10x+8} 
				&\Rightarrow x^2+5x-20=3x^2-10x+8\\
				&\Leftrightarrow 2x^2-15x+28=0
				\Leftrightarrow \hoac{&x=4\\&x=\dfrac{7}{2}.}
			\end{aligned}$\\
			Cả $x=4$ và $x=\dfrac{7}{2}$ đều thoả mãn phương trình $\sqrt{x^2+5x-20}=\sqrt{3x^2-10x+8}$ nên phương trình có tập nghiệm $S=\left\{4;\dfrac{7}{2}\right\}$.
		}
	\end{bt}
	
	\begin{bt}%[0T8Y1-1]%Dương Phước Sang%THPT Võ Thị Sáu
		Trên giá sách có $10$ quyển sách Toán khác nhau, $9$ quyển sách Văn khác nhau. Hỏi có bao nhiêu cách chọn ngẫu nhiên một quyển sách?
		\loigiai{
			Số cách chọn ngẫu nhiên $1$ quyển sách trên giá sách có $10$ quyển sách Toán khác nhau, $9$ quyển sách Văn khác nhau là $10+9=19$.
		}
	\end{bt}

	\begin{bt}%[0T8B1-2]%Dương Phước Sang%THPT Võ Thị Sáu
		Từ các chữ số $1$, $2$, $3$, $4$, $5$, $6$, $7$, $8$ có thể lập được bao nhiêu số tự nhiên có $4$ chữ số đôi một khác nhau sao cho $2$ chữ số cuối cùng là số chẵn?
		\loigiai{
			Đặt $X=\big\{1;2;3;4;5;6;7;8\big\}$.\\
			Xét số $\overline{abcd}$; với $a,b,c,d \in X$, chúng khác nhau và $c,d$ chẵn.\\
			Số cách chọn $c$ và $d$ khác nhau từ $\big\{2;4;6;8\big\}$ là $4\times 3$ (cách).\\
			Số cách chọn $a$ và $b$ khác nhau từ $X \setminus \big\{c;d\big\}$ là $6 \times 5$ (cách).\\
			Vậy có $4 \times 3 \times 6 \times 5=360$ số thoả mãn yêu cầu của bài toán.
		}
	\end{bt}
	
	\begin{bt}%[0T8B3-2]%Dương Phước Sang%THPT Võ Thị Sáu
		Xác định hệ số của $x^3$ trong khai triển nhị thức $\left(\dfrac{2}{3}x-\dfrac{1}{4}\right)^4$.
		\loigiai{
			Số hạng tổng quát của khai triển là $\mathrm{C}_4^k \left(\dfrac{2}{3}x\right)^{4-k}\left(-\dfrac{1}{4}\right)^k=\mathrm{C}_4^k \left(\dfrac{2}{3}\right)^{4-k} \left(-\dfrac{1}{4}\right)^k x^{4-k}$, $k \in \mathbb{N},k \leq 4$.\\
			Số hạng chứa $x^3$ là số hạng tương ứng với $4-k=3 \Leftrightarrow k=1$, đó là số hạng có hệ số bằng 
			$$\mathrm{C}_4^1 \left(\dfrac{2}{3}\right)^3 \left(-\dfrac{1}{4}\right)=-\dfrac{8}{27}.$$
		}
	\end{bt}
	
	\begin{bt}%[0T9B3-1]%Dương Phước Sang%THPT Võ Thị Sáu
		Trong hệ trục toạ độ $Oxy$, tìm tâm và bán kính của đường tròn $(C)$ có phương trình $x^2+y^2-4x+8y-5=0$.
		\loigiai{
			Xét $\heva{&-2a=-4\\&-2b=8\\&c=-5} \Leftrightarrow \heva{&a=2\\&b=-4\\&c=-5.}$\\
			Dựa vào đó ta tìm được toạ độ tâm $I(2;-4)$ và bán kính $R=\sqrt{2^2+(-4)^2-(-5)}=5$ của $(C)$.
		}
	\end{bt}

\begin{bt}%[0-TL-VoThiSau-HKII-NH22-23]%Nguyễn Ngọc Dũng%[0H9B3-3]
Trong mặt phẳng $Oxy$ cho đường tròn $(C)$ có tâm $I(2;-1)$ và bán kính $R=\sqrt{5}$. Viết phương trình tiếp tuyến của đường tròn $(C)$ song song với đường thẳng $\heva{&x=3-2t \\ &y=-2+t}, t\in \mathbb{R}$.
\loigiai{
Ta có $d\colon \heva{&x=3-2t \\ &y=-2+t} \Leftrightarrow d\colon \heva{&x=3-2t \\ &2y=-4+2t} \Leftrightarrow d\colon x+2y+1=0$.\\
Gọi tiếp tuyến cần tìm là $\Delta$.\\
Vì $\Delta\parallel d$ nên $\Delta\colon x+2y+m=0$, $(m\neq 1)$.\\
Ta có $R=\mathrm{d}(I,\Delta) \Leftrightarrow \sqrt{5} = \dfrac{|2+2\cdot (-1) +m|}{\sqrt{1^2+2^2}} \Leftrightarrow |m|=5 \Leftrightarrow m=\pm 5$ (nhận).\\
Vậy $\Delta\colon x+2y+5=0$ hoặc $\Delta\colon x+2y-5=0$.
}
\end{bt}

\begin{bt}%[0-TL-VoThiSau-HKII-NH22-23]%Nguyễn Ngọc Dũng%[0H9Y4-1]
Xác định độ dài các trục và tiêu cự của elip $(E)\colon \dfrac{x^2}{5} + \dfrac{y^2}{4} =1$.
\loigiai{
Ta có $a=\sqrt{5}$; $b=2$. Suy ra $c=\sqrt{a^2-b^2} = \sqrt{5-4} =1$.\\
Vậy độ dài trục lớn là $2a=2\sqrt{5}$, độ dài trục nhỏ là $2b=4$, tiêu cự là $2c=2$.	
}
\end{bt}

\begin{bt}%[0-TL-VoThiSau-HKII-NH22-23]%Nguyễn Ngọc Dũng%[0H9Y4-5]
Viết phương trình chính tắc của hypebol $(H)$, biết $(H)$ có tiêu cự bằng $16$ và độ dài trục ảo bằng $8$.
\loigiai{
Ta có tiêu cự bằng $16$ nên $2c=16\Rightarrow c=8$.\\
Độ dài trục ảo bằng $8$ nên $2b=8\Rightarrow b=4$.\\
Suy ra $a=\sqrt{c^2-b^2} = \sqrt{8^2-4^2} = 4\sqrt{3}$.\\
Vậy phương trình chính tắc của hypebol là $(H)\colon \dfrac{x^2}{48} - \dfrac{y^2}{16} =1$.	
}
\end{bt}

\begin{bt}%[0-TL-VoThiSau-HKII-NH22-23]%Nguyễn Ngọc Dũng%[0D8B2-2]
Đội tuyển học sinh giỏi Toán $12$ của trường $X$ gồm $8$ học sinh, trong đó có $5$ học sinh nam. Cần chọn $5$ học sinh đi thi học sinh giỏi cấp tỉnh. Có bao nhiêu cách chọn có cả nam và nữ và học sinh nam nhiều hơn học sinh nữ.
\loigiai{
\begin{enumerate}[\bf TH1.]
\item Chọn $3$ nam và $2$ nữ: có $\mathrm{C}^3_5\cdot \mathrm{C}^2_3=30$ (cách).
\item Chọn $4$ nam và $1$ nữ: có $\mathrm{C}^4_5\cdot \mathrm{C}^1_3=15$ (cách).
\end{enumerate}	
Vậy có tất cả $30+15=45$ (cách).
}
\end{bt}