
\de{ĐỀ THI GIỮA HỌC KỲ I NĂM HỌC 2023-2024}{THPT VĨNH LỘC}
\begin{bt}%[0D1N2-1]%[0D1H3-1]%[0D1H3-2]%[Dự án đề kiểm tra Toán 10 GHKI NH23-24- Quang Anh]%[THPT THPT Vĩnh Lộc]
	Cho các tập hợp sau $A=\left\{x \in\mathbb{Z}\mid 2 x\bigl(3 x^2-2 x-5\bigr)=0\right\}$;\\  $B=\{x \in \mathbb{N}\mid 8-5 x \geq-12\}$; $C=\{3; 4; 5; 6; a; b; c \}$;
	$D=\{b; d; 0; 4; 6\}$; $E=\{a; c; d; 3; 5; 6; 7\}$.
	\begin{enumerate}
		\item  Liệt kê các phần từ của tập hợp $A$, $B$. 
		\item Tìm $C\cap D, C\cup E, E\backslash D, E\backslash C$.
		\item Chứng minh rằng $C\backslash(E\cap D)=(C\backslash D) \cup(C\backslash E)$.
	\end{enumerate}
	\loigiai{
		\begin{enumerate}
			\item \begin{itemize}
				\item Xét tập $A$, ta có  $2 x\bigl(3 x^2-2 x-5\bigr)=0 \Leftrightarrow\hoac{&x=0\in\mathbb{Z}\\ &x=-1\in\mathbb{Z}\\ &x=\dfrac{5}{3}\notin\mathbb{Z}}\Rightarrow A=\{-1 ;0\}$.
				\item Xét tập $B$, với $x \in \mathbb{N}$ ta có $8-5 x \geq-12\Leftrightarrow x\leq 4\Rightarrow B=\{0;1;2;3;4\}$.
			\end{itemize}
			\item $C\cap D=\{b;4;6\}$, $C\cup E=\{a;b;c;d;3;4;5;6;7\}$, $E\backslash D=\{a;c;3;5;7\}$, $E\backslash C=\{d;7\}$.
			\item Ta có 
			\begin{align}
				&E\cap D=\{d;6\}\Rightarrow C\backslash(E\cap D)=\{a;b;c;3;4;5\} \tag{1}\\
				&C\backslash D=\{a;c;3;5\}, C\backslash E=\{b;4\}\Rightarrow(C\backslash D) \cup(C\backslash E)=\{a;b;c;3;4;5\} \tag{2}
			\end{align}
			Từ (1), (2) suy ra $C\backslash(E\cap D)=(C\backslash D) \cup(C\backslash E)$.
		\end{enumerate}
	}
\end{bt}
%%%==========================%%%
\begin{bt}%[0D1H3-3]%[0D1H3-4]%[Dự án đề kiểm tra Toán 10 GHKI NH23-24- Quang Anh]%[THPT THPT Vĩnh Lộc]
	Cho các tập hợp $A=(-7; 2]$; $B=[-3;+\infty)$. Tìm $A\cap B$, $A\cup B$, $A\backslash B$, $C_\mathbb{R}(A\cup B)$.
	\loigiai{
		Ta có 
		\begin{itemize}
			\item $A\cap B=[-3;2]$;
			\item $A\cup B=(-7; +\infty)$;
			\item $A\backslash B=(-7; 3)$;
			\item $C_\mathbb{R}(A\cup B)=(-\infty;-7]$.
		\end{itemize}	
	}
\end{bt}
%%%==========================%%%
\begin{bt}%[0D1V2-2]%[Dự án đề kiểm tra Toán 10 GHKI NH23-24- Quang Anh]%[THPT THPT Vĩnh Lộc]
	Cho tập hợp $A=\{x \in \mathbb{N}\mid|x|\leq 4)$; $B=\left\{x \in \mathbb{N^*}\mid x\left(x^2-1\right)=0\right\}$. Tìm tất cả tập $Y$ thỏa $Y\subset A; B\subset Y$ và $2 \notin Y$.
	\loigiai{
		\begin{itemize}
			\item Xét tập $A$, ta có $A=\{0;1;2;3;4\}$;
			\item Xét tập $B$, ta có  $x\left(x^2-1\right)=0 \Leftrightarrow\hoac{&x=0\notin \mathbb{N^*}\\ &x=-1\notin \mathbb{N^*}\\ &x=1\in\mathbb{N^*}}\Rightarrow B=\{1\}$;
		\end{itemize}	
		Để $B\subset Y\subset A$ và $2 \notin Y$ thì ta xét các trường hợp sau
		\begin{itemize}
			\item \textbf{TH1.} $Y$ có 1 phần tử: có $1$ tập $\{1\}$.
			\item \textbf{TH2.} $Y$ có 2 phần tử: có $3$ tập $\{1;0\}, \{1;3\}, \{1;4\}$.
			\item \textbf{TH3.} $Y$ có 3 phần tử: có $3$ tập $\{1;0;3\}, \{1;0;4\}, \{1;3;4\}$.
			\item \textbf{TH4.} $Y$ có 4 phần tử: có $1$ tập $\{0;1;3;4\}$.
		\end{itemize}	
		Vậy có tất cả $8$ tập hợp $Y$ thỏa yêu cầu bài toán.
	}
\end{bt}

\begin{bt}%[0H4H1-2]%[Dự án đề kiểm tra Toán 10 GHKI NH23-24- TheHung Nguyen]%[THPT VĨNH LỘC]
	Cho $\sin x=\dfrac{1}{3}$, $0^{\circ}<x<90^{\circ}$. Tính $\cos x, \tan x, \cot x$.
	\loigiai{
	Ta có $\sin^2x+\cos^2x=1\Rightarrow \cos^2x=1-\sin^2x=1-\dfrac{1}{9}=\dfrac{8}{9}$.\\
	Do $0^{\circ}<x<90^{\circ}$ nên $\cos x=\dfrac{2\sqrt{2}}{3}$.\\
	Khi đó $\tan x=\dfrac{\sin x}{\cos x}=\dfrac{\tfrac{1}{3}}{\tfrac{2\sqrt{2}}{3}}=\dfrac{\sqrt{2}}{4}$ và $\cot x=\dfrac{1}{\tan x}=2\sqrt{2}$.
}
\end{bt}

\begin{bt}%[0H4V1-3]%[Dự án đề kiểm tra Toán 10 GHKI NH23-24- TheHung Nguyen]%[THPT VĨNH LỘC]
	Chứng minh $\sqrt{\sin ^4 x+4 \cos ^2 x}+\sqrt{\cos ^4 x+4 \sin ^2 x}=3, \forall x$.
	\loigiai{
	Ta có 
	\begin{eqnarray*}
		& & \sqrt{\sin ^4 x+4 \cos ^2 x}+\sqrt{\cos ^4 x+4 \sin ^2 x}\\
		&= & \sqrt{\sin ^4 x+4(1-\sin ^2x)}+\sqrt{\cos ^4 x+4 (1-\cos^2x)}\\
		&= & \sqrt{\sin ^4 x-4\sin ^2x+4}+\sqrt{\cos ^4 x-4\cos^2x+4}\\
		&= & \sqrt{(\sin^2x-2)^2}+\sqrt{(\cos^2x-2)^2}\\
		&= &\left|\sin^2x-2\right|  + \left|\cos^2x-2\right|\\
		&= & 2-\sin^2x+2-\cos^2x~(\text{Do}~\sin^2x-2<0~ \text{và}~\cos^2x-2<0)\\
		&= & 4-(\sin^2x+\cos^2x)=4-1=3.
	\end{eqnarray*}
}
\end{bt}

\begin{bt}%[0H4H3-1]%[Dự án đề kiểm tra Toán 10 GHKI NH23-24- TheHung Nguyen]%[THPT VĨNH LỘC]
	Cho $\triangle ABC$ có $BC=4$cm, $ AB=2$cm, $ \widehat{B}=60^{\circ}$. Tính cạnh $AC$, $\widehat{A}$, diện tích, đường cao $AM$, bán kính đường tròn ngoại tiếp và nội tiếp của $\triangle ABC$.
	\loigiai{
	\immini{
	Theo định lý hàm số cosin ta có\\
	 $AC^2=AB^2+BC^2-2\cdot AB\cdot BC\cdot \cos B=2^2+4^2-2\cdot 2\cdot4\cdot\cos 60^\circ=12$.\\
	 Suy ra $AC=\sqrt{12}=2\sqrt{3}$.\\
	 Ta có $\heva{& BC^2=12 \\ & AC^2+AB^2=12+4=16}\Rightarrow BC^2=AC^2+AB^2$.\\
	 Suy ra $\triangle ABC$ vuông tại $A\Rightarrow \widehat{A}=90^\circ$.\\
	 Ta có $S=\dfrac{1}{2}\cdot AB\cdot AC=\dfrac{1}{2}\cdot 2\cdot 2\sqrt{3}=2\sqrt{3}$.\\
	 Ta có $S=\dfrac{1}{2}AM\cdot BC\Rightarrow AM=\dfrac{2S}{BC}=\dfrac{2\cdot 2\sqrt{3}}{4}=\sqrt{3}$.	 

}{\begin{tikzpicture}[scale=1, font=\footnotesize, line join=round, line cap=round, >=stealth]
			\draw (0,0)coordinate (A)--++(0:2)coordinate (B)--++(120:4)coordinate (C)--(A);
			\path 	($(B)!(A)!(C)$ ) coordinate (M);
			\draw (A)--(M);	
			\foreach \y/\g in {A/-180,B/0,C/90,M/10}  \fill (\y) circle (1pt)+(\g:.25)node {$\y$};
	\end{tikzpicture}}	
\noindent	Do $\triangle ABC$ vuông tại $A$ nên bán kính đường tròn ngoại tiếp  $R=\dfrac{BC}{2}=\dfrac{4}{2}=2$.	\\
Ta có $S=p\cdot r\Rightarrow r=\dfrac{S}{p}=\dfrac{2\sqrt{3}}{\dfrac{4+2+2\sqrt{2}}{2}}=\dfrac{2\sqrt{3}}{2+\sqrt{3}}=4\sqrt{3}-6$.
		
		
	}
\end{bt}


\begin{bt}%[0H4V3-2]%[Dự án đề kiểm tra Toán 10 GHKI NH23-24- TheHung Nguyen]%[THPT VĨNH LỘC]
	\immini{Từ vị trí $A$ người ta quan sát một cây cao.
		Biết $AH=4$, $HB=20$, góc $\widehat{BAC}=45^{\circ}$. Hỏi khi đó chiều cao của cây (làm tròn đến hàng phần mười) bằng bao nhiêu?}{\begin{tikzpicture}[scale=.55, font=\footnotesize, line join=round, line cap=round, >=stealth]
			%\hetruc(0,0)(10,10)
			\path (0,0) coordinate(H) (0,2) coordinate(A)   (6,0) coordinate(B)
			(6,6.5) coordinate(C);		
			\draw (A)--(B)--(C)--(A)--(H)--(B);
			\fill[green!50!gray] (4.75,1.75)--(7.25,1.75)--(C)--(4.75,1.75);
			\fill[gray] ($(B)+(-.1,0)$)--($(B)+(.1,0)$)--(C)--($(B)+(-.1,0)$);
			\foreach \y/\g in {A/-180,B/0,C/90,H/180}  \fill (\y) circle (1pt)+(\g:.35)node {$\y$};
			\draw (H) rectangle (.25,.25);
	\end{tikzpicture}}
	\loigiai{	\textbf{Cách 1:}\\
\immini{Trong $\triangle AHB$ vuông tại $H$,\\ có $AB^2=AH^2+HB^2\Rightarrow AB=\sqrt{4^2+20^2}=4\sqrt{26}$.\\
Kẻ $AK\perp BC$ tại $K$. Khi đó $AHBK$ là hình chữ nhật.\\ Suy ra $KB=AH=4$ và $AK=HB=20$.\\
Trong $\triangle AKC$ vuông tại $K$, ta có $AC^2=AK^2+CK^2$.
}{\begin{tikzpicture}[scale=.55, font=\footnotesize, line join=round, line cap=round, >=stealth]
		%\hetruc(0,0)(10,10)
		\path (0,0) coordinate(H) (0,2) coordinate(A)   (6,0) coordinate(B)
		(6,6.5) coordinate(C)  (6,2)coordinate(K);		
		\draw (A)--(B)--(C)--(A)--(H)--(B);
		\fill[green!50!gray] (4.75,1.75)--(7.25,1.75)--(C)--(4.75,1.75);
		\fill[gray] ($(B)+(-.1,0)$)--($(B)+(.1,0)$)--(C)--($(B)+(-.1,0)$);
		\foreach \y/\g in {A/-180,B/0,C/90,H/180,K/10}  \fill (\y) circle (1pt)+(\g:.25)node {$\y$};
		\draw (H) rectangle (.25,.25) (A)--(K)--(B);
		\draw pic[angle radius=3mm,draw=blue,angle eccentricity=2] {right angle = B--K--A};
\end{tikzpicture}}
Áp dụng định lý hàm số cosin trong $\triangle ABC$,ta có
\begin{eqnarray*}
	& & BC^2=AB^2+AC^2-2\cdot AB\cdot AC\cos BAC\\
	&\Leftrightarrow & (BK+CK)^2=416+AK^2+CK^2-2\cdot 4\sqrt{26}\cdot \sqrt{AK^2+CK^2}\cos45^\circ\\
	&\Leftrightarrow & (4+CK)^2=416+400+CK^2-2\cdot4\sqrt{26}\cdot\sqrt{400+CK^2}\cdot\dfrac{\sqrt{2}}{2}\\
	&\Leftrightarrow & \sqrt{13}\cdot\sqrt{400+CK^2}=100-CK\\
	&\Rightarrow & 12\cdot CK^2+200\cdot CK-4800=0\\
	&\Leftrightarrow &\hoac{& CK=\dfrac{40}{3} \\ & CK=-30~(\text{loại)}.}
\end{eqnarray*}
Ta lại có $BC=BK+CK=4+\dfrac{40}{3}=\dfrac{52}{3}\approx17{,}3$.\\
Vậy chiều cao cây là $17{,}3$m.\\
\textbf{Cách 2:}\\
Ta có $AB= \sqrt{AH^2 + BH^2} = \sqrt{4^2+20^2} = 4 \sqrt{26}$.\\
$\tan \widehat{HAB} = \dfrac{HB}{HA} = \dfrac{20}{4} = 5 \Rightarrow \widehat{HAB} \approx 78{,}69^{\circ}$.\\
Do $AH \parallel BC$ nên $ \widehat{ABC} = \widehat{HAB} \approx 78{,}69^{\circ}$.\\
$\widehat{ACB} = 180^{\circ} - 45^{\circ} - \widehat{ABC} \approx 56{,}31^{\circ}$.\\
Áp dụng định lí hàm số $\sin$ trong tam giác $ABC$ ta có
$$ \dfrac{BC}{\sin 45^{\circ}} = \dfrac{AB}{\sin 56{,}31^{\circ}} = \dfrac{4 \sqrt{26}}{\sin 56{,}31^{\circ}} \Rightarrow BC \approx  17{,}33.$$
}
\end{bt}




