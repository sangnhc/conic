
\de{ĐỀ THI HỌC KỲ I NĂM HỌC 2022-2023}{THPT Yên Mô - Ninh Bình}

\begin{center}
	\textbf{PHẦN 1 - TRẮC NGHIỆM}
\end{center}
\Opensolutionfile{ans}[ans/ans]
%Câu 1...........................
\begin{ex}%[0T1Y3-1]%[Dự án đề kiểm tra HKII NH22-23- Nguyễn Sĩ Đạt]%[THPT Yên Mô - Ninh Bình]
 Cho hai tập hợp $A=[-1;2]$, $B=(0;4)$. Tìm $A \cap B$?
 \choice
 {\True $A \cap B = (0;2]$}
 {$A \cap B = (0;2)$}
 {$A \cap B = [-1;0]$}
 {$A \cap B = [-1;4)$}
 \loigiai{
 	Ta có $A \cap B = (0;2]$.
 }
\end{ex}

%Câu 2...........................
\begin{ex}%[0T2B1-2]%[Dự án đề kiểm tra HKII NH22-23- Nguyễn Sĩ Đạt]%[THPT Yên Mô - Ninh Bình]
	Điểm nào sau đây thuộc miền nghiệm của bất phương trình $2x-y \geq 6$?
	\choice
	{$M(1;2)$}
	{$N(2;1)$}
	{\True $P(3;-1)$}
	{$Q(3;1)$}
	\loigiai{
		Ta có $2 \cdot 3 - (-1) = 7 \geq 6$ là mệnh đề đúng nên $P(3;-1)$ thuộc miền nghiệm của bất phương trình $2x-y \geq 6$.
	}
\end{ex}

%Câu 3...........................
\begin{ex}%[0T3Y1-2]%[Dự án đề kiểm tra HKII NH22-23- Nguyễn Sĩ Đạt]%[THPT Yên Mô - Ninh Bình]
	Tập xác định của hàm số $y=\dfrac{x-2}{x-1}$ là
	\choice 
	{$\mathscr{D}= \mathbb{R}$}
	{$\mathscr{D}=\mathbb{R} \setminus \{2\}$}
	{\True $\mathscr{D}=\mathbb{R} \setminus \{1\}$}
	{$\mathscr{D}=\mathbb{R} \setminus \{1;2\}$}
	\loigiai{
		Điều kiện xác định: $x-1 \neq 0 \Rightarrow x \neq 1$. \\
		Vậy tập xác định của hàm số đã cho là $\mathscr{D}=\mathbb{R} \setminus \{1\}$.
	}
\end{ex}

%Câu 4...........................
\begin{ex}%[0T3Y2-3]%[Dự án đề kiểm tra HKII NH22-23- Nguyễn Sĩ Đạt]%[THPT Yên Mô - Ninh Bình]
	Tọa độ đỉnh của Parabol $(P): y=x^2-2x+2$ là
	\choice
	{\True $I(1;1)$}
	{$I(1;-1)$}
	{$I(-1;5)$}
	{$I(2;2)$}
	\loigiai{
		Tọa độ đỉnh $I\left( -\dfrac{-2}{2 \cdot 1};-\dfrac{(-2)^2-4\cdot2}{4 \cdot 1}\right) $, hay tọa độ đỉnh của Parabol là $I(1;1)$.
	}
\end{ex}
%Câu 5...........................
\begin{ex}%[0T3B2-3]%[Dự án đề kiểm tra HKII NH22-23- Nguyễn Sĩ Đạt]%[THPT Yên Mô - Ninh Bình]
	\immini[thm]{Hàm số nào dưới đây có đồ thị là đường Parabol như trong hình bên? 
	\choice
	{$y=x-1$}
	{$y=-x-1$}
	{\True $y=2x^2-4x-1$}
	{$y=-2x^2+4x-1$}}
	{\begin{tikzpicture}[>=stealth,x=1cm,y=1cm,scale=1, font=\footnotesize]
			\def\a{2} 
			\def\b{-4}
			\def\c{-1}
			\draw[->] (-1,0) -- (3,0) node[below] {$x$}; \draw[->] (0,-4) -- (0,2.5) node[left] {$y$}; \draw (0,0)node[above right]{$O$};
			\draw[samples=150,domain=-0.5:2.5] plot(\x,{\a*(\x)^2+(\b)*\x+(\c)});
			\draw[dashed] (1,0)node[above]{$1$}|-(0,-3)node[left]{$-3$} (0,-1)node[left]{$-1$};
	\end{tikzpicture}}
	\loigiai{
		Ta có bề lõm Parabol hướng lên nên hàm số đồ thị có dạng $y=ax^2+bx+c$ với $a>0$.\\
		Vậy hàm số $y=2x^2-4x-1$ có đồ thị là đường Parabol như hình bên.
	}
\end{ex}

%Câu 6...........................
\begin{ex}%[0T7Y2-1]%[Dự án đề kiểm tra HKII NH22-23- Nguyễn Sĩ Đạt]%[THPT Yên Mô - Ninh Bình]
	Tập nghiệm của bất phương trình $2x^2-5x+2 \le 0 $
	\choice
	{$\left( -\infty;\dfrac{1}{2}\right]  \cup [2;+\infty)$}
	{\True $\left[ \dfrac{1}{2};2 \right] $}
	{$\left( \dfrac{1}{2};2 \right) $}
	{$\left( -\infty;\dfrac{1}{2}\right)  \cup (2;+\infty)$}
	\loigiai{
		Tam thức 
		$f(x)=2x^2-5x+2$ có $a=2 >0 $ và có hai nghiệm phân biệt $x_1=\dfrac{1}{2}$, $x_2=2$.\\
		Vậy $2x^2-5x+2 \le 0 \Rightarrow x\in \left[ \dfrac{1}{2};2 \right]$.
	}
\end{ex}

%Câu 7...........................
\begin{ex}%[0T7Y3-2]%[Dự án đề kiểm tra HKII NH22-23- Nguyễn Sĩ Đạt]%[THPT Yên Mô - Ninh Bình]
	Phương trình $\sqrt{3x+3}=3$ có nghiệm là:
	\choice
	{\True $x=1$}
	{ $x=2$}
	{$x=3$}
	{$x=0$}
	\loigiai{
		Ta có $\sqrt{3x+3}=3 \Rightarrow 3x+3=9 \Rightarrow x=2$.\\
		Thay $x=2$ vào phương trình thấy $x=2$ thỏa mãn.\\
		Vậy phương trình đã cho có nghiệm là $x=2$.
	}
\end{ex}

%Câu 8...........................
\begin{ex}%[0T4Y2-1]%[Dự án đề kiểm tra HKII NH22-23- Nguyễn Sĩ Đạt]%[THPT Yên Mô - Ninh Bình]
	Tính $A=\cos \alpha $ biết $\sin \alpha =\dfrac{3}{5}$ và $0<\alpha<90^\circ$
	\choice
	{$A=\dfrac{16}{25}$}
	{$A=\dfrac{2}{5}$}
	{$A=-\dfrac{4}{5}$}
	{\True $A=\dfrac{4}{5}$}
	\loigiai{
		Ta có $\cos^2{\alpha}+\sin^2{\alpha}=1 \Rightarrow \cos^2{\alpha}+\left(\dfrac{3}{5} \right)^2 = 1 \Rightarrow \cos^2{\alpha}=\dfrac{16}{25} \Rightarrow     
		\hoac{& \cos{\alpha}= \dfrac{4}{5}\\& \cos{\alpha}=-\dfrac{4}{5}.}
		$\\
		Do $0<\alpha<90^\circ$ nên $\cos{\alpha}= \dfrac{4}{5}$.
	}
\end{ex}
%Câu 9...........................
\begin{ex}%[0T4Y3-1]%[Dự án đề kiểm tra HKII NH22-23- Nguyễn Sĩ Đạt]%[THPT Yên Mô - Ninh Bình]
	Cho tam giác $ABC$. Khẳng định nào sau đây \textbf{sai}?
	\choice
	{$\dfrac{a}{\sin A}=2R$}
	{$a^2=b^2+c^2-2bc\cos A$}
	{$S=\dfrac{1}{2}ab \sin C$}
	{\True $S=\dfrac{abc}{R}$}
	\loigiai{
		Ta có công thức tính diện tích tam giác $S=\dfrac{abc}{4R}$.		
	}
\end{ex}

%Câu 10...........................
\begin{ex}%[0T9B1-1]%[Dự án đề kiểm tra HKII NH22-23- Nguyễn Sĩ Đạt]%[THPT Yên Mô - Ninh Bình]
	Trong hệ trục $Oxy$, cho tam giác $ABC$ biết $A(1;-4), B(2;1), C(3;0)$. Tìm tọa độ trọng tâm $G$ tam giác $ABC$?
	\choice
	{$G(6;-3)$}
	{$G(6;3)$}
	{\True $G(2;-1)$}
	{$G(2;1)$}
	\loigiai{
		Do $G$ là trọng tâm tam giác $ABC$ nên $\heva{&x_G=\dfrac{1+2+3}{3}=2 \\&
			y_G=\dfrac{(-4)+1+0}{3}=-1.}$\\
		Vậy $G(2;-1)$.}
\end{ex}

%Câu 11...........................
\begin{ex}%[0T9Y1-1]%[Dự án đề kiểm tra HKII NH22-23- Nguyễn Sĩ Đạt]%[THPT Yên Mô - Ninh Bình]
	Trong mặt phẳng tọa độ $Oxy$, cho $A(1;-2), B(3;2)$. Tính độ dài đoạn thẳng $AB$?
	\choice
	{$AB=2$}
	{$AB=\sqrt{5}$}
	{$AB=5$}
	{\True $AB=2\sqrt{5}$}
	\loigiai{
		Ta có $AB=\sqrt{(3-1)^2+[2-(-2)]^2}=2\sqrt{5}$.}
\end{ex}

%Câu 12...........................
\begin{ex}%[0T5Y4-1]%[Dự án đề kiểm tra HKII NH22-23- Nguyễn Sĩ Đạt]%[THPT Yên Mô - Ninh Bình]
	Trong hệ trục $Oxy$, cho $\vec{a}=(2;-3)$, $\vec{b}=(1;4)$. Tính $\vec{a}\cdot\vec{b}$?
	\choice
	{$\vec{a}\cdot\vec{b}=14$}
	{$\vec{a}\cdot\vec{b}=-5$}
	{$\vec{a}\cdot\vec{b}=10$}
	{\True $\vec{a}\cdot\vec{b}=-10$}
	\loigiai{
		Ta có $\vec{a}\cdot\vec{b}=2\cdot1+(-3)\cdot4=-10$.}
\end{ex}
%Câu 13...........................
\begin{ex}%[0T9Y2-2]%[Dự án đề kiểm tra HKII NH22-23- Nguyễn Sĩ Đạt]%[THPT Yên Mô - Ninh Bình]
	Trong mặt phẳng tọa độ $Oxy$, đường thẳng $\Delta$ đi qua $M(1;-3)$ và có hệ số góc $k=2$. Phương trình của đường thẳng $\Delta$ là
	\choice{$y=2x+1$}
	{$y=2x+5$}
	{$y=2x-1$}
	{\True $y=2x-5$}
	\loigiai{
		Vì $\Delta$ có hệ số góc $k=2$ nên $\Delta$ có dạng $y=2x+c$.\\
		Ta có $M(1;-3) \in \Delta \Rightarrow -3 = 2 \cdot 1 +c \Rightarrow c=-5$.\\
		Vậy phương trình đường thẳng $\Delta$ là $y=2x-5$.
	}
\end{ex}

%Câu 14...........................
\begin{ex}%[0T9B2-3]%[Dự án đề kiểm tra HKII NH22-23- Nguyễn Sĩ Đạt]%[THPT Yên Mô - Ninh Bình]
	Trong hệ trục $Oxy$, cho hai đường thẳng $d_1\colon y=mx+m-1$, $d_2\colon y=\dfrac{1}{4}x+1$. Tìm $m$ sao cho $d_1 \perp d_2$?
	\choice{\True $m=-4$}
	{$m=-1$}
	{$m=1$}
	{$m=4$}
	\loigiai{
		Ta có $d_1 \perp d_2$ nên $m \cdot \dfrac{1}{4} = -1$ hay $m=-4$.
	}
\end{ex}

%Câu 15...........................
\begin{ex}%[0T9Y2-1]%[Dự án đề kiểm tra HKII NH22-23- Nguyễn Sĩ Đạt]%[THPT Yên Mô - Ninh Bình]
	Trong hệ trục $Oxy$, cho đường thẳng $d$ có phương trình tham số $\heva{& x=2+3t\\& y=-1+t} (t \in \mathbb{R})$. Vectơ nào sau đây là một vectơ chỉ phương của đường thẳng $d$?
	\choice
	{$\Vec{v}_1=(2;-1)$}
	{\True $\Vec{v}_2=(3;1)$}
	{$\Vec{v}_3=(3;-1)$}
	{$\Vec{v}_4=(1;-3)$}
	\loigiai{
		Ta có $\Vec{v_2}=(3;1)$ là một vectơ chỉ phương của đường thẳng $d\colon \heva{& x=2+3t\\& y=-1+t} (t \in \mathbb{R})$.
	}
\end{ex}

%Câu 16...........................
\begin{ex}%[0T9Y2-5]%[Dự án đề kiểm tra HKII NH22-23- Nguyễn Sĩ Đạt]%[THPT Yên Mô - Ninh Bình]
	Trong hệ trục $Oxy$, khoảng cách từ điểm $M\left(1; 2\right)$ đến đường thẳng $\Delta\colon 3x+y+5=0$ bằng
	\choice
	{$5$}
	{ $\sqrt{5}$}
	{$10$}
	{\True$\sqrt{10}$}
	\loigiai{
		Ta có $d\left(M; \Delta\right)=\dfrac{|3\cdot 1+2+5|}{\sqrt{3^2+1^2}}=\sqrt{10}$. 
	}
\end{ex}
%Câu 17...........................
\begin{ex}%[0T9Y3-1]%[Dự án đề kiểm tra HKII NH22-23- Nguyễn Sĩ Đạt]%[THPT Yên Mô - Ninh Bình]
	Trong hệ trục $Oxy$, cho đường tròn $(C)\colon x^2+y^2-8x+2y+1=0$. Tìm tọa độ tâm $I$ và bán kính $R$ của đường tròn $(C)$?
	\choice
	{\True $I(4; -1)$, $R=4$}
	{ $I(-4; 1)$, $R=4$}
	{$I(-4; 1)$, $R=16$}
	{$I(4; -1)$, $R=16$}
	\loigiai{
		Ta có $a=\dfrac{-8}{-2}=4$, $b=\dfrac{2}{-2}=-1$, $ c=1$.\\
		Do đó tâm $I(4; -1)$ và $R=\sqrt{a^2+b^2-c}=\sqrt{4^2+(-1)^2-1}=4$.
	}
\end{ex}

%Câu 18...........................
\begin{ex}%[0T9B3-3]%[Dự án đề kiểm tra HKII NH22-23- Nguyễn Sĩ Đạt]%[THPT Yên Mô - Ninh Bình]
	Trong hệ trục $Oxy$, cho đường tròn $(C)\colon \left(x+1\right)^2+\left(y-2\right)^2=5$. Phương trình tiếp tuyến của đường tròn $(C)$ tại điểm $M(-2; 4)$ là 
	\choice
	{$-x+2y+6=0$}
	{ $2x+y=0$}
	{\True$x-2y+10=0$}
	{$x-2y-10=0$}
	\loigiai{
		Đường tròn $(C)$ có tâm $I(-1; 2)$.\\
		Ta có $\Vec{IM}=(-1; 2)$.\\
		Tiếp tuyến của đường tròn $(C)$ tại điểm $M(-2; 4)$ có phương trình là 
			\begin{center}
				$-1(x+2)+2(y-4)=0$ 
			hay $x-2y+10=0$. 
			\end{center}
	}
\end{ex}

%Câu 19...........................
\begin{ex}%[Dự án đề kiểm tra HKII NH22-23- Nguyễn Sĩ Đạt]%[THPT Yên Mô - Ninh Bình]
	Trong hệ trục tọa độ $Oxy$, cho elip $\dfrac{x^2}{13}+\dfrac{y^2}{9}=1$. Tiêu cự của elip là
	\choice
	{$F_1F_2=2$}
	{\True $F_1F_2=4$}
	{$F_1F_2=3$}
	{$F_1F_2=6$}
	\loigiai{
		Elip $\dfrac{x^2}{13}+\dfrac{y^2}{9}=1$ có $a^2=13$, $b^2=9$.\\
		Suy ra $c=\sqrt{a^2-b^2}=2$.\\
		Vậy $F_1F_2=2c=4$.
	}
\end{ex}

%Câu 20...........................
\begin{ex}%[0T9Y4-2]%[Dự án đề kiểm tra HKII NH22-23- Nguyễn Sĩ Đạt]%[THPT Yên Mô - Ninh Bình]
	Trong hệ trục tọa độ $Oxy$, cho elip $(E)$, biết một tiêu điểm là $F_2(3;0)$ và đi qua điểm $A(5;0)$. Phương trình chính tắc của elip $(E)$ là
	\choice
	{$\dfrac{x^2}{5}+\dfrac{y^2}{3}=1$}
	{$\dfrac{x^2}{5}+\dfrac{y^2}{4}=1$}
	{\True $\dfrac{x^2}{25}+\dfrac{y^2}{16}=1$}
	{$\dfrac{x^2}{25}+\dfrac{y^2}{9}=1$}
	\loigiai{
		Elip $(E)$ có phương trình chính tắc là $\dfrac{x^2}{a^2}+\dfrac{y^2}{b^2}=1$ $(a>b>0)$.\\
		Vì $F_2(3;0)$ là tiêu điểm của $(E)$ nên $c=3$.\\
		Vì $A(5;0) \in (E)$ nên $\dfrac{5^2}{a^2}=1$. Suy ra $a=5$ (do $a>0$).\\
		Suy ra $b=\sqrt{a^2-c^2}=4$.\\
		Vậy $(E)\colon \dfrac{x^2}{25} + \dfrac{y^2}{16}=1$.
	}
\end{ex}
%Câu 21...........................
\begin{ex}%[0T9Y4-5]%[Dự án đề kiểm tra HKII NH22-23- Nguyễn Sĩ Đạt]%[THPT Yên Mô - Ninh Bình]
	Trong hệ trục $Oxy$, phương trình nào sau đây là phương trình chính tắc của đường Hypebol?
	\choice
	{$\dfrac{x^2}{9}+\dfrac{y^2}{4}=1$}
	{\True $\dfrac{x^2}{9}-\dfrac{y^2}{4}=1$}
	{$x^2+y^2=1$}
	{$y^2=12x$}
	\loigiai
	{
		Phương trình chính tắc của Hypebol có dạng $\dfrac{x^2}{a^2}-\dfrac{y^2}{b^2}=1$, trong đó $b=\sqrt{c^2-a^2}$.
	}
\end{ex}

%Câu 22...........................
\begin{ex}%[0T8Y2-1]%[Dự án đề kiểm tra HKII NH22-23- Nguyễn Sĩ Đạt]%[THPT Yên Mô - Ninh Bình]
	Có bao nhiêu cách xếp $10$ người thành một hàng có thứ tự?
	\choice
	{$10$}
	{\True $10!$}
	{$2^{10}$}
	{$10^{10}$}
	\loigiai{
		Mỗi cách sắp xếp $10$ người theo một thứ tự là một hoán vị, do đó số cách sắp xếp là $10!$.
	}
\end{ex}

%Câu 23...........................
\begin{ex}%[0T8Y2-1]%[Dự án đề kiểm tra HKII NH22-23- Nguyễn Sĩ Đạt]%[THPT Yên Mô - Ninh Bình]
	Cho tập hợp $A$ có $12$ phần tử. Số tập con gồm $5$ phần tử của $A$ là
	\choice
	{$5^{12}$}
	{$12^5$}
	{\True $\mathrm{C}_{12}^5$}
	{$A_{12}^5$}
	\loigiai{
		Mỗi cách chọn $5$ phần tử trong $12$ phần tử được gọi là một tổ hợp chập $5$ của $12$, do đó số tập hợp con là $\mathrm{C}_{12}^{5}$.
	}
\end{ex}

%Câu 24...........................
\begin{ex}%[0T8B2-1]%[Dự án đề kiểm tra HKII NH22-23- Nguyễn Sĩ Đạt]%[THPT Yên Mô - Ninh Bình]
	Cho tập hợp $A=\{1 ; 2 ; 3 ; 4 ; 5 ; 6\}$. Có bao nhiêu số tự nhiên có ba chữ số khác nhau mà các chữ số thuộc $A$?
	\choice
	{\True $120$}
	{$20$}
	{$216$}
	{$150$}
	\loigiai{
		Gọi số cần tìm có dạng $\overline{abc}$.\\
		Mỗi cách chọn $3$ số trong $6$ số rồi xếp vào $3$ vị trí gọi là một chỉnh hợp chập $3$ của $6$, do đó có $\mathrm{A}_{6}^{3}=120$ số.
	}
\end{ex}

%Câu 25...........................
\begin{ex}%[0T8B2-2]%[Dự án đề kiểm tra HKII NH22-23- Nguyễn Sĩ Đạt]%[THPT Yên Mô - Ninh Bình]
Một lớp có $20$ nam và $22$ nữ. Số cách chọn ra hai bạn bất kì trong lớp là
\choice
{$440$}
{$\mathrm{C}^2_{20}.\mathrm{C}^2_{22}$}
{$\mathrm{C}^2_{20}+\mathrm{C}^2_{22}$}
{\True $\mathrm{C}^2_{42}$}
\loigiai
{
	Tổng số học sinh trong lớp là $20+22=42$ học sinh. \\
	Số cách chọn ra hai bạn bất kì trong lớp là số tổ hợp chập $2$ của $42$. \\
	Vậy số cách chọn ra hai bạn bất kì trong lớp là $\mathrm{C}^2_{42}$.
}	
\end{ex}
	
\begin{ex}
	Khai triển đúng của $\left( x+3 \right)^4$ là
	\choice
	{$x^4+4x^3+6x^2+4x+1$}
	{$x^4+3x^3+9x^2+27x+81$}
	{\True $x^4+12x^3+54x^2+108x+81$}
	{$x^4+12x^3+54x^2+27x+81$}
	\loigiai{
	Ta có \allowdisplaybreaks
	$ \begin{aligned}[t]
		\left( x+3 \right)^4&=\mathrm{C}^0_4\cdot x^4+\mathrm{C}^1_4\cdot x^3\cdot 3+\mathrm{C}^2_4\cdot x^2\cdot 3^2+\mathrm{C}^3_4\cdot x\cdot 3^3+\mathrm{C}^4_4\cdot 3^4 \\
		&= x^4+12x^3+54x^2+108x+81.
	\end{aligned} $}
\end{ex}

%Câu 27...........................
\begin{ex}%[0T0B2-2]%[Dự án đề kiểm tra HKII NH22-23- Nguyễn Sĩ Đạt]%[THPT Yên Mô - Ninh Bình]
	Một tổ có $6$ nam và $7$ nữ. Chọn ngẫu nhiên từ tổ ra $5$ bạn, xác suất sao cho $5$ bạn được chọn có $2$ nam và $3$ nữ là
	\choice
	{$\dfrac{50}{1287}$}
	{\True $\dfrac{175}{429}$}
	{$\dfrac{140}{429}$}
	{$\dfrac{100}{429}$}
	\loigiai{
		Gọi $\Omega$ là không gian mẫu. Số phần tử của không gian mẫu $n(\Omega)=\mathrm{C}_{13}^5=1287$.\\
		Gọi $A$ là biến cố: \lq\lq Trong số $5$ bạn được chọn có $2$ nam và $3$ nữ.\rq\rq \\
		Khi đó $n(A)=\mathrm{C}_{6}^2 \cdot \mathrm{C}_{7}^3 =525$.\\
		Vậy $\mathrm{P}(A)=\dfrac{n(A)}{n(\Omega)}=\dfrac{525}{1287}=\dfrac{175}{429}$.
	}
\end{ex}
%Câu 28...........................
\begin{ex}%[0T5B4-1]%[Dự án đề kiểm tra HKII NH22-23- Nguyễn Sĩ Đạt]%[THPT Yên Mô - Ninh Bình]
	Cho hình vuông $ABCD$ có cạnh bằng $a$. Tính $\vec{AB}\cdot \vec{AC}$?
	\choice
	{$\vec{AB}\cdot \vec{AC} = \dfrac{a^2 \sqrt{2}}{2}$}
	{$\vec{AB}\cdot \vec{AC} = a^2 \sqrt{2}$}
	{\True $\vec{AB}\cdot \vec{AC} = a^2$}
	{$\vec{AB}\cdot \vec{AC} = 2a^2$}	
	\loigiai{
		$ABCD$ là hình vuông nên $AC=a\sqrt{2}$.\\
		Ta có $\vec{AB}\cdot \vec{AC} = AB\cdot AC \cdot \cos BAC = a \cdot a\sqrt{2} \cdot \cos 45^\circ = a^2$.}
\end{ex}

%Câu 29...........................
\begin{ex}%[0T9B1-1]%[Dự án đề kiểm tra HKII NH22-23- Nguyễn Sĩ Đạt]%[THPT Yên Mô - Ninh Bình]
	Trong hệ trục $Oxy$, cho điểm $A(4;1)$ và đường thẳng $\Delta \colon \heva{&x = 2t\\& y = 3 + t &}$. Điểm $M$ thuộc $\Delta$ sao cho $AM = 5$. Biết $M(a;b)$ với $a>0$, tính $S = 12a + 5b$?
	\choice
	{$S=7$}
	{$S=17$}
	{\True $S=37$}
	{$S=65$}
	\loigiai{
		Ta có $\Delta\colon 2x-y+3=0$ và $M(a;b)$ thuộc $\Delta$ nên $M(a;2a+3)$.\\
		Vì $AM=5$ nên $AM^2 = 5^2 
		\Leftrightarrow (a-4)^2+(2a+2)^2=25 
		\Leftrightarrow a=1$. \\
		Suy ra $b=5$. \\
		Vậy $S=12\cdot 1+5\cdot 5 =37$.}
\end{ex}

%Câu 30...........................
\begin{ex}%[0T9K1-1]%[Dự án đề kiểm tra HKII NH22-23- Nguyễn Sĩ Đạt]%[THPT Yên Mô - Ninh Bình]
	 Trong hệ trục $Oxy$, cho tam giác $ABC$ biết $A(0;1), B(1;3), C(2;7)$. Diện tích tam giác $ABC$ bằng
	 \choice
	 {\True $1$}
	 {$2$}
	 {$4$}
	 {$8$}
	\loigiai{
		Diện tích tam giác $ABC$ bằng $S= \dfrac{1}{2}\cdot\left| (1-0)\cdot(7-1)-(2-0)(3-1)\right|=1$.
	}
\end{ex}


\begin{ex}%[0K9BQ-2]
	Chọn ngẫu nhiên một số từ tập hợp các số tự nhiên thuộc đoạn $[30;50]$. Xác suất để chọn được số có chữ số hàng đơn vị lớn hơn hàng chục bằng
	\choice
	{$\dfrac{13}{21}$}
	{\True $\dfrac{11}{21}$}
	{$\dfrac{10}{21}$}
	{$\dfrac{8}{21}$}
	\loigiai{
		Phép thử $T\colon$ \lq\lq Chọn ngẫu nhiên một số từ tập hợp các số tự nhiên thuộc đoạn $[30;50]$\rq\rq, $n(\Omega)=21$.\\
		Xét biến cố $A\colon$ \lq\lq Chọn được số có chữ số hàng đơn vị lớn hơn hàng chục\rq\rq, $n(A)=11$.\\
		Xác suất cần tìm $\mathrm{P}(A)=\dfrac{n(A)}{n(\Omega)}=\dfrac{11}{21}$.
	}
\end{ex} 

\begin{ex}%[0K4BK-5]
	Trong hệ trục $Oxy$, cho đường tròn $(C)\colon x^{2}+y^{2}-6x+2y+1=0$ và điểm $M(1; 0)$. Biết đường thẳng $\Delta$ qua $M$ luôn cắt $(C)$ tại hai điểm phân biệt $A$, $B$. Dây cung $AB$ có độ dài nhỏ nhất bằng
	\choice
	{$\sqrt{5}$}
	{$2\sqrt{5}$}
	{$2$}
	{\True $4$}
	\loigiai{
		Đường tròn $(C)$ có tâm $I(3;-1)$ và bán kính $R=3$.\\
		Kẻ $IH\perp AB$ tại $H$. Ta có $IH\le IM$.\\
		Có $AB^2=4\left(R^2-IH^2\right)\ge 4\left(R^2-IM^2\right)=4\left[3^2-\left(\sqrt{5}\right)^2\right]=16$. Do đó $\min AB=4$.
	}
\end{ex} 

\begin{ex}%[0K9BQ-3]
	Giải bóng đá Nam SEA Games $32$ được tổ chức tại Campuchia có $10$ đội bóng tham dự, trong đó có Việt Nam và Thái Lan. Ban tổ chức bốc thăm một cách ngẫu nhiên $10$ đội chia thành hai bảng $A$, $B$; mỗi bảng $5$ đội. Xác suất để Việt Nam và Thái Lan nằm ở cùng một bảng đấu bằng
	\choice
	{$\dfrac{1}{3}$}
	{$\dfrac{2}{9}$}
	{\True $\dfrac{4}{9}$}
	{$\dfrac{5}{9}$}
	\loigiai{
		Phép thử $T\colon$ \lq\lq Chia $10$ đội thành hai bảng\rq\rq, $n(\Omega)=\mathrm{C}_{10}^5$.\\
		Xét biến cố $A\colon$ \lq\lq Đội Việt Nam và Thái Lan nằm cùng một bảng đấu\rq\rq, $n(A)=\mathrm{C}_{2}^1\cdot \mathrm{C}_{8}^3$.\\
		Xác suất cần tìm là $\mathrm{P}(A)=\dfrac{n(A)}{n(\Omega)}=\dfrac{4}{9}$.
	}
\end{ex} 

\begin{ex}%[0K4BK-3]
	Trong hệ trục $Oxy$, cho điểm $A(1;-1)$ và đường tròn $(C)\colon (x+1)^{2}+(y-2)^{2}=4$ có tâm $I$. Biết từ $A$ kẻ được hai tiếp tuyến $AP$, $AQ$ với $(C)$, ($P$, $Q$ là các tiếp điểm). Tứ giác $APIQ$ có diện tích bằng
	\choice
	{\True $6$}
	{$3$}
	{$4$}
	{$2$}
	\loigiai{
		Đường tròn $(C)$ có tâm $I(-1;2)$, bán kính $R=2$.\\
		$\triangle API$ vuông tại $P$, $IP=R=2$, $AI=\sqrt{13}$ suy ra $AP=\sqrt{AI^2-IP^2}=3$.\\
		Ta có $S_{APIQ}=2S_{API}=AP\cdot IP=3\cdot 2=6$.
	}
\end{ex} 

\begin{ex}%[0K9BQ-3]
	Cho tập $A=\{1; 2; 3; \ldots; 30\}$ (tập gồm $30$ số nguyên dương đầu tiên). Lấy ngẫu nhiên hai số phân biệt từ tập $A$. Xác suất để tích hai số được chọn chia hết cho $10$ bằng
	\choice
	{$\dfrac{28}{145}$}
	{\True $\dfrac{8}{29}$}
	{$\dfrac{1}{5}$}
	{$\dfrac{1}{145}$}
	\loigiai{
		Phép thử $T\colon$ \lq\lq Lấy ngẫu nhiên hai số phân biệt từ tập $A$\rq\rq, $n(\Omega)=\mathrm{C}_{30}^2$.\\
		Xét biến cố $A\colon$ \lq\lq Tích hai số được chọn chia hết cho $10$\rq\rq, $n(A)=\mathrm{C}_{3}^2 + \mathrm{C}_{3}^1\cdot \mathrm{C}_{27}^1 + \mathrm{C}_{3}^1\cdot \mathrm{C}_{12}^1$.\\
		Xác suất cần tìm là $\mathrm{P}(A)=\dfrac{n(A)}{n(\Omega)}=\dfrac{8}{29}$.
	}
\end{ex} 



\Closesolutionfile{ans}
%\begin{center}
%	\textbf{ĐÁP ÁN}
%	\inputansbox{10}{ans/ans}	
%\end{center}


\begin{center}
	\textbf{PHẦN 2 - TỰ LUẬN}
\end{center}


\begin{bt}%[0,5 điểm]%[0K6BG-2]
	Cho biểu thức $f(x)=x^{2}+2x-3m+4$. Tìm tất cả các giá trị của tham số $m$ sao cho $f(x)\geq 0, \forall x \in \mathbb{R}$.
	\loigiai{
		Ta có $f(x)\geq 0, \forall x \in \mathbb{R} \Leftrightarrow \Delta'=1^2-1\cdot(-3m+4)\le 0 \Leftrightarrow m\le 1$.
	}
\end{bt} 

\begin{bt}[0,5 điểm]%[0K8BO-2]
	Tìm hệ số của $x^{3}$ trong khai triển của $(3x+4)^{5}$.
	\loigiai{
		Ta có $(3x+4)^{5}=\mathrm{C}_{5}^0\cdot (3x)^5+\mathrm{C}_{5}^1\cdot (3x)^4\cdot 4+\mathrm{C}_{5}^2\cdot (3x)^3\cdot 4^2+\mathrm{C}_{5}^3\cdot (3x)^2\cdot 4^3+\mathrm{C}_{5}^4\cdot(3x)\cdot 4^4+\mathrm{C}_{5}^5\cdot 4^5$.\\
		Vậy hệ số của $x^{3}$ trong khai triển là $\mathrm{C}_{5}^2\cdot (3)^3\cdot 4^2=4320$.
	}
\end{bt} 

\begin{bt}%[1,0 điểm]%[0K4BI-2]%[0K4BK-2]
	Trong hệ trục $Oxy$, cho hai điểm $A(1; 2)$, $B(3;-4)$.
	\begin{enumerate}[a)]
		\item Lập phương trình tham số của đường thẳng $AB$.
		\item Lập phương trình đường tròn $(C)$ có đường kính $AB$.
	\end{enumerate}
	\loigiai{
		\begin{enumerate}[a)]
			\item Ta có $\overrightarrow{AB}=(2;-6)$, chọn $\overrightarrow{u}=(1;-3)$ là véc-tơ chỉ phương của đường thẳng $AB$.\\
				Phương trình tham số của $AB\colon \heva{&x=1+t\\&y=2-3t} \quad (t\in \mathbb{R})$.
			\item Đường tròn $(C)$ có đường kính $AB$ có tâm $I(2;-1)$ và bán kính $R=\sqrt{10}$.\\
				Vậy phương trình đường tròn $(C)$ có đường kính $AB$ là $(x-2)^2+(y+1)^2=10$.
		\end{enumerate}
	}
\end{bt} 

\begin{bt}%[1,0 điểm]%[0K9BQ-3]
	Tại một phòng thi chọn học sinh giỏi lớp $10$ cấp trường có $24$ thí sinh, trong đó có $14$ học sinh thi môn Toán (gồm $8$ nam và $6$ nữ) và $10$ học sinh thi môn Văn toàn là nữ, mỗi thí sinh chỉ thi một môn. Xét phép thử: Giám thị chọn ngẫu nhiên $3$ học sinh trong phòng để vệ sinh phòng thi.
	\begin{enumerate}[a)]
		\item Tính $n(\Omega)$ và tính xác suất biến cố $A\colon$ \lq\lq Ba học sinh cùng thi môn Toán\rq\rq.
		\item Tính xác suất biến cố $B\colon$ \lq\lq Ba học sinh được chọn có cả học sinh thi Toán, có cả học sinh thi Văn đồng thời có cả nam và nữ\rq\rq.
	\end{enumerate}
	\loigiai{
		\begin{enumerate}[a)]
			\item Ta có $n(\Omega)=\mathrm{C}_{24}^{3}$.\\
				Biến cố $A\colon$ \lq\lq Ba học sinh cùng thi môn Toán\rq\rq, $n(A)=\mathrm{C}_{14}^{3}$.\\
				Xác suất cần tìm là $\mathrm{P}(A)=\dfrac{n(A)}{n(\Omega)}=\dfrac{\mathrm{C}_{14}^{3}}{\mathrm{C}_{24}^{3}}=\dfrac{91}{506}$.
			\item Biến cố $B\colon$ \lq\lq Ba học sinh được chọn có cả học sinh thi Toán, có cả học sinh thi Văn đồng thời có cả nam và nữ\rq\rq.\\
				Ta có $n(B)=\mathrm{C}_{10}^1\cdot \mathrm{C}_{8}^2 + \mathrm{C}_{10}^1\cdot \mathrm{C}_{8}^1\cdot \mathrm{C}_{6}^1 + \mathrm{C}_{10}^2\cdot \mathrm{C}_{8}^1$.\\
				Xác suất cần tìm là $\mathrm{P}(B)=\dfrac{n(B)}{n(\Omega)}=\dfrac{140}{253}$.
		\end{enumerate}
	}
\end{bt} 

